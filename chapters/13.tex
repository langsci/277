\documentclass[output=paper]{langsci/langscibook}
\ChapterDOI{10.5281/zenodo.4680302}
\author{Artemis Alexiadou\affiliation{Humboldt-Universität zu
    Berlin \& Leibniz-Zentrum Allgemeine Sprachwissenschaft} and Elena
Anagnostopoulou\affiliation{University of Crete}}
\title{Rethinking the nature of nominative case}

\abstract{In this squib, we investigate the nature of nominative\is{nominative
    case} and accusative case in \ili{Greek} from a cross-linguistic
    perspective in the light of recent discussion on the modes of case
    assignment, see \textcite{Baker2008,Baker2015}, \citet{Bobaljik2008},
    \citet{Zeijlstra2012}, \citet{Preminger2014} among others. We focus on
    \citeauthor{Baker2008}'s (\citeyear{Baker2008,Baker2015}) typology of Case
    and Agreement systems asking the question of where \ili{Greek} is situated
    in this typology. We argue that while accusative (\Acc) fits in the system,
    qualifying as \isi{dependent case} on the basis of \citegen{Baker2015}
    criteria, nominative\is{nominative case} (\Nom) is more problematic. On the
    one hand, \ili{Greek} \Nom{} behaves like unmarked case and is clearly not
    assigned under agreement with T in a number of environments. On the other
    hand, however, agreement always goes with \Nom{} when both are present.
    Crucially, the language pervasively shows long-distance chains involving a
    single in situ \Nom{} subject and many T heads fully agreeing with it. This
    is incompatible with \citegen{Baker2008} agreement and case typology.
    Building on \citet{AleAna1998}, we suggest that \ili{Greek} has T with
    interpretable \isi{φ-features} as a by-product of V \isi{raising}
    satisfying the \glsunset{EPP}\gls{EPP}.  This allows for the formation of
    long-distance chains between a single DP bearing unmarked \Nom{} and many
    fully agreeing Ts.  Turning to the question of why agreement always goes
    with \Nom{} in \ili{Greek}, this is compatible with the view that agreement
    is sensitive to unmarked case argued for by \citet{Bobaljik2008},
    \citet{Preminger2014}, \citet{Baker2015} and others.  We adopt this
    proposal and argue that the analysis of \ili{Greek} \Nom{} case in
    connection to agreement requires a separation of interpretability from
    valuation (\citealt{PesetskyTorrego2007}).  Finally, we address the
    implications of our proposal for the theory of pro and compare our analysis
to the Agree theory of pro proposed by \textcite{Roberts2010c,Roberts2010} and
\citet{Holmberg2010}.}


\begin{document}\glsresetall
\maketitle

\section{Introduction}

As is well known, there are two influential views \emph{on Case assignment}:
under view \eqref{ex:13.1}, all structural Case is assigned by functional
heads via Agreement \citep{Chomsky2001}. Under view \eqref{ex:13.2},
structural Case is assigned by the principles of \isi{dependent case}
assignment (\citealt{Marantz1991} and many others building on him).

On the nominative\is{nominative case} under \isi{Agree} perspective, an NP has nominative\is{nominative case} case (\Nom) if
and only if it is assigned that case by a T-like functional head that enters
into \isi{Agree} with it, see \eqref{ex:13.1} from \citet{Baker2015}, but cf.
\textcite{Sigurdsson2000}, who argues for a \emph{v}P based approach.

\ea\label{ex:13.1} Overt NP X has nominative\is{nominative case} case if and only if exactly
    one verbal form in the clause containing X agrees with it.
\z

On Case assignment under the principles of \isi{dependent case}, the situation is
different. \citet{Marantz1991} argues that the distribution of morphological
case is determined at PF, subject to the case realization hierarchy in
\REF{ex:13.2}:

\ea%2
    \label{ex:13.2} Case realization disjunctive hierarchy:\\
    (i) lexically governed case, (ii) \enquote{dependent} case (accusative and
    ergative), (iii) unmarked case (environment-sensitive), (iv) default case
\z

A lot of later literature has adopted the view that case distribution is
subject to \eqref{ex:13.2}, without necessarily also adhering to the view
that case realization is determined at PF (see e.g.
\citealt{Preminger2014,Baker2015} who argue that \eqref{ex:13.2} applies in
syntax). In this system, structural accusative and ergative is “dependent case”
subject to the definition in \eqref{ex:13.3}, from
\citet[74]{Baker2015}:\footnote{The domain is taken to involve two NPs within
    the same TP. See \citet{Schafer2012} for arguments that it involves NPs
within the same \emph{v}P.}

\ea%3
    \label{ex:13.3}
    \ea If NP\textsubscript{1} c-commands NP\textsubscript{2} and both are in
        the same domain, value NP\textsubscript{1}’s case as ergative.
    \ex If NP\textsubscript{1} c-commands NP\textsubscript{2} and both are in
        the same domain, value NP\textsubscript{2}’s case as accusative.
    \ex If NP has no other case feature, value its case as
        nominative/absolutive.
    \z
\z

Nominative/absolutive is unmarked/default in the verbal domain, while genitive\is{genitive case}
is unmarked/default in the nominal domain.

\citet{Baker2015} puts forth a typology of Case assignment, according to which,
case is not always assigned by \isi{Agree}, rather some structural Case is
assigned on the basis of the principles of \isi{dependent case}. From this
\enquote{mixed case} perspective, agreement (Agree) can assign case or
agreement is independent of case (see also \citealt{Baker2008} on the
relationship between case and agreement, and the discussion
below).\footnote{There is a third option, namely that nominative\is{nominative
case} (and perhaps also ergative/ accusative and perhaps also dative\is{dative case}, depending
on the language) \enquote{activates} a DP for agreement, i.e.\  agreement comes
after case \citep{Bobaljik2008}, we will come back to this.}

On the basis of the criteria discussed in \textcite{Baker2008,Baker2015}, it is
not immediately evident what the status of nominative\is{nominative case} is in \ili{Greek}, while it is
clear that accusative is \isi{dependent case}. In this squib, we will address the
following questions:

\begin{itemize}

    \item[(i)] What is the status of nominative\is{nominative case} and accusative in \ili{Greek}, and
        how does it pattern with or differ from other languages?

    \item[(ii)] If nominative\is{nominative case} is unmarked in the language
        and hence dissociated from \isi{Agree}, as evidenced from long-distance
        dependencies, among other properties, then why does agreement only go
        with nominative\is{nominative case} and never with some other case or
        category?

\end{itemize}

The squib is structured as follows: in~\Cref{sec:13.2,sec:13.3}, we
present Baker’s criteria to determine the two modes of Case assignment,
\isi{Agree} vs.\ \isi{dependent case}. In~\Cref{sec:13.4}, we apply these
criteria to \ili{Greek}.  In~\Cref{sec:13.4}, we address the issue of
parametric variation with respect to nominative\is{nominative case} case
assignment.

\section{Principles of Case assignment}\label{sec:13.2}

\subsection{Case under Agree}\label{sec:13.2.1}

\textcite[29f.]{Baker2015} provides evidence from Sakha that
nominative\is{nominative case} is assigned under \isi{Agree}. On this view,
agreement and nominative\is{nominative case} are two sides of the same coin, as
proposed in \citet{Chomsky2001}. The following environments make a clear case
for \Nom{} under \isi{Agree} assignment in Sakha. First, as shown in
\REF{ex:13.4}, we find an overt nominative\is{nominative case} subject when
the verb bears agreement, but not otherwise.

\ea%4
    \label{ex:13.4} \ili{Sakha} \parencite[29]{Baker2015}
	\ea
	\gll  Masha aqa-ta kinige-ni atyylas-ta.\\
	    Masha father-\Tsg.\Poss{}.\Nom{}  book-\Acc{}  buy-\Pst{}.\Tsg.\Sbj{}\\
	\glt     ‘Masha’s father bought the book.’
	\ex
	\gll  Uol uonna kyys kuorak-ka bar-dy-lar.\\
        boy and girl town-\Dat{}  go-\Pst{}-\Tpl.\Sbj{}\\
	\glt     ‘The boy and the girl went to the town.’
    \z
\z

As Baker points out, there are clause types in which agreement with the
subject\is{agreement!subject agreement}
is disrupted. This is the case in \isi{relative clauses} in \ili{Sakha}, which are formed
by using one of the participial forms available in the language preceding a
head noun. Importantly, the participle cannot \isi{Agree} with the subject, as shown
in \eqref{ex:13.5}.

\ea%5
    \label{ex:13.5}\ili{Sakha} \parencite[30]{Baker2015}\\
    \gll \llap{*}Masha cej ih-er-e caakky\\
        Masha tea drink-\Aor{}-\Tsg{} cup\\
    \glt ‘a cup that Masha drinks tea from’
\z

In order to construct a grammatical variant of \eqref{ex:13.5}, according to
Baker, one option is that the head noun of the relative clause (not the
participle) agrees with the subject of the relative clause, as in
\REF{ex:13.6}.\is{relative clauses}

\ea%6
    \label{ex:13.6}\ili{Sakha} \parencite[30]{Baker2015}\\
    \gll  Masha cej ih-er caakky-ta\\
          Masha-\Gen{}  tea drink-\Aor{}  cup-\Tsg.\Poss{}\\
    \glt  ‘a cup that Masha drinks tea from’
\z

In this case, however, the subject inside the relative clause bears genitive\is{genitive case}
and not nominative\is{nominative case} case morphology. Note that in Sakha genitive\is{genitive case} case is
syncretic with nominative\is{nominative case} (both are null) except after a possessive agreement
suffix as in \eqref{ex:13.7}.

\ea%7
    \label{ex:13.7}\ili{Sakha} \parencite[30]{Baker2015}\\
    \gll  [ Masha aqa-ty-\emph{n} ] atyylas-pyt at-\textbf{a}\\
          {} Masha father-\Tsg.\Poss{}-\Gen{} {} buy-\Ptcp{}  horse-\textbf{\Tsg.\Poss{}}\\
    \glt  ‘the horse that Masha’s father bought’  \citet[30]{Baker2015}
\z

Baker concludes that the contrast between \eqref{ex:13.4} and
\REF{ex:13.6} suggests that if a different head agrees with the subject in
\ili{Sakha}, then the case of the subject is distinct as well. In
\eqref{ex:13.4}, it is the verb that agrees with the subject, and the
subject bears nominative\is{nominative case}. In \REF{ex:13.6}, it is the
head of the relative clause that agrees with the subject, and the subject bears
genitive\is{genitive case}.\is{relative clauses}\is{agreement!subject
agreement}

The second possibility is that there is no overt agreement on either the
participle or on the head noun, and the subject of the clause is phonologically
null, see \REF{ex:13.8}:

\ea%8
    \label{ex:13.8}\ili{Sakha} \parencite[30]{Baker2015}\\
    \gll    cej ih-er caakky\\
            tea drink-\Aor{}  cup\\
    \glt    ‘a cup that one drinks tea from’
\z

This suggests that an agreement-bearing head in a relative clause structure is
not necessary in Sakha.

What seems to be, however, impossible is to have an overt NP in nominative\is{nominative case} case
as the subject of the relative clause, in the absence of any overt agreement,
as in \eqref{ex:13.9}, a fact indicating that there can be no nominative\is{nominative case} in the
absence of agreement in this language:\is{relative clauses}

\ea%9
    \label{ex:13.9}\ili{Sakha} \parencite[30]{Baker2015}\\
    \gll \llap{*}Masha cej ih-er caakky.\\
          Masha tea drink-\Aor{}      cup\\
    \glt  ‘a cup that Masha drinks tea from’
\z

A further correlation between nominative\is{nominative case} and agreement emerges when we look at
clauses that do not have a nominative\is{nominative case} subject. As Baker points out, the theme
argument of a passive\is{passive} verb in Sakha may be nominative\is{nominative case} or accusative. If it is
nominative, (\ref{ex:13.10}a), the passive\is{passive} verb must \isi{Agree} with it; if it
is not nominative\is{nominative case}, (\ref{ex:13.10}b), then the passive\is{passive} verb cannot agree
with it:

\ea%10
    \label{ex:13.10}\ili{Sakha} \parencite[32]{Baker2015}
	\ea
	\gll  Sonun-nar aaq-ylyn-ny-lar.\\
    news-\Pl{}  read-\Pass{}-\Pst{}-\Tpl.\Sbj{}\\
	\glt     ‘The news was read.’
	\ex
	\gll  Sonun-nar-y aaq-ylyn-na.\\
	    news-\Pl{}-\Acc{}  read-\Pass{}-\Pst{}.\Tsg.\Sbj{}\\
	\glt     ‘The news was read.’
    \z
\z

Baker takes these facts to suggest that \Nom{} is assigned under
Agree.\footnote{As pointed out by an anonymous reviewer, \citet{LevPre2015}
    argue that these facts are equally consistent with the view that
    nominative\is{nominative case} is the unmarked case in the language, and
    that agreement targets only NPs with nominative\is{nominative case} case;
We will come back to the issue of agreement targeting \Nom{} DPs.}

\section{Case assigned by different means}\label{sec:13.3}

\textcite[112f.]{Baker2015} presents evidence that if one NP is c-commanded by
another NP in the same clause, it is accusative in Sakha. This is
straightforwardly the case when both NPs are in the same domain, i.e.\ within
the same TP:

\ea%11
    \label{ex:13.11}\ili{Sakha} \parencite[112]{Baker2015}\\
    \gll  Erel  \textbf{kinige-ni}  atyylas-ta.\\
          Erel book-\Acc{}  buy-\Pst{}.\Tsg.\Sbj{}\\
    \glt  ‘Erel bought the book.’
\z

But if an NP is c-commanded by another NP in a higher clause in Sakha, it is
not necessarily accusative. For example, the matrix subject does not trigger
accusative case on the subject of its CP complement, as shown in
\REF{ex:13.12}. This is exactly what is expected, if CP is a
phase\is{phases} in \citegen{Chomsky2001} sense:

\ea%12
    \label{ex:13.12}\ili{Sakha} \parencite[113]{Baker2015}\\
    \gll  Min    [ sarsyn \textbf{ehigi-(*ni)}  kel-iex-xit dien ] ihit-ti-m.\\
          I.\Nom{} {} tomorrow you-(*\Acc)  come-\Fut{}-\Spl.\Sbj{}  that {} hear-\Pst{}-\Fsg.\Sbj{}\\
    \glt  ‘I heard that tomorrow you will come.’
\z

Importantly, in Sakha, the subject of an embedded clause can have accusative
case under certain conditions, as shown in \eqref{ex:13.13}, where the NP has
moved to the left edge of the embedded CP:

\ea%13
    \label{ex:13.13}\ili{Sakha} \parencite[114]{Baker2015}\\
    \gll  Min [ \textbf{ehigi-ni} [ bügün kyaj-yax-xyt dien ]]  erem-mit-im.\\
    I {} you-\Acc{} {} today win-\Fut{}-\Tpl.\Sbj{}  that {} hope-\Ptcp{}-\Fsg.\Sbj{}\\
    \glt  ‘I hoped that you would win today.’
\z

In \eqref{ex:13.13}, it is the presence of another NP in the matrix clause that
determines the case of the embedded subject. Evidence that the embedded subject
has moved to the left edge of the CP in \eqref{ex:13.13} comes from adverb
placement: if lower clause adverbs precede rather than follow it, then the
lower subject must be nominative\is{nominative case}, suggesting that it has not moved to the left
edge, and hence cannot bear accusative.

\ea%14
    \label{ex:13.14}\ili{Sakha} \parencite[115]{Baker2015}\\
    \gll    Min [ sarsyn ehigi-(*ni)  kel-iex-xit dien ]  ihit-ti-m.\\
    I.\Nom{} {} tomorrow you-(*\Acc)  come-\Fut{}-\Spl.\Sbj{} that {} hear-\Pst{}-\Fsg.\Sbj{}\\
    \glt    ‘I heard that tomorrow you will come.’
\z

This is a so-called edge effect, which is expected if the domains for dependent
case\is{dependent case} assignment are \isi{phases} in the sense of
\citet{Chomsky2001}.

Moreover, \citet{Baker2015} demonstrates that the one-to-one mapping of
nominative and agreement collapses if we look at a number of environments in a
different set of languages. For instance, in \ili{Oromo}, there are clauses
with more person-number-gender agreement than nominative\is{nominative case} subjects. This is the
case in periphrastic tenses consisting of a past or imperfective main verb and
an auxiliary\is{auxiliaries}. Here \emph{both verbs} \isi{Agree} with the subject in
\isi{φ-features}, including person, but presumably cannot both assign the
subject nominative\is{nominative case} case.

\ea%15
    \label{ex:13.15}\ili{Oromo} \parencite[99]{Baker2015}
	\ea
	\gll  Isaa-f xanni-\textbf{t}-é  tur-\textbf{t}-e.\\
	    him-\Dat{}  give-\Tsg.\Sbj{}-\Pst{}  was-\Tsg.\Sbj{}-\Pst{}\\
	\glt     ‘You HAVE given it to him.’
	\ex
	\gll  Joollée-n beelaw-\textbf{t}-é  hin-jír-\textbf{t-}u.\\
        Children-\Mnom{}  get.hungry-\glossF{}-\Pst{} \Neg{}-exist-\glossF{}-\Dep{}\\
	\glt     ‘The children haven’t gotten hungry.’
    \z
\z

Similarly, in \ili{Ingush} multiple heads \isi{Agree} with the same
absolutive\is{absolutive case} argument in the periphrastic progressive
\citep[71--72]{Baker2015} and also, like \ili{Tsez} \parencite{PolPot2001}, the
language tolerates long-distance agreement,\is{agreement!long-distance
agreement} where the matrix verb agrees with an NP inside an embedded clause):

\ea%16
    \label{ex:13.16}\ili{Ingush} \citep[263]{Nichols2011}\\
    \gll  Txy naana-z maasha b-ezhb-ar.\\
            our.\Gen{}  mother-\Erg{}  homespun.\Bb{} \Bb{}-make.\Cvb.\Sim{} \Bb{}-\Prog.\Pst\\
    \glt  ‘Our mother was making homespun (when I came in).’
\z

\ea%17
    \label{ex:13.17}\ili{Ingush} \parencite[551, 550]{Nichols2011}
	\ea
	\gll  Muusaa  [ zhwalii waaxar ]  qer.\\
            Musa.\Abs{} {} dog.\Abs{}  bark.\Vn{} {} fear\\
	\glt     ‘Musa is afraid the dog will bark.’
	\ex
	\gll  Waishet cec-j-ealar  [ Muusaa-z baq’  aalaragh ].\\
            Aisha.\Abs{}  surprise-\Jj-\Lv{}.\Pst{} {}  Musa-\Erg{} truth.\Abs{}  say.\Vn{}.\Lat{} {}\\
	\glt     ‘Aisha was surprised that Musa told the truth.’
    \z
\z

A related argument comes from the observation that Case assignment in
infinitival clauses works exactly as in finite ones in \ili{Burushaski},
exemplified below, but also in \ili{Shipibo}, \ili{Chukchi}, \ili{Greenlandic}
Inuit, \ili{Tamil}:

\ea%18
    \label{ex:13.18}\ili{Burushaski} \parencite[44]{Baker2015}
	\ea
	\gll   Já-a  [ \textbf{ún}  ní-as-e ]                r rái é-t-c-abaa.\\
    \Fsg-\Erg{} {} \Ssg.\Abs{}  go-\Inf{}-\Obl{} {} to want \Tsg.\Obj{}-do-\Npst-\Fsg{}.\Prs{}\\
	\glt     ‘I want you to go.’
	\ex
	\gll  Gús-e  [ \textbf{hir-e}  in mu-del-as-e ]                                     r rái a-é-t-c-ubo.\\
    woman-\Erg{} {}  man-\Erg{}  \Tsg{}.\Abs{} \Third.\glossF{}.\Obj{}-hit-\Inf{}-\Obl{} {} to want \Neg{}-\Tsg.\Obj{}-do-\Npst{}-\Third.\glossF.\Sbj{}.\Prs{}\\
	\glt     ‘The woman doesn’t want the man to hit her.’
    \z
\z

If T does not assign case to NP in the course of agreeing with it, then the
nominative case presumably comes from elsewhere.

Baker’s proposal is that languages of this type have unmarked/default
nominative or unmarked absolutive. Specifically, he links this to a parameter
discussed in \citeauthor{Baker2008} (\citeyear[155, (2)]{Baker2008}):

\ea%19
    \label{ex:13.19}\glsunset{CDAP}
    \emph{The Case-dependency of Agreement parameter}\\
    F agrees with DP/NP only if F values the case feature of DP/NP or vice
    versa.
\z

Combined with the directionality parameter in \eqref{ex:13.20} (his (1)),
\citet{Baker2008} derives a four-way typology of the agreement properties of
Tense:

\ea\label{ex:13.20}
    \emph{The direction of Agreement parameter}\glsunset{DAP}\\
    F agrees with DP/NP only if DP/NP asymmetrically c-commands F.
\z

This predicts certain language types, which can be described as follows,
according to \citet{Baker2008}: First, there are many \ili{Bantu} languages
that systematically obey \eqref{ex:13.20} but not \eqref{ex:13.19}, [No
\gls{CDAP}, Yes \gls{DAP}]. As a result, the finite verb agrees with whatever
precedes it, e.g.\ locatives or fronted objects:

\ea%21
    \label{ex:13.21}\ili{Kinande} \citep[158]{Baker2008}
	\ea
	\gll  Omo-mulongo mw-a-hik-a (?o)-mu-kali.\\
    \Loc.18-village.3 18\Sm-\Tns-arrive-\Fv{} \Aug{}-1-woman\\
	\glt     ‘At the village arrived a woman.’\\
    \ex
    \gll Oko-mesa kw-a-hir-aw-a ehilanga.\\
    \Loc.17-table 17\Sm-\Tns-put-\Pass{}-\Fv{} peanuts.19\\
    \glt ‘On the table were put peanuts.’
    \z
\z

Second, many \ili{Indo-European} languages systematically obey \eqref{ex:13.19} but
not \eqref{ex:13.20}, [Yes \gls{CDAP}, No \gls{DAP}]. As a result, the finite verb only
agrees with nominative\is{nominative case} DPs regardless of their position (preverbal or
postverbal)].

Third, there are languages such as \ili{Turkish} where both \eqref{ex:13.19}
and \REF{ex:13.20} are set positively, [Yes \gls{CDAP}, Yes \gls{DAP}]. As a
result, the finite verb only agrees with nominative\is{nominative case} DPs,
but only in SOV orders, not in inverted OSV orders which lack agreement.

Finally, \ili{Burushaski} (an isolate ergative language spoken in the Himalayas) is
argued to instantiate the fourth option, [No \gls{CDAP}, No \gls{DAP}]. This group of
languages have the following properties: nominative\is{nominative case} and ergative subjects
trigger the same form of agreement, unlike e.g.\ \ili{Hindi} where verbs \isi{Agree} only
with nominative\is{nominative case} subjects, and this is independent of word order, i.e.\ agreement
is always with the thematic subject and never e.g.\ with the fronted object in
inverted OSV orders.

In the next section, we turn to our investigation of Case assignment in
\ili{Greek} from the perspective of the above-sketched typology. We will show
that accusative\is{accusative case} is \isi{dependent case} and
nominative\is{nominative case} is unmarked case, i.e.\ not assigned under
\isi{Agree} with T, according to Baker’s criteria. Nevertheless, agreement
always goes with nominative\is{nominative case} arguments and never with
non-nominative ones, unlike e.g.\ \ili{Bantu} languages and like many Indo-European
languages. We will then explore how we can account for this.

\section{Case assignment in Greek}

\subsection{Accusative as dependent case in Greek}\largerpage[-1]

In \ili{Greek}, the subject of an embedded clause can have \Acc{} under certain
conditions \parencite{Iatridou1993,KotzPapa2007}. In (\ref{ex:13.22}a), the
subject of the embedded clause is assigned \Acc{} when it occurs at the edge of
the subjunctive. However, it is licensed by the negation in the subordinate
clause, which provides evidence that this is an \glsunset{ECM}\gls{ECM}\is{exceptional case marking} and not an object
control construction. As shown in (\ref{ex:13.22}b), object \isi{control}
constructions do not allow \glspl{NPI} licensed by negation in the embedded
clause.  Crucially, the adnominal modifier in the embedded clause bears
nominative\is{nominative case} obligatorily.\footnote{Mark Baker (personal
    communication) points out that a situation where the \gls{ECM}\is{exceptional case marking} subject
receives \Acc{} and the embedded modifier receives \Nom{}, as in
\eqref{ex:13.22}, does not arise in \ili{Sakha}, as far as he knows.}

\ea%22
    \label{ex:13.22}\ili{Greek}
    \ea[]{
	\gll Bika mesa ke me ekpliksi idha kanenan na min dulevi monos tu.                Oli ixan xoristi se omades.\\
    entered.I in and with surprise saw.I nobody.\Acc{} \Sbjv{} \Neg{} work.\Tsg{} alone his.\Nom{}. All had separated into teams.\\
	\glt ‘I entered and I saw to my surprise that nobody was working alone.  They had all separated into teams.’}
    \ex[*]{
	\gll Dietaksa kanenan na min figi apo edo\\
    ordered nobody.\Acc{}     \Sbjv{} \Neg{} leave.\Tsg{} from here\\
    \glt ‘I ordered that nobody leaves here.’}
    \z
\z

As in the other relevant languages discussed by Baker, the subject must move at
least to the edge of the CP and optionally also higher (presumably to the
Spec,\emph{v}P of the matrix clause) in order for it to be assigned accusative
case.  The relevant facts of \Acc{} vs. \Nom{} distribution in \ili{Greek} ECM
constructions are illustrated in \eqref{ex:13.23}. As \citet{KotzPapa2007} point
out, \Nom{} DP-subjects of the embedded predicates cannot surface on the left of
matrix adverbial material. On the contrary, this is possible with \Acc{}-marked
DPs, which may either precede matrix adverbials or follow them. When they
precede matrix adverbials, embedded \Acc{} subjects have presumably raised to the
matrix clause, while when they follow adverbials they remain at the edge of the
embedded subjunctive. In both positions, they can be assigned \Acc{} case. This
type of \Acc{} assignment is very local: \Acc{} subjects are not allowed to
surface below the edge of the subjunctive, in a position following the embedded
verb (arguably their \emph{v}P internal base position), where \Nom{} subjects are
possible.

\ea%23
\label{ex:13.23}\ili{Greek}\\
    \gll o Petros perimene \{*i Sofia / ti Sofia\} me laxtara \{i Sofia / tin Sofia\} na dhechti \{i Sofia / *tin Sofia\} tin protasi ghamu \\
    the Peter.\Nom{} expected.\Tsg{} \hphantom{\{*}the Sofia.\textbf{\Nom{}} {}
        the Sofia.\textbf{\Acc{}} with desire \hphantom{\{}the
            Sofia.\textbf{\Nom} {} the Sofia.\textbf{\Acc} \Sbjv{} accept-\Tsg{}
            \hphantom{\{}the Sofia.\textbf{\Nom} {} \hphantom{*}the
                Sofia.\textbf{\Acc} the proposal.\Acc{} wedding.\Gen{}\\
    \glt    \enquote*{It is with desire that Peter expected Sofia to accept the wedding proposal.} (\emph{matrix reading of PP})
\z

Similarly, in constructions involving secondary predication, where the subject
and the predicate must \isi{Agree} in \ili{Greek}, we see that no matter what
the case of the subject is (\Nom{} or \Acc), the embedded predicate always
bears nominative\is{nominative case} (data from \citealt{KotzPapa2007}):

\ea%24
\label{ex:13.24}\ili{Greek} \parencite{KotzPapa2007}
	\ea
	\gll  perimena o Janis na ine arostos / *arosto\\
        expected.\Fsg{} the John.\Nom{} \Sbjv{} be sick.\Nom{} {} \hphantom{*}*sick.\Acc{}\\
	\glt     ‘I expected John to be sick.’
	\ex
	\gll  perimena to Jani na ine arostos / *arosto\\
        expected.\Fsg{} the John.\Acc{} \Sbjv{} be sick.\Nom{} {} \hphantom{*}sick.\Acc{}\\
	\glt     ‘I expected John to be sick.’
    \z
\z

This suggests that accusative is \isi{dependent case} in \ili{Greek} and, moreover, that
\emph{dependent case can be assigned on top of a case assigned lower}, inside
the embedded clause, which is always nominative\is{nominative case} in \ili{Greek}.  As Baker notes,
there is cross-linguistic variation as to whether multiple cases can be
realized or not.

A particularly clear instance of case stacking, discussed in \citet{Baker2015},
is seen in Cuzco Quechua, where an NP can get genitive\is{genitive case} case
as the subject of a nominalized clause (i.e., as possessor of an NP), but then
move up into a higher clause and get \isi{accusative case} by being c-commanded
by the subject on top of its genitive\is{genitive case} case.

\ea%25
\label{ex:13.25}\ili{Cuzco Quechua} \parencite[116]{Baker2015}
	\ea
	\gll  Mariyacha muna-n  [ Xwancha-q platanu ranti-na-n-ta~].\\
        Maria want-\Third.\Sbj{} {} Juan-\textbf{\Gen{}}  banana buy-\Nmlz{}-\Third.\Poss{}-\Acc{}\\
	\glt     ‘Maria wants Juan to buy bananas.’
	\ex
	\gll  Mariyacha  Xwancha-q-ta muna-n [ platanu ranti-na-n-ta~].\\
        Maria    Juan-\textbf{\Gen{}-\Acc{}} want-\Third.\Sbj{} {} banana buy-\Nmlz{}-\Third.\Poss{}-\Acc{}\\
	\glt     ‘Maria wants Juan to buy bananas.’
    \z
\z

As we see in (\ref{ex:13.25}b), both the embedded and the matrix case are
realized, which is expected from \isi{dependent case} theory. In \ili{Greek},
accusative case can be assigned on top of nominative\is{nominative case}, but
only the higher case can be realized in case stacking configurations, unlike
the situation in Cuzco Quechua.

\citet{Baker2015} states the relevant morpho-syntactic parameter as follows:

\ea%26
    \label{ex:13.26}
    The case feature associated with nominal X can have a single value
    (\ili{Shipibo}, \emph{\ili{Greek}} \dots) or it can have a set of values
    (\ili{Quechua}, \ili{Korean}, some Australian languages).
\z

Our conclusion then is that accusative in \ili{Greek} is \isi{dependent case} assigned in
opposition to a higher argument at the CP-phase\is{phases} level.\footnote{See
    \citet{AnaSev2016} for evidence that Modern \ili{Greek} genitive/dative is also
    \isi{dependent case}, assigned in opposition to a lower argument at the \emph{v}P
level.} We turn to nominative\is{nominative case} next.

\subsection{Nominative case in Greek}\label{sec:13.3.2}

There is strong evidence that nominative\is{nominative case} is not assigned under \isi{Agree} with
finite T in \ili{Greek}. Specifically, nominative\is{nominative case} can be assigned in the absence of
finite T, as seen by the fact that it can appear in tenseless subjunctives in a
number of cases.

A first piece of evidence comes from \ili{Greek} \isi{raising} constructions
\parencite{AleAna1999}, shown in \eqref{ex:13.27}. In \eqref{ex:13.27},
we observe the absence of morphological and semantic Tense in the embedded
clause, as it is not possible to vary or modify the embedded verb by a temporal
adverb with independent reference, as shown in (\ref{ex:13.27}a) and
(\ref{ex:13.27}b), respectively:

\ea\label{ex:13.27}\ili{Greek}
    \ea[*]{
        \gll    {O Janis} arhizi na kolibise.\\
            John begins \Sbjv{} swam.\Tsg{}\\
    \glt    *\enquote*{John begins to have swum.}}
    \ex[*]{
        \gll    {O Janis} arhizi simera na kolibai avrio.\\
            John begins today \Sbjv{} swim.\Tsg{} tomorrow\\
    \glt    *\enquote*{John begins today to swim tomorrow.}}
    \z
\z

In these contexts, the nominative\is{nominative case} can appear in the
embedded clause, in spite of the absence of T. In this type of construction,
similar to the languages discussed in~\Cref{sec:13.2}, we have two verbs
that \isi{Agree} with one nominative\is{nominative case} obligatorily, a
\gls{LDA} phenomenon, see \textcite{AleAnaIorMar2012} for detailed
argumentation and arguments that this is not a covert \isi{raising}
construction but genuine \gls{LDA}:

\ea%28
\label{ex:13.28}\ili{Greek} \parencite[(36)]{AleAnaIorMar2012}\\
    \gll  Stamati\textbf{san} / *Stamatise [ na malo\textbf{nun} \textbf{i} \textbf{daskali} tus mathites ] \\
        stopped.\textbf{\Tpl} {}  \hphantom{*}stopped.\Tsg{} {} \Sbjv{} scold.\textbf{\Tpl} \textbf{the} \textbf{teachers}  the students\\
    \glt  ‘The teachers stopped scolding the students.’
\z

In these constructions, the subject resides in the embedded clause, but it
agrees both with the matrix and the embedded predicate obligatorily. Evidence
that the subject is truly embedded is provided by scope facts. The subject in
the embedded clause must take low scope (\ref{ex:13.29}a); on the other hand,
moved subjects must take wide scope (\ref{ex:13.29}b):

\ea%29
    \label{ex:13.29}\ili{Greek} \parencite[(41), (63)]{AleAnaIorMar2012}\\
	\ea \emph{stop} $>$ \emph{only}; *\emph{only} $>$ \emph{stop}\\
	\gll  Stamatise na perni  \textbf{mono} \textbf{i} \textbf{Maria}  kakus vathmus\\
	    stopped \Sbjv{} take only the Mary bad grades\\
	\glt ‘It stopped being the case that only Maria got bad grades.’\\
	\ex \makebox[0pt][r]{*}\emph{stop} $>$ \emph{only}; \emph{only} $>$ \emph{stop}\\
	\gll  \textbf{Mono} \textbf{i} \textbf{Maria} stamatise na perni kakus vathmus.\\
	    Only the Mary stopped \Sbjv{} take bad grades\\
	\glt   ‘Only Mary stopped getting bad grades.’
    \z
\z

Hence, these constructions violate \eqref{ex:13.1}, repeated here.

\begin{exe}
    \exi{\eqref{ex:13.1}}
    Overt NP X has nominative\is{nominative case} case if and only if \emph{exactly one verbal
    form} in the clause containing X agrees with it.
\z

The above facts lead to the conclusion that there is no one-to-one
correspondence between nominative\is{nominative case} case and verbal agreement
(a single nominative and many full agreements can co-occur) and that
nominative\is{nominative case} is realized in environments where \isi{Agree}
with a nominative\is{nominative case} assigning head does not take place (in
the \glsunset{ECM}\gls{ECM},\is{exceptional case marking} Raising\is{raising}
and \gls{LDA} constructions with embedded T lacking semantic and morphological
tense discussed above). These phenomena are reminiscent of the ones attested in
\ili{Burushaski}, \ili{Tamil}, \ili{Ingush}, \ili{Tsez}, which have been
analyzed by Baker in terms of unmarked nominative\is{nominative case}
(see~\Cref{sec:13.3}).

Further evidence for unmarked nominative\is{nominative case} in \ili{Greek} is
drawn from a series of environments where nominative\is{nominative case}
surfaces in the absence of agreement. For example:

\begin{enumerate}
\item Nominative assigned in the absence of agreement; \ili{Greek} free-adjunct
constructions including –\emph{ing} forms (\citealt{Tsimpli2000} and many
others call them \enquote{gerunds}) entirely lack subject agreement, but their
subjects bear nominative\is{nominative case} case:

\ea%30
    \label{ex:13.30}\ili{Greek}\\
    \gll  fevgondas i Maria \dots{}     eklise ti porta.\\
            leaving the Mary.\Nom{} {}  closed.\Tsg{} the door.\Acc{}\\
    \glt  ‘As Mary was leaving, she closed the door.’
\z

\item Nominative is the case on NPs that appear in HTLD, ellipsis etc.,
\citet{Schutze2001}:

\ea%31
    \label{ex:13.31}\ili{Greek}
	\ea
	\gll  O Janis, ton ematha kala ola afta ta hronia.\\
	    the John.\Nom{} him I learned well all these years\\
	\glt     ‘As for John, I got to know him very well after all these years.’
    \ex
	\gll  Pios theli na erthi?  Ego / *emena\\
            who wants to come  I {} \hphantom{*}Me\\
	\glt     ‘Who wants to come? Me.’
    \z
\z
\end{enumerate}

\section{Nominative Case and parametric variation}\label{sec:13.4}

Our conclusion leads to the following question: if nominative\is{nominative
case} is unmarked, then this means that \ili{Greek} is a [No \gls{CDAP}] language
like \ili{Bantu} or \ili{Burushaski}. But then why does the inflected verb in
\ili{Greek} only \isi{Agree} with nominative\is{nominative case} NPs and never
with anything else? Recall that \ili{Bantu} languages (which are, in addition,
[Yes \gls{DAP}] languages) show agreement between the finite verb and whatever precedes
it (locatives, objects etc.). On the other hand, \ili{Burushaski} (which is, in
addition, a [No \gls{DAP}] language) shows agreement with the thematic subject
regardless of the case of the subject (ergative or nominative\is{nominative
case}) and regardless of where the thematic subject is placed.

Note that, as is well-known, the nominative\is{nominative case} NP does not
need to be dislocated to Spec,TP in \ili{Greek}, i.e.\ \ili{Greek} clearly
qualifies as a [No \gls{DAP}] language (\citealt{AleAna1998}, i.a.):%\newpage

\ea%32
\label{ex:13.32}\ili{Greek}
	\ea
	\gll  O Janis / ta pedia agorase / agorasan to vivlio\\
    the John.\Nom{} {} the children.\Nom{} bought.\Tsg{} {} bought.\Tpl{}  the book.\Acc{}\\
	\glt     ‘John/the children bought the book.’
	\ex
	\gll  agorase / agorasan o Janis / ta pedia to vivlio\\
    bought.\Tsg{} {} bought.\Tpl{} the John.\Nom{} {} the children.\Nom{}  the book.\Acc{}\\
	\glt     ‘John/the children bought the book.’
    \z
\z

Crucially, verbal agreement is always with the nominative\is{nominative case}
argument and never with e.g.\ a higher locative or dative\is{dative case}
argument. \citet{Anagnostopoulou1999} provides evidence that dative\is{dative
case} experiencers in \ili{Greek} have subject status with respect to some
subjecthood criteria. For instance, the fact that they act as binders for
anaphors can be viewed as one argument for their subjecthood; nevertheless,
verbal agreement in this case is with the nominative\is{nominative case} and
not with the dative\is{dative case} argument:

\ea%33
    \label{ex:13.33}
    \gll    Ton pedion tus aresi o   eaftos tus\\
            The children.\Dat{} \Cl.\Dat{} like.\Tsg{} the self.\Nom{} theirs\\
    \glt    ‘The children like themselves.’
\z

Note, furthermore, that there are not even person restrictions in this kind of
quirky subject constructions in \ili{Greek}, unlike e.g.\ Icelandic, where the
verb is not allowed to \isi{Agree} with a nominative\is{nominative case} object
if this is first or second person
(\citealt{Anagnostopoulou2003,Anagnostopoulou2005} for \ili{Greek}, cf.\
\citealt{Sigurdsson1989,Taraldsen1995}, i.a. for Icelandic):

\ea%34
\label{ex:13.34}
    \ea \ili{Greek}\\
	\gll  Tis aresume / aresete / areso  / aresis    (emis / esis / ego / esi)\\
    her like.\Fpl{} {} like.\Spl{} {} like.\Fsg{} {} like.\Ssg{} \hphantom{(}we {} you.\Pl{} {} I {} you.\Sg{}\\
    \glt
    \ex \ili{Icelandic}\\
    \gll \llap{*}Henni leiddumst við\\
    She.\Dat{}  was bored.by.\Fpl{} us.\Nom{}\\
    \glt    ‘She was bored by us.’
    \z
\z

Similarly, in \gls{LDA} constructions under \isi{raising} predicates with an
experiencer argument agreement in person with the nominative\is{nominative
case} argument is possible:

\ea%35
\label{ex:13.35}\ili{Greek}\\
    \gll  Me apelise epidi den tu fenomun na dulevo  (ego)  kala\\
    me fired.\Tsg{} because \Neg{} him seemed.\Fsg{} \Sbjv{} work.\Fsg{} \hphantom{(}I well\\
    \glt  ‘He fired me because I seemed to him to not be doing a good job.’
\z

\ea%36
    \label{ex:13.36}\ili{Icelandic}
    \ea[]{
	\gll  Mér höfðu fundist  þær vera gáfaðar\\
	    Me.\Dat{}  had found they.\Nom{}   be intelligent\\
    \glt     ‘I had found them intelligent’}
    \ex[*]{
	\gll  Þeim höfum alltaf fundist við      vinna vel\\
	    Them.\Dat{}  have always found we.\Nom{}    work well\\
    \glt     ‘They have always thought that we work well.’}
    \z
\z

Thus, even though we have evidence from \gls{LDA}, Raising, and
\gls{ECM}\is{exceptional case marking} that \ili{Greek} behaves like a [No
\gls{CDAP}] language, we also have evidence that inflected verbs agree (fully)
with nominative\is{nominative case} arguments, just as in many Indo-European
languages which \citet{Baker2015} analyzes as \isi{Agree} and \citet{Baker2008}
analyses as [Yes \gls{CDAP}] languages.

The question then is what is the nature of the relevant parameter that can
account for the distribution of nominative\is{nominative case} case with respect to multiple
agreement in \ili{Greek} in long-distance agreement\is{agreement!long-distance
agreement} constructions of the type
discussed above. We would like to entertain the hypothesis that the
availability of such chains relates to the full pro drop status of Modern
\ili{Greek}. Suppose that full pro-drop languages have [+interpretable]
\isi{φ-features} on T, according to the hypothesis in \eqref{ex:13.37} (see
\citealt{Holmberg2005} who rejects it, \citealt{Barbosa2009} who argues for a
version of it, cf.\ \citealt{AleAna1998}):

\ea%37
    \label{ex:13.37}
    The set of \isi{φ-features} in T (Agr) is interpretable in \glspl{NSL}, and
    \emph{pro} is therefore redundant; Agr is a referential, definite pronoun,
    albeit a pronoun phonologically expressed as an affix. As such, Agr is also
    assigned a subject theta-role, possibly by virtue of heading a chain whose
    foot is in \emph{v}P,   receiving the relevant theta-role.
\z

It would follow from \eqref{ex:13.37} that T does not need to enter Agree
in order to license its \isi{φ-features}, and hence that \Nom{} Case will not be
assigned as a result of \isi{Agree} with the \isi{φ-features} of T. Thus, in such a
theory, the \isi{φ-features} of the lower T in \gls{LDA} configurations like
(\ref{ex:13.27}--\ref{ex:13.29}) are not deleted by entering \isi{Agree} with
\Nom{} arguments, and can thus form an \gls{LDA} chain with the \isi{φ-features} of
the higher T:

\ea%38
    \label{ex:13.38}
    \glspl{NSL} have T with interpretable \isi{φ-features} which are not
    deleted after checking, thus being able to form long-distance chains via
    \isi{Agree} (cf.  \citealt{Ura1994}).
\z

\textcite{TsaAnaAle2017,TsaAleAna2019}\footnote{\citeauthor{TsaAnaAle2017}
argue that apparent backward \isi{control} configurations also involve \gls{LDA}
chains of the type depicted in \eqref{ex:13.39}.} and \citet{AleAna2019} provide
further discussion of such \gls{LDA} chains in \ili{Greek}, which are schematically
represented in \eqref{ex:13.39}, as well as a discussion of the conditions under
which such chains are disrupted:

\ea\label{ex:13.39}
    {}[ Tφ\textsubscript{k} [\textsubscript{TP/CP} Tφ\textsubscript{k} DPφ\textsubscript{k} ]]
\z

Crucially for present purposes, overt subjects are expected to receive unmark\-ed
\Nom{} in \glspl{NSL} and not \Nom{} assigned by \isi{Agree} in such a theory.
In other words, the prediction of hypothesis~\eqref{ex:13.37} is the
unmarked status of nominative\is{nominative case} in NSLs.

This prediction seems to be borne out in \ili{Greek} and at least in
\ili{Romanian}, among other NSLs. \ili{Romanian} like \ili{Greek} has \gls{LDA}
\parencite{AleAnaIorMar2012}: as shown in \REF{ex:13.40}, the in situ DP
subject obligatorily agrees with both the matrix and the lower verb in person
and number, just like \ili{Greek}:

\ea\label{ex:13.40}\ili{Romanian}\\
    \gll {Au încetat} / {*A încetat} sǎ-i certe profesorii pe elevi.\\
    stopped.\Tpl{} {} \hphantom{*}stopped.\Tsg{} \Sbjv{}-\Cl{}.\Tpl{}.\Acc{} scold.\Tpl{} {the teachers} the students\\
    \glt ‘The teachers stopped scolding the students.’
\z

In situ subjects take narrow scope with respect to \isi{raising} verb and
matrix negation, as shown in \eqref{ex:13.41} (compare to \REF{ex:13.29}
above):\footnote{Note that the same judgements hold in \ili{Romanian} for the
    infinitival Raising constructions. We would like to point out here that
    with `seem' \ili{Romanian} only has the \emph{seem} $>$ \emph{only} reading,
irrespectively of the surface position of the subject, i.e.\ before the \isi{raising}
verb or in the embedded clause.}

\ea%41
    \label{ex:13.41}\ili{Romanian}
    \ea \emph{only} $>$ \emph{stop}\\
    \gll   Numai Maria {a încetat} sǎ   ia note slabe.\\
	    only Mary stopped \Sbjv{} get grades weak.\\
	\glt     ‘It is only Maria who stopped getting bad grades .’
    \ex \emph{stop} $>$ \emph{only}\\
    \gll   {A încetat} sǎ  ia numai Maria note slabe.\\
	    stopped \Sbjv{} get only Mary grades weak\\
	\glt     ‘It stopped being the case that only Mary got bad grades.’
    \z
\z

Like \ili{Greek}, \ili{Romanian} allows nominative\is{nominative case} \emph{in
gerunds}:

\ea\label{ex:13.42}\ili{Romanian} \citep{Alboiu2009}\\
    \gll fiind tu / *tine gata\\
    be-\Ger{}    \Ssg.\Nom{} {} \hphantom{*}\Ssg.\Acc{} ready\\
    \glt
\z

If we accept the above reasoning, it seems that at least some \glspl{NSL} have
unmarked nominative\is{nominative case}.

But what is it that ensures that the \isi{φ-features} of T always track/co-vary
with \Nom{} in \glspl{NSL}? Recall that \ili{Greek} (and \ili{Romanian}) is not
like a [No \gls{CDAP}] language. Typical [No \gls{CDAP}] languages dissociate
agreement from nominative in particular cases, for instance, \ili{Bantu}
languages show agreement between the finite verb and whatever precedes it
(locatives, objects etc.), while \ili{Burushaski} T agrees indiscriminately
with both ergative and absolutive subjects.\is{agreement!subject agreement}
\ili{Greek} instantiates the type of language, which \citet{Baker2008}
explicitly states should not exist: “No agreement with obliques; multiple
agreement OK” \parencite[223, (113d)]{Baker2008}. Multiple agreement in
\ili{Greek} and \ili{Romanian} suggests that (i) \Nom{} is not assigned under
Agree and (ii) agreement on T is not valued by Nom, which straightforwardly
follows from \eqref{ex:13.37} above.  Nevertheless, agreement can never trace
genitive\is{genitive case} DP \glspl{IO} or PPs but only \Nom{} DPs.

We can account for this puzzle, if we hypothesize that only DPs bearing
unmarked case (i.e.\ nominative\is{nominative case} case) are accessible for
phi-agreement (\citealt{Bobaljik2008}, \citealt{Preminger2014},
\citealt{Baker2015}) in \ili{Greek}. Under this hypothesis, even though the
\isi{φ-features} on T do not need to enter \isi{Agree} with a DP (see
\eqref{ex:13.37} above) and even though \Nom{} does not need to be licensed
by \isi{Agree}, when both agreement and a DP bearing \Nom{} are present,
agreement always targets DPs bearing \Nom{} and not e.g.\ DPs bearing
oblique/quirky \Gen. Naturally, this raises two further questions: (a) What
does \enquote{phi-agreement} mean, if this is not the reflex of \isi{Agree}?
What is the relationship between overt agreement and \isi{Agree}? (b) What
happens in pro-drop configurations where no overt DP bearing \Nom{} is present?

We are not going to fully address these questions here, but we would like to
suggest that the need for a separation of \isi{Agree} from agreement in order
to describe the state of affairs in \ili{Greek} reflects the need for a
separation of interpretability from valuation, argued for in
\citet{PesetskyTorrego2007} on independent grounds.

Suppose that the \isi{φ-features} on T are [+interpretable], thus not requiring
Agree to be licensed, as stated in \eqref{ex:13.37}, but at the same time
they are unvalued and need to receive a value. One way of receiving a value is
via an agreement operation copying the \isi{φ-features} of a DP onto T. Under the
hypothesis that only DPs bearing \Nom{} are accessible for agreement in \ili{Greek},
this will force agreement between \Nom{} and the lower T in configurations like
\REF{ex:13.39}. Once its \isi{φ-features} are valued, the lower T in \REF{ex:13.39}
will further value the \isi{φ-features} of the matrix T by copying its features
onto the higher T through the formation of an agreement chain with it. On this
view, \ili{Greek} has two key properties. On the one hand, agreement always goes with
a \Nom{} DP, similarly to e.g.\ \ili{English} and Sakha. This is due to the
fact that in all three languages, only \Nom{} DPs are accessible for agreement.
On the other hand, agreement and \Nom{} are not in a one-to-one relationship,
unlike \ili{Sakha} and \ili{English}. \ili{Greek} behaves similarly to
\ili{Oromo}, \ili{Ingush} and \ili{Tsez} in showing multiple fully inflected
for person and number verbal heads agreeing with a single \Nom{} DP
(\glsunset{LDA}\gls{LDA}). This is due to the fact that T in pro-drop
\ili{Greek} has [+interpretable] φ-features which do not have to be licensed
via \isi{Agree} with a \Nom{} DP, and, concomitantly, \Nom{} is unmarked case
and therefore possible also in environments lacking agreement (for instance,
gerunds).

The final issue to address concerns question b) raised above, namely, how to
analyze agreement in pro-drop configurations where no overt DP is present. We
already said that we adopt \eqref{ex:13.37} according to which, Agr on T is
[+ interpretable], phonologically expressed as an affix. As such, Agr is also
assigned a subject theta-role, by virtue of heading a chain whose foot is in
\emph{v}P (we could call it pro), receiving the relevant theta-role. The
question is what values the features of Agr in the absence of an overt DP
bearing \Nom{}. We believe that in these cases, valuation happens via a covert
Topic operator situated in the CP-periphery of the clause, along the lines of
proposals put forth in \citet{Frascarelli2007}, \citet{FraHin2007},
\citet{Miyagawa2017} and others.

\hspace*{-0.82156pt}This view on pro is very close to ideas in \citet{Holmberg2010} and
\textcite{Roberts2010c,Roberts2010}. \citet{Holmberg2010} and
\textcite{Roberts2010c,Roberts2010} take \glspl{NSL} to have a D feature T, see
also \citet{AleAna1998}. They assume that null pronouns are simply φPs, i.e.\
they are defective pronouns in the system of \citet{CarSta1999}. When T probes
a φP subject, its unvalued \isi{φ-features} are valued by the subject.  This results
in the union of the \isi{φ-features} of T and the subject, which in turn yields a
definite pronoun. Roberts and Holmberg take incorporation of a φP in T to be a
direct effect of \isi{Agree}. In particular, finite T has a set of unvalued
\isi{φ-features}, and probes for a category with matching valued features.  The
defective subject pronoun has the required valued \isi{φ-features}, and
therefore values T’s uφ-features. T values the subject’s unvalued case
feature. In this situation, according to \citet{Roberts2010b}, the probe and
the goal form a chain, the φP is not pronounced, but as the chain includes [D],
which is valued by the \isi{topic}, the result is a definite null
subject\is{null subjects} construction. The chain is pronounced in form of an
affix on the verb.  Specifically, in Holmberg's system the index-sharing
relationship between the null pronoun and the null Topic crucially involves T:
the \isi{topic} values the uD-feature of T, where the valuation consists of uD
copying the referential index of the \isi{topic}.

The difference between null pronouns and lexical DPs or D-pronouns is that they
value T’s uD-feature. However, in this case, T and the lexical subject DP,
while they share φ-feature\is{φ-features} values as a result of \isi{Agree},
they do not form a chain, and consequently the lexical subject is spelled out
and pronounced. In our analysis, though valuation is necessary, the additional
layer of [$+$interpretable] features leads to a situation, according to which
\Nom{} Case will not be assigned as a result of \isi{Agree} with the
\isi{φ-features} of T, and many fully agreeing Ts are possible.

\section{Conclusion and open questions}

In this squib, we investigated the nature of nominative\is{nominative case} and
\isi{accusative case} in Greek. We argued that while accusative qualifies as
\isi{dependent case} on the basis of \citegen{Baker2015} criteria,
nominative\is{nominative case} is problematic: while \ili{Greek} Nominative
behaves like unmarked case and is clearly not assigned under agreement in T in
a number of environments, unlike \ili{English}, agreement always goes with
\Nom{} when both are present, like \ili{English}. An important characteristic
of \ili{Greek} not shared by \ili{English} is that it pervasively shows
long-distance chains involving a single in situ \Nom{} subject and many T heads
fully agreeing with it. We suggested that \ili{Greek} has T with interpretable
\isi{φ-features} as a by-product of V \isi{raising} satisfying the \gls{EPP}.
This allows for the formation of long-distance chains between a single DP
bearing unmarked \Nom{} and many fully agreeing Ts. Turning to the question of
why agreement always goes with \Nom{} in Greek, we adopted the view that
agreement is sensitive to unmarked case and argued that the analysis of
\ili{Greek} nominative\is{nominative case} case in connection to agreement
requires a separation of interpretability from valuation, as in
\citet{PesetskyTorrego2007}.

Several issues arise from our proposal. First, an empirical question is whether
it is possible to find evidence from \gls{LDA} configurations under multiple
agreement pointing to the same conclusion for other pro-drop languages as well.
The first languages to look at would be pro-drop languages that have lost
infinitives and have replaced them with inflected clauses similar to \ili{Greek}
subjunctives, or pro-drop languages with inflected infinitives: several
languages of the Balkan Sprachbund and European Portuguese might be candidate
languages to look at.

Second, in a system where nominative\is{nominative case} and absolutive can either be assigned via
\isi{Agree} or be unmarked cases (see \citealt{LevPre2015} for arguments against this
dissociation), the more general question that arises is what determines which
case will be unmarked and/or default in a language and what determines
nominative/absolutive case assignment under \isi{Agree}. For instance, in \ili{English}
(but also \ili{Norwegian}), accusative is the default Case and \Nom{} is assigned via
\isi{Agree}, see \citet{Schutze2001} and \citet{McFadden2004}.\footnote{Thanks to Terje Lohndal
for raising this question.} A possible way of relating this particular
distribution of cases would be to propose that because nominative\is{nominative case} is assigned
via \isi{Agree} in \ili{English} and \ili{Norwegian}, another case must take over the role of
default case. Because of this, these languages have default accusative and not
default nominative\is{nominative case} case.  On the other hand, in a language like \ili{Greek} where
nominative is the unmarked case, default and unmarked case will have the same
realization in the clausal domain, since nominative\is{nominative case} always surfaces on NPs that
do not enter case competition.

\printchapterglossary{}

\section*{Acknowledgements}

We are indebted to two anonymous reviewers, Mark Baker, Terje Lohndal and the
editors of this volume for comments and suggestions. We are extremely happy to
present this squib to Ian who has influenced our thinking about grammar and
comparative syntax since the very early stages of our careers. AL 554/8-1
(Alexiadou) and the 2013 Friedrich Wilhelm Bessel Research Award
(Anagnostopoulou) are hereby acknowledged.

{\sloppy
\printbibliography[heading=subbibliography,notkeyword=this]
}

\end{document}
