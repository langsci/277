\documentclass[output=paper]{langsci/langscibook}
\author{Luigi Rizzi}
\title{Rethinking the ECP: Subject--object asymmetries as freezing effects}

% \chapterDOI{} %will be filled in at production

\abstract{The ECP had a major explanatory role in GB syntax. Conceptual and
technical difficulties with the principle diverted the focus of theoretical
attention from core ECP effects in minimalism. Nevertheless, the empirical
motivation for such effects remains robust across languages. In this article,
I would like to rethink core ECP effects such as subject--object asymmetries
in extraction contexts in terms of a different theoretical apparatus which
emerged in recent years in connection with cartographic studies. Criterial
positions determine freezing effects. If there is a Subject Criterion, subjects
will undergo such effects, and will be unmovable, unless special devices are
used by the language. We observe \emph{that}-trace effects with subjects but
not with objects because there is no general object criterion. This alternative
theoretical apparatus can be shown to be empirically advantageous with respect
to the ECP approach in connection with a number of phenomena discussed in the
classical ECP literature.}


\begin{document}\glsresetall
\maketitle

\section{The classical ECP approach}\glsunset{UG}

The \gls{ECP}\is{Empty Category Principle} played a major explanatory role in
Govern\-ment-and-Binding\glsunset{GB} (\gls{GB}) analyses.  First and foremost,
it captured different kinds of subject--object asymmetries in extraction
contexts: all other things being equal, subjects are harder to extract from
embedded domains than objects (or other complements).  The classical
illustration is the \emph{that}-trace effect. An object is extractable from an
embedded declarative introduced by \emph{that}, but a subject is not:

\ea%1
    \label{ex:key:21.1}
    \ea[*]{Who do you think [ that [  \_\_\_   will come ]]?}
    \ex[]{Who do you think  [ that [ Mary will meet \_\_\_ ]]?}
    \z
\z
%
Judgments gathered with controlled methods have confirmed such asymmetries,
while revealing new facets of the phenomenon.\footnote{For instance also in
    case of object extraction the optimal case is from a clause not introduced
    by an overt complementizer\is{complementizers}, but a clear contrast with
    subject extraction persists: \citet{Schippers2012}.} There are factors of
    empirical complexity, though: certain varieties of \ili{English} admit
    (\ref{ex:key:21.1}a) as acceptable, so that in such varieties the asymmetry
    tends to disappear (\citealt{Sobin2002,Schippers2012}); nevertheless, the
    constraints on extractability are not simply subjected to arbitrary
    variation: for instance, the asymmetry reappears, also for speakers who
    accept (\ref{ex:key:21.1}a), in other contexts, such as the extraction from
    indirect questions (here the contrast is in terms of relative
    acceptability, as extraction from the weak island\is{islands} is always
    degraded to some extent):

\ea%2
    \label{ex:key:21.2}
    \ea[*]{Who do you wonder if \_\_\_ will come?}
    \ex[??]{Who do you wonder if Mary will meet \_\_\-?}
    \z
\z
%
In other languages, things are even sharper. Subject extraction in
(\ref{ex:key:21.3}a) appears to be systematically excluded in \ili{French},
while object extraction in (\ref{ex:key:21.3}b) is possible
\citep{Berthelot2017}:

\ea%3
    \label{ex:key:21.3}\ili{French}
    \ea[*]{Qui    penses-tu      que \_\_\_ va venir?\\
    ‘Who do you think that will come ?’}
    \ex[]{Qui    penses-tu      que Marie va rencontrer \_\_\_ ?\\
    ‘Who  do you think that Marie will meet ?’}
    \z
\z
%
So, the asymmetries are a real, robustly attested phenomenon. The \gls{ECP}
tried to capture the asymmetries by appealing to independent properties
differentiating subjects and complements. According to the classical approach
of \citet{Chomsky1981}, traces must be lexically governed (or
antecedent-governed, an option that I do not discuss here). The object is
governed by a lexical element, the verb, while the subject is governed by a
functional head, the node Infl, or T, which is not sufficient to satisfy the
requirement. So, the asymmetry follows from the nature of the governing
element.

This analysis was extremely influential and gave rise to an important
literature both on the cross-linguistic scope of the phenomenon, the ways of
circumventing it (e.g.\ via complementizer\is{complementizers} deletion in \ili{English}), the exact
format of the principle, etc. (see, e.g.,
\citealt{Pesetsky1982,Kayne1984,Rizzi:1982,Rizzi1990}, a.o.) In spite of its
empirical success and its capacity to generate important syntactic research,
the \gls{ECP}\is{Empty Category Principle} approach was abandoned in Minimalism.

I think the main problem which led to this step is conceptual: Minimalism
permits a very limited variety of \gls{UG}\is{Universal Grammar} principles:
principles operating at the interfaces with sound and meaning, and somehow
enforced by the needs of the interface systems (e.g., linearization at the
\glsunset{PF}\gls{PF} side, the Theta Criterion at the LF side, etc.), and
principles of optimal computation, operating on the computing machine
(including principles of economy, locality, labeling,\is{labelling} etc.). The
\gls{ECP}\is{Empty Category Principle} does not naturally fit into any of these
categories, so it has no natural place in the minimalist universe.

There were also technical problems, due to the reliance of the \gls{ECP}\is{Empty Category Principle} on
government, a structural relation not assumed in Minimalism. Personally, I
never found such considerations compelling: government is minimal c-command,
i.e., c-command constrained by locality, and Minimalism must assume both some
form of c-command (perhaps derivatively from the extension condition or no
tampering) and some form of minimality, so that the ingredients for government
are there, even if a primitive government relation is not postulated. But, even
if the technical argument may be unconvincing for these reasons, the conceptual
argument remains compelling. So, research on the asymmetry was somehow demoted
from center stage in Minimalism.

Nevertheless, the facts are clear, and cross-linguistically robust. True, some
languages do not manifest the asymmetry, so that the phenomenon has sometimes
been qualified as “language specific”, and, as such, not bearing on
\gls{UG}\is{Universal Grammar}
principles. But this kind of reasoning is highly questionable. On the one hand,
systematic exceptions to \emph{that}-trace effects have turned out to be
amenable to independent principled explanations, such as the systematic absence
of the asymmetries in Null Subject Languages\is{null subject languages} \citep{Rizzi:1982}. On the other
hand, the cross-linguistic distribution is clearly constrained: we don’t seem
to find clear cases of the “mirror image” of \ili{English} or \ili{French}, i.e., a
language freely allowing subject extraction across an overt complementizer\is{complementizers} and
banning object extraction.

In classical discussions of such issues, poverty of stimulus considerations
were typically invoked to support the necessity of a principled explanation.
How does the learner of (the relevant variety of) \ili{English}, or \ili{French},
come to know that (\ref{ex:key:21.1}a), (\ref{ex:key:21.2}a),
(\ref{ex:key:21.3}a) are excluded? Why don’t all language learners analogically
generalize from cases of extraction they hear, (such as (\ref{ex:key:21.1}b)),
to all cases of extraction, assuming no asymmetries? An anonymous reviewer
observes that some qualification is needed here because through statistical
learning techniques it may be possible to infer the ungrammaticality of a
structure such as (\ref{ex:key:21.1}a) from its non-occurrence. The point is
well-taken, even though one should make sure that such techniques can be
selective enough, i.e., do not generalize from cases like (\ref{ex:key:21.1}b)
to cases like (\ref{ex:key:21.1}a), in the absence of any principled guidance.
More importantly, a  technical approach to these problems based on statistical
learning would remain too “local”:  why should the asymmetries be
systematically found across languages, and always in the same direction?
Somehow, the systematic higher difficulty with subject extraction, robustly
attested language after language, must come from some internal pressure and be
connected to a principled reason, exactly what the \gls{ECP}\is{Empty Category
Principle} approach assumed.

These considerations pave the way for the search of a principled alternative to
the \gls{ECP}\is{Empty Category Principle} to capture the asymmetries.

\section{Criterial freezing and the subject criterion}

According to the criterial approach  to scope-discourse semantics, the initial
periphery of the clause is populated of functional heads\is{functional items} such as Q, Top, Foc,
etc., which attract a phrase with matching feature, creating criterial
(Spec-head) configurations, and guide the interpretation of such criterial
configurations at the interfaces with sound and meaning \citep{Rizzi1997}.

One salient property of such criterial configurations is that the attracted
element is frozen in the criterial position, i.e., it cannot be attracted to a
higher position. The canonical example is the case of a wh-phrase satisfying
the Q-criterion in an embedded interrogative, selected by a verb like
\emph{wonder}. In such cases, the wh-element cannot be moved further:

\ea%4
    \label{ex:key:21.4} \textcite{LasnikSaito1992}, \textcite{Boskovic2008}
    \ea[]{Bill wonders [\emph{which} \emph{book} \emph{Q} [ she read \_\_\_ ]]}
    \ex[*]{\emph{Which} \emph{book}  Q does Bill wonder [ \_\_\_  \emph{Q} [ she read \_\_\_ ]] ?}
    \z
\z
%
While obvious options come to mind to rule out (\ref{ex:key:21.4}b) (one could
invoke interface problems with the derived representations, or an
“inactivation” analysis à la \citealt{Boskovic2008}), more complex cases
discussed in \textcite{rizzicriterial,Rizzi2011} and much subsequent work suggest that the
problem is deeper. So, a descriptive principle like the following seems to
hold:

\ea%5
    \label{ex:key:21.5}
    \isi{Criterial freezing}: An XP meeting a criterion is frozen in place.
\z
%
In fact the phrase meeting a criterion is not completely frozen: if the phrase
is complex, part of it can be subextracted. E.g., taking Italian
(\ref{ex:key:21.6}a) as baseline, focalization of the PP \emph{di Piero} is
possible, with subextraction and clefting:\is{clefts}

\ea%6
    \label{ex:key:21.6}\ili{Italian}
    \ea Non è chiaro   [[ \emph{quanti libri di Piero} ] Q siano stati censurati ]\\
        ‘It isn’t clear       how many book by Piero Q have been censored’
    \ex E’ \emph{di Piero} che non è chiaro [[ \emph{quanti libri} \_\_\_ ] \emph{Q}  siano stati censurati] (non di Gianni)\\
        ‘It is by Piero that it is not clear how many books Q have been censored, not by Gianni’
    \z
\z
%
The formulation in \REF{ex:key:21.5} should be refined to permit this kind of
subextraction. In fact, the element of the specifier which is frozen is the
carrier of the criterial feature, the criterial goal, if we assume that the
criterial head enters into a probe--goal relation with the attracted phrase
\citep{Chomsky2000}. So, \REF{ex:key:21.5} should be refines as follows:

\begin{exe}
    \exi{(5$'$)} \isi{Criterial freezing}: In a criterial configuration, the criterial
    goal is frozen in place.\label{ex:key:21.5prime}
\z

See \textcite{Chomsky2013,Chomsky2015}, \citeauthor{Rizzi2015}
(\citeyear{Rizzi2015}; \citeyear{Rizzi2015b}; \citeyear{Rizzi2016a}) for
attempts to derive the effects of (5$'$) from the labeling\is{labelling} algorithm. I will not
address this important point here, and will just assume a descriptive
formulation like (5$'$).

Criterial Freezing\is{criterial freezing} separates specifier positions
targeted by \isi{movement} into two classes: \textsc{halting} positions, and
\textsc{transiting} positions. The \isi{criterial positions} are halting
positions, where \isi{movement} stops; transiting positions are specifier positions
from which \isi{movement} can (and in fact must) continue, for instance the C-system
of a verb like \emph{think}, which can function as an escape-hatch for a
wh-phrase, but not as the final landing site of wh-movement.

If we now turn to the system of A-movement, the typical halting position of
A-movement chains is the subject position of finite clauses (as opposed to
transiting A-positions, such as the subject positions of \isi{raising} clauses,
participial constructions, etc.). If halting positions are equated to criterial
positions, these considerations lead us to assuming a criterial position for
A-movement, a subject criterion (\citealt{rizzicriterial}, and much subsequent
work).

Criteria typically go with scope-discourse interpretive effects, such as the
\isi{topic}--comment or \isi{focus}--presupposition articulations. So, what could be an
analogous interpretive effect for subjects? Interpretively, the subject
position designates the referent “about which” the event is presented.
Active-passive pairs clearly differ in this aboutness property. The following
sentences are both appropriate in “all new” contexts, e.g., as answers to
questions like “What happened?”, or, with a narrower contextualization, “How
did the battle start?”:

\ea%7
\label{ex:key:21.7}\ili{Italian}
    \ea Un aereo ha attaccato un incrociatore\\
         ‘A plane   attacked      a cruiser’
    \ex Un incrociatore è stato attaccato  da un aereo\\
         ‘A cruiser          was attacked       by a plane’
    \z
\z
%
Both sentences felicitously depict an attacking event in the given context, but
(\ref{ex:key:21.7}a) depicts the event as being about a plane, the agent, and
(\ref{ex:key:21.7}b) depicts it as being about a cruiser, the patient. The
choice of the aboutness subject has consequences for discourse organization.
For instance, as \citet{Calabrese1986} pointed out, the choice affects anaphora
resolution in null subject languages,\is{null subject languages} in that a \emph{pro} subject in the
following sentence picks out the aboutness subject. So, if \REF{ex:key:21.8} is
uttered immediately after (\ref{ex:key:21.7}a), the intended interpretation is
that the plane asked for backup; if \REF{ex:key:21.8} is uttered after
(\ref{ex:key:21.7}b), it’s the cruiser which did (see also
\citealt{BellettiEtAl2007} on this effect):

\ea%8
    \label{ex:key:21.8}\ili{Italian}\\
    \dots{} poi, \emph{pro} ha chiesto rinforzi\\
    ‘\dots{} then \_\_ asked for backup’
\z
%
In much current work initiated in \citet{rizzicriterial}, and building on
\citet{Cardinaletti:2004a}, I have assumed that a nominal head Subj is an
obligatory component of the clausal spine. This head occurs immediately under
the lowest head of the complementizer\is{complementizers} system Fin, hence higher than T, so that
we have a partial map of the high part of the IP as follows:

\ea%9
    \label{ex:key:21.9}
    \dots{} Fin \dots{} Subj\tss{[$+$N]} \dots{} T \dots{}
\z
%
In syntax, Subj\tss{[$+$N]} attracts the closest nominal expression to its Spec.
At the interface, it triggers an interpretive routine along the following
lines: “interpret my Spec as the argument which the predicate is about, and my
complement as the predicate”.

\section{$+$N as an attracting feature to the subject position}

Why should $+$N be the relevant feature here? The obvious intuition is that the
system needs a nominal expression, capable of referring to an argument, to
trigger the appropriate aboutness interpretation. An alternative that comes to
mind, perhaps more in line with standard assumptions, is that the attracting
feature could be the set of Phi features.

One motivation for assuming $+$N to be the attractor is provided by the
widespread existence of quirky subject constructions, in which a non-nominative
nominal expression occupies a subject position (typically with psych-verbs and
a few other verbal classes in some languages):

\ea%10
    \label{ex:key:21.10}\ili{Italian}\\
    A Gianni    piacciono   queste idee\\
    ‘To Gianni  please        these ideas’
\z
%
In such cases, it is not very plausible that the attracting features would be
the Phi set, as the clause initial nominal does not enter into an agreement
relation with the inflected verb, whereas if the attracting feature is $+$N,
quirky subjects are expected. In languages like Icelandic, the quirky subject
with an inherent case\is{case!inherent case} may be a KP, in languages like
Italian it could be a KP or a PP, but in any event it plausibly is an “extended
projection” the nominal element, in \citegen{Grimshaw:1991} sense, hence
accessible to being attracted by a $+$N attractor.\footnote{That the
    dative\is{dative case} experiencer is in subject position, and not a
    \isi{topic}, is shown, among other things, by the fact that it does not
    interfere at all with A-bar extraction, whereas a genuine \isi{topic} does:
    \citealt{Calabrese1986,BellettiRizzi1988}. The special properties of
    expletives as elements formally satisfying the subject criterion are
    discussed in \citet{RizziShlonsky2007}.}

The point is relevant in the context of this paper because the hypothesis that
the attractor is $+$N may help explain other subject--object asymmetries
originally ascribed to the ECP.  One has to do to do with \emph{en}
cliticization in \ili{French}. As was shown by \citet{Ruwet1972}, the clitic
\emph{en} can pronominalize both a PP (in partitive constructions such as [
\emph{la première partie} [\tss{PP} \emph{de ce roman} ]] ‘the first part of
this novel’) and an NP (contained in a larger structure headed by a numeral,
such as [ \emph{trois} [\tss{NP} \emph{romans} ]] ‘three novels’):

\ea%11
    \label{ex:key:21.11}\ili{French}
    \ea Jean  en a publié [ la première partie \_\_\_] en 1968\hfill (de ce roman : en = pro-PP)\\
        ‘Jean of-it published the first part in 1968\hfill (of this novel)’
    \ex Jean  en a publié [ trois \_\_\_ ] en 1968\hfill (romans : en = pro-NP)\\
        ‘Jean of-them published three in 1968\hfill (novels)’
    \z
\z
%
But if the DP is in subject position, e.g., in the passivized versions of
\eqref{ex:key:21.11}, only PP extraction is possible, and NP extraction is
barred:

\ea%12
    \label{ex:key:21.12}\ili{French} (adapted from \citealt{Ruwet1972})
    \ea[]{a [ la première partie \_\_\_ ] en a été publiée \_\_\_ en 1968\\
        ‘The first part of-it was published in 1968’}
    \ex[*]{[ Trois \_\_\_ ] en ont été publiés \_\_\_           en 1968\\
    ‘Three         of-them have been published in 1968’}
    \z
\z
%
Why this asymmetry? In \citet[37--38]{Rizzi1990} I proposed an \gls{ECP}
analysis: in object position both traces are lexically governed, by the noun
\emph{partie} in (\ref{ex:key:21.11}a), and by the verb in (\ref{ex:key:21.11}b)
(under the definition of government adopted there). In (\ref{ex:key:21.12}a) the
trace is still lexically governed by the noun, but in (\ref{ex:key:21.12}b) there
is no lexical governor available, hence the structure is excluded.

How can this asymmetry be captured without appealing to the ECP? Under the
assumption that the attractor of subject is Subj\tss{[$+$N]}, the contrast
between (\ref{ex:key:21.12}a) and (\ref{ex:key:21.12}b) also follows: in the
derivation of (\ref{ex:key:21.12}a), after \emph{en} has been extracted, the
remnant DP still contains a nominal part, and can be attracted; in
(\ref{ex:key:21.12}b), the nominal part has been entirely extracted by
\emph{en} cliticization, hence the remnant DP is not extractable any longer
(under the copy theory of traces the trace of \emph{en} is still there, but
traces typically are not attractable elements).

It should also be noticed that the asymmetry shown by \REF{ex:key:21.12}
disappears under A$'$-movement of the object after \emph{en} cliticization:

\ea%13
    \label{ex:key:21.13}\ili{French}
    \ea {}[ Combien de parties \_\_\_ ] il en a publiées \_\_\_ en 1968?\\
        ‘How many parts \_\_\_  he of-it published in 1968)?’
    \ex {}[ Combien \_\_\_ ]  il en a publiées \_\_\_ en 1968?\\
            ‘How many        he of-them published in 1968?’
    \z
\z\largerpage[-5]
%
Here the $+$N analysis may have an advantage over the \gls{ECP}\is{Empty Category Principle} analysis:
according to the latter, there is no obvious reason why the lexical government
requirement could be lifted in the case of the output of A$'$-movement, as
(\ref{ex:key:21.13}b). By contrast, the alternative involving $+$N as an attractor
captures the contrast between (\ref{ex:key:21.12}b) and (\ref{ex:key:21.13}b): in
(\ref{ex:key:21.13}b) the attractor is +Q, and \emph{combien} clearly carries the
Q feature, so the fact that the NP has been extracted is irrelevant, and the
remnant can undergo A$'$-movement.\footnote{An anonymous reviewer observes that
    the contrast between (\ref{ex:key:21.12}a) and (\ref{ex:key:21.12}b) is
    reproduced if the clause is embedded under an “exceptional case marking”
    verb like \emph{laisser} (let) in \ili{French}:

\begin{exe}
    \exi{(i)}[*]{Il a laissé [ trois \_\_\_ ]  en être publiées.\\
        ‘He let three of-them+to+be published’}
    \exi{(ii)}[]{Il a laissé la premiere partie en être publiée.\\
        ‘He let [ the first part \_\_\_ ] of-it+to+be published’}
\end{exe}
%
The reviewer observes that the \gls{ECP}\is{Empty Category Principle} would not draw the right distinction
in this case because the trace of \emph{en} would be lexically governed by
\emph{laisser} in (i). The contrast follows from the analysis proposed in the
text if infinitival clauses of this kind also involve a  Subj\tss{[$+$N]} head.}

A somewhat analogous, but also different case of an asymmetry previously
connected to the \gls{ECP}\is{Empty Category Principle} concerns the fact that \emph{that} deletion cannot
affect a moved sentential complement:

\ea%14
    \label{ex:key:21.14}
    \ea[]{Bill didn’t say (that) John could win}
    \ex[*]{(that) John could win, Bill didn’t say \_\_\_}
    \ex[*]{(that) John could win wasn’t said by anyone}
    \z
\z
%
Here, contrary to \emph{en} extraction in \ili{French}, both A- and A$'$-movement
affect the structure. \citet{Stowell1981} originally observed that the
asymmetry in \REF{ex:key:21.14} recalls the \gls{ECP}, and \citet{Pesetsky1995}
captured this intuition by assuming that the deleted complementizer\is{complementizers} is in fact
(abstractly) cliticized to the main verb, so that the complementizerless
clauses do involve a trace of \isi{head movement}, arguably in the scope of the ECP.

An alternative to the \gls{ECP}\is{Empty Category Principle} analysis, still based on the Stowell-Pesetsky
insight, could be the following: the clause, in order to move in
(\ref{ex:key:21.14}a)-b must be attracted, but its head, the complementizer\is{complementizers}, has
already been attracted and incorporated into the verb; so, if traces are not
attractable, the whole clause cannot undergo \isi{movement}, and must remain in
complement position, as in  (\ref{ex:key:21.14}a). Notice that this analysis
implies that \isi{head movement} (however it is implemented) is part of narrow
syntax, as argued for in \citet{Roberts2010}, against the frequently made
assumption that \isi{head movement} is post-syntactic. The difference between
\emph{that} deletion and \emph{en} cliticization is that in the latter case the
head of the construction (the numeral, or possibly a higher abstract
determiner) is not affected by cliticization, so that there is no general ban
on \isi{movement} of the whole phrase, but only a selective ban linked to the $+$N
attractor. In case of \emph{that} deletion, the head of the whole construction
has been moved and has become a trace, so that the whole configuration is
unmovable.

\section{Subject--object asymmetries in extraction contexts}

We can now come back to subject--object asymmetries under A$'$-movement. If
criterial configurations are frozen, and there is a subject criterion, nominal
elements which reach Subj will be frozen there. I.e., the attempt of deriving a
sentence like (\ref{ex:key:21.1}a) would go through an intermediate
representation like \REF{ex:key:21.15}:

\ea%15
    \label{ex:key:21.15}
    You think [ that [ who Subj\tss{[$+$N]} will come \_\_\_ ]]
\z
%
Where \emph{who} will be frozen and will become inaccessible to further
movement. No similar effect arises in the case of object extraction
(\ref{ex:key:21.1}b), as there is no object criterion. The asymmetry thus
follows from \isi{criterial freezing} and the subject criterion, which provide
an alternative to the classical \gls{ECP}\is{Empty Category Principle}
analysis.

As usual, it is important to look for empirical differences between competing
analyses. One class of facts (originally pointed out to me by Paul
Hirschbühler) which seems to support the freezing analysis is the following.
The wh operator \emph{combien} in \ili{French} can be extracted from an object,
or pied-pipe the whole object, as in \REF{ex:key:21.16}:

\ea%16
    \label{ex:key:21.16}\ili{French}
    \ea Combien de personnes veux-tu rencontrer \_\_\_?\\
        ‘How many of people do you want to meet ?’
    \ex Combien veux-tu rencontrer [ \_\_\_ de personnes ]?\\
        ‘How many do you want to meet of people ?’
    \z
\z
%
Extraction of \emph{combien de NP} from an embedded subject position gives rise
to ungrammaticality (as in (\ref{ex:key:21.17}a)), but subextraction of
\emph{combien} from subject position is only mildly degraded, as in
(\ref{ex:key:21.17}b) (\citealt{Obenauer1976,Kayne1984}):

\ea%17
    \label{ex:key:21.17}\ili{French}
    \ea[*]{Combien de personnes veux-tu [ que [ \_\_\_  Subj viennent à ton anniversaire]] ?\\
        ‘How many people do you want that                     come to your birthday?’}
    \ex[?]{Combien veux-tu que [ [ \_\_\_ de personnes ] Subj viennent à ton anniversaire ] ?\\
        ‘How many do you want that of people                come to your birthday?’}
    \z
\z
%
Under the \gls{ECP}\is{Empty Category Principle} analysis, the ungrammaticality of (\ref{ex:key:21.17}a) is
expected, but (\ref{ex:key:21.17}b) would be predicted to be equally ill-formed:
if there is no lexical governor for a trace in subject position, a fortiori
there should not be a lexical governor for a trace in the specifier of the
subject. So, the improvement manifested by (\ref{ex:key:21.17}b) is not expected.

The freezing analysis, by contrast, predicts the ill-formedness of
(\ref{ex:key:21.17}a) as a violation of \isi{criterial freezing}, whereas it
makes no claim on (\ref{ex:key:21.17}b), which does not fall under the scope of
formulation  \eqref{ex:key:21.5prime}: only the criterial goal, the nominal
part of the DP, is frozen in the criterial configuration with Subj\tss{[$+$N]}.
The marginality of the example will be linked to other factors constraining
extractions from left branches (on such factors, and their interplay with
criteria, see \citealt{Lohndal2010}, \citealt{Berthelot2017}).

Other cases of special behavior of subjects may be amenable to the same
analysis. The complex inversion construction in \ili{French}
(\citealt{Kayne1972,RizziRoberts1989}, and subsequent work) involves a wh
element (or a null yes/no operator), a subject DP and the inflected verb with
an encliticized subject clitic, doubling the subject, as in \REF{ex:key:21.18}:

\ea%18
    \label{ex:key:21.18}\ili{French}\\
    Où        Jean est-il allé?\\
         ‘Where John did-he go?’
\z
%
If the inversion is a reliable cue that I to C (or, in current terms, T to Fin)
has occurred, the subject must sit in a special subject position higher than
Fin, hence in the left periphery.

Among the many noticeable properties of the construction there is the fact that
the left peripheral subject must be distinct from the wh-element, i.e., the
following is impossible:

\ea%19
    \label{ex:key:21.19}\ili{French}\\
    \llap{*}Qui    est-il  parti?\\
        ‘Who did-he leave?’
\z
%
\textcite{RizziRoberts1989}, following a suggestion due to Marc-Ariel
Friedemann, analyzed \REF{ex:key:21.19} as an \gls{ECP}\is{Empty Category
Principle} violation: \isi{movement} from the left-peripheral subject position
to the landing site of wh-movement would violate the head-government
requirement of the \gls{ECP}. How does this analysis translate into the system
developed here?

Evidently, in this construction, an extra subject position is licensed in the
lower part of the left periphery. One possible way to go is to assume that
I--to--C movement can carry along the Subj head to the left periphery, where it
remains active to license an A-specifier. If it is so, the subject criterion
configuration is reconstituted in the left periphery, yielding a representation
like the following:

\ea%20
    \label{ex:key:21.20}
    Où Foc  [ Jean est+Subj+Fin [ il \dots{} allé ]]
\z
%
If this derivational option is taken, and the subject is a wh-element, we would
obtain an intermediate representation like:

\ea%21
    \label{ex:key:21.21}
    Foc  [ qui est+Subj+Fin [ il \dots{} parti ]]
\z
%
But here \emph{qui} satisfies the subject criterion, therefore under criterial
freezing it cannot move further to the landing site of a wh-element,
Foc.\footnote{As for the possibility of local subject questions in general,
    \emph{qui est parti?}, \emph{who left?}, etc., one of the “skipping
devices” assumed in \textcite{RizziShlonsky2007} must be operative.} The
impossibility of \REF{ex:key:21.19} can thus be captured, and another case for
which the \gls{ECP}\is{Empty Category Principle} had been evoked can fall under the freezing approach.

\section{Conclusions}

The \gls{ECP}\is{Empty Category Principle} had a broad explanatory role in \gls{GB} syntax, where it offered
a coherent account of different constraints on \isi{movement} across languages. The
core case was the asymmetries between subject and object extraction from
embedded domain, the former being more severely constrained than the latter,
all other things being equal. Starting from the analysis of the core cases, a
very large array of phenomena across languages turned out to be amenable to an
\gls{ECP} analysis.

Under minimalist guidelines, the \gls{ECP}\is{Empty Category Principle} showed problematic features both
conceptual and technical: on the one hand, it did not seem to naturally fit the
principled typology of principles foreseen by minimalism; on the other hand,
its crucial reliance on government was problematic in a framework explicitly
attempting to do away with the government relation. So the principle was
abandoned, and the vast body of empirical discoveries connected to the
\gls{ECP} fell out of center stage in the minimalist literature.

In this article I have tried to show that certain important effects analyzed in
terms of the \gls{ECP}\is{Empty Category Principle} in previous literature (including my own work) could be
advantageously reanalyzed in different terms, relying on cartographic work and
on the system of criteria in particular. Criterial configurations are Spec-head
configurations which go with special interpretive instructions of the
scope-discourse kind.  So, criterial heads such as Top, Foc, Q, Rel, etc.
attract phrases with matching features to the specifier position, and guide the
interpretation of the structure, e.g., as expressing the \isi{topic}--comment or
focus--presupposition articulation, or explicitly marking the scope of
operators. One remarkable syntactic property of \isi{criterial positions} elucidated
in the recent literature is the freezing effect: a phrase meeting a criterion
(or, more accurately, the criterial goal) is frozen in the criterial
configuration and cannot undergo further \isi{movement}. \isi{criterial positions} thus are
“halting” sites for syntactic \isi{movement}. In a number of articles starting from
\citet{rizzicriterial} I have argued that freezing plays a key role in the
explanation of classical \gls{ECP}\is{Empty Category Principle} effects. If there is a Subject Criterion,
the halting character of subject positions is immediately captured. The
difficulty of extracting subjects, the prototypical case of which is the
\emph{that}-trace effect, can be made to follow from freezing. Subject--object
asymmetries follow from the fact that there is a subject criterion but not (in
typical cases) an object criterion.

In certain cases, the freezing approach is empirically advantageous compared to
the \gls{ECP}\is{Empty Category Principle} approach. We have seen a number of syntactic phenomena showing
asymmetries (\emph{en} cliticization, \emph{beaucoup} extraction in
\ili{French}, etc.) in which a requirement of lexical government seems to be
too weak, whereas a freezing analysis correctly captures the facts.

No attempt is made here (or in related work of mine) to capture the whole array
of \gls{ECP}\is{Empty Category Principle} phenomena in terms of freezing. For instance, the whole chapter of
\gls{ECP} effects at \glsunset{LF}\gls{LF}, and many of the “ECP extensions”,
in \citegen{Kayne1984} sense are not addressed. Nevertheless, it is important
to stress that some core \gls{ECP}\is{Empty Category Principle} effects are naturally and advantageously
amenable to an explanation in terms of tools provided by recent syntactic
theorizing. This offers the promise that also other aspects of the vast and
varied \gls{ECP}\is{Empty Category Principle} phenomenology may regain the focus of attention and offer new
grounds to test the explanatory capacities of current syntactic theory.

\printchapterglossary{}

\section*{Acknowledgements}

This research was supported by the ERC Advanced Grant n. 340297 “SynCart”.

{\sloppy
\printbibliography[heading=subbibliography,notkeyword=this]
}

\end{document}
