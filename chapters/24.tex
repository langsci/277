\documentclass[output=paper]{langsci/langscibook}
\ChapterDOI{10.5281/zenodo.4680324}
\author{Christina Sevdali\affiliation{Ulster University} and Michelle Sheehan\affiliation{Anglia Ruskin University}}
\title{Rethinking partial control: New evidence from finite control clauses}

\abstract{In this squib, we provide evidence that finite \isi{control} languages like
    Greek and \ili{Romanian} display \gls{PC}, albeit in very limited contexts,
    contrary to what has previously been claimed in the literature.  This fact
    poses problems for existing theories of \isi{control} which predict a fundamental
    incompatibility between \gls{PC}\is{control!partial control} and [+Agr]
    complements. These finding can be considered welcome, however, inasmuch as
    the ban on \gls{PC}\is{control!partial control} in [+Agr] contexts appears
    stipulative in the context of \citegen{Landau2015} approach.  They are also
    consistent with the claim that European Portuguese inflected infinitives,
    which are also [+Agr] also permit obligatory \isi{control}
\parencite{Sheehan2018a,Sheehan2018b}.}


\begin{document}\glsresetall
\renewcommand{\lsChapterFooterSize}{\footnotesize}
\maketitle

\section{Introduction}\label{sec:24.1}

\Gls{PC}\is{partial control|see{control}}\is{control!partial control} is a
phenomenon whereby a singular subject is able to function as the controller of
a reciprocal verb which, where matrix, would require a semantically plural
subject (see \citealt{Landau2000}).\footnote{In fact, even non-reciprocal verbs
    can be “coerced” into the \gls{PC}\is{control!partial control}
    interpretation, e.g. \textit{John wanted to apply for the grant together}. We limit
    ourselves to reciprocal verbs here as it makes \gls{PC}\is{control!partial
control} into a matter of grammaticality rather than interpretation.} Consider
the contrasts in grammaticality in (\ref{ex:24.1}a,b):\largerpage

\ea\label{ex:24.1}
	\ea The couple / John and Mary / *John broke up.
	\ex John didn’t want to break up.
	\z
\z

Whereas both the semantically plural group noun \emph{the couple} and the
syntactically plural co-ordination \emph{John and Mary} can function as the
subject of ‘break up’ in a simple monoclausal environment, the semantically and
syntactically singular \emph{John} cannot. This restriction is suspended in
the \isi{control} context in (\ref{ex:24.1}b), however, where the interpretation of the embedded
null subject (PRO) is such that it comprises John plus some other unspecified
person or persons, recovered from the context. \gls{PC}\is{control!partial
control} has been described in a number of languages (e.g. \ili{Russian},
\ili{European Portuguese}, \ili{Icelandic}, \ili{German} and more
controversially \ili{French} and \ili{Italian}) as
illustrated by the following examples:

\ea\label{ex:24.2} \ili{Russian} \parencite[909]{Landau2008}\\
	\gll    Ona   poprosila   predsedatelja [ sobrat’sja   vsem/*vsex   v šest’ ].\\
    she.\Nom{}   asked  chair.\Acc{} {} gather.\Inf{}   all.\Dat{}/*\Acc{}  at six\\
	\glt    ‘She asked the chair to all gather at six.’
\ex\label{ex:24.3} \ili{European Portuguese} \parencite[34]{Sheehan2018b}\\
	\gll Os professores   persuadiram   o director  [ a    reunir(em)=se     mais tarde].\\
    The teachers   persuaded   the headteacher {} \textsc{a} meet.\Inf{}.\Tpl=\textsc{se}.\Third{}   more late\\
	\glt ‘The teachers persuaded the headteacher to meet later on.’
\ex\label{ex:24.4} \ili{Icelandic} \parencite[149]{Sheehan2018b}\\
	\gll Hann bað    Ólaf     [að   hittast     einir/*eina]\\
    he      asked Olaf.\Acc{}    to  meet.\textsc{st}   alone.\Nom.\M.\Pl{} alone.\Acc.\M.\Pl{}\\
	\glt ‘He asked Olaf to meet alone.\Pl{}.’
\ex\label{ex:24.5} \ili{German} \parencite[45]{Landau2000}\\
	\gll Hans sagte der Maria dass er  es   bedauerte   letzte Nacht [ gemeinsam   gearbeitet   zu haben ]\\
    Hans said   the Maria that    he it   regretted   last night {} together     worked   to have\\
	\glt ‘Hans told Maria that he regretted having worked together last night.’
\ex\label{ex:24.6} \ili{French} \parencite[85]{Landau2000}\\
	\gll Jean a    dit      à Marie  qu’   il  veut    correspondre plus  souvent.\\
		Jean has said   to Marie that he wants correspond    more often\\
	\glt ‘John told Mary that he wants to correspond more often.’\pagebreak
\ex\label{ex:24.7} \ili{Italian}, adapted from \parencite[46]{Landau2000}\\
	\gll Maria pensava che Gianni avesse dimenticato di esser=si baciati alla festa.\\
		Maria thought that Gianni had.\Sbjv{} forgotten of be=\textsc{se}.\Third{} kissed.\Pl{} at.the party\\
	\glt ‘Maria thought that John had forgotten having kissed at the party.’
\z

In all of these languages the acceptability of \gls{PC}\is{control!partial
control} appears to be sensitive to the matrix \isi{control} predicate.\footnote{The
controversy surrounding the status of \gls{PC}\is{control!partial control} in
\ili{French} and Italian concerns the fact that in addition to being sensitive to the
matrix \isi{control} predicate, these languages also show sensitivity to the embedded
controlled predicate. In \ili{French} at least the generalisation seems to be that
\gls{PC}\is{control!partial control} is only possible where the embedded verb
is comitative \citep{Sheehan2014c,AuthierReed2018,PitteroffSheehan2018}.
\textcite{PitteroffEtAl2017a,PitteroffEtAl2017b} argue that German also shows
such a sensitivity.}  Following \textcite{Landau2000,Landau2004}, we
can thus make a distinction between \emph{PC predicates}, which permit either
partial or exhaustive \isi{control} into their complements and \emph{exhaustive
control predicates}, which permit only \gls{ExC}.\is{exhaustive
control|see{control}}\is{control!exhaustive control}

In \citeauthor{Landau2000}’s \isi{Agree}-based model
(\citeyear{Landau2000,Landau2004} et seq.) the difference between
\gls{PC}\is{control!partial control} and \gls{ExC}\is{control!exhaustive
control} predicates is regulated by their ability to support independent
temporal reference in their non-finite complement: \gls{PC}\is{control!partial
    control} predicates (including desideratives, factives, interrogatives and
    epistemics) allow this and so are [+T], whereas
    \gls{ExC}\is{control!exhaustive control} predicates (aspectuals, modals and
    implicatives) do not and so are [−T]. \citet{Pearson2016} however, claims
    that \gls{PC}\is{control!partial control} predicates are better defined as
    attitude predicates reporting on the mental state or a communicative act of
    some individual (e.g. \emph{believe}, \emph{want}, \emph{hope} but also
    \emph{say}, \emph{promise} and \emph{claim}):

\ea\label{ex:24.8} Non-attitude predicates
    \ea[*]{John started to break up.\hfill [aspectual]}
    \ex[*]{John must break up.\hfill [modal]}
    \ex[*]{John managed to break up.\hfill [implicative]}
	\z
\ex\label{ex:24.9} Attitude predicates
	\ea John hoped to break up.\hfill [desiderative]
	\ex John hated to break up.\hfill [factive]
	\ex John wondered whether to break up.\hfill [interrogative]
	\z
\z

There is a class of languages, however, which is claimed not to permit
\gls{PC}\is{control!partial control} at all, namely those languages which make
extremely restricted use of non-finite complementation and instead display
finite \isi{control}. Amongst these are the languages of the Balkan Sprachbund (e.g.\
\ili{Greek}, \ili{Romanian}, \ili{Bulgarian} etc.). In this paper, we re-evaluate this claim,
providing data which calls it into question. While it is generally the case
that obligatory \isi{control} in finite-control languages is limited to the
complements of \gls{ExC}\is{control!exhaustive control} predicates, we
nonetheless show that, under the root modal ‘can’, obligatory \isi{control}
complements permit \gls{PC}\is{control!partial control} for many speakers. The
structure of the squib is as follows. \Cref{sec:24.2} reviews the treatment
of finite \isi{control} in previous analyses, notably
\textcite{Landau2004,Landau2015}. \Cref{sec:24.3} reviews the evidence for
\gls{PC}\is{control!partial control} in Greek. \Cref{sec:24.4} identifies
similar such cases in \ili{Romanian}. \Cref{sec:24.5} concludes by discussing the
theoretical implications of the existence of partial \isi{control} in finite \isi{control}
languages.

\section{Finite control in previous approaches}\label{sec:24.2}

It is often claimed that Balkan languages lack \gls{PC}\is{control!partial
control} (see \citealt{Alboiu2007} on \ili{Romanian}). With the  exception of
\citet{Spyropoulos2007b} to whom we will return shortly, this claim is echoed
with respect to Modern Greek (see \citealt[95]{AlexiadouEtAl2010}, citing
\citealt{Varlokosta1994} on Greek).\footnote{When discussing Greek, we refer to
Standard Modern Greek, unless stated otherwise.}  Indeed,
\citeauthor{Landau2004}’s (\citeyear{Landau2004,Landau2015}) analyses of
obligatory \isi{control} attempts explicitly to derive the fact that
\gls{PC}\is{control!partial control} is not possible in these languages.

In all of its instantiations, \citeauthor{Landau2000}'s
(\citeyear{Landau2000,Landau2004,Landau2015}) model distinguishes two types of
control: \gls{PC}\is{control!partial control} and
\gls{ExC}\is{control!exhaustive control}. In earlier versions of the theory,
these are the result of two different operations: direct \isi{control} of PRO by an
antecedent from the main clause in the cases of \gls{ExC}\is{control!exhaustive
control}, and \isi{control} of PRO via C in the cases of PC. Crucially, the
distribution of the two kinds of \isi{control} is claimed by Landau to be regulated
by the features [+/−T] and [+/−Agr]. \gls{ExC}\is{control!exhaustive control}
arises in [+/−Agr, −T] contexts and \gls{PC}\is{control!partial control} in
[−Agr, +T] contexts. As finite complements in languages like Greek and \ili{Romanian}
are characterised by being [+Agr], these languages are therefore expected to
lack \gls{PC}\is{control!partial control} as they lack [−Agr] clauses
altogether.  \citet[7]{Landau2015} summarises the findings of his early work in
the “obligatory control--no control” generalisation in \eqref{ex:24.10}:

\ea\label{ex:24.10} The obligatory control--no \isi{control} generalisation\\
In a fully specified clause (in a clause in which the I head carries slots for both [T] and [Agr])
	\ea If the I head carries both semantic tense and agreement, [no control] obtains.
	\ex Elsewhere, [obligatory control] obtains.
	\z
\z

He presents evidence in support of this prediction from finite \isi{control} in
Balkan languages. Building on \citet{Varlokosta1994}, he argues that Balkan
subjunctives come in two types: \emph{controlled} and \emph{free} subjunctives
(C- and F-subjunctives respectively, exemplified below) distinguished by the
interpretation of their subjects, expressed here as the distinction between PRO
and \emph{pro}. As \citet[827]{Landau2004} further notes, C-subjunctives
display the diagnostic properties of obligatory \isi{control}, despite their
finiteness:

\ea\label{ex:24.11} C-subjunctive, \ili{Greek}\\
    \gll I Maria\textsubscript{1}   kseri PRO\textsubscript{1/*2} na   diavazi \\
		the Mary   know.\Tsg{} {}  \Ptcl{}   read.\Tsg{}\\
	\glt ‘Mary knows how to read’
\ex\label{ex:24.12} F-subjunctive, \ili{Greek} \parencite[(21)]{Varlokosta1994}\\
	\gll O Yianis\textsubscript{1}   elpizi  \emph{pro}\textsubscript{1/2} na   figi. \\
		the John   hopes.\Tsg{} {}    \Ptcl{}   wins.\Tsg{} \\
	\glt ‘John\textsubscript{1} hopes that he\textsubscript{1/2} will win’
\ex\label{ex:24.13} C-subjunctive, \ili{Romanian} \parencite[6]{Alboiu2007}\\
	\gll Victor\textsubscript{1}  încearcă   [ să          PRO\textsubscript{1/*2} cânte ].\\
        Victor    try.\Prs.\Tsg{} {} \Sbjv{} {}    sing.\Sbjv.\Tsg{}\\
	\glt ‘Victor is trying to sing.’
\ex\label{ex:24.14} F-subjunctive, \ili{Romanian}\\
	\gll Ionuț\textsubscript{I} vrea  [ sǎ     \emph{pro\textsubscript{i/j}}\textsubscript{}  cânte ].\\
            Ionuț    wants {}  \Sbjv{} {} play.\Sbjv{}.\Tsg{}\\
	\glt ‘Ionuț wants him/PRO to sing.’
\z

As Landau notes, many \gls{ExC}\is{control!exhaustive control} predicates seem
to require C-subjunctives where\-as \gls{PC}\is{control!partial control}
predicates usually take F-subjunctives and hence fail to display obligatory
control. This follows if their complements are [+Agr, +T], leading to the
possibility of referential subjects.

\citet{Landau2015} revises his early approach to the \gls{PC}/\gls{ExC}
distinction, drawing on \citegen{Pearson2016} idea that the defining property
of \gls{PC}\is{control!partial control} predicates is that they are attitudinal
(\emph{hope}, \emph{want}, \emph{regret}) unlike
\gls{ExC}\is{control!exhaustive control} predicates which are not
(\emph{start}, \emph{manage}, \emph{try}). He proposes that whereas attitude
predicates select a larger non-finite complement containing a logophoric
\emph{pro} in its edge which mediates \isi{control}, \gls{ExC}\is{control!exhaustive
control} predicates select a smaller complement and \isi{control} arises from direct
predication. The generalisation in \eqref{ex:24.10} now equates to that in
\eqref{ex:24.15}:\glsunset{OC}

\ea\label{ex:24.15} The \gls{OC}--NC generalisation (final)\\
    {}[+Agr] blocks logophoric \isi{control} but not predicative \isi{control}.
\z

Landau proposes to derive \eqref{ex:24.15} from the fact that variable \isi{binding} requires
feature sharing and this is blocked where a pronoun is involved. In finite
control languages, then, logophoric \isi{control} will always be blocked as every
clause is [+Agr].

\section{Partial control in Greek}\label{sec:24.3}

The phenomenon of \gls{PC}\is{control!partial control} in \ili{Greek} has been
discussed very little in the literature. This is because, as noted above,
\gls{PC}\is{control!partial control} predicates tend to select F-subjunctives
and so apparent instances of \gls{PC}\is{control!partial control} can always, in
principle, be cases of accidental (partial) co-reference between main and
embedded subject.\footnote{The following Greek verbs take what Landau calls
    \enquote{F-subjunctive complements}: \emph{elpizo} `hope',
    \emph{pistevo}, 'believe', \emph{nomizo} `think', \emph{apofasizo}
    `decide', \emph{protimo} 'prefer', \emph{thelo} `want', \emph{perimeno}
    `expect', \emph{efchome} 'wish', \emph{sxediazo} `arrange',
    \emph{prospatho} `try', \emph{frontizo} `arrange', \emph{kataferno}
    `succeed', \emph{pitho} `persuade', \emph{zito} `ask', \emph{apagorevo}
    `forbid', \emph{diatazo} 'order'. Verbs taking \enquote{C-subjunctive}
    complements include aspectuals (\emph{arxizo} `start', \emph{teliono}
    `finish', \emph{sinexizo} `continue') and other
    \gls{ExC}\is{control!exhaustive control} verbs (\emph{distazo} `hesitate',
    \emph{dokimazo} `try', \emph{matheno} `learn'), but also a set of verbs
    which one expect be \gls{PC}\is{control!partial control} predicates,
    including: \emph{xerome} `be pleased', \emph{ipofero} `suffer',
    \emph{fovame} `fear', \emph{ksero} `know', \emph{erchome} `come', \emph{ime
ipochreomenos} `be obliged', \emph{ekana to lathos} `make the mistake'
\parencite[Ch.\ 4]{Varlokosta1994}.\label{fn:24.4}} Consider, by way of
example, the apparent cases of \gls{PC}\is{control!partial control} given by
\textcite{Spyropoulos2007b}, cited also by \textcite{Kapetangianni2010}, with
object \isi{control} verbs like \emph{pitho} `to persuade' and \emph{diatazo} `to
order':

\ea%16
    \label{ex:24.16}\ili{Greek}\\
	\gll o Yianis\textsubscript{i}       epise         ti Maria\textsubscript{j}       na     pane ec\textsubscript{i+j} jia psonia tin Triti\\
		the John.\Nom{}   persuade.\Tsg.\Pst{}    the Mary.\Acc{}    \Sbjv{} go.\Tpl{} {} for shopping the Tuesday\\
	\glt ‘John persuaded Mary to go (John and Mary) shopping on Tuesday.’
\ex%17
    \label{ex:24.17} \ili{Greek} \parencite[(34a), ((35a)]{Spyropoulos2007b}\\
	\gll episa       ti Maria\textsubscript{i}   na   pane ec\textsubscript{i+}   jia psonia tin Triti\\
        persuade.\Fsg{}.\Pst{}   the Mary.\Acc{} \Sbjv{} go.\Tpl{} {}  for shopping the Tuesday\\
	\glt ‘I persuaded Mary that they should go for shopping on Tuesday.’\\
\z

The problem with these examples is that, as \citet{Varlokosta1994} notes, these
verbs take F-subjunctive rather than C-subjunctive complements: they permit
overt nominative\is{nominative case} subjects, strict and sloppy readings under
ellipsis, and non \emph{de se} readings.

Looking beyond these examples, however, we find that Greek displays
\gls{PC}\is{control!partial control} with modal \emph{mporo} ‘can’ which
selects a C-subjunctive:

\ea%18
    \label{ex:24.18} \ili{Greek}\\
	\gll Chthes   \textbf{mporusa}  akoma na     sinandithume tin alli Triti\\
        yesterday  could.\Fsg{}  still     \Sbjv{} meet.\Sbjv{}.\Fpl{}  the other Tuesday\\
	\glt ‘Yesterday I was still able for us to meet next Tuesday.’
\z

In \eqref{ex:24.18}, we see not only that the embedded subject of the subjunctive clause
can be interpreted as partially controlled by the main subject, but also that
the two clauses are indeed temporally distinct, as they allow two separate
temporal adverbials `yesterday' modifying the main clause event, and `the
following Tuesday' modifying the embedded clause event. Data like this appear
to challenge the link between \gls{PC}\is{control!partial control} and
attitude-predicates: example \eqref{ex:24.18}, an apparent case of
\gls{PC}\is{control!partial control} in a finite \isi{control} language is found in a
case of temporal independence (and therefore a +T environment) under a
non-attitude predicate.

One of the key diagnostics that we use to distinguish between F-subjunctives
and C-subjunctives and hence between \gls{PC}\is{control!partial control} and
accidental partial co-reference (NC), following \citet{Varlokosta1994} and
\citet{Landau2004} is the possibility of an overt or covert
nominative\is{nominative case} subject with disjoint reference from any matrix
argument. Example \eqref{ex:24.19} is ungrammatical in Greek (as it is in
Romanian, cf.\ \Cref{sec:24.4}), suggesting that \eqref{ex:24.18} is a
genuine instance of PC:

\ea%19
    \label{ex:24.19}\ili{Greek}\\
	\gll \llap{*}Mporo   na   fas\\
        can.\Fsg{}  \Sbjv{}   eat.\Ssg{}\\
    \glt intended: ‘I can you to eat.’
\z

There is certainly a difference between \emph{mporo} and verbs which freely
permit an F-subjunctive complement, such as those listed in \cref{fn:24.4}. It
is possible, however, to coerce a disjoint reading with \emph{mporo}, as
suggested to us by Vina Tsakali and Despoina Oikonomou (p.c.).

\ea%20
    \label{ex:24.20}\ili{Greek}\\
	\gll Mporis  na   erthi     i Pinelopi     sto   parti   tu  Felix?\\
		can.\Ssg{} \Sbjv{}   come.\Tsg{}   the Penelope.\Nom{}      to.the   party   the Felix.\Gen{}\\
	\glt ‘Can you arrange / allow for Penelope to come to Felix’s party?’
\z

In fact, an anonymous reviewer suggests that this is even possible in
\eqref{ex:24.21} if we add ‘at my home’ to the example:

\ea%21
    \label{ex:24.21}\ili{Greek}\\
	\gll Mporo   na   fas   spiti mu\\
    can.\Fsg{}   \Sbjv{}   eat.\Ssg{}  home my\\
	\glt ‘It is possible for me that you eat at my place.’
\z

One possible conclusion, then is that \emph{mporo} allows for a complement
clause with a disjoint reference subject, and therefore an F-subjunctive, so
that \eqref{ex:24.18} is not an instance of \gls{PC}\is{control!partial control} after all.
There are however, two objections to this line of argumentation: firstly such
examples are indeed quite labored and require a very elaborate context. As an
anonymous reviewer notes, such contexts usually involve some relationship
between a matrix argument and something in the embedded clause, something which
is not required with verbs which freely select for F-subjunctives. Moreover,
partial \isi{control} verbs always seem to allow coercion of this kind with overt
subjects: unlike \gls{ExC}\is{control!exhaustive control} predicates. Consider
for example the following example from \ili{English}:

\ea%22
    \label{ex:24.22}
    I persuaded Mary for her children to wear a coat.
\z

In \eqref{ex:24.22}, \emph{persuade}, which usually favours an obligatory
control reading, permits disjoint reference in exactly the same kind of context
discussed as in \eqref{ex:24.20} and \eqref{ex:24.21}. The fact that
\emph{mporo} permits coercion of this kind therefore actually makes it look
like a \gls{PC}\is{control!partial control} predicate from a comparative
perspective.  To this extent, then, examples like \eqref{ex:24.20} and
\eqref{ex:24.21} do not undermine the point made here about a finite
control language exhibiting PC. A remaining question is why can
\gls{PC}\is{control!partial control} be coerced into allowing for disjoint
reference whereas \gls{ExC}\is{control!exhaustive control} cannot. This seems
to point towards treating the two phenomena as distinct, and not one as a
subclass of the other, but a detailed formulation of this intuition lies beyond
the scope of this work for reasons of space (though see
\citealt{Cinque2006,Landau2000,Landau2008,Landau2015,Sheehan2018b} for
different implementations of this idea).\footnote{An anonymous reviewer notes
    that if languages with finite \isi{control} permit coercion more easily than
    languages with non-finite \isi{control}, then this might be taken to support a
    weakened version of \citegen{Landau2015} proposal. The facts are not so
    clear to us, though, as \ili{English} appears to allow coercion with
\gls{PC}\is{control!partial control} verbs just as easily as Greek does. In any
case, a problem remains for Landau’s general approach if there is a [+Agr, +T]
context in which the default reading is \isi{control}.}

\section{Partial control in Romanian}\label{sec:24.4}

In \ili{Romanian} too, the vast majority of matrix verbs selecting a
\emph{C-subjunctive} (with forced co-reference) are
\gls{ExC}\is{control!exhaustive control} predicates in \citegen{Landau2000}
sense (\emph{şti} ‘know’, \emph{începe} ‘begin’, \emph{încearca} ‘try’ and
\emph{reuş}\emph{i} ‘manage’). Conversely, the vast majority of
\gls{PC}\is{control!partial control} predicates select an \emph{F-subjunctive}
in \ili{Romanian} with a referential subject, which, given the lack of obviation
effects, can also be co-referential with the matrix subject, but need not be
(see \citealt{Alboiu2007,AlexiadouEtAl2010,Hill2012,Nicolae2013}
on \ili{Romanian} and examples \eqref{ex:24.13} and \eqref{ex:24.14}
above). F-subjunctive complements can optionally be introduced by the
subjunctive complementiser \emph{ca} (cf.\
\citealt{GrosuHorvath1987,Hill2012}). The \emph{ca sǎ} subjunctive complements
display typical \ili{Romance} obviation effects but, the bare \emph{sǎ} complements
do not \citep{AlexiadouEtAl2010}:

\ea%23
    \label{ex:24.23}\ili{Romanian}
	\ea
		\gll Ionuț\textsubscript{i} vrea     \textbf{sǎ} EC\textsubscript{i/j} cânte     la violoncel\\
        Ionuț  wants     \Sbjv{}   {}  play.\Sbjv{}.\Tsg{}  at cello\\
	\ex	\gll Ionuț\textsubscript{i}  vrea   \textbf{ca} \textbf{sǎ}   EC\textsubscript{*i/j} cânte     la violoncel\\
        Ionuț  wants   that   \Sbjv{}  {}   play.\Sbjv{}.\Tsg{}  at cello\\
	\z
\z

As in \ili{Greek}, the C-subjunctives display the properties of obligatory
control
\parencite{Landau2004,Alboiu2007,AlexiadouEtAl2010,Hill2012,Nicolae2013}:

\ea%24
    \label{ex:24.24}\ili{Romanian}\\
	\gll \llap{*}Victor încearcă     [ Mihai     să   cânte ]\\
        Victor try.\Prs.\Tsg{} {} (*Mihai) \Sbjv{} sing.\Sbjv.\Tsg{}\\
    \glt ‘Victor is trying (*Mihai) to sing.’
\z

This is not restructuring, however: the embedded clause can contain negation,
can be modified by an adverb and does not always permit \isi{clitic climbing}
\parencite{Alboiu2007,AlexiadouEtAl2010}:

\ea%25
    \label{ex:24.25}\ili{Romanian} \citep[8]{Alboiu2007}
	\ea
		\gll Li=a         putut     vedea?\\
        \Cl{}.\Tsg.\M.\Acc{}=\Aux.\Tsg{}   could.\Ptcp{}   see.\Tsg{}\\
		\glt ‘Could s/he see him?’
	\ex
		\gll Nu (*li)-a             încercat    [ să-li       vadă ].\\
        not \hphantom{(*}\Cl.\Tsg.\M.\Acc{}=\Aux.\Tsg{} try.\Ptcp{} {} \Sbjv=\Cl.\Tsg.\M.\Acc{} see.\Tsg\\
		\glt ‘S/he didn’t try to see him.’
	\z
\z

There is disagreement in the literature over whether this is \isi{raising} or \isi{control}
(see \citealt{Nicolae2013}). We assume they at least \emph{can} be obligatory
control contexts here, partly on the basis of the \gls{PC}\is{control!partial
control} evidence below.

It has been claimed that \ili{Romanian} lacks partial \isi{control}.
\citet{AlexiadouEtAl2010} claim that \ili{Romanian} lacks partial \isi{control} based on
the following data (‘learn’ is an obligatory \isi{control} verb, as in Greek):

\ea%26
    \label{ex:24.26} \ili{Romanian}
    \ea[*]{
		\gll Eu   am   invătat sǎ   inotăm\\
        I   have   learnt   \Sbjv{}   swim.\Sbjv{}.\Fpl{}\\
    \glt}
    \ex[*]{
		\gll Ion a     zis   ca   \textbf{tu} ai   invătat sǎ   inotati.\\
        John has   said   that   you.\Sg{}   have   learnt   \Sbjv{} swim.\Sbjv{}.\Spl{}\\
        \glt}
	\z
\z

\citet[10]{Alboiu2007} claims the same thing on the basis of the following
examples:

\ea%27
\label{ex:24.27}\ili{Romanian}\\
	\gll \llap{*}Eu vreau   [ să   plec      împreună ]\\
        I   want.\Fsg{} {} \Sbjv{}   leave.\Sbjv{}.\Fsg{}  together\\
\ex%28
\label{ex:24.28}\ili{Romanian}\\
	\gll \llap{*}Vreau     [ să   plecăm       eu  împreună ]\\
    want.\Fsg{} {}  \Sbjv{}   leave.\Sbjv{}.\Fpl{}   I  together\\
\z

There are, however, independent explanations as to why these examples are
ungrammatical. In \eqref{ex:24.27}, a predicate with a singular subject is modified by
together and in \eqref{ex:24.30} there is a mismatch between the plural verb form and
singular subject. Alboiu also notes that the following is permitted:

\ea%29
\label{ex:24.29}\ili{Romanian}  \citep[10]{Alboiu2007}\\
	\gll Eu vreau     [ să   plecăm      împreună ]\\
    I   want.\Fsg{} {} \Sbjv{}   leave.\Sbjv{}.\Fpl{}   together\\
	\glt ‘I want (us) to leave together.’
\z

The problem is that, as noted in relation to Greek, and as she notes, we cannot
tell whether \eqref{ex:24.29} involves partial \isi{control} or accidental co-reference as
\emph{a vrea} ‘to want’ (like other desiderative predicates) takes an
F-subjunctive which does not force obligatory coreference:

\ea%30
    \label{ex:24.30}\ili{Romanian}  \citep[11]{Alboiu2007}\\
	\gll \emph{pro}\textsubscript{1}   vrea       [ \emph{pro\textsubscript{1/2}} să   plece ]\\
        {} want.\Prs.\Tsg{}  {} {} \Sbjv{}   leave.\Sbjv{}.\Third{}\\
	\glt ‘S/he wants (for her/him/them) to leave.’
\z

The problem, then, is that the vast majority of obligatory \isi{control} verbs in
Romanian happen to be exhaustive \isi{control} predicates, which fail to allow
partial \isi{control} in any language (see \citealt{Landau2000,Landau2004,Landau2015}
and the discussion above).

Like in Greek, however, there is one \gls{ExC}\is{control!exhaustive control}
predicate which takes C-subjunctive complements and nonetheless permits PC: the
modal \emph{putea} ‘can’. Example \eqref{ex:24.31} shows that \emph{putea} takes a
C-subjunctive and not an F-subjunctive. Examples \eqref{ex:24.32} and \eqref{ex:24.33} show that
partial \isi{control} is nonetheless permitted here with either a \Fsg{} or \Tsg{}
subject controlling a \Fpl{} verb form (based on judgments from four speakers):

\ea%31
    \label{ex:24.31} \ili{Romanian}\\
	\gll \llap{*}Tu   poți   să   meargă   mâine.\\
    you   can.\Ssg{}   \Sbjv{}   go.\Sbjv{}.\Third{} tomorrow\\
    \glt
\ex%32
    \label{ex:24.32} \ili{Romanian}\\
	\gll Pot   să   ne     întâlnim     mâine.\\
    can.\Fsg{}  \Sbjv{}  \textsc{se}.\Fpl{}  meet.\Sbjv{}.\Fpl{}  tomorrow\\
	\glt ‘I can meet tomorrow.’
\ex%33
    \label{ex:24.33} \ili{Romanian}\\
	\gll Pot   să   ne    căsătorim         doar la anul,      când fac      18 ani.\\
    can.\Fsg{}  \Sbjv{}  \textsc{se}.\Fpl{} marry.\Sbjv{}.\Fpl{} only to year.\Def{} when make 18 years\\
	\glt ‘I can marry only next year, when I turn 18.’
\ex%34
    \label{ex:24.34} \ili{Romanian}\\
	\gll Ea  poate   să   ne   căsătorim         doar la anul,      când face   18 ani.\\
    she can   \Sbjv{}   \textsc{se}.\Fpl{} marry.\Sbjv{}.\Fpl{} only  to year.\Def{}      when makes   18 years\\
	\glt ‘She can marry only next year, when she turns 18.’
\z

This is particularly interesting because, unlike Greek, \ili{Romanian} retains
limited usage of non-finite clauses and one context where the latter occur is
precisely under this same verb:

\ea%35
    \label{ex:24.35} \ili{Romanian} \citep[136]{PanaDindelegan2013}\\
	\gll El poate alerga\\
		he can run.\Inf{}\\
	\glt ‘He can run’
\z

Bare infinitives of this kind probably involve restructuring as \isi{clitic climbing}
and long passives are permitted here \citep[194, 196]{Dragomirescu2013}:

\ea%36
    \label{ex:24.36}\ili{Romanian}\\
	\gll Cartea     o       pot   citi   acum\\
        book.\Def{}.\Acc{}   \Cl.\Acc{}.\glossF.\Tsg{}   can.\Fsg{}   read.\Inf{} now\\
	\glt ‘I can read the book now’
\ex%37
    \label{ex:24.37}\ili{Romanian}\\
	\gll Cartea     se          poate      citi   de {către oricine} ȋntr-o zi\\
    book.\Def{}.\Nom{}   \Cl.\Refl.\Pass{} can.\Tsg{} read.\Inf{}   by anyone   in=one day\\
	\glt ‘The book can be read by anyone in one day’
\z

Until the 19th century, \emph{putea} also freely selected an infinitive
complement introduced by \emph{a}, but nowadays this possibility is restricted
to complements which are negated \citep{Dragomirescu2013}. No \isi{clitic climbing}
is possible where \emph{a} is present:

\ea%38
    \label{ex:24.38}\ili{Romanian} (\citealt[194]{PanaDindelegan2013}, citing \citealt[60]{Jordan2009})\\
	\gll El putea       a   nu-l       primi\\
    he can.\Ipfv{}.\Tsg{}   \textsc{a}   not=\Cl.\Acc{}.\M.\Tsg{}   receive.\Inf{}\\
	\glt ‘He could not receive it.’
\z

Even where \emph{a} is present, however, \gls{PC}\is{control!partial control} is not possible with a non-finite complement:

\ea%39
    \label{ex:24.39}\ili{Romanian}\\
	\gll \llap{*}Tu   poți   a vă     căsători   la anul.\\
        you   can.\Ssg{} \textsc{a} \textsc{se}.\Spl{}  marry.\Inf{}   to year.\Def{}\\
    \glt ‘You can marry next year.’
\z

This minimal contrast between finite and non-finite complements suggests that
this is a matter of syntax and not semantics as presumably the modal has the
same meaning in both contexts. Like in Greek, then, there is at least one
\gls{ExC}\is{control!exhaustive control} predicate which appears to permit
\gls{PC}\is{control!partial control} in finite \isi{control} contexts.

\section{Theoretical discussion and tentative conclusions}\label{sec:24.5}

A very important question is whether the examples of
\gls{PC}\is{control!partial control} in Greek and \ili{Romanian} mentioned above are
genuine instances of PC. \citet{Poole2015} notes that a similar phenomenon is
possible also in \ili{English} with the root modal ‘can’, but he claims that it is
not an instance of \gls{PC}\is{control!partial control} (pace
\citealt{Rodrigues2007}). He proposes, rather, that apparent instances of
\gls{PC}\is{control!partial control} under ‘can’ in \ili{English} actually involve a
covert comitative, based on the fact that only comitative verbs can surface in
the complement to \emph{can} in instances of \gls{PC}:\largerpage

\ea%40
    \label{ex:24.40} \textcite[14]{Poole2015}
	\ea[*]{John can gather tomorrow.}
	\ex[*]{John can disperse next week.}
	\z
\z

He therefore proposes the following analysis (see also \citealt{Sheehan2014c} on
\enquote{fake \gls{PC}} in some \ili{Romance} languages):

\ea\label{ex:24.41} Modal-meet construction schema \parencite[15]{Poole2015}\\%41
    XP\textsubscript{1} can [ t\textsubscript{1} meet (with y) ]
\z

The core idea here is that the plural reading of \emph{meet} arises from the
exceptional possibility of a covert comitative and not from partial \isi{control}. In
fact, \emph{can} is analysed as a \isi{raising} predicate on his
analysis.

This account however clearly does not carry over to the \ili{Romanian} and
\ili{Greek} facts. In these languages, the embedded subject clearly differs in \isi{φ-features}
from the matrix subject so the effect cannot reduce to \isi{raising} (or \gls{ExC}).
Moreover, the plural reading of the embedded predicate marry/meet cannot be
attributed to the presence of a covert comitative as the embedded verb is
itself inflected as plural. Finally, note that examples involving an overt
comitative are possible with these verbs, but the comitative cannot be omitted
in these contexts.

Many verbs in \ili{Romanian} undergo the comitative alternation (\emph{a se certa}
‘to argue’, \emph{a se întâlni} ‘to meet’, \emph{a se săruta} ‘to kiss’,
\emph{a se împăca} ‘to make up’):

\ea%42
    \label{ex:24.42}\ili{Romanian}\\
	\gll \llap{*}Alex     se   întâlnește\\
		Alex     \textsc{se}   meet.\Tsg{}\\
\ex%43
    \label{ex:24.43}\ili{Romanian}\\
	\gll Alex     se   întâlnește   cu   Adina\\
		Alex     \textsc{se}   meet.\Tsg{}  with  Adina\\
\ex%44
    \label{ex:24.44}\ili{Romanian}\\
	\gll Alex și Adina   se   întâlnesc\\
		Alex and Adina   \textsc{se}   meet.\Tpl{}\\
\z

These verbs can occur in \isi{control} contexts with a singular antecedent, but the
\Tsg{} and \Tpl{} forms of the subjunctive are identical, so it is impossible
to tell whether the comitative can be omitted in the equivalent to \eqref{ex:24.45}:\largerpage

\ea%45
    \label{ex:24.45}\ili{Romanian}\\
	\gll Vrea    să   se   întânească    mâine     (cu   ea)\\
    wants.\Tsg{} \Sbjv{}   \textsc{se}   meet.\Sbjv{}.\Third{}   tomorrow \hphantom{(}with   her\\
	\glt ‘He wants to meet (with her) tomorrow.’
\ex%46
    \label{ex:24.46}\ili{Romanian}\\
	\gll Vrea     să   se   certe     din   când   în   când   (cu   ea)\\
    wants.\Tsg{} \Sbjv{}   \textsc{se}   argue.\Sbjv{}.\Third{}   from   when to   when  \hphantom{(}with   her\\
	\glt ‘He wants to argue (with her) from time to time.’
\ex%47
    \label{ex:24.47}\ili{Romanian}\\
	\gll Vrea    să   se   sărute       curând     (cu   ea)\\
    wants.\Tsg{} \Sbjv{}   \textsc{se}   kiss.\Sbjv{}.\Third{}      soon \hphantom{(}with   her\\
	\glt literally ‘He wants to kiss (with her) soon.’
\ex%48
    \label{ex:24.48}\ili{Romanian}\\
	\gll Vrea   să   se   împace     (cu   ea)\\
    wants.\Tsg{} \Sbjv{}   \textsc{se}   make.up.\Sbjv{}.\Third{}  \hphantom{(}with  her\\
	\glt ‘He wants to make up (with her) soon.’
\z

If the subject is first or second person, however, the number distinction is
morphologically expressed and it is clearly not possible to omit the comitative
in such cases (based on a survey of 21 speakers):

\ea%49
    \label{ex:24.49}\ili{Romanian}\\
	\gll Vreau   să   mă     întâlnesc    mâine     *(cu ea)\\
		want.\Fsg{}  \Sbjv{}  \textsc{se}.\Fsg{}  meet.\Sbjv{}.\Fsg{}  tomorrow \hphantom{*(}with her\\
	\glt ‘I want to meet (with her) tomorrow.'
\ex%50
    \label{ex:24.50}\ili{Romanian}\\
	\gll Vrei   să   te     întâlneşti     mâine     *(cu ea)\\
		want.\Ssg{}  \Sbjv{}  \textsc{se}.\Ssg{}  meet.\Sbjv{}.\Ssg{}  tomorrow \hphantom{*(}with her\\
	\glt ‘You want to meet (with her) tomorrow.'
\z

This shows that the kind of \gls{PC}\is{control!partial control} observed in
Romanian does not involve a covert comitative. The situation in Greek is
exactly the same, with agreement interacting with the comitative alternation
where the presence of a comitative phrase induces singular agreement on the
verb \eqref{ex:24.49}, but the lack of the comitative phrase is only allowed when the verb
has plural agreement \eqref{ex:24.50}:

\ea%51
    \label{ex:24.51}\ili{Greek}\\
	\gll \llap{*}O Yianis sinantithike\\
		the John  met.\Tsg{}\\
    \glt
\ex%52
    \label{ex:24.52}\ili{Greek}\\
	\gll O Yianis sinantithike me ton Petro\\
		the John met.\Tsg{}    with the Peter\\
	\glt ‘John met with Peter.’
\ex%53
    \label{ex:24.53}\ili{Greek}\\
	\gll O Yianis ki o Petros sinantithikan\\
		the John and the Peter met.\Tpl{} \\
	\glt ‘John and Peter met.’
\z

Greek has the full agreement paradigm in subjunctive forms, so examples like
Romanian (\ref{ex:24.47}--\ref{ex:24.52}) display no ambiguity.
Indeed, \gls{PC}\is{control!partial control} cases with \emph{thelo} ‘want’
cannot involve a covert comitative exactly because the embedded verb appears in
the singular when there is a comitative phrase and in the plural without
it.\footnote{We use a verb which selects an F-subjunctive here because it is
    our intention to show that comitatives cannot be omitted in subjunctive
    contexts. The patterns are the same if the matrix verb is \emph{can}. Thanks to an
anonymous reviewer for querying this.}

\ea%54
    \label{ex:24.54}\ili{Greek}\\
	\gll Thelo / mporo    na   sinantithume   avrio\\
        want.\Fsg{} {} can.\Fsg{}  \Sbjv{}   meet.\Fsg{}   tomorrow\\
	\glt ‘I want to meet (plural) tomorrow.’, ‘I can meet (plural) tomorrow.’
\ex%55
    \label{ex:24.55}\ili{Greek}\\
	\gll Thelo  / mporo   na   sinantitho   me tin Stefania   avrio\\
    want.\Fsg{} {}  can.\Fsg{} \Sbjv{}  meet.\Fsg{}   with the Stefania   tomorrow\\
	\glt ‘I want to meet with Stefania tomorrow.’, ‘I can meet with Stefania tomorrow’
\z\largerpage[3]

To sum up, in this squib we have provided some preliminary evidence that finite
control languages like Greek and \ili{Romanian} display \gls{PC}\is{control!partial
control} in very limited contexts, contrary to what has previously been claimed
in the literature. Moreover, the very existence of this phenomenon inside
[+Agr], [+T] complements of non-attitude predicates is incompatible with any
mainstream theory of \gls{PC}\is{control!partial control} that predicts it to
be incompatible with [+Agr]. Data problematic for this claim can also be found
in European Portuguese, which appears to permit obligatory \isi{control} into
inflected infinitives, at least for some speakers
\parencite{Sheehan2018a,Sheehan2018b},
though this is somewhat controversial (see \citealt{Barbosa2017}). We have
dismissed, somewhat tentatively, the idea that apparent cases of
\gls{PC}\is{control!partial control} in Greek and \ili{Romanian} might be instances
of coercion of a C-subjunctive into an F-subjunctive or of
\gls{ExC}\is{control!exhaustive control} with a covert comitative. The next
step for this investigation is to survey the extent of this phenomenon in Greek
and \ili{Romanian} and establish whether it can be unambiguously found with
predicates other than ‘can’. If this can be established, then an alternative
theory of \isi{control} must be explored which captures the fact that
\gls{PC}\is{control!partial control} is in fact compatible with [+Agr] clauses,
without overgenerating. It is worth noting in this regard that the
incompatibility is somewhat stipulative in \citegen{Landau2015} approach, so
this may not be as difficult as first appears.{\interfootnotelinepenalty10000\footnote{There are, for example,
    other verbs which \citeauthor{Varlokosta1994} claims take C-subjunctives
    which appear to allow \gls{PC}:

\begin{exe}
    \exi{(i)} \ili{Greek}\\
    \gll Tha charo     na   vrethume \\
    \Fut{} please.\Fsg{}   \Sbjv{} meet.\Fpl{}\\
    \glt ‘I will be pleased us to meet tomorrow\dots{}’
\end{exe}}}\pagebreak

\printchapterglossary{}

{\sloppy
\printbibliography[heading=subbibliography,notkeyword=this]
}
\end{document}
