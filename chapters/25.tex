\documentclass[output=paper]{langsci/langscibook}
\author{Jonathan David Bobaljik\affiliation{Harvard University}}
\title{Absolutive control and absolute universals}

% \chapterDOI{} %will be filled in at production

\abstract{It is widely held, across frameworks, that complement \isi{control}
    universally targets the subject function, cross-cutting major \isi{alignment}
    divisions. Whether case follows an accusative or ergative or other
    \isi{alignment}, it is consistently the subject of a non-finite complement clause
    that is normally unexpressed and understood as coreferent with a matrix
    argument. This squib examines a recent challenge to that characterization
    from \ili{Belhare} [byw], a \ili{Kiranti} (Sino-Tibetan) language, which is alleged to
    have a pattern of \isi{control} targeting the absolutive argument of the
    complement clause, regardless of its grammatical function. I argue that the
    challenge from \ili{Belhare} is mis-characterized, and that even on the primary
description of the relevant \ili{Belhare} data, the facts are consistent with the
universal characterization of \isi{control} as syntactically targeting subjects.}


\begin{document}\glsresetall
\maketitle

\section{Introduction}

Across a variety of theoretical traditions, something along the following lines
has been held to constitute a syntactic universal:

\ea \label{scu} When case and grammatical function diverge, it is the function
\textsc{subject} and not a case category (nominative, absolutive, ergative, etc.)
that determines which argument in a non-finite clause is subject to \isi{control}.
\z

In this squib, I will examine an alleged counter-example, from \ili{Belhare}
(Sino-Tibetan), which has been taken \citep{bicknich01,malchukov14} to show an
instance of \isi{control} on an ergative--absolutive \isi{alignment}, and thus that
(\ref{scu}) represents only a strong trend, and not a true universal. I argue
that the conclusion is hasty, and that even on Bickel's own analysis, the data
do not in fact challenge (\ref{scu}).\footnote{I restrict the discussion here
    to \ili{Belhare}. \citet{bicknich01} argue that Chechen shows a similar pattern
    to \ili{Belhare}, but is subsumed under the same analysis, without the
    complicating factor of morphological \isi{deponence}. In her survey of complement
    \isi{control} cross-linguistically, \citet{stiebels07} presents Austronesian and
    Mayan as showing a different type of challenge to (\ref{scu}).  In these
    languages, \isi{control} may single out either the grammatical subject or the
    logical subject (agent/actor/external argument) on her analysis,  sometimes
    with variation across constructions in a single language (see also
    \citealp{Kroeger:1993,wurmbrand13tag} on \ili{Tagalog}).  Whether these challenge
    (\ref{scu}) as phrased depends in large part on how \enquote{subject} is
    defined, a matter of no small controversy in particular in Austronesian. I
    am unable to address these examples within the confines of a squib, but my
    narrow goal here is to defend the claim that case is never the determining
    factor as to which argument will be PRO, and the Austronesian and Mayan
    examples are thus orthogonal to that narrow point. For additional
discussion of \isi{control} in ergative languages, and some important additional
qualifications, see \citet[104--109]{polinsky16}.}

\section{Control is not case}

To begin, it may help to have a brief review of the standard evidence for
(\ref{scu}).

In the canonical \isi{control} configuration, a designated argument in a non-finite
complement is obligatorily unexpressed, and obligatorily coreferent with an
argument in a higher clause. A long established tradition represents the
controlled element as PRO. In \ili{English}, PRO is always the subject, never the
object of the non-finite clause. There are of course also ways of representing
this dependency without a null element in the syntax, but for current purposes,
the important observation is that it is the subject of the non-finite clause
that is shared/coreferent with an argument (subject in (\ref{leo}) or object,
as with \emph{ask}, \emph{tell}, etc.) of the matrix clause.

\ea \label{leo}	\begin{xlista}
	\ex[]{Leo tried [ PRO$_{\Nom{}}$ / *Mika / *himself  to win ].}
	\ex[]{Leo tried [ {PRO$_{\Nom{}}$} to choose his teacher$_{\Acc{}}$ ].}
	\ex[*]{Leo tried [ his teacher$_{\Nom{}}$ to choose {PRO$_{\Acc{}}$} ].}
	\end{xlista}
\z

In a canonical nominative--accusative \isi{alignment}, where there is a direct
correspondence between nominative\is{nominative case} case and the grammatical function subject, it
is not possible to decide on simple empirical grounds whether the distribution
of PRO should be stated in terms of case or subjecthood. Famously,
\citet{Vergnaud1977,Chomsky1980} proposed that the distribution of \isi{control} is
reducible to the distribution of case: \ili{English} nominative\is{nominative case} case is restricted to
finite subjects, thus lexical subjects are excluded from the subject position
of non-finite clauses (unless they are \enquote{exceptionally} assigned case by
a higher verb or preposition).  If this view were correct, one would expect to
find that variation in case patterns~-- which is amply attested -- would
correlate with variation in the distribution of PRO. That expectation, it turns
out, is resoundingly false.

For example, in an ergative--absolutive \isi{alignment}, the transitive subject
bears \isi{ergative case} while the object and intransitive subject share the
typically unmarked \isi{absolutive case}. Case and subjecthood do not align:
there is no case that is assigned to all and only subjects. If
\citeauthor{Vergnaud1977} were right, and the account of \isi{control} was that
some designated case is unavailable in non-finite contexts, then the
\isi{control} pattern should track case, rather than subjecthood.  For example,
an absolutive pattern\is{absolutive alignment} of \isi{control} would look like
the following, with PRO as the object, but not the subject, of a transitive
infinitive:

\ea \label{abscntr} 	\nobreak\begin{xlista}
	\ex[]{Leo tried [ PRO\tss{\Abs}  to win ].}
	\ex[*]{Leo tried [ PRO\tss{\Erg} to choose his teacher\tss{\Abs} ].}
	\ex[]{Leo tried [ his teacher\tss{\Erg} to choose PRO\tss{\Abs} ].}
	\end{xlista}
\z

Such a pattern has been prominently claimed not to exist. For example, in an
important survey of ergativity, \citet[134--135]{Dixon1994} notes:

\begin{quote} Whenever [concepts such as `can', `try', `begin', `want' ``and
    similar verbs''] are realised as lexical verbs, taking an object complement
    clause construction which involves another verb, the two verbs must have
    the same subject (S or A) irrespective of whether the language is
    accusative or ergative at morphological and/or syntactic levels...

    This is a universal, relating to the universal category of subject.
\end{quote}

There are, of course, many questions one can ask about which constructions
should and shouldn't fall under the scope of such a universal (for example,
where to draw the line between \isi{control} and \isi{raising}, and whether adjunct \isi{control}
and complement \isi{control} should be grouped together, and whether this should
include \isi{control} by the matrix object). For the narrow goals of this squib, we
may abstract away from some of these important issues.

The \ili{Tsez} (Nakh-Dagestanian) examples in (\ref{tsez}) from
\citet[319]{polinsky16} illustrate Dixon's observation nicely. \ili{Tsez} has an
ergative \isi{alignment}\is{ergative alignment} in case and agreement, but the distribution of PRO{} cannot
be characterized uniquely in terms of case. It is neither all-and-only
ergative\is{ergative case}
NPs nor all-and-only absolutive NPs that correspond to PRO. Rather, PRO
corresponds to the NP that stands in the subject function of the infinitive
complement to \emph{-et-} `want/need', whether that NP would be
ergative\is{ergative case}
(\ref{tsez}a) or absolutive (\ref{tsez}b).\footnote{I reproduce Polinsky's
glosses here; see \citet[319]{polinsky16} on the different readings of
\emph{-et-} as `want' versus `need'.}

\ea \label{tsez} \ili{Tsez}
	\begin{xlista}
	\ex \gll \ldots pro$\tss{\emph{i}}$ [ PRO$\tss{\emph{i}}$ gulu b-exad-a ] r-eti-n.\\
	\ldots \Fsg.\Lat{} {} PRO.\Erg{} horse.\Abs.\Iii{} \Iii{}-slaughter-.\Inf{} {}.\Iv{} \Iv{}-want-\Nw{}\\
	\glt `I need to slaughter the horse.'
	\ex \gll D\"a-r$\tss{\emph{i}}$ [ PRO$\tss{\emph{i}}$ \v{z}ek'u-de kec-a ] r-eti-n.\\
	\Fsg.\Lat{} {} PRO.\Abs{} man.\Apud{} sleep-\Inf{} {}.\Iv{} \Iv{}-want-\Nw{} \\
	\glt `I needed to sleep with a man.'
	\end{xlista}
\z

And even in nominative--accusative languages, it is known that case and
grammatical function can sometimes diverge, as famously documented for
\enquote{quirky} (i.e., non-nominative) subjects in \ili{Icelandic}
\citep{andrews76,ZaenenEtAl1985,sigurdsson91}. When the subject would be dative\is{dative case}
and the object nominative\is{nominative case}, it is the subject, not the nominative\is{nominative case} NP, that is
obligatorily suppressed and coreferent with a higher NP, i.e.,
PRO:\footnote{That PRO here is indeed dative\is{dative case} is well documented;
    \citet{sigurdsson91} showed for example that elements which agree with the
unexpressed subject, such as floating quantifiers, are obligatorily dative\is{dative case}
exactly when PRO replaces a subject that would be dative\is{dative case} if it were overt, and
analogously for all other cases.}

\ea \ili{Icelandic} \citep[116]{jonsson96}\\
    \gll J\'on vonast til [ a{\dh} PRO lika \th{essa} b\'ok ].\\
	Jon.\Nom{} hopes for {} to PRO.\Dat{} like this book.\Nom{} \\
	\glt `Jon hopes to like this book.'
\z

The evidence from \ili{Icelandic} and ergative languages\is{ergative alignment}
provides a compelling reason to believe that it is quite generally subjecthood,
not case, that determines the distribution of \isi{control} effects regardless
of language type.\footnote{\citet{Legate2008} defends a version of Case theory
    with its roots in the Vergnaud--Chomsky approach. Legate concedes that Case
    is not responsible for the distribution of PRO, but argues that there is
    nevertheless a connection between Case and finiteness that includes
    ergative languages. Space precludes a full engagement with Legate's
    proposals, but it is relevant to observe that the majority of her arguments
    show that non-finite clauses in ergative languages distinguish absolutive
    subjects from absolutive objects. From this, she concludes that absolutive
    subjects are actually nominative\is{nominative case} (and objects aren't),
    maintaining a role for Case. However, in all of the languages she considers
    (with an additional qualification for some, but not all, speakers of
    \ili{Warlpiri}) the absolutive subjects pattern together with ergative
    subjects wherever testable, reflecting, as Dixon maintained, that it is the
    (possibly derived) notion of subject that is relevant for the effects
considered, rather than a case category.}

\section{Object unification and restructuring}

In the context of the quoted passage above, Dixon notes that there are two
patterns shown across languages by this class of predicates. The canonical
complement \isi{control} pattern, in which the subjects are shared, is one such
pattern. There is a second pattern, which could be described as unification or
sharing of the entire argument structure of both predicates, i.e., as clause
union or restructuring. In such contexts, in addition to a shared subject, if
the lower predicate is transitive, the embedded object may behave in various
ways as if it is the object of the matrix predicate (see
\citealp{Wurmbrand2001}). As we will see below, this patterning of the embedded
object in a restructuring configuration will turn out to be the key to
understanding the alleged \ili{Belhare} counter-example to (\ref{scu}).

A famous example of a clause union effect, cited by \citet{bicknich01}, is
clitic-climbing in \ili{Romance}. In (\ref{clcl}), the object clitic
corresponding to the object of the subordinate clause attaches to the matrix
verb \emph{quiero} `I want', in this sense behaving as if it were the matrix
object.

\ea \label{clcl} \ili{Spanish}\\
    \gll Lo=quiero [ ver a Juan ].\\
	\Tsg{}.\M{}.\Acc{}=want.\Fsg{} {} see.\Inf{} \Acc{} Juan {} \\
	\glt `I wanted to see Juan.'
\z

In addition to \isi{clitics}, long-distance
agreement\is{agreement!long-distance agreement} in restructuring clauses is
attested for languages that display object agreement.\is{agreement!object
agreement} In \ili{Itelmen}, the modal \emph{utu-} `be.unable' may (optionally)
inherit the argument structure of its complement, inflecting intransitively if
the complement is intransitive (\ref{itl}a) or transitively, if the complement
is transitive (\ref{itl}b):

\ea \label{itl} \ili{Itelmen}
	\begin{xlista}
	\ex \gll kəmma t'-utu-s-ki\v{c}en [ ŋekse-kaz ].\\
    I \Fsg{}-be.unable-\Prs-\Fsg.\Sbj{} {} sleep-\Nfin{} {}\\
	\glt `I can't sleep.' (Field notes: SA6-A)
	\ex \gll kəmma t'-utu-z-in [ əl\v{c}qu-aɬ-iɬ ].\\
    I \Fsg{}-be.unable-\Prs-\Ssg.\Obj{} {} see-\Fut-\Nfin{} {}\\
	\glt `I can't see you.' (Field notes: S3:19)
	\end{xlista}
\z

Note that this restructuring construction is a special case of
control;\footnote{Or \isi{raising}, if modals are always \isi{raising} configurations, see
\citet{wurmbrand99}.} the subjects are shared in both the transitive and
intransitive contexts. A quirk of Chu\-kotko-Kamchatkan languages is the
curiously absolutive-like agreement suffix position on the matrix predicate: in
(\ref{itl}b), the matrix verb inflects transitively, and the suffix agrees with
the object (of the embedded clause), while in (\ref{itl}a), the suffix (as well
as the prefix) agrees with the local subject. This is not particular to
restructuring -- the double agreement in intransitives is a regular feature of
Chukotko-Kamchatkan verbs \citep{bobaljik98}. As a result of this morphological
quirk, the matrix suffix comes to agree with the object of a transitive
complement, but the subject of an intransitive complement, an apparently
absolutive pattern in a language that otherwise lacks an \isi{ergative alignment}.
But the absolutive pattern\is{absolutive alignment} is epiphenomenal: the analysis of (\ref{itl}b)
proposed in \citet{BobWur2005} is given in (\ref{tree}). Subject sharing is
represented, as is standard, as \isi{control} (PRO) or \isi{raising} (\emph{t}), but this
is not crucial to the argument and a representation without a null
subject\is{null subjects}
argument in the embedded infinitive would work just as well. What is important,
following \citet{Wurmbrand2001,wurmbrand15} and many others, is the proposal that
what sets restructuring complements apart from non-restructuring complements is
that the infinitival clause (α) is transparent to clause- (or phase-)
bounded phenomena, such as clitic movement and agreement (and others).

\ea \label{tree}
    \begin{tikzpicture}[baseline]
    \Tree
            [
                \node(s){\Sbj{}\tss{\emph{i}}};
                [
                    \node(m1){\Agr-\textsc{modal}-\Agr};
                    [.{α}
                        ({PRO/\emph{t}})\tss{\emph{i}}
                        [.{}
                            \node(o){\Obj{}};
                            V
                        ]
                    ]
                ]
            ]

        \draw [arrow, ->, bend right=90] (s.south) to (m1.200);
        \draw [arrow, ->, bend left=60] (o.south) to (m1.340);

    \end{tikzpicture}
\z

The intransitive complement is represented as in (\ref{treein}), with the
characteristic double agreement with the intransitive subject:

\ea \label{treein}
    \begin{tikzpicture}[baseline]
    \Tree
            [
                \node(s){\Sbj{}\tss{\emph{i}}};
                [
                    \node(m1){\Agr-\textsc{modal}-\Agr};
                    [.{α}
                        ({PRO/\emph{t}})\tss{\emph{i}}
                        V
                    ]
                ]
            ]

        \draw [arrow, ->, bend right=90] (s.south) to (m1.200);
        \draw [arrow, ->, bend right=70] (s.south) to (m1.340);

    \end{tikzpicture}
\z

Again, despite the morphological pattern on the matrix subject, there is no
absolutive pattern of \isi{control} here. Both arguments of the embedded clause are
in effect shared -- the subject is controlled and the object becomes a matrix
object by clause union/restructuring. The apparent \isi{absolutive alignment} is an
artifact of how agreement works generally in Chukotko-Kamchatkan.

\ili{Belhare}, which we will turn to in the next section, also shows predicates like
\ili{Itelmen} \emph{utu-}, in which transitivity of the matrix predicate is
determined by the transitivity of its non-finite complement. The complement is
uninflected (non-finite), and the matrix predicate agrees with the subject and
with the embedded object as if it were its own:

\ea \label{belhi} \ili{Belhare}\begin{xlista}
    \ex \gll [ hit mett-a ] \{ ka-hiu-ka / *hiu-ka \}   i?\\
    {} look \Caus-\Sbjv{} {} {} \Fsg{}-be.able-\Second.\Su{} {} \hphantom{*}be.able-\Second.\Su{} {} \glossQ{} \\
	\glt `Can you show me the way?'
	\ex \gll  unna han lu-ma n-lapt-he-ga i?\\
	\Tsg.\Erg{} \Ssg.\Abs{} tell-\Inf{} \Third.\Aa{}-be.about.to-\Pst.\Second.\Su{} \glossQ{} \\
	\glt `Was s/he about to tell you?'
	\end{xlista}
\z

\citet{bicknich01} refer to this as \enquote{agreement climbing} to highlight
the parallel to clitic-climbing, citing examples from other languages as well.
Their analysis is not expressed in phrase structure terms, but is directly
comparable to (\ref{tree}). They treat the matrix predicate as a light verb
whose argument structure is labile, and which thus inherits its arguments via
unification with its non-finite complement, and in addition, they argue that
the embedded object remains in the embedded clause, as in (\ref{tree}). Within
the notation of \citet{bicknich01}, (\ref{tree}) corresponds to the following
(their (13a)). Working up from the bottom: \emph{hir-} `be.able.to' in
(\ref{belhi}) has a labile argument structure. In this example it is bivalent
\tuple{A,O}, which unifies with the bivalent argument structure of its
complement \tuple{a,o} (capital versus small letters are simply for keeping
track of matrix versus embedded frames). In the syntax, \emph{hir-} is
transitive, with A (subject) and O (object) corresponding to the shared
arguments with the embedded predicates. The morphology (agreement) is faithful
to the syntax, and both arguments of the embedded predicate are expressed on
the matrix predicate.

\ea \label{bn13} \begin{tabular}[t]{l l c c c c c}
	Morphology:	& 		&     A      & O \\
		    &				&      |      &  | \\
	Syntax: & 				&     A      & O \\
		    &				&     |       &  | \\
    Arg Str: & \tuple{a,o}+\tuple{A,O}	& $\langle$ A=a & , O=o $\rangle$ \\
\end{tabular}
\z

The key correspondences among the frameworks are that subject
\enquote{sharing} is implemented as \isi{control} or \isi{raising} and that
unification is represented phrase structurally as a transparent domain (α).
Object \enquote{sharing} is not represented directly in (\ref{tree}) although
it could have been. In (\ref{tree}), I have represented the object as remaining
in the embedded clause, and syntactically related to the matrix verb via
agreement, but the transparency of the node α effectively encodes the effect
that the embedded object stands in the object-of relation to both verbs
simultaneously.

It is not central to the argument here that the object remain in the embedded
clause, or that the subject raise -- the object could raise (as in
\citealp{BobWur2005}) or both subject and object could in principle remain in the
embedded clause with matrix agreement targeting both, as in backwards
raising/control. \citet[159-160]{bickel04} presents the following examples to
argue that the object of a light verb remains in the embedded clause (a), while
expressing it in the matrix clause (b) results in ``questionable
grammaticality''. This contrasts with the shared subject in a related light
verb construction, which may occur in the matrix clause (c). The data provided
are sparse and open to other interpretations.

\ea \label{belhmv}\ili{Belhare}\begin{xlista}
	\ex[]{\gll [ ŋka lu-ma ] nui-ʔ-ŋa.\\
	{} \Fsg{} tell-\Nfin{} {} may-\Npst-\Excl{} \\
	\glt `I may be told.'}
	\ex[?]{\gll [ \emph{t$\tss{\emph{i}}$} lu-ma ]  ŋka$\tss{\emph{i}}$ nui-ʔ-ŋa.\\
	{} \emph{t$\tss{\emph{i}}$} tell-\Nfin{} {} \Fsg{}  may-\Npst-\Excl{} \\
	\glt `I may be told.'}
	\ex[]{\gll [ lu-ma ]  ŋka khei-ʔ-ŋa.\\
	{} tell-\Nfin{} {} \Fsg{}  must-\Npst-\Excl{} \\
	\glt `I must tell him/her.'}
	\end{xlista}
\z

To this point, everything presented has been consistent with (\ref{scu}). The
important interim conclusion is this: \isi{control} (or possibly \isi{control} and raising)
always involves subject sharing, with a subset of \isi{control} constructions also
involving a sharing-like dependency between the matrix predicate and the
embedded object. The subject is always shared, and if transparency obtains,
then the embedded object may also behave as local to the matrix
clause.\footnote{It may be possible to have transparency of the infinitive
without \isi{control}, a point I leave for future discussion.}

\section{Belhare -- absolutive control?}

\citet{bickel04} identifies a range of light verbs in \ili{Belhare}, with meanings
corresponding to: `may', `must', `begin', `stop', `finish', `can', `forget',
`know', `be about to', `already', `be able to', `want', `think'. These fall
squarely in the cross-linguistically expected class of \isi{raising} and
restructuring predicates. A number of these verbs behave as illustrated above
in (\ref{belhi}) -- that is, they are unexceptional restructuring or clause
union (or transitive raising) predicates: in one way or another both arguments
of the embedded predicate are treated as arguments of the matrix predicate.
Bickel notes in addition that two of the \ili{Belhare} modal light verbs in the list
above have a slightly different morphological pattern, illustrated here with
\emph{nus-} `may':

\ea \label{belh}
	\begin{xlista}
	\ex \gll Khon-ma nui-ka \\
	play-\Inf{} may.\Npst-\Second.\Su{} \\
	\glt `You may play.'
	\ex \gll Lu-ma nui-ka \\
	tell-\Inf{} may.\Npst-\Second.\Su{} \\
	\glt `I/she/he/they may tell you.' or `You may be told.'\\Not: `You may tell someone/them.'
	\end{xlista}
\z

It is this pattern that is held to show an absolutive pattern\is{absolutive alignment} of \isi{control},
contravening (\ref{scu}). I understand the relevant observation to be this: the
matrix predicate \emph{nui-ka} `may' agrees with only the \Ssg{} argument,
which corresponds to the absolutive NP in the infinitive -- the subject in
(\ref{belh}a) and the object in (\ref{belh}b). The ergative argument is not
expressed via agreement on the modal, even when the paradigm has (non-zero)
affixes to do so.

Note that the object (and subject) may be overt in the embedded clause
(\ref{belhbc}), but apparently resists expression in the matrix clause, as we
have seen above. Thus considering this in terms of \isi{control} requires relaxing
one of the canonical criteria (that the argument be obligatorily unexpressed)
and that this be considered a case of \enquote{backwards control}. I return to
this observation in the final section, but set it aside for now.

\ea \label{belhbc}
       \gll han lu-ma nui-ka \\
       \Second{} tell-\Inf{} may.\Npst-\Second.\Su{} \\
	\glt `[They] may tell you.' \citep[156]{bickel04}
\z

So the question is: is this an absolutive pattern\is{absolutive alignment} of \isi{control}, in the sense that
is relevant for (\ref{scu})? \citet{bicknich01} contend that it is, with
specific reference to Dixon's quoted passage above. Following them,
\citet{malchukov14} refers to this pair to argue that \isi{control} may, if rarely,
follow an \isi{ergative alignment}.

Yet Bickel's and Nichols's analysis of the facts gives room for pause.
Syntactically, their analysis is in relevant respects analogous to the analysis
of \ili{Itelmen} in (\ref{tree}) in which the apparent ergative--absolutive
pattern is a quirk of agreement morphology and not a matter of the syntax of
\isi{control}.  \citet{bickel04} and \citet{bicknich01} argue that \emph{nus-}
in (\ref{belh}) shows in fact the same argument unification pattern as the
other light verbs considered above in (\ref{belhi}). What sets \emph{nus-} and
(on one reading) \emph{khes-} `must' aside from the other light verbs is a
morphological quirk -- although they undergo argument unification, they are
morphologically deponent, a notion familiar from \ili{Latin} and \ili{Greek}
\citep{baermanetal07}: syntactically transitive, but morphologically
intransitive. More specifically, their agreement follows an \isi{absolutive
alignment}. Their analysis of the representation of \emph{nus-} with a
transitive complement, (\ref{belh}b) is given here:

\ea \label{bn2} \begin{tabular}[t]{l l c c c c c}
	Morphology:	& 		&           & O \\
		    &				&            &  | \\
	Syntax: & 				&     A      & O \\
		    &				&     |       &  | \\
    Arg Str: & \tuple{a,o}+\tuple{A,O}	& $\langle$ A=a & , O=o $\rangle$ \\
\end{tabular}
\z

This represents the following claims: \emph{nus-} has a bivalent argument
structure \tuple{A,O}, which unifies with the bivalent argument structure of
its complement \tuple{a,o}. On their analysis, in the (line labeled) syntax,
\emph{nus-} is bivalent, i.e., transitive, with A (subject) and O (object)
corresponding to the shared arguments with the embedded predicates. But in the
morphology, where verbs like \emph{hir-} in (\ref{belhi}) would express both
arguments via agreement, \emph{nus-} is deponent, and only cross-references one
argument, namely the absolutive (the normal pattern for a verb in an ergative
alignment that cross-references a single argument, see
\citealt{Bobaljik2008}).\footnote{\citet{bicknich01} note in support of this
    analysis that \ili{Belhare} has other deponent verbs, including experiencer
    predicates that take two syntactic actants but inflect intransitively and
    other light verbs which show the reverse morphology--syntax mismatch,
    inflecting transitively whether they have one or two arguments. They note
    also that case patterns support a deponent analysis of \emph{nus-} and
    \emph{khes-}, which would otherwise be the only instances of a transitive
    case array (\Erg--\Nom{}) with an intransitive predicate.}

The key observation here is that in (\ref{bn2}), there is no ergative (or
absolutive) \isi{alignment} in the syntax, i.e., the portion of the notation that
represents \isi{control}. Leaving out the Morphology line, (\ref{bn2}) is
indistinguishable from (\ref{bn13}). Using the same correspondences across
frameworks, the phrase structure implementation of Bickel and Nichol's analysis
is (\ref{25.tree2}a), identical to (\ref{tree}) except that it lacks agreement
with the transitive subject. The corresponding intransitive is given in
(\ref{25.tree2}b) (cf. \ref{treein}).

\ea\label{25.tree2}
    \ea
    \begin{tikzpicture}[baseline]
    \Tree
            [
                \node(s){\Sbj{}\tss{\emph{i}}};
                [
                    \node(m1){\textsc{modal}-\Agr};
                    [.{α}
                        ({PRO/\emph{t}})\tss{\emph{i}}
                        [.{}
                            \node(o){\Obj{}};
                            V
                        ]
                    ]
                ]
            ]

        \draw [arrow, ->, bend left=60] (o.south) to (m1.330);

    \end{tikzpicture}
    \ex
    \begin{tikzpicture}[baseline]
    \Tree
            [
                \node(s){\Sbj{}\tss{\emph{i}}};
                [
                    \node(m1){\textsc{modal}-\Agr};
                    [.{α}
                        ({PRO/\emph{t}})\tss{\emph{i}}
                        V
                    ]
                ]
            ]

        \draw [arrow, ->, bend right=70] (s.south) to (m1.330);

    \end{tikzpicture}
    \z
\z

Expressed in terms of phrase structure, as in (\ref{25.tree2}), nothing in this
pattern falls afoul of (\ref{scu}), as can readily be seen by examining the
structures. In all relevant examples, the unexpressed, referentially dependent
element in the non-finite clause is the subject. Restructuring/clause union/α
transparency makes available an additional morphosyntactic dependency between
the matrix predicate and the embedded object. Agreement is free to follow an
ergative--absolutive pattern,\is{ergative alignment}\is{absolutive alignment} even in clause-union configurations and it is
independently known to do so. But (\ref{scu}) is not intended to be read so as
to constrain agreement relations, and so no issue arises if the
ergative--absolutive agreement sits atop a clause-union configuration which
itself shows sharing (i.e., \isi{control}) of the subject. As far as I can see, this
is indeed the state of affairs that \citet{bicknich01} argue for, namely, that
the apparent ergative (absolutive) \isi{alignment} in the \ili{Belhare} \isi{control}
configuration is a property of the morphosyntax of agreement, not of the
syntactic representation of \isi{control}.\footnote{Andrej Malchukov asks whether the
    ``passive'' paraphrases such as `You may be told' in (\ref{belh}b)
    indicates a kind of \isi{passive} syntax in which the embedded object is
    syntactically represented as the matrix subject. Note that this is not a
    property of the analysis in \citet{bicknich01} or \citet{bickel04}, nor in
    my translation of their analysis into phrase structural terms. In all these
    analyses, the embedded object remains in the embedded clause. Bickel
    glosses examples of this sort variously as `I/she/he/they may tell you.' or
    `You may be told.' or `Someone may tell you.' Of these, the \isi{passive} version
    seems to most closely convey the meaning of an impersonal matrix subject,
    but at the cost of an unfaithful rendering of the \ili{Belhare} (morpho-)syntax.
    The \isi{passive} construction in \ili{English} allows \emph{you} to be the subject of
    \emph{may}, associating it with the deontic force of the modal. However,
    this syntax is not required. We know from \ili{English} and other languages that
    the deontic force of a modal need not be directed to the matrix subject, as
    illustrated by examples such as \emph{The cookies may/must be eaten (by
    Paul)} \citep{warner93,wurmbrand99}. In the absence of counter-evidence, I
    take it that the choice of paraphrase here represents an attempt to render
    the meaning in \ili{English} as closely as possible, given that a literal
    paraphrase of this construction (with or without an overt matrix subject)
would be ungrammatical in \ili{English}, but that no particular syntactic analysis
should be read into the paraphrase.} The \ili{Belhare} facts (and for that matter the
\ili{Itelmen} suffixes) challenge (\ref{scu}) only if the various aspects of the
morphosyntactic representation are not factored out in this way, and
(\ref{scu}) is held to range over all aspects of the representation, including
agreement.

\section{Postscript: Belhare control}

Above, I have argued that the alleged \isi{absolutive alignment} in \ili{Belhare} \isi{control}
is an artifact of agreement morphology and not a property of the syntactic
representation of \isi{control}. Since Bickel observes that the \enquote{controlled}
NPs are not obligatorily unexpressed, and may remain in the embedded clause in
the light verb constructions, the discussion was predicated on allowing
\enquote{control} to include \enquote{backwards} \isi{control}, a configuration
involving apparent argument-sharing, but in which the shared argument is in the
embedded, not the matrix clause. As it happens, \ili{Belhare} does have non-finite
contexts which show a more canonical \isi{control} configuration: a designated
argument is obligatorily unexpressed and is coreferent with a matrix argument,
i.e., PRO. These are the non-finite verb forms apparently in adverbial or
purpose-clause function, marked by the suffixes \emph{-sa} or \emph{-si}. Under
Bickel's description, these show  exactly the \ili{Tsez}-like pattern expected of
control in a language with an ergative case system: neither an S (absolutive)
nor an A (ergative) argument may be overt, while any other argument, including
the absolutive object, may be overt. The PRO gap is necessarily understood as
coreferent with an argument of the matrix clause.

\ea \begin{xlista}
	\ex \gll [ *(un) khatd-e yuŋ-sa ] mai-lur-he.\\
    {} \hphantom{*(}\Third{} bed-\Loc{} sit-\Ss{} {} \Fsg.\Uu{}-tell-\Pst{} \\
		\glt `He told it to me while (*he) sitting on the bed.'
	\ex \gll [ *(un-chik-ŋa) dhol te\~i-sa ] la ŋŋ-us-e.\\
    {} \hphantom{*(}\Third{}-\Nsg{}-\Erg{} drum beat-\Ss{} {} dance \Third.\Nsg{}.\Su{}-dance-\Pst{} \\
		\glt `They danced (*they) beating the drum.' \citep[147]{bickel04}
	\end{xlista}
\z

Thus, while \ili{Belhare} has a rich array of light verb constructions, some of
which have intriguing agreement patterns, it also has far more canonical
syntactic \isi{control} constructions, with an obligatory gap (PRO), and these
adhere, without any complications, to the universal pattern that it is the
function \textsc{subject} and not a case category (nominative, absolutive,
ergative, etc.) that determines which argument in a non-finite clause is
subject to \isi{control}.

\printchapterglossary{}

\section*{Acknowledgements}

I'm pleased to be able to offer this brief contribution in Ian's honour. The
squib is an extended reply to a question from Andrej Malchukov at the ReCoS
workshop in Arezzo. For lively discussion, in addition to Ian, Andrej, Susi
Wurmbrand, and the other ReCoS participants, I thank audiences in mini-courses
in Frankfurt and Leipzig (especially Barbara Stiebels), and for comments on an
earlier draft, I am grateful to Susi Wurmbrand, Andrej Malchukov, and two
reviewers for this volume. This paper was written while I was visiting the
Hebrew University of Jerusalem and the Leibniz-Zentrum Allgemeine
Sprachwissenschaft (Berlin), whose support, along with the financial support of
the Guggenheim Foundation, I gratefully acknowledge.

{\sloppy
\printbibliography[heading=subbibliography,notkeyword=this]
}

\end{document}
