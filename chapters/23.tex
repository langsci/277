\documentclass[output=paper]{langsci/langscibook}
\ChapterDOI{10.5281/zenodo.4680322}
\author{Susi Wurmbrand\affiliation{University of Vienna}}
\title{Rethinking implicit control}

\abstract{This paper discusses Visser’s generalization effects in light of the
    question whether control involves a direct relation between the embedded
    PRO subject and a matrix controller, or an indirect relation mediated by a
    functional head\is{functional items} in the matrix clause. Based on certain
    case restrictions and effects of additional \emph{by}-phrases, it is
suggested that both types of licensing may be necessary.}


\begin{document}\glsresetall
\maketitle

\noindent Approaches to control which assume an embedded PRO subject differ
regarding the relation PRO has with the argument supplying the interpretation.
The traditional view is that PRO is licensed \emph{directly} by a matrix DP via
some form of \isi{binding}. More recent approaches postulate a mediated form of
binding: PRO is only \emph{indirectly} connected to the actual controller in
that it is identified/bound by a functional head\is{functional items} of the matrix clause (e.g., T
or \emph{v}) which itself is licensed by the controller. In this squib, I
suggest based on data involving implicit control\is{control!implicit control} that both forms of
identification of PRO exist.

In \textcite{vanUrk2013} evidence for a mediated approach to control is
provided via a novel observation regarding \emph{Visser’s generalization}
effects in languages that have been assumed to not show such effects. As shown
in \eqref{ex:23.1}, \ili{Dutch} and \ili{German} allow implicit matrix
agents of verbs like \emph{promise} to control PRO. The interpretation of these
sentences is such that the person promising is also the person initiating the
embedded event.\newpage

\ea\label{ex:23.1}
    \textsc{implicit.agent}  DP.\Dat{}  V  [\textsubscript{\Inf{}}  PRO\textsubscript{ …}  ]\\
    \ea\label{ex:23.1a} \ili{Dutch} \parencite[171, (8)]{vanUrk2013}\\
		\gll Er  werd  mij  beloofd / aangeboden  om  me  op de hoogte  te houden.\\
                there  was  I.\Dat{}  promised {} offered  \Comp{}  me  on the height  to keep.\textsc{\Inf{}}\\
		\glt ‘It was promised/offered to me to keep me informed.’
    \ex\label{ex:23.1b} \ili{German} \parencite[171, (9a)]{vanUrk2013}\\
		\gll Mir  wurde  versprochen,  mir  noch heute  den Link für das Update  zu schicken.\\
			I.\Dat{}  was  promised  I.\Dat{}  still today  the link for the update  to send\\
		\glt ‘It was promised to me to send me the link for the update today.’
	\z
\z
%
Such implicit control\is{control!implicit control} in ditransitive matrix contexts is restricted, however,
to predicates like \emph{promise} in \eqref{ex:23.1} that combine with a dative\is{dative case} argument
(in addition to the infinitive). Implicit control is impossible when the matrix
predicate combines with a structurally case marked object realized as
accusative in the active and nominative\is{nominative case} in the passive\is{passive}.
This is shown in \eqref{ex:23.2} for \ili{Dutch} and \eqref{ex:23.3} for German. The (a) examples
illustrate that in active statements, subject control is possible in
appropriate contexts with these predicates. The same interpretations are lost,
i.e., implicit control\is{control!implicit control} is impossible, when the matrix predicate is passivized
as in the (b) examples.\largerpage[2]

\ea\label{ex:23.2} *\textsc{implicit.agent}  DP.\Acc{} $\to$ \Nom{}  V  [\textsubscript{\Inf{}}   PRO \textsubscript{ …}  ]\\
	\ea\label{ex:23.2a} \ili{Dutch} (P. Fenger, p.c.)\\
    \gll De kinderen  hebben  de leraren  overtuigd  om  ze  te mogen  kietelen.\\
		the children  have  the teachers\textsubscript{L}  convinced  \Comp{}  them\textsubscript{L}  to may  tickle\\
	\glt ‘The children convinced the teachers to be allowed to tickle them.’ (PRO=children \ding{51})\\
		‘The children convinced the teachers that they (the children) would
    be allowed to tickle them (the teachers).’
    \ex\label{ex:23.2b} \ili{Dutch} \parencite[171, (10b)]{vanUrk2013}\\
    \gll \llap{*}De leraren  werden overtuigd  om ze  te mogen kietelen.\\
		the teachers\textsubscript{L}  were  convinced  \Comp{}  them\textsubscript{L}  to may  tickle\\
	\glt Lit. ‘The teachers were convinced to be allowed to tickle them.’\\
			‘The teachers were convinced that they/someone would be allowed to tickle them (the teachers).’
	\z
\ex\label{ex:23.3} *\textsc{implicit.agent}  DP.\Acc{} $\to$ \Nom{}  V  [\textsubscript{\Inf{}}  PRO \textsubscript{…}  ]\\
	\ea\label{ex:23.3a} \ili{German} (personal knowledge)\\
		\gll Die Kinder  haben  den Lehrer  gebeten,  ihn  kitzeln  zu dürfen.\\
			the children  have  the.\Acc{} teacher\textsubscript{L}  begged  him\textsubscript{L}  tickle  to may\\
		\glt ‘The children begged the teacher to be allowed to tickle him.’ (PRO=children \ding{51})\\
			‘The children begged the teacher that they (the children) would be allowed to tickle him.’
    \ex\label{ex:23.3b} \ili{German} \parencite[171, \eqref{ex:23.10b}]{vanUrk2013}\\
        \gll \llap{*}Der Lehrer  wurde  gebeten,  ihn  kitzeln  zu dürfen.\\
			the.\Nom{} teacher\textsubscript{L}  was  begged  him\textsubscript{L}  tickle  to may\\
        \glt Lit. ‘The teacher was begged to be allowed to tickle him.’
	\z
\z
%
Given that implicit control\is{control!implicit control} is, in principle,
possible in these languages, a direct control approach faces the question of
how to distinguish between \eqref{ex:23.1} and
\eqref{ex:23.2}/\eqref{ex:23.3} if implicit control\is{control!implicit
control} is established as a direct dependency between an implicit
argument\is{implicit arguments} (e.g., \emph{pro}) and PRO. On the other hand,
if control is mediated by matrix T, the difference can be implemented since, as
suggested in van Urk’s \emph{revised Visser’s generalization} in
\eqref{ex:23.4}, a difference arises in whether T agrees,
\eqref{ex:23.2}/\eqref{ex:23.3}, or does not agree,
\eqref{ex:23.1}, with a matrix argument not connected to the control
dependency.

\ea\label{ex:23.4} \emph{Revised Visser’s generalization} \parencite[172, (12)]{vanUrk2013}\\
    Obligatory control by an implicit subject is impossible if an overt DP
    agrees with T.
\z

\begin{figure}
    \begin{floatrow}
    \ffigbox[.5\textwidth]
    {\begin{tikzpicture}[baseline=(root.base), align=center]
        \Tree 	[.\node(root){};
                    \node (t) {T\\{}[DP]};
                    [.\emph{v}P
                        \node (phi) {φ(P)};
                        [.VP
                            DP.\Dat{}
                            [.{}
                                V
                                [.\node(inf){\Inf}; \edge [roof]; {PRO} ]
                            ]
                        ]
                    ]
                ]

        \draw [arrow, <->, bend right=50]
            (t.south) to node [midway] (source) {} (phi.south);
        \node [below=.5cm of inf] (PRO) {\hphantom{PRO}};
        \draw [arrow, ->, bend right]
            (source.north) to (PRO.west);
    \end{tikzpicture}}
    {\caption{Implicit control across a dative\label{fig:ex:23.5}}}%
    \ffigbox[.5\textwidth]
    {\begin{tikzpicture}[baseline=(root.base)]
    \Tree 	[.\node(root){};
                \node (t) {T\\{}[DP]};
                [.\emph{v}P
                    \node (phi) {φ(P)};
                    [.VP
                        \node (dp) {DP.\Nom{}};
                        [.{}
                            V
                            [.\node(inf){\Inf}; \edge [roof]; {PRO} ]
                        ]
                    ]
                ]
            ]

    \draw [arrow, <->, bend right]
        (t.south) to node [midway] (source) {} (dp.west);
    \node [below=.5cm of inf] (PRO) {\hphantom{PRO}};
    \draw [arrow, ->, bend right] (source.north) to (PRO.west);
    \end{tikzpicture}}
    {\caption{Failure of implicit control\label{fig:ex:23.6}}}
\end{floatrow}
\end{figure}

A possible account of \eqref{ex:23.4} (this is a modified version of van Urk’s
suggestion)\largerpage[2]\relax{} is illustrated in \figref{fig:ex:23.5}. I
assume that implicit passive\is{passive} arguments are syntactically
represented as weak deficient pronouns, and, more specifically, as
φ-feature\is{φ-features} bundles without a D-layer  (see among many others
\citealt{CarSta1999,DecWil2002,Landau2010,Roberts2010,Roberts2010c}). I leave
open here whether these φ-bundles are projected as independent arguments or as
part of \emph{v} (see \citealt{Legate2012,Legate2014} for the latter). Due to
the lack of D-layer, which is required to receive a referential interpretation,
implicit passive\is{passive} arguments are not able to control (or bind) on
their own.  Instead, following the works in \citet{BibHolRobShee2010}, I assume
that weak pronouns can acquire referential properties or grounding through an
Agree dependency with T, for instance, via a D-feature in T as indicated in
\Cref{fig:ex:23.5}, or via referential anchoring to the speech context through
the dependency with T.\footnote{The latter option may be preferred, since the
languages under consideration here (German and Dutch) are not null-subject
languages for which the D-feature in T has been proposed in the works cited in
the text.} In other words, although the implicit subject lacks a D-layer and
can thus not refer on its own, referential properties can be transmitted from T
or C through the Agree relation with T. After the features of the implicit
subject are strengthened by T (i.e., they acquire a D-property through T),
either of these elements can control PRO, depending on one’s ultimate control
mechanism. Thus, similar to agreement-based approaches to control as suggested
in \citet{Borer1989} and developed in \citeauthor{Landau2000}
(\citeyear{Landau2000} et seq.), Agree with T is essential for an implicit
argument to control PRO.

The failure of implicit control\is{control!implicit control} in \eqref{ex:23.2} and \eqref{ex:23.3}
is illustrated in \figref{fig:ex:23.6}. Since the matrix argument in these
constructions is not a lexical dative\is{dative case} DP but a structurally Case marked DP, it
has to Agree with T in passive\is{passive} contexts. This relation with T, I suggest, then
precludes any further dependency between T and another argument. In other
words, in \figref{fig:ex:23.6} T cannot enter an additional Agree relation with
the implicit subject since this would lead to referential identity between the
nominative\is{nominative case} argument and the implicit subject (i.e., a
non-existing reflexive interpretation – ‘the teachers begged/convinced
themselves’ in \eqref{ex:23.2}/\eqref{ex:23.3}). Similarly, T cannot
Agree with the implicit subject first since this would either leave the object
without Case or create two conflicting referential dependencies. As a result,
implicit control\is{control!implicit control} is impossible and the only
control relation that can be established in these contexts is control by the
nominative\is{nominative case} argument (which is in general possible in
passive\is{passive} contexts such as \eqref{ex:23.2}/\eqref{ex:23.3};
in the specific examples above, it would be excluded due to the resulting
\isi{binding} violation between PRO and the embedded pronouns).\largerpage

In both \ili{Dutch} and German, the difference in the availability of implicit
control between \eqref{ex:23.1} and \eqref{ex:23.2}/\eqref{ex:23.3}
disappears when an overt \emph{by}-phrase corresponding to the implicit agent
is present. As shown in \eqref{ex:23.7} and \eqref{ex:23.8}, the
interpretation that is impossible in \eqref{ex:23.2} and
\eqref{ex:23.3} becomes available when PRO can be understood to be
controlled by the \emph{by}-phrase.\largerpage

\ea\label{ex:23.7}
	\ea\label{ex:23.7a} \ili{Dutch} (P. Fenger, p.c.)\\
		\gll De leraren  werden  door de kinderen  overtuigd  ze  te mogen  kietelen.\\
			the teachers\textsubscript{L}  were  by the children  convinced  them\textsubscript{L}  to may  tickle\\
		\glt ‘The teachers were convinced by the children that they (the
        children) would be allowed to tickle them (the teachers).’
	\ex\label{ex:23.7b} German\\
		\gll Der Lehrer  wurde  von den Kindern  gebeten,  ihn  kitzeln  zu dürfen.\\
			the.\Nom{} teacher\textsubscript{L}  was  by the children  begged  him\textsubscript{L}  tickle  to may\\
		\glt Lit. ‘The teacher was begged to be allowed to tickle him.’
	\z
\ex\label{ex:23.8}
	\ea\label{ex:23.8a} \ili{Dutch} (P. Fenger, p.c.)\\
		\gll De leraar  werd  door de kinderen  gesmeekt niet  weer  hun best  te hoeven  doen.\\
			the teacher  was  by the children  begged not  again  their best  to have  do\\
		\glt ‘The teacher was begged by the children that they wouldn’t have to do their best again.’
	\ex\label{ex:23.8b} German\\
		\gll Der Lehrer  wurde  von den Kindern  angefleht, nicht  wieder  ihr Bestes  geben  zu müssen.\\
			the.\Nom{} teacher  was  by the children  beseeched not  again  their best  give  to must\\
		\glt ‘The teacher was beseeched by the kids that they wouldn’t have to give their best again.’
	\z
\z
%
There are two ways control by \emph{by}-phrase agents could be
achieved—directly via the DP within the \emph{by}-phrase or mediated by an
implicit Agent (which I assume is present in passive\is{passive} independently of whether
there is a \emph{by}-phrase agent or not). The first option, direct licensing
by the \emph{by}-phrase DP, is given in \figref{fig:ex:23.9}a. The c-command relation could be
established by covert movement of the DP outside the \emph{by}-PP, by
assuming that the \emph{by}-PP is transparent for c-command (e.g., by
treating the \emph{by}-PP as a DP in syntax and the preposition as a pure PF
element which is inserted as a last resort to license the DP), or by a strict
left-to-right branching structure for PPs as in \citet{Pesetsky1995}. The
second option in \figref{fig:ex:23.9}b is for the \emph{by}-phrase DP to anchor the deficient
implicit argument\is{implicit arguments} referentially (e.g., via \isi{binding}), which would then make the
implicit subject strong enough to license PRO.\largerpage

\begin{figure}
    \begin{subfigure}[b]{.5\linewidth}
    \begin{tikzpicture}[baseline=(root.base)]

        \Tree 	[.\node(root){};
                    T
                    [.\emph{v}P
                        \node (by) {(\emph{by}) DP};
                        [.\emph{v}P
                            φ(P)
                            [.VP
                                DP.\Nom{}
                                [.{}
                                    V
                                    [.\node(inf){\Inf}; \edge [roof]; {PRO} ]
                                ]
                            ]
                        ]
                    ]
                ]

        \node [below=.5cm of inf] (PRO) {\hphantom{PRO}};
        \draw [arrow, ->, bend right] (by.south) to (PRO.west);

    \end{tikzpicture}
    \caption{Control by the overt \emph{by}-DP}
    \end{subfigure}%
    \begin{subfigure}[b]{.5\linewidth}
    \begin{tikzpicture}[baseline=(root.base)]

        \Tree 	[.\node(root){};
                    T
                    [.\emph{v}P
                        \node (by) {(\emph{by}) DP};
                        [.\emph{v}P
                            \node (phi) {φ(P)};
                            [.VP
                                DP.\Nom{}
                                [.{}
                                    V
                                    [.\node(inf){\Inf}; \edge [roof]; {PRO} ]
                                ]
                            ]
                        ]
                    ]
                ]


        \draw [arrow, ->, bend right] (by.south) to (phi.west);
        \node [below=.5cm of inf] (PRO) {\hphantom{PRO}};
        \draw [arrow, ->, bend right] (phi.south) to (PRO.west);

    \end{tikzpicture}
    \caption{Control by the implicit agent}
    \end{subfigure}
    \caption{\emph{by} \textsc{agent} DP.\Acc{} $\to$ \Nom{} V [\tss{\Inf} PRO \dots{} ]\label{fig:ex:23.9}}
\end{figure}

Importantly, both options in \figref{fig:ex:23.9} involve \emph{direct control}
which cannot be mediated by T. In the examples in \eqref{ex:23.7} and
\eqref{ex:23.8}, T is still engaged in a Case and agreement dependency with
the overt DP argument of the matrix clause, which is referentially independent
from the implicit/\emph{by}-phrase Agent and PRO. Thus, T cannot be involved in
the control relation in these cases, and control is established directly by the
antecedent.

At this point, one may wonder whether it is possible to have a unified
mechanism for control based on direct licensing. Taking the option in
\figref{fig:ex:23.9}b, one could imagine that it is always the implicit passive
subject that licenses PRO directly, however, it can only do so when supplied
with a D-property through Agree with T or association with a \emph{by}-DP.
While this is attractive for its uniformity, the data below may suggest that
there is still a difference between licensing of PRO mediated by T vs.\ the
\emph{by}-phrase Agent. As shown in \eqref{ex:23.10}, in both \ili{Dutch}
and \ili{German} implicitly controlled PRO in a \emph{promise} context (i.e., a
context where the implicit argument\is{implicit arguments} can be associated with T) cannot bind lower
possessive pronouns, as would be required in the \emph{to do one’s best}
construction.\footnote{The same restriction is also found in simple passive
    statements like (i). As in the case of control discussed below in the text,
    bound possessors become possible when an overt \emph{by}-phrase Agent is
    added as in (ii). For a comparison of \isi{binding} in the \emph{to do one’s
    best} construction with other \isi{binding} relations (apparently) established by
    an implicit passive\is{passive} argument, see \citet{Wurmbrand2016}.

\begin{exe}
    \exi{(i)} \ili{German}\\
    Wie haben sich die Kinder heute verhalten?
    \begin{xlist}
    \ex[*]{
    \gll    Es  wurde sein / ihr Bestes gegeben.\\
            it  was his {} their best  given.\\
    \glt    intended: ‘They did their best.’}
    \ex[]{
    \gll    Es  wurde  von jedem  sein Bestes  gegeben.\\
            it  was  by everyone  his best  given\\
    \glt    ‘Everyone did their best.’, literally ‘The best was given by everyone.’}
    \end{xlist}
\end{exe}} In \eqref{ex:23.11a} it is shown that even when the implicit
Agent is contextually very salient, the interpretation in which the possessive
pronoun (and PRO) refer to the implicit matrix subject is impossible. In
contrast, if the matrix clause includes a \emph{by}-phrase Agent, the implicit
control and \isi{binding} relation becomes possible again.\largerpage[2]

\ea\label{ex:23.10}
	\ea\label{ex:23.10a} \ili{Dutch} (P. Fenger, p.c.)\\
        \gll \llap{*}Mij  werd  beloofd  (om)  zijn / haar / hun  best  to doen.\\
            I.\Dat{}  was  promised  \hphantom{(}\Comp{}  his {} her {} their  best  to do\\
        \glt lit.\ ‘I was promised to do his/her/their best.’ \\
			‘I was promised that they would do their best.’
    \ex\label{ex:23.10b} \ili{German}\\
        \gll \llap{*}Mir  wurde  versprochen / angeboten,  sein / ihr  Bestes  zu geben.\\
            I.\Dat{}  was  promised {} offered  his {} her=their  best  to give\\
        \glt intended: ‘I was promised/offered that they would do their best.’
	\z
\ex\label{ex:23.11} John just returned from a meeting with his boss. What happened?\\
    \ea\label{ex:23.11a}\ili{German}\\
		\gll Dem Hans  wurde  angeboten / versprochen,  seine Beleidigung  zurückzunehmen  /  nächstes Mal  sein Bestes  zu geben.\\
            the.\Dat{} John  was  offered {} promised his insult  away.to.take {}  next time  his best  to give\\
		\glt possible: ‘John was offered/promised to retract his (=John’s) insult/do his (John’s) best next time.’\\
			*intended: ‘John was offered/promised that he (the boss) would retract his (the boss’) insult/do his (the boss’) best next time.’
    \ex\label{ex:23.11b} \ili{German}\\
		\gll Dem Hans  wurde  von seinem Chef  angeboten / versprochen, seine Beleidigung  zurückzunehmen  /  nächstes Mal  sein Bestes  zu geben.\\
         the.\Dat{} John  was  by his boss  offered {} promised his insult  away.to.take {}  next time  his best  to give\\
		\glt ‘John was offered/promised that he (the boss) would retract his
        (the boss') insult/do his (the boss’) best next time.’
	\z
\z
%
One way to derive this difference is to differentiate between direct control by
a referential DP vs.\ control by a non-referential argument which is (merely)
anchored to the context via T. This then allows us to formulate the following
generalizations:

\ea
	\ea Implicit passive\is{passive} arguments cannot control on their own.
	\ex Implicit passive\is{passive} arguments can control when anchored to the context via an Agree dependency with T, but such control does not transmit referential properties.
	\ex Overt DPs can control and transmit referential properties.
	\z
\z
%
The above thus points to a hybrid approach—control is established either as a
direct (syntactic and semantic) \isi{binding} relation between a referential DP and
PRO, or a non-referential φ-feature\is{φ-features} dependency between a weak implicit subject
pronoun (or subject features on \emph{v}) and PRO, which is only possible when
the subject is anchored to the context via T.\footnote{As pointed out by a
    reviewer, this approach may be extended to implicit control\is{control!implicit control} (as in \emph{It
    is/was difficult to catch an early train}) which shows differences in the
interpretation of the embedded subject (generic or specific) depending on the
value of matrix tense (present vs.\ past).}

In conclusion, the data presented in this squib support the view of deficient
pronouns as φ-bundles, the relevance of T in referentially licensing weak
subject pronouns, and the involvement of T in certain cases of control. More
specifically regarding control, the least it seems we can conclude from the
data presented here is that control does not always require a dependency
between PRO and matrix T but can also be established as a direct relation
between the reference supplying DP and PRO. Whether control can always be
established as a direct antecedent—PRO dependency is left for another occasion.
\pagebreak
\printchapterglossary{}

{\sloppy\printbibliography[heading=subbibliography,notkeyword=this]}

\end{document}
