\documentclass[output=paper]{langsci/langscibook}

\author{Norbert Corver\affiliation{Utrecht Institute of Linguistics-OTS,
Utrecht University}}
\title{Inflected intensifiers:\newlineCover{} The structure-dependence of parasitic agreement}

% \chapterDOI{} %will be filled in at production

\abstract{This article examines parasitic agreement\is{agreement!parasitic agreement} in \ili{Dutch}, that is, the
appearance of an inflection whose existence is dependent on the presence of a
\enquote{real} inflection. Specifically, an intensifying degree word\is{degree words} (optionally)
carries an inflection that is associated with a gradable attributive adjective.
The article lays bare various properties of, and constraints on, the phenomenon
of parasitic agreement\is{agreement!parasitic agreement}. An important conclusion that follows from the analysis
of parasitic agreement\is{agreement!parasitic agreement} is that this phenomenon is structure dependent, just
like the parasitic gap\is{parasitic gaps} phenomenon. The structural configuration that is claimed
to be at the basis of parasitic agreement\is{agreement!parasitic agreement} is the Spec-head relationship.}


\begin{document}\glsresetall
% % \multicolsep=.25\baselineskip
\maketitle

%\textbf{Keywords:} intensifier\is{intensifiers}, adjective, inflection, parasitism, structure dependence, Dutch

\section{Parasitism in human language}

Research on \isi{parasitic gaps} has made us familiar with the phenomenon of
parasitism in syntax, that is the phenomenon that the presence of a symbol of
type α in a syntactic representation is dependent (i.e., parasitic) on the
presence of another symbol of type α in that same representation; see among
others \textcite{Ross1967}, \textcite{Taraldsen1981}, \textcite{Chomsky1982},
and \textcite{Engdahl:1983}. Example \eqref{ex:18.1} is an illustration of the
parasitic gap phenomenon:

\ea%1
    \label{ex:18.1}
    {}[\textsubscript{CP} \emph{Which articles} did [\textsubscript{TP} John
    [\textsubscript{VP} file \emph{e}\textsubscript{\Rg{}}] [without
    reading \emph{e}\textsubscript{\Pg{}}]]]]?
\z

The gap (\emph{e}\textsubscript{\Pg{}}) in the adjunct clause depends on the
existence of another gap (the \enquote{real} gap: \emph{e}\textsubscript{\Rg{}}) in the
main clause, sharing with it the direct object wh-phrase \emph{which articles}.
If the object noun phrase of the main clause is \emph{in situ}, the appearance
of \emph{e}\textsubscript{\Pg{}} in the adjunct clause is impossible:
\emph{*John filed this book without reading}). In that case, presence of an
overt element is required: \emph{\dots{}without reading it}. Obviously,
the presence of the pronoun \emph{them} in the adjunct clause in
\eqref{ex:18.1} is also possible.

Research on parasitic gap\is{parasitic gaps} constructions led to an important conclusion: the
appearance of the parasitic gap\is{parasitic gaps} is structure-dependent\is{structure-dependence}.\footnote{See e.g.
    \citet{Chomsky1975} for the notion of structure dependence. See also
    \citet{EveraertEtAl2015} for various illustrations of the structure
dependence of grammatical rules.} Specifically, the parasitic gap
(\emph{e}\textsubscript{\Pg{}}) may not be linked to a real gap
(\emph{e}\textsubscript{\Rg{}}) that is in a structurally higher position.  In
more formal terms: \emph{e}\textsubscript{\Pg{}} cannot be c-commanded by
\emph{e}\textsubscript{\Rg{}}. This anti-c-command requirement is met in
\eqref{ex:18.1}: \emph{e}\textsubscript{\Rg{}}, which is dominated by VP, is in
a structurally lower position than \emph{e}\textsubscript{\Pg{}}, which is part
of an adjunct clause higher up in the clausal structure. The anti-c-command
requirement is violated, however, in \eqref{ex:18.2}, where
\emph{e}\textsubscript{\Rg{}}, the \enquote*{trace} of the \textsc{wh}-moved subject
noun phrase, c-commands \emph{e}\textsubscript{\Pg{}} in the adjunct clause.

\ea[*]{[\textsubscript{CP} Who [\textsubscript{TP}
    \emph{e}\textsubscript{\Rg{}} [\textsubscript{VP} met you] [before you
    recognized \emph{e}\textsubscript{\Pg{}}]]]?}\label{ex:18.2}
\z

The case study on \isi{parasitic gaps} raises the question whether other instances of
syntactic parasitism can be found in natural language syntax. That is, are
there other phenomena in which the appearance of symbol α depends on the
existence of another symbol α? And to what extent is the appearance of the
parasitic symbol subject to a structure dependent requirement? In this article,
I present a case study on morpho-syntactic parasitism in \ili{Dutch}.
Specifically, an adjectival agreement suffix (\emph{-e}, pronounced schwa) can
optionally appear on an adjectival degree word\is{degree words} (an intensifier) that modifies
an overtly inflected attributive adjective (see
\citealt{Verdenius1939,Royen1948,Corver1997a}). An example is given in
\eqref{ex:18.3}.

\ea%3
    \label{ex:18.3}\ili{Dutch}\\
    \gll een erg(-e) leuk-e auto\\
         a very-(\emph{e}) nice-\emph{e} car\\
    \glt
\z

The article is organized as follows: \Cref{sec:18.2} introduces the phenomenon
of parasitic agreement\is{agreement!parasitic agreement}. \Cref{sec:18.3} discusses semantic and categorial
restrictions on the intensifier\is{intensifiers} that carries the parasitic agreement\is{agreement!parasitic agreement} morpheme.
In \Cref{sec:18.4}, multiple parasitism is discussed, that is, the appearance
of more than one parasitic agreement\is{agreement!parasitic agreement} morpheme within the adjectival projection.
\Cref{sec:18.5} discusses a string-based analysis of parasitic agreement\is{agreement!parasitic agreement}, and
\Cref{sec:18.6} discusses a structure-based approach according to which the
intensifier and the gradable adjective are represented as separate attributive
modifiers within the noun phrase.  \Cref{sec:18.7} presents the analysis
adopted in this article: parasitic agreement\is{agreement!parasitic agreement} as a manifestation of the
Spec-head agreement configuration. In \Cref{sec:18.8} the phenomenon of
parasitic agreement is associated with emphasis of information.
\Cref{sec:18.9} concludes the article.

\section{Augmented degree words}\label{sec:18.2}

Consider the inflectional paradigm of \ili{Dutch} attributive adjectives:

\begin{multicols}{2}
\ea%4
\label{ex:18.4}\ili{Dutch}
	\ea
	\gll de  leuk-\textbf{e}  auto\textsubscript{[−neuter]}\\
		the    nice-\textsc{e}  car  \\
	\ex
	\gll een  leuk-\textbf{e}  auto\textsubscript{[−neuter]}\\
		a        nice-\textsc{e}  car\\
	\ex
	\gll  (de)  leuk-\textbf{e}  auto's\textsubscript{[−neuter]}\\
		(the)  nice-\textsc{e}  cars \\
	\z
\ex%5
\label{ex:18.5}\ili{Dutch}
	\ea
	\gll het leuk\textbf{e}  huis\textsubscript{[+neuter]}\\
		the    nice    house\\
	\ex
	\gll een leuk  huis\textsubscript{[+neuter]}\\
		a        nice  house\\
	\ex
	\gll  (de)  leuk\textbf{e}  huizen\textsubscript{[+neuter]}\\
		 (the)  nice    houses\\
	\z
\z\end{multicols}

As (\ref{ex:18.4}--\ref{ex:18.5}) show, attributive adjectives in \ili{Dutch}
normally carry the adjectival inflection \emph{–e} (i.e., /ə/), as in
\emph{leuk}\textbf{\emph{e}}. However, when the attributive adjective modifies
a noun phrase with the feature constellation [+neuter], [+singular],
[−definite], as in (\ref{ex:18.5}b), the attributive adjective is
morphologically bare (\emph{leuk}), in the sense that there is no overt
inflection attached to the adjective. I assume that, in that case, a zero-affix
is attached to the adjective: \emph{leuk-$\varnothing$}; see
\Cref{sec:18.8} for an argument in support of the presence of this
zero-affix.

Consider next the examples in \eqref{ex:18.6}, in which the attributive
adjectival expression contains an intensifying degree modifier that specifies
the degree to which the property denoted by the gradable adjective
(\emph{dure}) holds. As indicated, this degree word\is{degree words} can optionally carry a
schwa. From now on, this augmentative schwa, which is typically found in
colloquial speech, is represented as \textsc{-e}. This way, it is
orthographically easily distinguishable from the adjectival inflection
\emph{-e} on the attributive adjective.

\ea%6
    \label{ex:18.6}\ili{Dutch}\\
    \gll een  [ erg(-\textsc{e})  /  afgrijselijk(\textsc{-e})  /  ongelofelijk(\textsc{-e})    dur-\textbf{e} ]                  auto\\
    a  {}    very(-\textsc{e})  {}  horrible(-\textsc{e})  {}  unbelievable(\textbf{\textsc{-e}})  expensive-\Agr{} {} car \\
    \glt ‘a very / horribly / unbelievably expensive car’
\z

The appearance of \textsc{-e} on the degree word\is{degree words} is dependent
(\emph{parasitic}) on the appearance of overt inflectional morphology (i.e.,
\emph{-e}) on the modified adjective. This is clear from the examples in
\eqref{ex:18.7} and \eqref{ex:18.8}. Only if \emph{-e} is attached to
\emph{leuk} can the degree modifier be augmented with \textsc{e}. If there is
no overt inflectional morphology (i.e., \emph{-e}) present on the attributive
adjective, \textsc{e} cannot appear on the degree modifier.\footnote{See,
though, \Cref{sec:18.8}.} This is shown by (\ref{ex:18.8}a), where we
have an attributive adjective within a [−definite, +singular, +neuter] noun
phrase. As illustrated by (\ref{ex:18.8}b), augmentative \textsc{-e} is
permitted when the attributive adjectival occurs in a noun phrase specified as
[−definite, −singular, +neuter]. In that nominal environment, the attributive
adjective carries overt inflection.

\begin{multicols}{2}\ea%7
    \label{ex:18.7}\ili{Dutch}
	\ea
	\gll een  [erg(\textbf{\textsc{-e}})    leuk-e]      auto\\
	         a      very-\textsc{e}      nice-\Agr{}    car \\
	    \glt
	\ex
	\gll [erg(\textbf{\textsc{-e}})  leuk-e]      auto's\\
		very-\textsc{e}    nice-\Agr{}    cars\\
	\glt
	\z
\ex%8
    \label{ex:18.8}\ili{Dutch}
	\ea
	\gll een  [erg(*\textbf{\textsc{-e}})    leuk]    huis\\
	        a        very-\textsc{e}        nice      house\\
	    \glt
	\ex
	\gll erg(\textsc{-e})      leuk-e        huizen\\
		very(\textsc{-e)}    nice-\Agr{}    houses\\
	\glt
	\z
\z\end{multicols}

A further illustration of the fact that the appearance of \textsc{-e} on the
degree word is parasitic on the presence of inflectional \emph{–e} on the
(gradable) adjective, comes from NP-ellipsis constructions. As shown by the
contrast between (\ref{ex:18.9}a) and (\ref{ex:18.9}b), \emph{-e} typically
appears on an attributive adjectival modifier when the nominal head of the
indefinite neuter singular noun phrase has been elided
(\citealt{Kester1996,CorvervanKoppen2011}).\largerpage[-2]

\ea%9
    \label{ex:18.9}\ili{Dutch}
	\ea
	\gll Jan   heeft  [ een  [ heel  lief ] konijn ] en  Marie  heeft  [ een  [ heel  stout ]      konijn ].\\
    Jan    has    {}  a  {}  very    sweet {} rabbit {} and  Marie  has  {} a  {}  very    naughty {}    rabbit\\
	\glt
	\ex
	\gll Jan    heeft  [ een  [ heel  lief ]      konijn ] en  Marie  heeft  [ een  [ heel    stout\textbf{e} ] $\varnothing$ ].\\
    Jan    has  {}  a    {}  very    sweet {} rabbit  {}  and  Mary  has    {}
    a   {}   very    naughty-\Agr{}\\
	\glt \enquote*{Jan has a very sweet rabbit and Mary has a very naughty one.}
	\z
\z

Notice now that the inflected attributive adjective (\emph{stoute}) in the
NP-ellipsis pattern licenses the appearance of \textsc{-e} on the degree word\is{degree words}
(yielding \emph{hel}\textsc{e}); see (\ref{ex:18.10}b). As shown by
(\ref{ex:18.10}a), \emph{hel}\textsc{e} is impossible when NP-ellipsis has not
applied to the nominal expression.

\ea%10
\label{ex:18.10}\ili{Dutch}
    \ea[*]{
	\gll Jan heeft een heel lief konijn en Marie heeft [ een [ hel\textbf{\textsc{e}} stout ] konijn ].\\
        Jan has a very sweet rabbit and Marie has {} a {} very-\textsc{e} naughty rabbit\\}
         \ex[]{
	\gll Jan    heeft    een    heel    lief        konijn    en  Marie  heeft  [een    [hel\textbf{\textsc{e}}      stout\textbf{e}] ${\varnothing}$].\\
		 Jan    has        a       very    sweet    rabbit    and  Marie  has a
     very-\textsc{e}    naughty-\Agr{} {}\\}
	\z
\z

A third observation that suggests that the appearance of augmentative
\textsc{-e} is parasitic on the presence of (overt) adjectival inflection
(i.e., \emph{-e}) on the adjective comes from predicatively used APs.
Predicative APs, as opposed to attributive ones, do not display any (overt)
inflection on the adjectival head, as is exemplified in \REF{ex:18.11}.
Observe that it is impossible to have an augmentative \textsc{-e} on the
adjectival degree word\is{degree words} that modifies the predicative adjective:

\ea%11
\label{ex:18.11}\ili{Dutch}\\
    \gll Deze    auto    is  erg(*-\textsc{e})    leuk.\\
        this      car      is  very(\textsc{-e})  nice\\
    \glt \enquote*{This car is really nice.}
\z

\section{Semantic and categorial restrictions on parasitic
agreement}\label{sec:18.3}\largerpage[-2]

Besides the morpho-syntactic requirement that the modified attributive
adjective carry the adjectival inflection \emph{–e}, there are a number of
other restrictions on the appearance of augmentative \textsc{-e}. From a more
interpretative point of view, augmentative \textsc{-e} typically occurs on
intensifiers that belong to the subtype of amplifiers; that is, \isi{degree words}
that scale upwards from some tacitly assumed standard value or norm (see
\citealt[104]{Broekhuis2013}). Besides the \isi{intensifiers} \emph{erg},
\emph{afgrijselijk}, and \emph{ongelofelijk} in \eqref{ex:18.6}, this subtype
also includes modifiers such as \emph{vreselijk} \enquote*{extremely},
\emph{ontzettend} \enquote*{terribly}, \emph{ongelofelijk}
\enquote*{unbelievably}, \emph{waanzinnig} \enquote*{insanely}, \emph{geweldig}
\enquote*{tremendously}, \emph{verschrikkelijk} \enquote*{terribly},
\emph{belachelijk} \enquote*{absurdly}, \emph{behoor\-lijk}
\enquote*{quite/rather}.\footnote{For some speakers \textsc{-e} is also
    acceptable on downtoners (i.e., down-scaling degree words) such as
    \emph{tamelijk} \enquote*{rather} and \emph{redelijk}
    \enquote*{reasonably}, as in \emph{een} \emph{tamelijk-}\textsc{e}
    \emph{lompe} \emph{opmerking} (a quite-\textsc{e} rude-\Agr{} remark) and
\emph{een} \emph{redelijk-}\textsc{e} \emph{snelle} \emph{auto} (a
reasonable-\textsc{e} fast-\Agr{} car \enquote*{a reasonably fast car}).}

As shown by (\ref{ex:18.12}a,b), modifiers of absolute adjectives -- i.e.,
adjectives that are not scalar but rather imply the endpoint of a scale -- tend
to be less easily combinable with \textsc{-e}. The same holds for the
approximative modifier \emph{praktisch} in (\ref{ex:18.12}c). It should be
noted, though, that instances of such patterns can be found on the internet
(Google search), whence the judgment \textsuperscript{\%}, which means
acceptable for some speakers but not for others.

\ea%12
    \label{ex:18.12}\ili{Dutch}
	\ea
	\gll  een    compleet / \textsuperscript{\%}complet-\textsc{e}    leg-e            kamer\\
    a        complete {} \hphantom{\textsuperscript{\%}}complete-\textsc{e}      empty-\Agr{}  room\\
	\glt \enquote*{a completely empty room}
	\ex
	\gll een    volledig / \textsuperscript{\%}volledig-\textsc{e}        naakt-e        vrouw\\
    a        complete {} \hphantom{\textsuperscript{\%}}complete-\textsc{e} naked-\Agr{}  woman\\
    \glt \enquote*{a completely naked woman}
	\ex
	\gll  een    praktisch / \textsuperscript{\%}praktisch-\textsc{e}    leg-e            kamer\\
            a        virtual {} \hphantom{\textsuperscript{\%}}virtual-\textsc{e} empty-\Agr{}  room\\
    \glt \enquote*{a practically empty room}
	\z
\z

As indicated by the examples in \REF{ex:18.13} modal, temporal or evaluative
modifiers are never augmented with \textsc{-e}.

\ea%13
    \label{ex:18.13}\ili{Dutch}\\
    \gll een [ vermoedelijk(*\textbf{\textsc{-e}})  /  tijdelijk(*\textbf{\textsc{-e}})    /  [ gelukkig(*\textbf{\textsc{-e}})    goedkop-\textbf{e} ] fiets\\
    a  {}    presumable-\textsc{e}  {}  temporary-\textsc{e} {} {} fortunate-\textsc{e}  cheap-\Agr{} {} bike\\
    \glt \enquote*{a presumably / temporarily / fortunately cheap bike}
\z

Having shown that augmentative -\textsc{e} typically occurs on (amplifying)
intensifiers, I now turn to a second restriction on the word that functions as
a host for \textsc{-e}. Categorially, the host must be adjectival in nature.
Importantly, in line with \citegen{Bowers1975} and \citegen{Emonds1976} claim
that \ili{English} \enquote{adverbs} such as \emph{extremely} and \emph{terribly} are
actually adjectives, I propose that adverbially used degree modifiers such as
\emph{erg}, \emph{afgrijselijk}, and \emph{ongelofelijk} in \eqref{ex:18.6} are
actually adjectives.\footnote{Thus, I do not claim that the modifiers in
\eqref{ex:18.6}, and also those in \REF{ex:18.12}, (categorially) are adverbs
that can be turned into adjectives by means of affixation of \textsc{-e}. These
modifiers are adjectives that can be used adverbially, in the spirit of
\citet{Bowers1975} and \citet{Emonds1976}.} Evidence in support of their
adjectival nature comes from their distributional behavior. As illustrated in
\REF{ex:18.14}, these intensifying elements occur in syntactic positions that
are typically (though not exclusively) occupied by adjectives. For example,
they occur as attributive modifiers of nouns, complements of copular
verbs,\is{copulas} and
complements of verbs like \emph{vinden}, which  select a predicative
complement:

\ea%14
    \label{ex:18.14}\ili{Dutch}
	\ea
	\gll  een    afgrijselijke  blunder\\
		 a        horrible          mistake\\
	\ex
	\gll  Deze    film      is    afgrijselijk.\\
		this      movie  is    horrible\\
	\ex
	\gll Ik    vind    die    muziek    afgrijselijk.\\
		  I      find    that  music    horrible\\
	\z
\z

Consider now the degree modifiers \emph{zeer} \enquote*{very} and \emph{vrij}
‘rather/fairly’, which are, respectively, an amplifying
intensifier\is{intensifiers} and a downtoning one. As \REF{ex:18.15} shows,
augmentation with –\textsc{e} is impossible.\footnote{\citet{Verdenius1939}
    gives the form \emph{eine zere nette miensj} (a very-\textsc{e}
    decent-\Agr{} person, \enquote*{a very decent person}) for Limburgian
    \ili{Dutch}. The augmented form \emph{zere} suggests that in this variety
of \ili{Dutch} \emph{zeer} is adjectival.}

\ea%15
    \label{ex:18.15}\ili{Dutch}\\
    \gll een   zeer/*zer-\textsc{e}    dure                      auto\\
        a        very/very-\textsc{e}    expensive-\Agr{}    car\\
    \glt
\z

As shown in \REF{ex:18.16}, the degree modifier \emph{zeer} does not appear in
positions where adjectives are typically found.

\ea%16
    \label{ex:18.16}\ili{Dutch}
    \ea[*]{
	\gll  een  zer-e        blunder\\
    a        horrible    mistake\\}
    \ex[*]{
	\gll De    pijn  was  zeer.\\
    the    pain  was  very\\}
    \ex[*]{
	\gll Ik    vond    de  pijn    zeer.\\
    I      found    the  pain    very\\}
	\z
\z

\section{Multiple parasitism}\label{sec:18.4}

\textsc{e-}augmentation can sometimes apply to more than one degree word\is{degree words} within
the extended adjectival projection. This phenomenon of multiple parasitism is
typically found in (inflected) attributive adjectival phrases featuring the
complex modifier \emph{heel} \emph{erg} (very much). An example is given in
\REF{ex:18.17}:

\ea%17
    \label{ex:18.17}\ili{Dutch}\\
    \gll  een    [\textsubscript{AxP} hel-\textsc{e}    erg-\textsc{e}
    dur-e ]          fiets\\
    a   {}         very-\textsc{e}    much-\textsc{e}  expensive-\Agr{} {} bike\\
    \glt \enquote*{a really very expensive bike}
\z

This \enquote{spreading} of schwa is not an arbitrary process. As shown in
\REF{ex:18.18}, it is impossible to \enquote{skip} a potential carrier of
augmentative schwa. In a way, a non-augmented degree word\is{degree words} counts as an
intervener for leftward spreading of augmentative schwa (see also
\citealt{Corver1997a,Broekhuis2013}).\footnote{A reviewer points out
    that the restriction on \enquote{spreading} in \REF{ex:18.18} is
    reminiscent of the weak-strong alternation in \ili{German}, where mixed endings
    are acceptable, but the endings can never go \enquote{back and forth}
    between the paradigms:

\begin{exe}
    \exi{(i)}\ili{German}
    \begin{xlist}
        \ex[]{
    	\gll mit  kühl\textbf{em},  frisch\textbf{en},  lecker\textbf{en}  Bier\\
            with  cool      fresh        nice          beer  \\}
        \ex[*]{mit kühl\textbf{em}, frisch\textbf{en}, lecker\textbf{em} Bier}
    \end{xlist}
\end{exe}}

\ea%18
    \label{ex:18.18}\ili{Dutch}
    \ea[]{
	    \gll een  [\textsubscript{AxP}  heel      erg      dure ]                \textbf{auto}\\
        a  {}            real      very    expensive-\Agr{} {} car\\
        \glt \enquote*{a really very expensive car}}
    \ex[?]{een [\textsubscript{AxP} heel erg-\textsc{e} dure] auto}
    \ex[]{een [\textsubscript{AxP} hel-\textsc{e} erg-\textsc{e} dure] auto}
    \ex[*]{een [\textsubscript{AxP} hel-\textsc{e} erg dur\textbf{e}] auto}
	\z
\z

Another pattern in which the phenomenon of multiple parasitism is found is
given in \REF{ex:18.19}:

\ea%19
    \label{ex:18.19}\ili{Dutch}
    \ea[]{
	    \gll een erg      erg      dure            auto\\
	        a        very    very    expensive  car \\
        \glt \enquote*{a really very expensive car}}
    \ex[?]{een erg, erg-\textsc{e} dur-e auto}
    \ex[]{een erg-\textsc{e} erg-\textsc{e} dure auto}
    \ex[*]{een erg-\textsc{e} erg dure auto}
	\z
\z

In these examples, we have an iterative pattern: repetition of the degree
modifier amplifies the intensifying meaning.

\section{Parasitic agreement: A string-based approach?}\label{sec:18.5}

From the parasitic agreement\is{agreement!parasitic agreement} phenomena discussed so far one might draw the
conclusion that augmentation of the intensifier\is{intensifiers} with \textsc{-e} is a
string-based “surface-structure” effect. That is, \textsc{e}-augmentation is a
pure \glsunset{PF}\gls{PF}-phenomenon that results from \emph{linear}-based spreading of the
adjectival inflection of the attributive adjective onto the \emph{linearly
adjacent} adjectival degree word\is{degree words}. More specifically, the affix \emph{-e} of the
attributive adjective gets copied onto the adjectival degree word\is{degree words} under linear
adjacency, a process reminiscent of \citegen{EmbNoy2001}
post-syntactic (morphological merger) rule of \emph{local dislocation}.
Schematically, we have the process as depicted in \REF{ex:18.20}, where
\emph{α * ${\beta}$} means that the elements \emph{α} and \emph{${\beta}$} are
linearly adjacent. Augmentation applies in a right to left direction, where the
agreement morpheme \emph{-e} on \emph{dure} gets copied onto the immediately
left adjacent instance of \emph{erg}, yielding \emph{erg\textsc{e}}, whose
inflection is subsequently copied onto the leftmost instance of \emph{erg},
resulting in the sequence \emph{erg\textsc{e}} \emph{erg\textsc{e}}
\emph{dure}.

\ea%20
    \label{ex:18.20}
	\ea \emph{een}   \emph{*} \emph{erg}   \emph{*} \emph{dur}\textbf{\emph{e}} \emph{*} \emph{auto}                ${\rightarrow}$ \\
              \emph{een} \emph{*} \emph{erg}\textbf{\textsc{e}} \emph{*}
              \emph{dur}\textbf{\emph{e}} \emph{*} \emph{auto}
              (\emph{een} \emph{erg\textsc{e}} \emph{dure} \emph{auto})
	\ex \emph{een} \emph{*} \emph{erg} \emph{*} \emph{erg} \emph{*} \emph{dur}\textbf{\emph{e}} \emph{*} \emph{auto}      ${\rightarrow}$ \\
              \emph{een} \emph{*} \emph{erg} \emph{*} \emph{erg}\textbf{\textsc{e}} \emph{*} \emph{dur}\textbf{\emph{e}} \emph{*} \emph{auto}    ${\rightarrow}$ \\
              \emph{een} \emph{*} \emph{erg}\textbf{\textsc{e}} \emph{*}
              \emph{erg}\textbf{\textsc{e}} \emph{*}
              \emph{dur}\textbf{\emph{e}} \emph{*} \emph{auto}
              (\emph{een} \emph{erg\textsc{e}} \emph{erg\textsc{e}} \emph{dure}
              \emph{auto})
	\z
\z

A first potential problem for this string-based analysis is the fact that
degree word augmentation is possible if linguistic material linearly
intervenes. Specifically, the parenthetical word \emph{ja} \enquote*{yes},
expressing the speaker's reinforced affirmation of the high degree, may
separate the members of a sequence of iterated \isi{degree words} like
(\ref{ex:18.20}b). This is exemplified in \REF{ex:18.21}, where
(\ref{ex:18.21}a) represents the non-augmented pattern and (\ref{ex:18.21}b) the
augmented pattern. If degree word\is{degree words} augmentation applied only under strict linear
adjacency with a following lexical item carrying \emph{-e}, then the
intervening \emph{ja} should block the \enquote{spreading} of schwa, but it
doesn't.

\ea%21
    \label{ex:18.21}\ili{Dutch}
	\ea
	\gll een  erg   \textbf{ja}      erg      \textbf{ja}      erg      goeie    grap\\
		a        very    yes    very    yes    very    good    joke\\
	\glt \enquote*{a really, yes, really good joke!}
	\ex een erg-\textsc{e} \textbf{ja} erg-\textsc{e} \textbf{ja} erg\textsc{-e} goeie grap
	\z
\z

A second potential argument against a linear, purely \gls{PF}-based analysis of
augmentative schwa comes from patterns in which \textsc{-e} is present on the
adjectival degree word\is{degree words} even though there is no overt adjectival inflection
\emph{-e} present on the gradable adjective that heads the adjectival
projection. The existence of such patterns suggests that augmentative schwa
does not simply result from a copying process that applies at the sound
surface; that is, \emph{-e} as part of an attributive adjective gets \gls{PF}-copied
onto a linearly adjacent adjectival degree word\is{degree words}.

Some relevant facts are given in \REF{ex:18.22}:

\ea%22
    \label{ex:18.22}\ili{Dutch}
	\ea
	\gll een  erg(-\textsc{e})    verlegen(*-e)  man\textsubscript{[−neuter]}\\
		a        very        shy                    man\\
	\ex
	\gll   een    erg(-\textsc{e})    belezen(*-e)    man\textsubscript{[−neuter]}\\
		 a        very        well-read        man\\
	\ex
	\gll  een  erg(-\textsc{e})    open(*-e)    samenleving\textsubscript{[−neuter]}\\
		 a        very        open            society\\
	\z
\z

The adjectives \emph{verlegen}, \emph{belezen}, and \emph{open} end in
\emph{–en} in written language but are pronounced as schwa in spoken (Standard)
\ili{Dutch}. Possibly, the absence of attributive adjectival inflection is
somehow related to the fact that the adjectival root ends with the sound schwa
(see also \citealt{Broekhuis2013}).

Importantly, the examples in \REF{ex:18.22} show that, in spite of the
presence of the right morphosyntactic feature constellation – i.e., [−neuter,
−definite, +singular] – the attributive adjectives do not display the
attributive adjectival inflection \emph{-e}. Nevertheless, it is possible to
add augmentative \textsc{-e} to the adjectival degree word\is{degree words}.
This suggests that the appearance of \textsc{-e} is not simply a matter of
(string-based) \gls{PF}-copying of an overt inflectional marker. Rather, what
really matters is the abstract feature constellation associated with the
attributive adjective.

For the sake of completeness, observe also the following examples, in which the
attributive adjective phrase is contained within a noun phrase having the
feature constellation [−definite, +singular, +neuter].\largerpage[-1]

\ea%23
    \label{ex:18.23}\ili{Dutch}
	\ea
	\gll een  erg(*-\textsc{e})    verlegen    kind\textsubscript{[+neuter]}\\
		 a        very            shy              child\\
	\ex
	\gll een  erg(*-\textsc{e})    belezen      kind\textsubscript{[+neuter]}\\
		a        very          well-read  child\\
	\ex
	\gll een  erg(*-\textsc{e})    open  volk\textsubscript{[+neuter]}\\
		a        very          open  nation (i.e., group of people)\\
	\glt \enquote*{very open-minded people}
	\z
\z

As we saw in (\ref{ex:18.5}b), the adjectival head never displays the overt
inflection \emph{‑e} in those contexts. Example (\ref{ex:18.8}a) further showed
that augmentative -\textsc{e} never appears on the degree word\is{degree words} in those
environments. The obligatory absence of augmentative –\textsc{e} in
\REF{ex:18.23} is completely in line with (\ref{ex:18.8}a). Importantly, the
patterns in \REF{ex:18.22} and \REF{ex:18.23} suggest that what matters for
\textsc{e}-augmentation is not the Spell-out (i.e., overt phonological
realization) of the adjectival inflection, but rather the abstract feature
complex that underlies Spell-out.

Let me now turn to a third argument against a string-based “surface” approach
to augmentative schwa. The argument comes from participles that are used
attributively. Consider the following examples featuring an inflected
attributive present participle:

\ea%24
    \label{ex:18.24}\ili{Dutch}
	\ea
	\gll een [ maandenlang  over    zijn  toekomst    erg(*\textsc{-e})      twijfelende ]      leerling\\
    a  {}    months.long      about  his    future        much(\textsc{-e})    doubting-\Agr{} {} student\\
	\glt \enquote*{a student who has been very much in doubt about his future for months}
	\ex
	\gll een  [ zich  al            jaren    daarop    erg(*\textsc{-e})    verheugende ]          man\\
    a {} \Refl{}  already  years    that.to      much(-\textsc{e})    look.forward-\Agr{} {} man\\
	\glt \enquote*{a man who has been rejoiced at that for many years}
	\z
\z

These examples show that the participles \emph{twijfelend} and
\emph{verheugend} can carry an attributive adjectival inflection \emph{-e} and
be modified by a degree modifier (\emph{erg}). As indicated, the degree
modifier cannot be augmented with –\textsc{e} even though it is linearly
adjacent to the inflected present participle. The ill-formedness of the
augmented form \emph{erg-}\textsc{e} suggests that \textsc{e}-augmentation is
not a surface process based on string-adjacency.

A similar conclusion can be drawn on the basis of the examples in
\REF{ex:18.25}, where the degree word\is{degree words} modifies a past/passive participle:

\ea%25
    \label{ex:18.25}\ili{Dutch}
	\ea
	\gll een [ toendertijd    door    iedereen    erg(*-\textsc{e})    gehat-e ]        dictator\\
        a  {} at.the.time        by        everyone    very(-\textsc{e})      hated-\Agr{} {}   dictator\\
	\glt ‘a dictator who was hated very much by everyone at the time’
	\ex
	\gll een [ toendertijd    door    iedereen    erg(*-\textsc{e})    gewantrouwd-e ]    president\\
    a   {}   at.the.time        by        everyone    very-\textsc{e}        distrusted-\Agr{} {}     president\\
	\glt \enquote*{a president who was distrusted very much by everyone at the time}
	\z
\z

The examples in \REF{ex:18.24} and \REF{ex:18.25} show that
\textsc{e}-augmentation of a degree word\is{degree words} is not possible when the degree word\is{degree words}
modifies a (linearly adjacent) present or past/passive participle. At this
point, it should be noted, though, that there are patterns in which
\textsc{e}-augmentation of the degree word\is{degree words} does seem to be possible when it
modifies a participle. Consider the following examples:

\ea%26
    \label{ex:18.26}
	\ea
	\gll een [ erg(\textsc{-e})    opwindende ]    gebeurtenis\\
        a {} very(-\textsc{e})    exciting-\Agr{} {} event\\
	\glt \enquote*{a very exciting event}
	\ex
	\gll een [ erg(-\textsc{e})  geïnteresseerde ]    student\\
        a {}   very-\textsc{e}      interested-\Agr{} {}     student\\
	\glt \enquote*{a very interested student}
	\z
\z

So, what underlies the contrast between \REF{ex:18.24} versus
(\ref{ex:18.26}a), and \REF{ex:18.25} versus (\ref{ex:18.26}b)?

From a string-based perspective, there is no difference as regards the distance
between the inflected present participle and the modifying degree
word\is{degree words}. So there must be another factor that is at the basis of
the contrast. This factor might very well be related to the categorial nature
of participles.  Specifically, the categorial nature of the participles in
(\ref{ex:18.24}--\ref{ex:18.25}) is verbal, while that of the
participles in \REF{ex:18.26} is adjectival (see also
\citealt{Broekhuis2013} for discussion). The verbal nature of the participles
in (\ref{ex:18.24}--\ref{ex:18.25}) is clear from their aspectual
properties.  The present participles in \REF{ex:18.24} express durative
aspect, as is clear from the presence of the modifiers \emph{maandenlang} and
\emph{al jaren}. The participle designates an ongoing event. Note that this
durative meaning is absent in (\ref{ex:18.26}a): \emph{opwindend} refers to
the property (a state of affairs) of being excited. The past/passive
participles in \REF{ex:18.25} express perfective aspect: we are dealing
with an event that has been completed. In (\ref{ex:18.26}b), on the
contrary, the participle \emph{geïnteresseerde} refers to the property of being
interested. In other words, it semantically acts like a true adjective.

Note that the adjectival nature of \emph{opwindend} and \emph{geïnteresseerd}
in \REF{ex:18.26} is confirmed by a number of diagnostics for adjectival
status (see also \citealt{Broekhuis2013}). Firstly, synthetic comparative
formation (\emph{-er}) can apply to these forms, as in \REF{ex:18.27}.

\ea%27
    \label{ex:18.27}\ili{Dutch}
	\ea
	\gll  een [ nog    opwindend-er-e ]            gebeurtenis\\
    an {} even    exciting-\Cmpr-\Agr{} {} event\\
	\glt ‘an even more exciting event’
	\ex
	\gll  een  [ nog  geïnteresseerd-er-e ]        student\\
    an {} even  interested-\Cmpr-\Agr{} {}  student\\
	\glt ‘an even more interested student’
	\z
\z

Secondly, as shown in \REF{ex:18.28}, these participles can be prefixed by
means of the negative morpheme \emph{on-}, which is typically found on
adjectives (e.g., \emph{aardig} \enquote*{kind}, \emph{onaardig}
\enquote*{unkind}).

\ea%28
    \label{ex:18.28}\ili{Dutch}
	\ea
	\gll  een [ onopwindende ]  gebeurtenis\\
    an {} unexciting-\Agr{} {}   event\\
	\glt ‘an unexciting event’
	\ex
	\gll  een [ ongeïnteresseerde ]  student\\
    an {} uninterested-\Agr{} {}   student\\
	\glt ‘an uninterested student’
	\z
\z

Thirdly, the participles in \REF{ex:18.26} can be modified by the intensifier\is{intensifiers}
\emph{heel} ‘very’ (see \ref{ex:18.29}), an intensifier\is{intensifiers} that can combine with
adjectives but not with verbs.\footnote{For example, it is impossible to say:
*\emph{Dat windt hem heel op} (that excites him much \Ptcl{},
\enquote*{that excites him a lot}).}

\ea%29
    \label{ex:18.29}
	\ea
	\gll een    [heel  opwindende]  gebeurtenis\\
		 a        very    exciting-\Agr{}  event\\
	\glt ‘a very exciting event’
	\ex
	\gll  een    [heel  geïnteresseerde]  student\\
		a        very    interested-\Agr{}      student\\
	\glt ‘a very interested student’
	\z
\z

None of these adjectival properties apply to the participles in
(\ref{ex:18.24}--\ref{ex:18.25}). In \REF{ex:18.30}, this is exemplified for
\emph{twijfelend} in \REF{ex:18.24}:

\ea%30
    \label{ex:18.30}\ili{Dutch}
    \ea[*]{
	\gll een  nog    twijfelend-er-e                student\\
		an      even  doubting-\Cmpr-\Agr{}  student\\
    \glt ‘a student who is even more in doubt’}
    \ex[*]{
	\gll een  ontwijfelende        student\\
    an    un-doubting-\Agr{}  student\\}
    \ex[*]{
	\gll een  heel    twijfelende        student\\
    a          very    doubting-\Agr{}    student\\}
	\z
\z

On the basis of the above-mentioned contrasts it can be concluded that
participles can display verbal or adjectival grammatical behavior. When the
participle is adjectival, parasitic agreement\is{agreement!parasitic agreement} is attested: that is, the
inflection \emph{-e} (= schwa) on the participle can license the appearance of
\textsc{-e} (= schwa) on the adjectival degree modifier. When the participle is
verbal, however, parasitic agreement\is{agreement!parasitic agreement} is impossible: -\textsc{e} cannot appear
on the adjectival degree modifier despite the presence of an inflection on the
linearly adjacent participle. As a final illustration of this contrast,
consider also the following minimal pair:

\ea%31
    \label{ex:18.31}\ili{Dutch}
	\ea
	\gll een   [ hem    erg(*-\textsc{e})  opwindend-e ]    jurk\\
    a      {}  him    very(-\textsc{e})    exciting-\Agr{} {} dress\\
	\glt \enquote*{a dress that excites him a lot}
	\ex
	\gll een   [ erg(-\textsc{e})    opwindend-e ]    jurk\\
    a    {}  very(-\textsc{e})    exciting-\Agr{} {}  dress\\
	\glt \enquote*{a very exciting dress}
	\z
\z

In (\ref{ex:18.31}a), \emph{opwindend} is a verbal participle, while, in
(\ref{ex:18.31}b), it is an adjectival participle. parasitic agreement\is{agreement!parasitic agreement} is
possible in (\ref{ex:18.31}b), but not in (\ref{ex:18.31}a).

Although I have related the absence of parasitic agreement\is{agreement!parasitic agreement} to the verbal nature
of participles in \REF{ex:18.24}, \REF{ex:18.25} and (\ref{ex:18.31}a), the
question remains why the inflection \emph{-e} on the participle cannot spread
onto the degree modifier. Related to that question: if the participle in these
examples is verbal, how does that match with a clearly adjectival property,
namely the presence of adjectival inflection? In what follows (see
\Cref{sec:18.7}), I propose that the adjectival participle and the verbal
participle have a different underlying syntactic structure. To make things
concrete, the participle \emph{opwindend} in (\ref{ex:18.31}b) is an adjectival
word. Specifically, it has the syntactic representation in (\ref{ex:18.32}b).
The verbal participle \emph{opwindend} in (\ref{ex:18.31}a), on the contrary,
has a composite syntactic structure, consisting of a verbal part (\emph{hem
opwind-}) and an adjectival part (\emph{-end}); see (\ref{ex:18.32}a). It will
be argued that this difference in phrasal structure is at the basis of the
contrast between (\ref{ex:18.24}--\ref{ex:18.25}), on the one hand, and \REF{ex:18.26}, on
the other hand.\footnote{Also for \ili{German} it has been argued that participial
endings are homophonous between \enquote{completely verbal} and
\enquote{completely adjectival uses}, i.e., participles are not
\enquote{hybrids} with mixed properties, but switch around between clear-cut
categories. See, for example, \citet{Toman1986} for discussion.}

\ea%32
    \label{ex:18.32}
    \ea {}[\textsubscript{AP} [\textsubscript{VP} hem opwind-] -end]
	\ex {}[\textsubscript{A(P)} opwindend]
	\z
\z

\section{Parasitic agreement: Inflected intensifiers as attributive
adjectives?}\label{sec:18.6}

In the previous section it was shown that the phenomenon of parasitic agreement\is{agreement!parasitic agreement}
cannot be analyzed in terms of string-based leftward spreading of the overt
adjectival inflection \emph{-e}. A structure-based approach seems more
plausible. In this section, one implementation of such an approach will be
sketched and rejected.

Starting from the idea that the appearance of \textsc{-e} on an AP-internal
degree modifier is unusual, this structure-based approach hypothesizes that in
a construction like \emph{een erg-\textsc{e} dur-e auto}, the adjectival degree
word \emph{erg-}\textsc{e} is not located within the attributive adjectival
expression at all but rather behaves like an AP-external attributive AP that
somehow has scope over the gradable adjective that follows it, see
(\ref{ex:18.33}a). Under such an analysis, \emph{afgrijselijke dure} in
(\ref{ex:18.33}a) has the same structural analysis as \emph{mooie dure} in
(\ref{ex:18.33}b). Being in an attributive position, the adjectival degree word\is{degree words}
\emph{afgrijselijk} receives an adjectival inflection (here represented as
\textsc{-e}), just like the \enquote{normal} attributive adjective \emph{dure}.

\ea%33
    \label{ex:18.33}\ili{Dutch}
	\ea
	\gll  [\textsubscript{DP} een    [\textsubscript{NP}  afgrijselijk\textsc{-e}    [\textsubscript{NP}  dure              [\textsubscript{NP} fiets ]]]]\\
		{} a  {}                  horrible-\textsc{e}    {}        expensive-\Agr{}  {}    bike\\
	\glt \enquote*{a horribly expensive bike}
	\ex
	\gll  [\textsubscript{DP}  een    [\textsubscript{NP}  mooie        [\textsubscript{NP}  dure                [\textsubscript{NP}  fiets ]]]]\\
		{} a  {}     beautiful-\Agr{}  {}    expensive-\Agr{} {}   bike\\
	\glt \enquote*{a beautiful expensive bike}
	\z
\z

It can easily be shown that this approach towards augmentative \textsc{-e} does
not work. First of all, as shown in \REF{ex:18.34}, \textsc{-e} can also
appear on a degree word\is{degree words} that clearly forms a conjunct (and therefore a
constituent) together with the modified adjective:

\ea%34
    \label{ex:18.34}\ili{Dutch}\\
    \gll een [ niet alleen [\textsubscript{AP} afgrijselijk\textsc{e} dure ] maar ook [\textsubscript{AP} afgrijselijk\textsc{e} lelijke ]] fiets\\
    a {} not only {} horrible-\textsc{e} expensive {} but also {} horrible-\textsc{e} ugly {} bike\\
    \glt ‘a horribly expensive but also horribly ugly bike’
\z

Secondly, patterns like \REF{ex:18.35} are possible, in which a PP that is
selected by the adjective precedes the augmented degree word\is{degree words}.

\ea%35
    \label{ex:18.35}\ili{Dutch}\\
    \gll  een    daarvan  erg-\textsc{e}    afhankelijk-e    jongen\\
         a        that.on      very-\textsc{e}  dependent-\Agr{}    boy\\
    \glt \enquote*{a boy who is very dependent on that}
\z

If the augmented degree word\is{degree words} occupied a separate attributive position, as in
(\ref{ex:18.33}a), the PP-complement would have to be moved from within the
second attributive AP to a position preceding the first (supposedly)
attributive AP, as depicted in \REF{ex:18.36}.

\ea%36
    \label{ex:18.36}\ili{Dutch}\\
    \gll  [\textsubscript{DP}  een    [daarvan\textsubscript{i}  [\textsubscript{NP}  erg\textbf{\textsc{-e}}  [\textsubscript{NP}  [t\textsubscript{i}    afhankelijk\textbf{e}]  [\textsubscript{NP}  jongen]]]]]\\
      {}  a              that-on   {}         very-\textsc{e}      {}  {}    dependent-\Agr{}   {}   boy\\
    \glt \enquote*{a boy very much dependent on that}
\z

Such a displacement operation, however, is impossible, as shown by the
ill-formed example (\ref{ex:18.37}b), where the PP-complement \emph{daarvan} has
been moved from within the attributive AP headed by \emph{afhankelijke} (see
(\ref{ex:18.35})) to a position preceding the attributive AP
\emph{vriendelijke}.

\ea%37
    \label{ex:18.37}\ili{Dutch}
    \ea[]{
    \gll {}[\textsubscript{DP}  een    [\textsubscript{NP}  vriendelijk\textbf{e}  [\textsubscript{NP}  [daarvan    afhankelijk\textbf{e}]  [\textsubscript{NP} jongen]]]]\\
		{} a {}   friendly-\Agr{} {}  that-on      dependent-\Agr{}  {}    boy\\
    \glt \enquote*{a friendly boy who is very much dependent on that}}
    \ex[*]{[\textsubscript{DP}  een [ daarvan\textsubscript{i} [\textsubscript{NP} vriendelijk\textbf{e} [\textsubscript{NP} [t\textsubscript{i} afhankelijk\textbf{e}] [\textsubscript{NP} jongen]]]]]\\
    \enquote*{a friendly boy who is very much dependent on that}}
	\z
\z

Given the above-mentioned problems, I conclude that the phenomenon of parasitic
agreement cannot be explained in terms of an attributive adjectival analysis of
the augmented degree word\is{degree words}.

\section{Parasitic agreement as a Spec-Head relationship}\label{sec:18.7}

The paradigms in \eqref{ex:18.4} and \eqref{ex:18.5} made clear that
three features play a role in determining the appearance of overt adjectival
inflection on \ili{Dutch} attributive adjectives: $\pm$definite, $\pm$singular,
and $\pm$neuter. When the noun phrase has the feature specification [−definite,
+singular, +neuter], the attributive adjective is morphologically bare, which
was interpreted as the presence of a zero-affix on the attributive adjective.
In all other cases we find the inflectional affix \emph{-e}. I will take these
affixal manifestations to be spell-outs (externalizations) of the feature
complex that is associated with the adjective entering into an agreement
relationship -- concord -- with the noun phrase. If augmentative (i.e.,
parasitic) \textsc{‑e} is a manifestation of adjectival agreement, then the
question arises how this agreement can appear on the adjectival degree
modifier.

From the examples in \REF{ex:18.22} and \REF{ex:18.23} we may conclude that
appearance of parasitic agreement\is{agreement!parasitic agreement} is dependent on the abstract feature
constellation of the attributive adjective rather than on the overt
manifestation of this feature complex. That is, there are patterns in which
\emph{-e} is absent on the attributive adjective but nevertheless (optionally)
present on the degree modifier (represented here as \textsc{-e}). This suggests
that parasitism regards first and foremost the abstract feature constellations
that form the input to Spell-Out.

Besides the feature constellation of the attributive adjective, the structural
relationship between the attributive adjective and the degree modifier matters
for the appearance of parasitic agreement\is{agreement!parasitic agreement}. Specifically, I propose that
parasitic agreement is an instance of Spec--head agreement\is{agreement!Spec--head agreement}. I assume that the
adjectival intensifier\is{intensifiers} occupies the Spec-position of the lexical head A, which
means that the intensifier\is{intensifiers} is structurally close to the attributive gradable
adjective:\footnote{A reviewer raises the question as to whether \textsc{-e}
could simply be interpreted as phonological (meaningless) \enquote*{junk},
which is still available as an adverbial remnant of older varieties of
\ili{Dutch}. This remnant \emph{-e} is still available in fixed expressions
such as \emph{van verre} (from far-\emph{e}, \enquote*{from a distance}) and
\emph{nog} \emph{lange} \emph{niet} (yet long-\emph{e} not, \enquote*{not
yet}). That \textsc{-e}  in patterns such as \emph{een erg(\textsc{-e}) leuk-e
auto} (a very\textsc{-e} nice-e car, \enquote*{a very nice car}) is not simply
the appearance of a historical inflectional remnant but rather results from
contextually determined morphosyntax comes from the observation that this
phenomenon of parasitism is also attested in partitive genitive\is{genitive
case} constructions. For example, besides \emph{iets erg doms} (something very
stupid\emph{-s}) and \emph{iets vreselijk ingewikkelds} (something extremely
complicated\emph{-s}), one also comes across patterns such as \emph{iets
erg\textbf{s} dom\textbf{s}} and \emph{iets vreselijk\textbf{s}
ingewikkeld\textbf{s}}, where both the modifier and the adjective carry the
bound morpheme \emph{-s} (see \citealt{Royen1948}). Notice, by the way, that
\textsc{-e} never appears on the modifier in these structural environments:
\emph{iets erg(*-\textsc{e}) doms}, \emph{iets vreselijke(*\textsc{-e})
ingewikkelds.} The distribution of \emph{-s} on modifiers in partitive
genitive\is{genitive case} constructions needs further investigation. The bound
morpheme \emph{-s}, for example, never appears on the modifier \emph{heel}, as
in \emph{iets heel(*\textbf{-s}) mooi\textbf{s}} (see also
\citealt[423]{Broekhuis2013}). As shown in \REF{ex:18.17}, \emph{heel} cán
carry \textsc{-e}.}\textsuperscript{,}\footnote{Note that the structure in (\ref{ex:18.38}a) is
identical to the one in (\ref{ex:18.38}b). This structural identity is what we
find also in parasitic gap\is{parasitic gaps} constructions. That is, the overall structure of
\emph{Which book did you file without reading?} is similar to the structure of
\emph{Which book did you file without reading it?} The only difference regards
the (derivation of) the object position in the adjunct clause; i.e. pronoun
(\emph{it}) versus parasitic gap.}

\ea%38
    \label{ex:18.38}
    \ea {}[\textsubscript{AP} [\textsubscript{AP} erg]
    leuk\textsubscript{\tuple{1,G}}-\Agr{}]\footnote{Subscript 1 represents the
    external argument of \emph{leuk} and subscript G represents the lexical
property of being gradable; see \textcite{Corver1997a}.} (no parasitic agreement)
	\ex {}[\textsubscript{AP} [\textsubscript{AP} erg-\Agr{}] leuk\textsubscript{\tuple{1,G}}-\Agr{}]          (parasitic agreement)
	\z
\z

A reason for placing the degree modifier in a structurally close relationship
with the attributive adjective is the fact that the scalar/gradable property of
the adjective is a lexical property of the adjective \emph{leuk}, here
represented with the subscript G(radable). I assume that this lexical property
must be locally satisfied, meaning within the lexical projection AP. Empirical
support for the structural proximity of the adjectival intensifier\is{intensifiers} and the
gradable adjective comes, first of all, from complex attributive adjective
phrases containing multiple modifiers. As shown in \REF{ex:18.39}, the degree
word is always closest to the gradable adjective:

\ea%39
    \label{ex:18.39}\ili{Dutch}
    \ea[]{
    \gll een [ \{vermoedelijk / tijdelijk / gelukkig\} [ vreselijk goedkop-\textbf{e} ]] fiets\\
    a {} \hphantom{\{}presumably {} temporarily {} fortunately {} extremely cheap-\Agr{} {} bike\\
	\glt \enquote*{a presumably / temporarily / fortunately extremely cheap bike}}
    \ex[*]{een [vreselijk \{vermoedelijk / tijdelijk / gelukkig\} goedkop-e] fiets}
	\z
\z

Secondly, the PP-complement of a regular (i.e., non-deverbal) adjective like
\emph{blij} \enquote*{happy} cannot intervene between the gradable adjective
and the degree word\is{degree words}, neither in predicative APs nor in attributive
ones:\footnote{As opposed to the predicative AP in (\ref{ex:18.40}a), the
    attributive AP in (\ref{ex:18.40}b) does not permit the pattern in which the
    PP-complement follows the adjective. That is, the (inflected) adjective
    must be linearly adjacent to the noun. This restriction on the placement of
    PP within an attributive adjectival phrase has been attributed to a ban on
right recursion for (certain) phrases occurring on left branches. For
discussion, see among others \citet{Emonds1976}, \citet{Williams1981}, and
\citet{fofc}.}

\ea%40
    \label{ex:18.40}\ili{Dutch}
	\ea
	\gll [\textsubscript{AP} \tuple{Daarmee} erg \tuple{*daarmee} blij \tuple{daarmee} ] was Jan.\\
        {} \phantom{〈}that.with very {} happy {} {} was Jan\\
	\glt \enquote*{Jan was very happy with that.}
	\ex
	\gll    een [\textsubscript{AP} \tuple{daarmee} erg \tuple{*daarmee} blije \tuple{*daarmee} ] man\\
            a {} \phantom{〈}that.with very {} happy-\Agr{} {} {} man\\
	\glt \enquote*{a man who is very happy with that}
	\z
\z

As shown by the following examples, other types of modifiers can reasonably
well be separated from the adjective by an intervening PP-complement:

\ea%41
    \label{ex:18.41}\ili{Dutch}
	\ea
	\gll een [ \tuple{daarmee}  gelukkig    \tuple{\textsuperscript{?}daarmee}    erg    \tuple{*daarmee}    blije]        man\\
        a {} \phantom{〈}that.with      fortunately        {}                  very         {}             happy-\Agr{}    man\\
	\ex
	\gll een [ \tuple{daarmee}  vermoedelijk  \tuple{\textsuperscript{?}daarmee}  erg    \tuple{*daarmee}    blije]    man\\
        a {}         \phantom{〈}that.with      presumably       {}                     very      {}            happy-\Agr{}    man\\
	\ex
	\gll een [ \tuple{daarmee}  slechts    tijdelijk    \tuple{\textsuperscript{?}daarmee}  erg    \tuple{*daarmee}    blije]    man\\
        a   {}     \phantom{〈}that.with        only    temporarily      {}
        very         {}           happy-\Agr{}    man\\
	\z
\z

Having shown that there are good reasons for assuming that the adjectival
intensifier occupies a syntactic position that is structurally close to the
(attributive) adjective, let us next turn to the pattern in \REF{ex:18.18},
repeated here as \REF{ex:18.42}:

\ea%42
    \label{ex:18.42}\ili{Dutch}
    \ea[]{
	\gll  een  [\textsubscript{AxP}  heel  erg    dur\textbf{e}] fiets\\
        a {}                real  very  expensive-\Agr{}  bike\\
    \glt \enquote*{a really very expensive bike}}
    \ex[?]{een [\textsubscript{AxP} heel erg\textbf{-\textsc{e}} dur\textbf{e}] fiets}
	\ex[]{een [\textsubscript{AxP} hel-\textsc{e} erg-\textsc{e} dur\textbf{e}] fiets}
    \ex[*]{een [\textsubscript{AxP} hel-\textsc{e} erg dur\textbf{e}] fiets}
	\z
\z

Before giving an analysis of the (multiple) parasitic
agreement\is{agreement!parasitic agreement} phenomenon in
(\ref{ex:18.42}c), let me point out that the amplifier \emph{heel} can be
followed only by the amplifying degree word\is{degree words} \emph{erg}. Other
\isi{degree words} such as \emph{vreselijk} \enquote*{extremely},
\emph{ontzettend} \enquote*{terribly} etc.\ cannot occur in combination with
\emph{heel}, as is exemplified in \REF{ex:18.43}:

\ea%43
    \label{ex:18.43}\ili{Dutch}\\
    \gll \llap{*}een    [heel  vreselijk/ontzettend    dure]              fiets\\
        a            very    extremely/terribly  expensive-\Agr{}  bike\\
\z

From the possible cooccurrence of \emph{heel} and \emph{erg} I conclude that
they form a syntactic unit that acts as a modifier of the gradable adjective.
Schematically:

\ea%44
    \label{ex:18.44}
    een [\textsubscript{AP} [\textsubscript{AP} heel erg] dure] fiets
\z

The question, obviously, arises why \emph{erg} is the only amplifying degree
word that can be modified by \emph{heel}. Possibly, \emph{erg} can function as
a pure marker of upward scalarity. That is, it refers to a point on the implied
scale that is higher than the standard value, but it does not so much express
the size of the interval between the standard value and that higher point. In
this respect, \emph{erg} differs from amplifiers such as \emph{vreselijk}
\enquote*{extremely} and \emph{ontzettend} \enquote*{terribly} in
\REF{ex:18.43}, which express that the size of the interval between the
standard value and the higher degree is \enquote{really big}. It seems that the
amplifier \emph{heel} in \REF{ex:18.44} marks the (big) size of the interval
between the standard value and the higher point on the scale.

Let us return to the patterns in \REF{ex:18.42} and see how the
(im)possibility of parasitic \textsc{-e} can be accounted for. In
(\ref{ex:18.42}a), there is no parasitic agreement\is{agreement!parasitic agreement}. The attributive adjective is
the only element carrying adjectival inflection (\emph{-e}) as a result of
concord with the noun phrase. Specifically, \emph{-e} is an externalization of
the feature constellation [−definite, +singular, −neuter].

Consider next (\ref{ex:18.42}b), which for most people is acceptable but a
little deviant. In this example, parasitic inflection is overtly realized on
the head of the modifying AP:

\ea%45
    \label{ex:18.45}
    een [\textsubscript{AP} [\textsubscript{AP} heel erg-\textsc{e}] mooi\textsubscript{\tuple{1,G}}-e] auto
\z

In (\ref{ex:18.42}c), the amplifying adjective \emph{heel} carries parasitic
\textsc{-e} as a result of the Spec--head agreement\is{agreement!Spec--head agreement} relationship with
\emph{erg-}\textsc{e}. Thus, \emph{hel-}\textsc{e} carries the attributive
adjectival inflection by transitivity; that is, via \emph{erg-}\textsc{e},
which heads the AP in which the modifier \emph{heel} is
embedded.\footnote{Thus, the agreeing AP headed by \emph{erge} is taken to be
structurally closer to the modified noun than is the modifier \emph{heel},
which is embedded within the agreeing attributive AP. As a reviewer points out,
one might want to adopt a \isi{bare phrase structure} approach here. Under such an
approach, the distribution of \textsc{-e} in (\ref{ex:18.42}c) can be accounted
for as follows: The label of the modifying phrase as a whole would be
\emph{erg-}\textsc{e} itself, with \emph{erg-}\textsc{e}, arguably, in the
right configuration for agreement with \emph{mooie}, and \emph{hel-}\textsc{e}
in the right configuration for agreement with \emph{erg-}\textsc{e}. Pattern
(\ref{ex:18.42}d) is ruled out because \emph{hel-}\textsc{e} is embedded too
deeply in (the phrase labeled) \emph{erg} to be available for licensing by
\emph{dure}.}

The ill-formedness of (\ref{ex:18.42}d) follows straightforwardly: \emph{heel}
can never be augmented with \textsc{-e} since it does not enter into a
Spec-head relationship with the inflected attributive adjective. Thus,
parasitic agreement between the \enquote{host} — the carrier of \enquote{real}
agreement – and the \enquote{parasite} – the carrier of parasitic agreement\is{agreement!parasitic agreement} –
is only possible when the two stand in a structurally local relationship with
each other: the parasite must be the specifier of the host.

Keeping this locality restriction in mind, consider next the examples in
\REF{ex:18.31}, repeated here as \REF{ex:18.46}:

\ea%46
    \label{ex:18.46}\ili{Dutch}
	\ea
	\gll een [ hem  erg(*-\textsc{e})  opwindend-e ]    jurk\\
    a   {}     him  very(-\textsc{e})    exciting-\Agr{} {} dress\\
	\ex
	\gll een [ erg(-\textsc{e})    opwindend-e ]    jurk \\
    a   {}   very(-\textsc{e})    exciting-\Agr{} {} dress\\
	\z
\z

Recall that it was argued that the present participle \emph{opwindend} in
(\ref{ex:18.46}a) has a different categorial make-up from the one in
(\ref{ex:18.46}b). Specifically, \emph{opwindend} in (\ref{ex:18.46}b) was
analyzed as an adjectival element: \emph{[\textsubscript{A} opwindend]};
\emph{opwindend} in (\ref{ex:18.46}a), on the contrary, was claimed to have a
composite syntactic structure, consisting of a verbal part (\emph{hem opwind-})
and an adjectival part (the participial ending \emph{-end}); see
(\ref{ex:18.32}a). As shown in (\ref{ex:18.47}b), \emph{erg} is in a Spec-head
relationship with the inflected adjective \emph{opwindende}. Consequently,
\emph{erg} can display parasitic agreement\is{agreement!parasitic agreement}: \emph{erg-}\textsc{e}. In
(\ref{ex:18.47}a), however, the degree modifier \emph{erg} is part of the verbal
layer and does not stand in a Spec-head relationship with the inflected
adjectival part, viz., \emph{‑ende}. Since the degree word\is{degree words}
does not stand in a local Spec-head relation with the inflected participial
ending \emph{-ende}, it is not able to display parasitic
agreement\is{agreement!parasitic agreement} morphology.

\ea%47
    \label{ex:18.47}
	\ea een [\textsubscript{AP} [\textsubscript{VP} hem erg(*-\textsc{e}) opwind-] -end-e] jurk
	\ex een [\textsubscript{A(P)} erg(\textsc{-e}) [\textsubscript{A} opwindende]] jurk
	\z
\z

From the minimal pair in \REF{ex:18.46} and the structure in \REF{ex:18.47}
it can be concluded that it is hierarchical structure rather than linear order
that matters for the licensing of parasitically agreeing (adjectival) degree
words.

The relevance of hierarchical structure for the appearance of parasitic
agreement is also clear from a number of other adjectives that turn out to be
structurally ambiguous. The adjectives I have in mind are the deverbal
adjectives in \REF{ex:18.48}. The characterization \enquote*{deverbal} comes
from two observations: firstly, some of these adjectives display
(past/passive-)participial morphology and as such are formally similar to
verbal forms (e.g., \emph{\textbf{ge}stel\textbf{d}},
\emph{verknoch\textbf{t}}). Secondly, some of these adjectives are
derivationally related to a verb. For example, \emph{afhankelijk (van)}
\enquote*{dependent (on)} is clearly related to the verb \emph{afhangen (van)}
\enquote*{to depend (on)}.\newpage

\ea%48
    \label{ex:18.48}\ili{Dutch}
	\ea
	\gll een [ \tuple{daarvan} erg \tuple{daarvan} afhankelijke ] man\\
    a {} \phantom{〈}that.on very {} dependent-\Agr{} {} man\\
	\glt \enquote*{a man who is very dependent on that}
	\ex
	\gll een [ \tuple{daarop} erg \tuple{daarop} gestelde ] man\\
    a {} \phantom{〈}that.on very {} keen-\Agr{} {} man\\
	\glt \enquote*{a man who is very keen on that}
	\ex
	\gll een [ \tuple{daaraan} erg \tuple{daaraan} verkochte ] man\\
    a {} \phantom{〈}that.to very {} devoted-\Agr{} {} man\\
	\glt \enquote*{a man who is very devoted to that}
	\z
\z

As shown in \REF{ex:18.48}, the PP-complement can appear either at the left
periphery of the adjectival projection or in between the degree modifier and
the attributive adjective. Especially the latter syntactic position is
remarkable, since, as was shown in \REF{ex:18.40}, the PP-complement cannot
occur in between the degree word\is{degree words} \emph{erg} and an attributive adjective, when
the latter is a \enquote{regular} (i.e., non-deverbal) adjective. This
asymmetry between the patterns in \REF{ex:18.48} and those in \REF{ex:18.40}
suggests that the deverbal adjectives in \REF{ex:18.48} have, or can have, an
underlying structure which differs from that of \enquote{regular} adjectives
such as \emph{blij} \enquote*{happy} and \emph{trots} \enquote*{proud}. I
propose that, analogously to the structural ambiguity of the form
\emph{opwindend} in \REF{ex:18.47}, the deverbal adjectives in \REF{ex:18.48}
can have two different structural representations, namely an adjectival one
(\ref{ex:18.49}a) and a deverbal one (\ref{ex:18.49}b):

\ea%49
    \label{ex:18.49}
	\ea
	\gll een [\textsubscript{AP} daarvan [\textsubscript{A} afhankelijke ]] man\\
        a {} that.on {} dependent-\Agr{} {} man\\
	\ex een [\textsubscript{AP} [\textsubscript{VP} daarvan afhang-] -elijke] man
	\z
\z

An elaborate motivation for this structural distinction falls beyond the scope
of the present paper. Let me nevertheless give one argument that supports the
ambiguous status of \emph{afhankelijk}, namely its possible co-occurrence with
two types of modifiers: \emph{heel} \enquote*{very}, which typically modifies
(gradable) adjectives, and \emph{voldoende} \enquote*{sufficiently}, which
typically modifies verbs (see also \citealt{Broekhuis2013}). Let me start with
\emph{heel}.

As shown in (\ref{ex:18.50}a,b), \emph{heel} only occurs as a modifier of
(gradable) adjectives and never modifies verbs that can combine with degree
modifiers (e.g., \emph{erg}). The fact that \emph{heel} can modify
\emph{afhankelijke}, as in (\ref{ex:18.50}c), suggests that \emph{afhankelijke}
behaves like a non-deverbal adjective in that case. Note in passing that the
PP-complement \emph{daarvan} can only occur at the left periphery of the
adjectival phrase and not in a position in between the degree word\is{degree words} and the
adjective. This distributional behavior of the PP-complement is completely in
line with that of PP-complements selected by a \enquote{regular} (i.e.
non-deverbal) adjective like \emph{blij} \enquote*{happy}; compare with
\REF{ex:18.40}.

\ea%50
    \label{ex:18.50}\ili{Dutch}
	\ea
	\gll een [ daarmee heel blije ] man\\
    a {} that.with very happy-\Agr{} {} man\\
	\glt \enquote*{a man who is very happy with that}
	\ex
    \gll Jan verheugde zich {erg / *heel} op haar komst.\\
        Jan looked.forward \Refl{} very to her arrival\\
	\glt \enquote*{Jan very much looked forward to her arrival.}
	\ex
	\gll een [ \tuple{daarvan} heel \tuple{*daarvan} afhankelijke ] man\\
        a {} that.on very {} dependent-\Agr{} {} man\\
	\glt \enquote*{a man who is very dependent on that}
	\z
\z

Consider next the modifier \emph{voldoende} \enquote*{sufficiently}. As shown
in (\ref{ex:18.51}a), combining \emph{voldoende} with a regular adjective like
\emph{trots} \enquote*{proud} yields a pattern which is quite marked.
Combination with a (gradable) verb is completely natural; see (\ref{ex:18.51}b).
As illustrated in (\ref{ex:18.51}c), \emph{voldoende} can easily combine with
the adjective \emph{afhankelijk}, which is expected if \emph{afhankelijk} can
have a \enquote{verbal flavor}. Note in passing that, under this verbal
behavior of \emph{afhankelijk}, the possible placement of the PP-complement in
between the degree word\is{degree words} and the adjective is entirely expected. As shown in
(\ref{ex:18.51}b), the PP-complement can also be placed in between the degree
word \emph{voldoende} and the gradable verb.

\ea%51
    \label{ex:18.51}\ili{Dutch}
    \ea[]{
    \gll \llap{??}Jan is [ voldoende trots op haar ].\\
        Jan is {} sufficiently proud of her\\}
    \ex[]{
	\gll Jan heeft zich \tuple{daarop} voldoende \tuple{daarop} verheugd.\\
        Jan has \Refl{} that.on sufficiently {} looked.forward\\
    \glt \enquote*{Jan has looked forward to that sufficiently.}}
    \ex[]{
	\gll Jan is [ \tuple{daarvan} voldoende \tuple{daarvan} afhankelijk \tuple{daarvan} ]\\
    Jan is {} that.on sufficiently {} dependent {}\\
    \glt \enquote*{Jan is sufficiently dependent on that.}}
	\z
\z\newpage

If I am right in saying that \emph{heel} acts as a modifier of an
\emph{adjectival} projection and \emph{voldoende} as a modifier of a
\emph{verbal} projection, then the adjectival structures in (\ref{ex:18.50}c)
and (\ref{ex:18.51}c) can be represented as (\ref{ex:18.52}a) and
(\ref{ex:18.52}b), respectively:\footnote{As indicated, I assume that the
PP-complement has been moved from a postadjectival position to the left
periphery of the AP.}

\ea%52
    \label{ex:18.52}\ili{Dutch}
	\ea
	\gll een [\textsubscript{AP} daarvan\textsubscript{i} [\textsubscript{AP}
    heel [\textsubscript{A$'$} afhankelijke t\textsubscript{i} ]]] man\\
    a {} that.on {} very {} dependent-\Agr{} {} {} man\\
	\ex een [\textsubscript{AP} [\textsubscript{VP} \tuple{daarvan} voldoende \tuple{daarvan} afhang-] -elijke] man
	\z
\z

Having shown that a deverbal adjective like \emph{afhankelijk} has an ambiguous
status, let us return to the phenomenon of parasitic agreement\is{agreement!parasitic agreement}. Consider,
specifically, the following contrast:

\ea%53
    \label{ex:18.53}\ili{Dutch}
	\ea
	\gll een [ erg(*-\textbf{\textsc{e}})    daarvan    afhankelijk\textbf{e}]      jongen        (MOD-\textsc{e} PP A-e)\\
    a   {}     very-\textsc{e}        that-on    dependent-\Agr{}    boy\\
	\glt \enquote*{a boy who is very dependent on that}
	\ex een [ daarvan erg(-\textbf{\textsc{e}}) afhankelijk\textbf{e}] jongen                  (PP MOD-\textsc{e} A-e)
	\z
\z

(53a) shows that parasitic agreement\is{agreement!parasitic agreement} is blocked when the PP-complement
\emph{daarvan} intervenes between the degree modifier \emph{erg} and the
attributive adjective \emph{afhankelijke}. As indicated by (\ref{ex:18.53}b),
parasitic agreement is possible when the PP-complement is at the left periphery
of the adjectival projection and, consequently, does not intervene between the
degree word and the attributive adjective. One might interpret this contrast as
support for a linear approach towards parasitic agreement\is{agreement!parasitic agreement} (see
\Cref{sec:18.5}). That is, the inflected attributive adjective and the
adjectival degree word\is{degree words} must be linearly adjacent for inflection to spread onto
the degree word\is{degree words}. As I have argued in \Cref{sec:18.5}, however, there are
good reasons for rejecting such a string-based approach to parasitic agreement\is{agreement!parasitic agreement}.
A structure-dependent\is{structure-dependence} account is preferred. Analogously to my account of the
contrast between (\ref{ex:18.46}a) and (\ref{ex:18.46}b), I propose that the
adjectival expressions in (\ref{ex:18.53}a) and (\ref{ex:18.53}b) have different
internal structures. Specifically, (\ref{ex:18.53}a) has the structure in
(\ref{ex:18.54}a), and (\ref{ex:18.53}b) the one in (\ref{ex:18.54}b).

\ea%54
    \label{ex:18.54}\ili{Dutch}
	\ea
	\gll een [\textsubscript{AP} [\textsubscript{VP} erg(*-\textsc{e}) daarvan afhang-] -elijke] man\\ a {} {} very-\textsc{e} that.on depend- -ent-\Agr{} man\\
	\ex
	\gll een [\textsubscript{AP} daarvan\textsubscript{i} [\textsubscript{AP} erg(-\textsc{e}) [\textsubscript{A$'$} afhankelijke t\textsubscript{i}]]] man\\
        a {} that.on {} very-\textsc{e} {} dependent-\Agr{} man\\
	\z
\z

In (\ref{ex:18.54}a), the deverbal adjective \emph{afhankelijk} has a composite
structure consisting of a verbal part, viz., the VP \emph{erg(*-\textsc{e})
daarvan afhang-}, and an adjectival part, viz., the adjectival suffix plus the
adjectival inflection: \emph{elijk-e}. Since \emph{erg} is contained within the
(AP-internal) verbal domain, it does not enter into a Spec-head relationship
with the adjectival inflection associated with \emph{-elijke}. Consequently,
appearance of –\textsc{e} on the degree word\is{degree words} will not be licensed. In
(\ref{ex:18.54}b), on the contrary, licensing of –\textsc{e} is possible. Here
\emph{afhankelijke} is a non-composite adjective (just like \emph{trots}
‘proud’, for example) which has the degree word\is{degree words} erg(-\textsc{e)} in its
specifier position. In other words, we have the right structural configuration
for parasitic inflection to appear on the adjectival degree word\is{degree words}.

\section{Parasitic \textsc{-e} as a marker of expressive
emphasis}\label{sec:18.8}

So far I have examined the phenomenon of parasitic agreement\is{agreement!parasitic agreement} from the
perspective of syntax. I argued that the adjectival degree word\is{degree words} can be
augmented with \textsc{-e} (schwa) if it stands in a Spec-head relationship
with an attributive adjective carrying a feature constellation that
externalizes as \emph{-e} (schwa). The question, obviously, arises why
\textsc{-e} should appear, since the \textsc{e-}less pattern is also
well-formed. So what information is it that \textsc{-e} encodes and
contributes? I tentatively propose that \textsc{-e} is a marker of (expressive)
emphasis\is{expressive emphasis}. It adds expressive force to the amplifying meaning of the adjectival
degree word. Expressive emphasis\is{expressive emphasis} is obtained by duplication of information in
syntax – namely, duplication of agreement information via Spec--head agreement\is{agreement!Spec--head agreement} –
and multiple Spell-out\is{multiple spell-out} (externalization) at the Syntax-Sensorimotor interface.
An adjectival affix that normally remains silent when the adjectival host
fulfills an adverbial function, as in \emph{een erg-${\varnothing}$ mooie auto}
(a very beautiful-\Agr{} car), externalizes as \textsc{-e} in order to make the
intensified meaning expressed by the adjectival degree word\is{degree words} more
prominent/salient at the \emph{sound} surface. In other words, adding
expressive force or prominence should be interpreted here as a property of
externalization.

At this point, it may be useful to point out that this expressive-emphatic use
of \textsc{-e} (i.e. schwa) is also found on certain \ili{Dutch} pronouns (see
e.g.\ \citealt[237--238]{HaeserynEtAl1997}; \citealt{Hoeksema2000,Zwart2001}).
This is exemplified in \REF{ex:18.55}:

\begin{exe}
    \ex\label{ex:18.55}
    \begin{tabularx}{.40\textwidth}[t]{llll}
        a. & \emph{ik} & a$'$. & \emph{ikke}\\
           & I & & I-\emph{e}\\
        b. & \emph{dat} & b$'$. & \emph{datte}\\
           & that & & that-\emph{e}\\
        c. & \emph{dit} & c$'$. & \emph{ditte}\\
           & this & & this-\emph{e}\\
        d. & \emph{what?} & d$'$. & \emph{watte?}\\
           & what & & what-\emph{e}\\
    \end{tabularx}
\end{exe}

As noted in \citet{Zwart2001}, an augmented form like \emph{ikke} can be
interpreted as standing in a contrastive relationship with an alternative
individual, as in (\ref{ex:18.56}a), or as a highly intensified/emphatic
form (i.e., intensity accent), as in (\ref{ex:18.56}b).

\ea%56
    \label{ex:18.56}\ili{Dutch}
	\ea
	\gll Jij   krijgt    geen  ijsje            maar  ik(-e)    wel!\\
		you  get        no      ice-cream  but      I(-\emph{e})      \Ptcl{}\textsubscript{positive}\\
	\glt \enquote*{You won}t get an ice cream, but I will!'
	\ex
	\gll \textbf{A:} Wie    wil        er        een    ijsje?            \textbf{B:}  Ik(-e)!\\
		{}  who    wants    there  an    ice-cream    {}  I(-\emph{e})\\
	\glt \textbf{A:} \enquote*{Who would like to have an ice cream?} \textbf{B:} \enquote*{Me!}
	\z
\z

An in-depth analysis of these augmented pronouns falls beyond the scope of this
article. In the spirit of my analysis of \textsc{-e} on adjectival degree
words, one might propose that \emph{-e} in \REF{ex:18.55} is licensed by the
presence of a functional element within the structure of the pronoun. In line
with \citet{DechWil2002}, for example, one might take pronouns to
have the layered structure [\textsubscript{DP} D [\textsubscript{φP} φ
[\textsubscript{NP} N]]], where \emph{ik} is the realization of \emph{φ(P)},
the locus of person and number features, and \emph{-e} an affixal realization
of D, which possibly gets inherited by (i.e. copied onto) φ(P). Schematically:
[\textsubscript{DP} D [\textsubscript{φP} φ (= ik)+D (= -e) [\textsubscript{NP}
N\textsubscript{${\varnothing}$}]]].\footnote{In certain varieties of
    \ili{Dutch}, the affixal article \emph{-e} \enquote*{the} is also found on
    certain nouns. Take, for example, the following examples from Oldambt
    \ili{Dutch} \citep[101]{Schuringa1923}.

\begin{exe}
    \exi{(i)} Oldambt Dutch
    \begin{xlist}
    \ex
	\gll noar kerk-e\\
		to church-\emph{e}\\
	\glt \enquote*{to church}
	\ex
	\gll  Lamp-e wil nait bran'n.\\
		lamp-\emph{e} will not light \\
	\glt \enquote*{The lamp won't light.}
    \end{xlist}
\end{exe}} It goes without saying that this structural analysis of
expressive-emphatic schwa in pronominal phrases needs further investigation.

Summarizing, I have argued that \textsc{-e} adds emphasis to the adjectival
degree word (the intensifier) that modifies the      gradable adjective. The
emphatic marker \textsc{-e} is, actually, an adjectival inflection that is
licensed under Spec--head agreement\is{agreement!Spec--head agreement} with the inflected (\emph{-e}) attributive
adjective. Thus, syntax (i.e., the structural Spec-head relation) provides the
right context for parasitic agreement\is{agreement!parasitic agreement}, and externalization of that structure
yields a pattern featuring \textsc{-e}.

I close this section with a brief discussion of a phenomenon that seems
unexpected under the approach towards parasitic agreement\is{agreement!parasitic agreement} taken so far. It
turns out that there are patterns in which \textsc{-e} appears on an
intensifier, even though there is no gradable adjective present, which carries
the inflection \emph{–e}. Before turning to those patterns, recall that
\textsc{-e} does not appear on the degree word\is{degree words} when the latter modifies an
attributive adjective carrying the feature constellation [−definite, +singular,
+neuter], as in (\ref{ex:18.8}a), repeated here as (\ref{ex:18.57}a). Nor does
\textsc{-e} appear when the adjective is used predicatively, as in the
copula\is{copulas}
construction in \eqref{ex:18.11}, repeated here as
(\ref{ex:18.57}b):\footnote{Similar patterns can be found in Frisian.
    \citet{Verdenius1939}, for example, gives the following sentences:

\begin{exe}
    \exi{(i)}\ili{Frisian}\\
	\gll 't is al skandalig(e) let\\
		it is already scandalous(\textsc{-e}) late\\
	\glt \enquote*{It is already very late!}
    \exi{(ii)}\ili{Frisian}\\
	\gll Hy kaem skandalig(e) let\\
		he came scandalous(\textsc{-e}) late\\
	\glt \enquote*{He arrived terribly late!}
\end{exe}\label{fn:18.16}}

\ea%57
    \label{ex:18.57}\ili{Dutch}
	\ea
	\gll  een    [erg(*\textbf{\textsc{-e}})    leuk]  huis\\
		a          very(\textsc{-e})    nice    house\\
	\ex
	\gll Deze  auto    is    erg(*\textsc{-e})  leuk.\\
		this    car      is    very(\textsc{-e})    nice\\
	\glt \enquote*{This car is really nice.}
	\z
\z\newpage

Consider now the adjectival expressions in the following examples:

\ea%58
    \label{ex:18.58}\ili{Dutch}
	\ea
	\gll  Jan  heeft    [een    [\textsubscript{AP}  verdomd(\textsc{-e})  leuk ]    huis!\\
		Jan    has        a         {}       damned-\textsc{e}      nice {}   house\\
	\glt \enquote*{Jan has a really nice house.}
	\ex
	\gll  Jan  heeft  [een    [\textsubscript{AP}  verrekt(\textsc{-e})  leuk ]  huis!\\
    Jan    has      a      {}          damned-\textsc{e}    nice {} house\\
	\glt \enquote*{Jan has a really nice house.}
	\z
\z

\ea%59
    \label{ex:18.59}\ili{Dutch}
	\ea
	\gll Deze    auto    is    [verdomd(\textsc{-e})  leuk].\\
		this      car      is    damned-\textsc{e}        nice\\
	\glt \enquote*{This car is really nice!}
	\ex
	\gll Deze    auto    is    [verrekt(\textsc{-e})    leuk]\\
		this      car      is    damned-\textsc{e}      nice\\
	\glt \enquote*{This car is really nice!}
	\z
\z

What is remarkable about these examples is that \textsc{-e} appears on an
intensifier (\emph{verdomd}, \emph{verrekt}) within an adjectival context that
normally does not license the appearance of \textsc{-e}; see \REF{ex:18.57}.
The question therefore arises as to what licenses the presence of \textsc{-e}
in these examples. And related to that question: what distinguishes
intensifiers such as \emph{verdomd} and \emph{verrekt} from \isi{intensifiers} such
as \emph{erg} \enquote*{very}, \emph{vreselijk} \enquote*{extremely},
\emph{ontzettend} \enquote*{terribly} etc.?\largerpage[-1]

I propose that the distinct behavior of the \isi{intensifiers} \emph{verdomd}
and \emph{verrekt} has to do with their status as \emph{expressive} modifiers
in the sense of \textcite{Potts2005}; see also \textcite{Potts2007} and
\citet{Morzycki2008}. As Potts points out, \ili{English} expressive modifiers such as
\emph{damn} and \emph{fucking}, as in \emph{the damn Republican} or \emph{the
fucking car}, do not express truth-conditional, restrictive meaning. In this
respect they behave differently from descriptive adjectives such as \emph{rich}
and \emph{beautiful}, which clearly contribute restrictive meaning to the noun
phrase: \emph{a rich Republican}, \emph{a beautiful car}. As Potts argues,
expressive modifiers typically convey the speaker's commentary on and attitude
towards what is being said. As such, the expressive modifier has a more
appositional or \enquote{additional} (i.e., non-restrictive) meaning, one which
is directly connected to the utterance situation itself. In a way, then,
descriptive modifiers such as \emph{rich} and \emph{beautiful} represent a
different dimension of meaning than do expressive modifiers such as \emph{damn}
and \emph{fucking}. I refer the reader to \textcite{Potts2005,Potts2007} for
further details.\footnote{The idea that descriptive meaning and expressive
    meaning represent different layers of interpretation raises the question as
    to whether this interpretative difference has a counterpart in syntax. That
    is, are descriptive modifiers integrated differently in syntactic structure
    than expressive modifiers? Building on a suggestion by Chris Kennedy,
    \citet{Morzycki2008}, for example, tentatively proposes that phrase
    structure may contain a specific layer — E(xpressive)P(hrase) – for
    encoding expressive information. Under such an analysis, \emph{the damn
    Republican} would look like: [\textsubscript{DP} the [\textsubscript{EP}
    damn [\textsubscript{E$'$} E [\textsubscript{NP} Republican]]]]. In this
    article, I won't consider this option and assume that \isi{intensifiers}
such as \emph{verdomd} and \emph{verrekt} occupy the same position as
\isi{intensifiers} such as \emph{erg} and \emph{vreselijk}.}

Now what is it that allows expressive modifiers such as \emph{verdomd} and
\emph{verrekt} to be augmented with \textsc{-e}  in spite of the absence of
overt adjectival inflection? One might hypothesize that the answer simply lies
in the expressive nature of words such as \emph{verdomd} and \emph{verrekt}. In
other words, it is an intrinsic property (say, their expressive semantics) of
these lexical items that permits augmentation with \textsc{-e}. Although
expressiveness obviously matters for the appearance of \textsc{-e} in
\REF{ex:18.58}-(59), it cannot be the whole story. Under such an analysis, one
would expect that these words can be augmented with \textsc{-e} when they occur
in an AP-external context. It turns out, though,  that \textsc{-e} is
impossible in such contexts. Consider, for example, the following utterances,
in which \emph{verrekt} and \emph{verdomd} occur as independent utterances and
clearly have an expressive meaning but cannot be augmented with
\textsc{–e.}\footnote{\citet{Verdenius1939} observes the same for Frisian.
    Recall from \cref{fn:18.16} that the intensifier\is{intensifiers} \emph{skandalig}
    (scandalously, \enquote*{terribly}) can be augmented with \textsc{-e} when
    it is contained within an AP. The appearance of \textsc{-e} is blocked,
    however, when \emph{skandalig} acts as a modifier of a verb. For example:

\begin{exe}
    \exi{(i)}\ili{Frisian}\\
    \gll Hy liicht skandalig(*\textsc{-e})\\
    	 he lies scandalous(\textsc{-e})\\
    \glt \enquote*{He lies terribly!}
\end{exe}}

\ea%60
    \label{ex:18.60}\ili{Dutch}
	\ea
	\gll Verrekt(*\textsc{–e})!  Je      hebt    gelijk!\\
		damned            you  have  right\\
	\glt ‘Gosh! You are right!’
	\ex
	\gll Verdomd(*\textsc{–e})!  Je      hebt    gelijk!\\
		damned                you  have  right\\
	\glt ‘Gosh! You are right!’
	\z
\z

The contrast between \REF{ex:18.58}-(59), on the one hand, and
\REF{ex:18.60}, on the other hand, suggests that some property of the gradable
adjective plays a role in licensing the appearance of  \textsc{–e} on the
expressive intensifier\is{intensifiers}. In view of what we have seen before, it does not seem
implausible to claim that this property is the Spec--head agreement\is{agreement!Spec--head agreement} relationship
between the gradable adjective and the degree word\is{degree words}. This would mean that, even
if the adjective does not carry any overt inflection (i.e., \emph{-e}), the
adjective can still enter into an agreement relationship with the degree
modifier in its Spec-position. Under such an analysis, one would be forced to
say that morphologically bare adjectives do carry an inflectional morpheme, but
that this morpheme is silent; that is, it is a null suffix.

The idea that Spec--head agreement\is{agreement!Spec--head agreement} does not have to become manifest by means of
overt inflectional morphology but can remain hidden under the (sound) surface
as a result of zero-morphology makes it possible to extend the phenomenon of
parasitic agreement to the attributive \emph{erg leuk} in (\ref{ex:18.57}a) and
the predicative AP \emph{erg leuk} in (\ref{ex:18.57}b). That is, there can be
parasitic agreement between the degree modifier and the gradable adjective but
the agreement does not surface audibly/visibly as a result of zero-morphology
(represented as ${\varnothing}$) on both items. Schematically:

\ea%61
    \label{ex:18.61}\ili{Dutch}
	\ea
	\gll een [ erg-${\varnothing}$\textsubscript{[-def,+sg,+neut]} leuk-${\varnothing}$\textsubscript{[-def,+sg,+neut]} ] huis\\
    a   {}     very                        nice                         {}     house\\
	\ex
	\gll Deze  auto    is  [ erg-${\varnothing}$    leuk-${\varnothing}$ ].\\
        this    car      is {} very        nice {}\\
	\glt \enquote*{This car is really nice.}
	\z
\z

If we follow this line of analysis, \emph{verdomd leuk} in
\REF{ex:18.58}--\REF{ex:18.59} would have the structure in
(\ref{ex:18.62}a), and \emph{verdomde leuk} the one in (\ref{ex:18.62}b):

\ea%62
    \label{ex:18.62}
    \ea {}[verdomd-${\varnothing}$  leuk-${\varnothing}$ ]
	\ex {}[verdomd-\textsc{e}    leuk-${\varnothing}$ ]
	\z
\z

Thus, both patterns feature the \enquote{abstract} Spec--head agreement\is{agreement!Spec--head agreement}
relationship between the expressive intensifier\is{intensifiers} and the gradable adjective, but
the externalization of the agreement relationship is symmetric
(-${\varnothing}$ -${\varnothing}$) in (\ref{ex:18.62}a) but asymmetric
(\textsc{-e} -${\varnothing}$) in (\ref{ex:18.62}b). Possibly, the asymmetric
Spell-out of the agreement relationship is a formal manifestation of
expressivity on the side of the speaker. In a way, the formally asymmetric
manifestation of the Spec--head agreement\is{agreement!Spec--head agreement} relationship constitutes a
deviant/marked or \enquote{imperfect} externalization. As argued in
\textcite{Corver2013,Corver2016}, such deviations from regular linguistic
patterns have a high information/surprise value as a result of their
unexpectedness. By means of this unexpected linguistic symbol at the sound
surface, the speaker provides a cue/signature of his internal emotional
state.\footnote{Other examples of expressive/affective signatures at the sound
    surface arguably are the following:  First, the appearance of \emph{-e}
    (schwa) on attributively used monosyllabic adjectives in \ili{Afrikaans}.
    Under a neutral reading, these adjectives do not bear any overt
    inflectional morphology (as opposed to bisyllabic ones), which I take to be
    an instance of zero-morphology ($\varnothing$); e.g.\ \emph{n mooi konyn}
    (a beautiful rabbit). In their expressive/affective use, however, they
    become augmented with \emph{-e}: \emph{'n mooie konyn} (\enquote*{a really
    beautiful rabbit}). A second illustration might be the (optional)
    augmentation with \emph{-e} (schwa) of \ili{Dutch} superlative adjectives,
as in \emph{Jan reed 't hardste} (Jan drove the/it\textsubscript{neuter}
fastest-\emph{e}, \enquote*{Jan drove fastest}).}

\section{Conclusion}\label{sec:18.9}

The parasitic gap\is{parasitic gaps} phenomenon has made us familiar with the phenomenon of
parasitism in syntax, that is the phenomenon that the presence of a symbol of
type α in a syntactic representation is dependent (i.e., parasitic) on the
presence of another symbol of type α in that same representation. Research on
parasitic gaps led to an important conclusion: the appearance of the parasitic
gap is structure-dependent\is{structure-dependence}. Specifically, the parasitic gap
(\emph{e}\textsubscript{\Pg{}}) may not be linked to a real gap
(\emph{e}\textsubscript{\Rg{}}) that is in a structurally higher position. In
this article, I have tried to add another phenomenon to the list of linguistic
parasitism, viz. parasitic agreement\is{agreement!parasitic agreement}; that is, the appearance of an inflection
whose existence is dependent on the presence of a \enquote{real} inflection.
Specifically, an intensifying degree word\is{degree words} (optionally) carries an inflection
which is associated with the gradable adjective. Crucially, it was shown that
the appearance of the parasitic inflection depends on hierarchical structure
and not on sequential or linear structure. In other words, parasitic agreement\is{agreement!parasitic agreement},
just like the parasitic gap\is{parasitic gaps} phenomenon, is structure dependent. The structural
configuration that was claimed to be at the basis of parasitic agreement\is{agreement!parasitic agreement} is the
Spec-head relationship.

In short, rethinking the phenomenon of linguistic parasitism from the
perspective of agreement leads to the same conclusion as research on parasitism
from the perspective of gaps: Hierarchical structure matters!

\printchapterglossary{}

\section*{Acknowledgements}

I would like to thank a reviewer for very useful comments on an earlier version
of this article.

{\sloppy
\printbibliography[heading=subbibliography,notkeyword=this]
}

\end{document}
