\documentclass[output=paper]{langsci/langscibook}
\author{Mark C. Baker\affiliation{Rutgers University} and Nadezhda Vinokurova\affiliation{Institute for Humanities Research and Indigenous Studies of the North, Siberian Branch, Russian Academy of Sciences}}
\title{Rethinking structural case: Partitive case in Sakha}

% \chapterDOI{} %will be filled in at production

\abstract{The \ili{Sakha} language has a special \isi{partitive case} used only
on nonspecific direct objects in imperative\is{imperatives} sentences. This is neither a
canonical structural case, nor a canonical inherent case. We show that its
basic properties can be explained within a configurational case theory by
assuming that partitive\is{partitive case} is unmarked case assigned to any NP within the VP
complement of \emph{v}\textsubscript{\Imp}, a special \emph{v} head found only
in the scope of imperative\is{imperatives} (Jussive) heads and a few semantical similar items.
This theory is briefly contrasted with one in which partitive\is{partitive case} is assigned by
agreement with a special v, and one in which partitive\is{partitive case} is the feature V copied
onto a nearby NP.}


\begin{document}\glsresetall
\maketitle

\section{Introduction}

Within the generative program, Case theory has normally gotten started by
making a sharp distinction between so-called structural cases, like nominative\is{nominative case}
and accusative\is{accusative case}, and inherent or semantic
cases,\is{case!inherent case} like locative, ablative or
instrumental, syntactic theory being more integrally concerned with the
structural cases. However, it is not clear that this distinction is so
well-defined, or that the boundaries between the two phenomena have necessarily
been drawn in the right place.

As a case in point, consider the so called \isi{partitive case} in \ili{Sakha}, exponed by
the suffix -\emph{tA}. A relic of the Old Turkic locative case, in \ili{Sakha} this
is a very specialized case, used only on some objects of verbs in imperative\is{imperatives}
sentences, as in~\eqref{ex:12.1}.

\ea\label{ex:12.1}Sakha \parencite[421, 429]{StachowskiMenz1998}
	\ea
		\gll Kiliep-te  sie.\\
			bread-\Part{}  eat.\Imp{}\\
		\glt ‘Eat some bread.’ or ‘Eat some of the bread.’
	\ex
		\gll Kinige-te  atyylas.\\
			book-\Part{}  buy.\Imp{}\\
		\glt ‘Buy any book.’  (Not: \#‘Buy some of the book.’)
	\z
\z
This partitive\is{partitive case} is certainly not on the list of normal
structural cases, apparently having little in common with nominative\is{nominative case} and
accusative\is{accusative case}. On the contrary, it is used in a semantically
well-defined context (imperatives), where it expresses a kind of semantic
notion (an indefinite having narrow scope with respect to the
imperative\is{imperatives} operator). However, it is not a canonical
inherent\is{case!inherent case}
case either, in that it does not express the equivalent of a PP in English, nor
is there a particular thematic role associated with it. Syntactic structure
seems relevant to the partitive\is{partitive case}, in that it is found only on
direct objects, not on subjects or indirect objects. Sakha’s
partitive\is{partitive case} is thus rather far from the prototypes for both
structural case and inherent/semantic case.\is{case!inherent case} It could be a hint that this
traditional distinction needs to be rethought, and along with it the basic
principles of case assignment themselves.

\largerpage[2]
\begin{sloppypar}
In this short paper, we discuss how the major properties of \isi{partitive
case} in \ili{Sakha} can be analyzed within a theory in which much of case assignment
is configurational -- determined by an NP’s syntactic position with respect to
other grammatical elements -- not by agreement with designated functional heads\is{functional items}
(the structural case prototype) or by theta-role assignment from particular
lexical heads (the inherent/semantic/lexical\is{case!inherent case} case prototype). In doing this, we
extend our earlier theory of structural case in \ili{Sakha}
(\citealt{BakVin2010}, hereafter B\&V) to this very specialized case. More
specifically, we propose that there is a special functional head\is{functional items} in imperative\is{imperatives}
clauses that we call \emph{v}\textsubscript{\Imp}. This is a special flavor of
the v/Voice head that is licensed semantically in imperative\is{imperatives} sentences (and a
few others), and as such it is a phase\is{phases} head that triggers the spell out of its
VP complement. What is special about \emph{v}\textsubscript{\Imp} is that it
stipulates that any NP not otherwise marked for case within the spelled-out VP
gets a special unmarked case, namely partitive\is{partitive case}. On this analysis, partitive\is{partitive case} in
\ili{Sakha} finds a place alongside nominative\is{nominative case}, which is the unmarked case for
NPs inside a spelled out TP in many languages, and genitive\is{genitive case},
which is the unmarked case for NPs inside a spelled out DP in some languages.
This is similar to \citeauthor{Baker2015}'s (\citeyear[140--145]{Baker2015})
analysis of \isi{partitive case} in Finnish, except that partitive\is{partitive case} is only
assigned in the complement of this one particular \emph{v} head in \ili{Sakha},
not in the VP complement of any \emph{v} head, as in Finnish.
\end{sloppypar}

\section{Partitive case in Sakha in context}
\largerpage[2]

One telling reason for saying that partitive\is{partitive case} in \ili{Sakha} is a special kind
of structural case is that it participates in alternations. \ili{Sakha} is a
\gls{DOM}\is{differential object marking} language: definite or specific objects are marked with
\isi{accusative case}; nonspecific indefinite objects are unmarked for case
(morphologically indistinguishable from nominative\is{nominative case}; see
\citealt{Vinokurova2005}, B\&V).  Interestingly, both of these possibilities
can also be found in \isi{imperatives}, alongside the partitive\is{partitive case} option
in~\eqref{ex:12.1}, each with what seems to be its usual semantic value:

\ea\label{ex:12.2}Sakha
	\ea
		\gll Kilieb-i  sie.\\
			bread-\Acc{}  eat.\Imp{}\\
		\glt ‘Eat the bread.’
	\ex
		\gll Kiliep  sie.\\
			bread eat.\Imp{}\\
		\glt ‘Eat bread.’
	\z
\z
So \ili{Sakha} actually has a three-way rather than a two-way
\gls{DOM}\is{differential object marking} distinction in this limited
grammatical environment, with~(\ref{ex:12.1}a),~(\ref{ex:12.2}a),
and~(\ref{ex:12.2}b) all possible. ~(\ref{ex:12.2}a) is quite different
semantically from~(\ref{ex:12.1}a): in (\ref{ex:12.2}a) the object has
a definite or specific reading, whereas in~(\ref{ex:12.1}a) it has a
partitive\is{partitive case} or nonspecific indefinite reading. The bare object
in~(\ref{ex:12.2}b), however, is very close in meaning to the
partitive\is{partitive case} objects in (1a,b); it also has what is broadly
speaking a nonspecific indefinite meaning.\footnote{An anonymous reviewer asks
    how exactly a bare NP object like the one in~(\ref{ex:12.2}b) differs
    semantically or pragmatically from a partitive\is{partitive case} object
    like the one in~(\ref{ex:12.1}a), given that both have narrow-scope
    indefinite readings.  Unfortunately, we cannot give a fully helpful or
    insightful answer; it is hard to articulate a clear and consistent
    difference.  One possible hint is that~(\ref{ex:12.1}a) with
    \isi{partitive case} seems to imply that there should be some bread left
    over (perhaps so that the speaker can eat some too),
    whereas~(\ref{ex:12.2}b) allows the addressee to eat all the bread.}
    We return to this below.

Sakha also has explicit partitive\is{partitive case} constructions, which it shares with other
Turkic languages, including \ili{Turkish} (see
\citealt{Kornfilt1990,Kornfilt1996} for
detailed discussion of the \ili{Turkish} analogs). In these constructions, the
NP expressing the whole from which the part is taken bears ablative case,
\emph{not} \isi{partitive case}. If a nominal head expressing the part is
overt, as in~(\ref{ex:12.3}a), it bears a normal direct object
case -- accusative or (in \isi{imperatives} only) partitive\is{partitive case}. The nominal head of
this partitive\is{partitive case} construction can also be null, giving a kind of bare partitive\is{partitive case}
construction, in which it looks like the direct object itself has ablative
case. In the spirit of Kornfilt’s studies, we assume that this is a relatively
straightforward variant of the construction in (3a), which happens to have a
null head.

\ea\label{ex:12.3}Sakha
	\ea
		\gll Jablaka-ttan   ikki-ni / ikki-te sie. (*Jablaka-ta...)\\
			apple-\Abl{}      two-\Acc{} / two-\Part{}  eat.\Imp{} (*apple-\Part{})\\
		\glt ‘Eat two of the apples!’
	\ex
		\gll Jablaka-ttan sie!\\
			apple-\Abl{} eat.\Imp{}\\
		\glt ‘Eat some of the apple/apples.’
	\z
\z
Like~\eqref{ex:12.2} and unlike~\eqref{ex:12.1}, these expressions of the object are equally possible
in ordinary declarative sentences. Calling the –\emph{tA} case marker in~\eqref{ex:12.1}
“partitive” might now seem like a bit of a misnomer, since the case is not used
in explicit partitive\is{partitive case} constructions like~(\ref{ex:12.3}a), and since some examples with
partitive case do not naturally have a partitive\is{partitive case} translation (e.g.,~\ref{ex:12.1}b).
However, this is the term now used in \ili{Sakha} grammar studies, and the case does
express partitive\is{partitive case} meanings in some examples (e.g.,~\ref{ex:12.1}a); it also does have
similarities with the Finnish partitive\is{partitive case}. Therefore, we maintain this
terminology here.\footnote{An older term for this case, used for example by
Otto Boehtlingk in the mid 19th century, was “accusative
indefinite.”}

It is also worth noting that (as far as is known) the direct object of any
transitive verb in \ili{Sakha} can bear \isi{partitive case} if the following
conditions are met: if the clause is imperative\is{imperatives}, and the object
permits a nonspecific indefinite reading.  In this sense, \isi{partitive case}
is no less a structural case than overt accusative or bare
accusative\is{accusative case} is. The use of this case is limited
syntactically, but not lexically, in contrast with standard instances of
inherent case.\is{case!inherent case}

\section{Partitive case as case for NPs inside VP}

With these comparisons in mind, we now build our case that partitive\is{partitive case} is an
unmarked case assigned to NPs that stay inside VP in imperative\is{imperatives} clauses.

The possibility of~(\ref{ex:12.2}a) in particular tends to point away from
an alternative idea within the configurational case theory, according to which
what is special about \isi{imperatives} is that they have some special kind of covert
subject, one with distinctive grammatical features of some kind. One might
imagine a variant of a \isi{dependent case} theory \citep{Marantz1991} in which
an NP has \isi{partitive case} if and only if it is c-commanded in the local domain
by another NP that has these special features. But this alternative view makes
it rather mysterious why \isi{accusative case} on the object is also an option in
imperative clauses. B\&V argue in detail that \isi{accusative case} in \ili{Sakha} is
the result of the object being locally c-commanded by an ordinary NP subject.
It is far from clear, then, how c-command by the same subject could cause both
accusative case on the object in~(\ref{ex:12.2}a) and \isi{partitive case}
in~\eqref{ex:12.1}.

Another objection to a view in which partitive\is{partitive case} is a special \isi{dependent
case} is the fact that \isi{imperatives} in \ili{Sakha} can have normal overt
subjects as well as covert ones. Although these overt subjects have no obvious
special features, the object can still be partitive\is{partitive case}.~(\ref{ex:12.4}a) shows
this with an overt NP serving as the addressee, as is possible in all varieties
of \ili{English};~(\ref{ex:12.4}b) shows it with a kind of third person
imperative, where the addressee is exhorted to have a third person expressed as
the subject accomplish some act, as is possible in some idiolects of
\ili{English} \citep{Zanuttini2008}.

\ea\label{ex:12.4}Sakha
	\ea
		\gll Masha  salamaat-ta sie.\\
			Masha  porridge-\Part{}  eat.\Imp{}\\
		\glt ‘Masha (you) eat some porridge!’ (command addressed to Masha)
	\ex
		\gll Masha  salamaat-ta sie-tin.\\
			Masha  porridge-\Part{}  eat-\Imp{}.\Tsg.\Sbj{}\\
		\glt ‘Have Masha eat some porridge!’ (command addressed to someone other than Masha)
	\z
\z
We conclude, then, that Sakha’s partitive\is{partitive case} is not a specialized type of
dependent case.

The examples in~\eqref{ex:12.4} also suggest that it is only the direct
object that can be partitive\is{partitive case} in an imperative\is{imperatives}; overt subjects are nominative\is{nominative case},
as in other clauses.  This is true even if the agentive subject of the
imperative is an indefinite nominal, semantically compatible with partitive\is{partitive case}, as
shown in~\eqref{ex:12.5} (see also~\eqref{ex:12.12} below on the
nonagentive subjects of unaccusative\is{unaccusativity} verbs).

\ea\label{ex:12.5}Sakha\\
	\gll    Oqo-(\#to)  yllaa-tin!\\
		    child-(*\Part{})  sing-\Imp{}.\Tsg.\Sbj{}\\
	\glt    ‘Have a/any child sing!’
\z
Put in structural terms, it is only an NP inside VP (that is not otherwise case
marked, e.g. with dative\is{dative case}) that can be partitive\is{partitive case}. This fits our idea that
partitive is an unmarked case for NPs in a VP domain.

The idea that partitive\is{partitive case} is a case for NPs inside VP fits the observed facts in
another respect as well. The interpretative properties of partitive\is{partitive case} objects
suggest that they remain inside the VP, in that they get only weak indefinite
readings.  For example, in a negative imperative\is{imperatives}, the partitive\is{partitive case} object can only
be interpreted as an existential that takes narrow scope with respect to
negation (as well as with respect to the imperative\is{imperatives} operator itself).

\ea\label{ex:12.6}Sakha\\
	\gll Kiliep-te  sie-me.\\
		bread-\Part{} eat-\Imp{}.\Neg{}.\Ssg{}S\\
	\glt Only: ‘Do not eat any bread at all.’\\
		 {}[IMP [Neg [${\exists}$x bread (x) [you eat x]]]]\\
		(Not: ‘Make sure there is some bread that you don’t eat.’)
\z
This is quite different from a command with an accusative\is{accusative case} object, where the
object does have (the equivalent of) wide scope with respect to negation.

\ea\label{ex:12.7}Sakha\\
	\gll Kilieb-i  sie-me.\\
		bread-\Acc{}  eat-\Imp{}.\Neg{}.\Ssg{}S\\
	\glt ‘Do not eat that bread.’\\
		Bread (x) [IMP [Not [you eat x]]]\\
		(‘There might be other bread around which you do eat, but not \textit{that} bread.’)
\z
This fits well with the idea that NPs that shift out of VP and get strong
readings in accordance with \citegen{Diesing1992} \isi{mapping hypothesis} come into
the domain of the subject and are assigned dependent \isi{accusative case} in \ili{Sakha}.
In contrast, NPs that stay inside the VP and receive weak indefinite readings
get \isi{partitive case}.  This also explains the fact that proper names and nominals
with a demonstrative cannot be in \isi{partitive case} when used as the direct object
of an imperative\is{imperatives} verb in \ili{Sakha}:

\ea\label{ex:12.8}Sakha\\
    \gll    \llap{*}Sargy-ta / *bu kinige-te  bul.\\
            Sargy-\Part{} {} \hphantom{*}this book-\Part{}  find.\Imp{}\\
    \glt    ‘Find Sargy/this book!’
\z
These nominals are intrinsically definite, so they have to move out of VP and
receive accusative\is{accusative case}; they never remain in the VP-internal position where
partitive is assigned.

It is worth recalling in this connection that the reading of the partitive\is{partitive case}
object is very similar to the reading of the bare object (see~\ref{ex:12.1}a and~\ref{ex:12.2}b).
This is also seen by comparing~\eqref{ex:12.6} with~\eqref{ex:12.9}, where the object definitely stays
inside VP; the two naturally receive the same \ili{English} translation, because the
structures are the same in this regard.

\ea\label{ex:12.9}Sakha\\
	\gll    Kiliep  sie-me.\\
		    bread  eat-\Imp{}.\Neg{}.\Ssg{}S\\
	\glt    ‘Do not eat (any) bread.’\\
		    {}[IMP [Neg [${\exists}$x bread (x) [you eat x]]]]
\z
Although bare NPs and NPs with \isi{partitive case} are very similar in meaning,
there is a clear structural difference between them. Bare objects have to be
strictly left-adjacent to the verb in \ili{Sakha}, whereas partitive\is{partitive case} objects can be
separated from the verb by an adverb or resultative phrase, as seen in
(10).\footnote{\textcite{Kornfilt1990,Kornfilt1996} shows that bare
    ablative-partitives like~(\ref{ex:12.3}b) also must be strictly adjacent to the verb in
    \ili{Turkish}, like bare objects.  This confirms that the so-called ablative
partitives should have a different sort of analysis from objects with partitive\is{partitive case}
case in \ili{Sakha}.}

\ea\label{ex:12.10}Sakha
	\ea
		\gll    Kumaaqy-ta / *kumaaqy  xoruopka-qa ug-uma.\\
                paper-\Part{} {} \hphantom{*}paper  case-\Dat{}  put-\Neg.\Imp{}\\
		\glt    ‘Don’t put any paper(s) in the case!’
	\ex
		\gll    Kiliep-te / *kiliep  türgennik sie!\\
                bread-\Part{} {} \hphantom{*}bread  quickly  eat.\Imp{}\\
		\glt    ‘Eat some bread quickly!’
	\z
\z
For this and other reasons, \citet{Baker2014} argues in detail that bare
objects in \ili{Sakha} are the result of pseudo-incorporation applying between the
head of the direct object and the verb. This requires strict linear adjacency
in \ili{Sakha} (and in other languages in which the verb does not move to T,
according to Baker).  In contrast, partitive\is{partitive case} objects are not
pseudo-incorporated, and do not need to be next to the verb, either because a
lower resultative phrase intervenes (in~\ref{ex:12.10}a), or because the object has
undergone short scrambling within VP over a VP adverb, as in~(\ref{ex:12.10}b).  This then
gives an account of the three-way distinction among objects in \ili{Sakha}: objects
that undergo object shift out of VP are accusative\is{accusative case}; objects that are
pseudo-incorporated with the verb either do not undergo case marking at all\largerpage{}\
(because they are “hidden” inside the verb) or have their case feature deleted;
objects that stay in VP but do not incorporate get \isi{partitive case}.  These
structural distinctions correspond to semantic distinctions given Diesing’s
\isi{mapping hypothesis} and the special semantics that goes with
pseudo-incorporation (see \citealt{Dayal2011}).

The examples in~\eqref{ex:12.10} also show that having \isi{partitive case}
on the object in \ili{Sakha} is perfectly compatible with there being other material
inside the VP.  That is true for directional/resultative phrases like ‘in the
case’, which are lower than the object in syntactic structure. It is also true
for goal/recipient phrases which are higher than the object in syntactic
structure, as shown in~\eqref{ex:12.11} (see \citealt{BakVin2010} on higher
goal NPs with structural \isi{dative case} in \ili{Sakha}).\footnote{We thank an
anonymous reviewer for asking about (11) and pointing out its potential
theoretical significance.}~(\ref{ex:12.11}a) shows this for a goal
intrinsically selected by the verb ‘give’,~(\ref{ex:12.11}b) for a freely
added benefactive expression.

\ea\label{ex:12.11}Sakha
	\ea
		\gll At-tar-ga ok-kut-una bier-din-ner.\\
			horse-\Pl{}-\Dat{} hay-\Ssg{}P-\Part{} give-\Imp{}-\Tpl{}.\Sbj{}\\
		\glt ‘Have them give the horses some of your hay.’
	\ex
		\gll Miexe  kiliep-te atyylas.\\
			me.\Dat{}  bread-\Part{}  buy.\Imp{}\\
		\glt ‘Buy me some bread.’
	\z
\z
This is theoretically significant for distinguishing a view in which partitive\is{partitive case}
is unmarked case assigned when VP is spelled out from a Chomsky-style analysis
in which \isi{partitive case} is assigned to the object by a special \emph{v} found in
imperative clauses.  The goal phrases in~\eqref{ex:12.11} intervene structurally between v
and the theme, which should block \emph{v} from entering into Agree with the theme. If
partitive case assignment depended on Agree, it should be blocked in~\eqref{ex:12.11},
contrary to fact.  In contrast, our proposal that partitive\is{partitive case} is unmarked case
for any NP inside VP that is not already case marked correctly predicts that
partitive is possible in~\eqref{ex:12.11}, since this assignment rule does not depend in
any way on details about where the NP is relative to other VP-internal items.

Overall, then, it is precisely those NPs that are generated inside VP (objects
as opposed to subjects) and that stay inside VP (nonspecific indefinite objects
as opposed to specific/definite objects) that get \isi{partitive case} in
imperatives. Thus our core proposal that partitive\is{partitive case} is a case for NPs inside VP
that are not otherwise case marked fits the facts well.\largerpage[2]

This raises the question of what happens with the theme arguments of
unaccusative verbs.  Like direct objects, these are generated inside VP, under
standard assumptions.  Hence, one might expect that
unaccusative\is{unaccusativity} subjects could get \isi{partitive case}, in
contrast with unergative\is{unergativity} subjects, \eqref{ex:12.5}. In
fact, this is impossible in \ili{Sakha}, as shown in~\eqref{ex:12.12}.

\ea\label{ex:12.12}Sakha
    \ea[]{
		\gll    Morkuop-(*ta)  üün-nün!\\
			    carrot-(*\Part{})  grow-\Imp{}.\Tsg.\Sbj{}\\
        \glt    ‘Let some carrots grow.’}
    \ex[\#]{
		\gll    Oqo-to  yaldy-ba-tyn.\\
			    child-\Part{}  get.sick-\Neg{}-\Imp{}.\Tsg.\Sbj{}\\
		\glt    Not OK as ‘Don’t let any child get sick!’\\
                (OK as ‘Don’t let his child get sick’, with -\emph{to} = \Tsg.\Poss{})}
	\z
\z
This fact fits with our hypothesis as long as we assume that \ili{Sakha} has a
strong \glsunset{EPP}\gls{EPP} feature, such that some suitable NP must move to
SpecTP (or at least to Spec\emph{v}\textsubscript{\Imp}P; see \cref{fn:12.8}).
Since the theme is the only NP in these unaccusative\is{unaccusativity} structures, it must be the
one to move. This takes the theme out of VP, bleeding \isi{partitive case}
assignment, just as object shift out of VP does.  In contrast, unaccusative\is{unaccusativity}
subjects can get \isi{partitive case} in Finnish, because in that language EPP
properties are absent or can be satisfied in other ways (see
\citealt{Baker2015}: 142 and references cited there).\footnote{Unaccusative
    predicates also allow their subjects to have the bare ablative partitive\is{partitive case} in
    \ili{Turkish}, according to \textcite{Kornfilt1990,Kornfilt1996} -- another
difference between the two so-called partitive\is{partitive case} constructions.}

\section{The structure of imperative clauses}\largerpage[1]

The major remaining question, then, is how to relate the fact that NPs inside
VP get a special \isi{partitive case} in \isi{imperatives} only to the overall syntax of
imperative clauses. On the latter topic, we take as our starting point the
theory of the syntax of \isi{imperatives} in \citet{Zanuttini2008} and
\citet{ZanPakPor2012}, a theory with crosslinguistic aspirations which fits
well with the basic facts about \ili{Sakha}. On this view, imperative\is{imperatives} clauses have a
special Jussive head that is not present in other clause types. This head has
intrinsic interpretable second \isi{person features} that relate to the fact that
imperatives are enjoined on the addressee of the utterance in a special way.
The head is assumed to be high in the clausal structure, above TP and most of
the rest of the functional structure of the clause. In \ili{Sakha}, this fits with
the fact that the imperative\is{imperatives} operator in an example like~\eqref{ex:12.6} necessarily has
scope over negation:~\eqref{ex:12.6} means ‘you have the obligation not to eat bread’, not
‘you don’t have the obligation to eat bread’. Similarly, \ili{Sakha} has a special
future tense imperative\is{imperatives} seen in~\eqref{ex:12.13}; here imperative\is{imperatives} has scope over the future
tense.

\ea\label{ex:12.13}Sakha\\
	\gll Kinige-te  atyylah-aar.\\
        book-\Part{}  buy-\Fut{}.\Imp{}.\Ssg{}\\
	\glt ‘You have an obligation (now) to buy a book in the future.’\\
		(Not: ‘In the future, you will have an obligation to buy a book.’)
\z
Furthermore, according to \citet{Zanuttini2008}, if T in an imperative\is{imperatives} clause
has person agreement features of its own, it can license a subject distinct
from the addressee; this is what we find in examples like~(\ref{ex:12.4}b) in \ili{Sakha}.
However, T in imperative\is{imperatives} clauses can also lack a person agreement feature. In
that case, the Jussive head can itself agree with the subject, endowing it with
its intrinsic second person feature.\is{person features} In this way, a null second person pronoun
can be licensed in the subject position of \isi{imperatives} even in the absence of
rich agreement, as in examples like~\eqref{ex:12.1}, and a second person reading can be
imposed on a nominal that otherwise would not have one, as in~(\ref{ex:12.4}a). Overall,
then, Zanuttini’s theory of the syntax of \isi{imperatives} is a good fit for \ili{Sakha}.

But there is a significant problem when it comes to the licensing of partitive\is{partitive case}
case in \ili{Sakha} \isi{imperatives}, since the Jussive head is too high in the clause to
trigger this case on the direct object in any contending theory of case
assignment. Clearly Jussive should not be able to assign partitive\is{partitive case} to the
object under Agree, because the subject intervenes structurally between the
two. But essentially the same problem arises for our view that partitive\is{partitive case} is an
unmarked case assigned at Spell out. One could stipulate that Jussive is a
phase head, and that \isi{partitive case} is assigned to un-case-marked NPs inside
its spelled-out complement. But the complement of Jussive is (at least) TP,
which also includes the subject, and partitive\is{partitive case} is not possible on the subject
(see~\ref{ex:12.5} and~\ref{ex:12.12}). Moreover, Jussive embeds a TP that itself contains a normal
\emph{v}P structure. Since \emph{v} is a (hard) phase\is{phases} head in \ili{Sakha}, which spells out its VP
complement but does not provide an unmarked case for NPs inside that
complement, NPs inside VP that are not otherwise case-marked are forced to
undergo pseudo-incorporation, showing up as bare nominals. By the time the
derivation reaches the Jussive head, then, there should be no object NP visible
inside its complement to get partitive\is{partitive case}, VP already having been spelled out.
Therefore, Jussive could have no direct case marking effect on the VP-internal
object.

Therefore, we are led to propose that the structure of imperative\is{imperatives} clauses in
Sakha is a bit more complex. We suggest that \ili{Sakha} has a special flavor
of \emph{v}, called \emph{v}\textsubscript{\Imp}, which is licensed in the scope of
the Jussive head, as expressed in~\eqref{ex:12.14}.

\ea\label{ex:12.14}
\emph{v}\textsubscript{\Imp} is licensed only in the semantic scope of Jussive or a similar operator.
\z
Semantically \emph{v}\textsubscript{\Imp} introduces an agent and says that that agent
is obligated to perform the predicate expressed by its VP complement. Like
other vs, it triggers the spell out of its VP complement. However, this \emph{v} is
special in that it supplies \isi{partitive case} as an unmarked case for that
complement. This is stated explicitly in~\eqref{ex:12.15}.

\ea\label{ex:12.15}
Assign Partitive to an NP not marked for case in the domain spelled out by \emph{v}\textsubscript{\Imp}.
\z
Then the result follows that \isi{partitive case} is licensed on direct objects
that remain inside VP in \ili{Sakha}, but not on agentive subjects or on direct
objects that move out of VP to get a strong/definite/specific
reading.\footnote{One might think that positing \emph{v}\textsubscript{\Imp} in
    addition to Jussive would also make possible a view in which
    \emph{v}\textsubscript{\Imp} assigns partitive\is{partitive case} case to the direct object
    under Agree.  However, this view would find it difficult to explain why
    accusative\is{accusative case} objects are also possible in \isi{imperatives}
    (see~(\ref{ex:12.2}a)), since the normal accusative-assigning \emph{v}
    would not be present in \isi{imperatives}, by hypothesis. Our
    configurational account readily accommodates both: if the object stays
    inside VP, it gets unmarked \isi{partitive case} when VP is spelled out; if
    it moves out of VP, it gets \isi{dependent case} by being locally c-commanded by
    the subject (see Figures~\ref{fig:ex:12.21a} and~\ref{fig:ex:12.21b}). (For another problem with this
alternative, see the discussion of~\eqref{ex:12.11}.)}

Possible independent evidence that \isi{imperatives} in \ili{Sakha} involve a
special v head as well as Jussive is the fact that there seems to be
interference between imperatives and the most obvious overt v/Voice head in the
language, namely the passive morpheme -\emph{IlIn}. Passives formed with this
morpheme cannot be used in the imperative\is{imperatives};
hence~\eqref{ex:12.16} is ungrammatical.

\ea\label{ex:12.16}Sakha\\
    \gll \llap{*}Tal-ylyn!\\
		choose-\Pass{}.\Imp{}\\
    \glt ‘Be chosen!’ (e.g., for some honor or prize)
\z
This contrasts with \ili{English}, where \isi{passive} \isi{imperatives} are grammatical
under certain conditions (e.g., \emph{Be examined by a doctor!}). We can
account for this if we say that \emph{v}\textsubscript{\Imp} and
v\textsubscript{Pass} compete for the same \emph{v} position in \ili{Sakha},
and only one can be used at a time.\footnote{In contrast,
    unaccusative\is{unaccusativity} verbs are possible as commands: \emph{öl-ö
    oʁus!} (‘die quickly!’ spoken to a bug), \emph{yaldj-ima}
    (get.sick-\Neg.\Imp{} ‘Don’t get sick!’). These do not need any special v
to suppress an agent argument. Apparently the theme argument can move from
inside VP to the Spec\emph{v}\textsubscript{\Imp}P position in sentences like
these.}

Since~\eqref{ex:12.14} is semantic in nature, it allows for the possibility
that other heads might be close enough in meaning to Jussive to semantically
license a \emph{v}\textsubscript{\Imp} projection, and hence \isi{partitive
case} on the object. In fact, \ili{Sakha} also has certain so-called
necessitive constructions, in which \isi{partitive case} can be observed on the
object. An example is~\eqref{ex:12.17}.

\ea\label{ex:12.17}Sakha\\
    \gll Kiliep-te  aʁal-ɯax-xa  naada.\\
    bread-\Part{}  get-\Pros{}.\Ptcp{}-\Dat{}  necessary\\
    \glt ‘It is necessary to get some bread.’ \parencite[429]{StachowskiMenz1998}
\z

\begin{sloppypar}
\noindent Our rough idea about this is that the adjective \emph{naada} ‘necessary’ is
similar enough semantically to the functional head\is{functional items} Jussive that it too can
license a \emph{v}\textsubscript{\Imp} projection in its scope.
\end{sloppypar}

We also do not say precisely how close \emph{v}\textsubscript{\Imp} must be
syntactically to Jussive (or \emph{naada}) in order to be licensed. Here we
have in mind an analogy with \glspl{NPI} licensed in the scope of negation: in
some languages, the \gls{NPI} must be in the same clause as the licensing
negation, but in others the \gls{NPI} can be at some distance, within an
embedded clause (e.g., \ili{English}: \emph{I don’t} \emph{want to eat
anything}.).  \emph{v}\textsubscript{\Imp} licensing in \ili{Sakha} seems to be
like NPI licensing in \ili{English} in this respect. Thus, all speakers allow
an imperative\is{imperatives} matrix clause to license partitive\is{partitive case} inside the embedded clause in
nonfinite control-like\is{control} complements, as shown in~\eqref{ex:12.18}.

\ea\label{ex:12.18}Sakha\\
	\gll Kiliep-te  sii-r-gin  umnu-ma!\\
		bread-\Part{}  eat-\Aor{}-\Ssg{}.\Acc{}  forget-\Neg{}.\Imp{}\\
	\glt ‘Don’t forget to eat some bread.’
\z
Some speakers even allow Jussive in the matrix clause to license \emph{v}\textsubscript{\Imp} inside a fully finite complement clause, permitting \isi{partitive case} on the object of the embedded clause as in~\eqref{ex:12.19}, whereas for other speakers this is ruled out.

\ea\label{ex:12.19}Sakha\\
	\gll Masha  kiliep-(\%te) atyylah-ya dien eren-ime.\\
		Masha  bread-\Part{}  buy-\Fut{}.\Tsg.\Sbj{}  that  hope-\Neg{}.\Imp{}\\
	\glt ‘Don’t hope that Masha will buy any bread.’
\z
In contrast, nobody allows a matrix imperative\is{imperatives} to license
\isi{partitive case} on the subject of an embedded clause in a sentence
like~\eqref{ex:12.20}.

\ea\label{ex:12.20}Sakha\\
	\gll Masha-ny   byrdax-(*ta)         ɯst-ɯaʁ-a    dien  eren-ime.\\
		Masha-\Acc{}  mosquito-(*\Part{})  bite-\Fut{}-\Tsg.\Sbj{}  that    hope-\Neg{}.\Imp{}\\
	\glt ‘Don’t hope that a(ny) mosquito bites Masha.’
\z
We can account for this curious pattern using the idea that Jussive doesn’t
license partitive\is{partitive case} NPs directly; rather it licenses
\emph{v}\textsubscript{\Imp}, which in turn triggers the assignment of
\isi{partitive case} locally inside its complement. On the one hand, if Jussive
in~\eqref{ex:12.20} licenses \emph{v}\textsubscript{\Imp} in the embedded
clause, then the embedded subject is not in the complement of this
\emph{v}\textsubscript{\Imp}, so it cannot be partitive\is{partitive case}. On the other hand, if
Jussive in~\eqref{ex:12.20} licenses \emph{v}\textsubscript{\Imp} in the
matrix clause, then the subject is in the c-command domain of
\emph{v}\textsubscript{\Imp}, but it is already spelled out on the CP
phase\is{phases} headed by \emph{dien} ‘that’, so it cannot get partitive\is{partitive case} in
those circumstances either.\footnote{There is one other possibility: the
    embedded subject could shift to the edge of the CP phase\is{phases}. This
    is possible in \ili{Sakha}, resulting in accusative\is{accusative case} subjects in many kinds
    of clauses (\citealt{Vinokurova2005}, B\&V). But we assume that CP actually
    extraposes out of VP as well.  This takes the subject at the edge of CP
outside the domain of \isi{partitive case} assignment within the matrix clause,
so these subjects can get accusative\is{accusative case} but not partitive\is{partitive case}.\label{fn:12.8}}  The
distribution of partitive\is{partitive case} NPs in embedded clauses can thus be accounted for
using~\eqref{ex:12.14}.

We summarize our proposal for \ili{Sakha} \isi{imperatives} in
Figures~\ref{fig:ex:12.21a} and~\ref{fig:ex:12.21b}. \figref{fig:ex:12.21a} contains the structure of a simple
imperative with an indefinite object, which gets partitive\is{partitive case} when the VP
complement of \emph{v}\textsubscript{\Imp} is spelled
out. \figref{fig:ex:12.21b} contains the structure of an imperative\is{imperatives} with a definite
object that moves out of VP and gets accusative\is{accusative case} because it is c-commanded by
the subject inside the same phase\is{phases}.  Either construction can have a
third person subject licensed by agreement with T, or a second person subject
licensed by Jussive if T does not bear agreement.  It is also possible for an NP
inside VP to pseudo-incorporate with V, in which case it surfaces as a bare
noun adjacent to the verb.  This accounts for all the major versions of the
imperative in \ili{Sakha}.

\begin{figure}
    \caption{\color{red}Please insert a caption\label{fig:ex:12.21a}}
    \begin{tikzpicture}[baseline=(root.base), align=center]
        \Tree   [.\node(root){JussP};
                    [.TP
                        [.\emph{v}P
                            [.NP
                                \node (masha) {(Masha)\\\emph{pro}};
                            ]
                            [.\emph{v}$'$
                                [.\node(vp){VP};
                                    [.NP \node (bread) {bread\\\Part{}}; ]
                                    [.V \node (eat) {eat}; ]
                                ]
                                \emph{v}\textsubscript{\Imp}
                            ]
                        ]
                        {T\\\Tsg}
                    ]
                    \node (juss) {Juss\\2nd};
                ]
        \node [draw, solid, fit = (vp) (eat) (bread), inner sep=0mm] (box) {};
        \draw [arrow, ->, bend right=70, font=\small, align=center]
            (juss.150) to node [above left= .1cm] {Agree iff\\no Phi on T} (masha.130);
    \end{tikzpicture}
\end{figure}

\begin{figure}
    \caption{\color{red}Please insert a caption\label{fig:ex:12.21b}}
    \begin{tikzpicture}[baseline=(root.base), align=center]

        \Tree   [.\node(root){JussP};
                    [.TP
                        [.\emph{v}P
                            [.NP
                                \node (masha) {(Masha)\\\emph{pro}};
                            ]
                            [.\emph{v}$'$
                                [.NP {bread\\\Acc} ]
                                [.\emph{v}$'$
                                    [.\node(vp){VP};
                                        [.NP \node (bread) {\tuple{\sout{bread}}}; ]
                                        [.V \node (eat) {eat}; ]
                                    ]
                                    \emph{v}\textsubscript{\Imp}
                                ]
                            ]
                        ]
                        {T\\\Tsg}
                    ]
                    \node (juss) {Juss\\2nd};
                ]

        \node [draw, solid, fit = (vp) (eat) (bread), inner sep=0mm] (box) {};
        \draw [arrow, ->, bend right=70, font=\small, align=center]
            (juss.150) to node [above left= .1cm] {Agree iff\\no Phi on T} (masha.130);

    \end{tikzpicture}
\end{figure}

Sakha is not the only language thought to have a special unmarked case for NPs
inside VPs. One analog is \citegen[140--145]{Baker2015} analysis of partitive\is{partitive case}
case in Finnish; see also \citet{Baker2017} for a similar analysis of the
so-called accusative\is{accusative case} indefinite case in \ili{Evenki}. But there is a significant
theoretical difference: in Finnish, \isi{partitive case} can be assigned within any
VP, so \isi{partitive case} has a much wider distribution, and is found in
declarative clauses as well as in \isi{imperatives}. We are led to say, then, that
all vs license partitive\is{partitive case} on NPs inside their complement in Finnish, whereas
only the special head \emph{v}\textsubscript{\Imp} does so in \ili{Sakha}.

This difference is theoretically interesting because it seems to point away
from \citegen{Pesetsky2013} attractive proposal that case features\is{case!case features} are not a
separate kind of feature provided by Universal Grammar,\is{Universal Grammar} but rather category
features copied onto an NP from a nearby head. At first glance,
\citegen{Baker2015} theory of partitive\is{partitive case} seems similar to this: saying that
partitive is an unmarked case assigned within VP could be recast as saying that
partitive case is the V feature being copied onto NP inside VP. But this
possible equation does not carry over so well to \ili{Sakha}, given that in
\ili{Sakha} \isi{partitive case} is not assignable in all VPs, but only in the VP
complements of one particular \emph{v}-like head. This issue for Pesetsky’s view is
further compounded by the fact that \ili{Sakha} allows accusative\is{accusative case} as well as
partitive on direct objects, so it has two distinct cases associated with
dependents of VP, and they cannot both plausibly be copies of the same V
feature.  Of course there may well be ways to enrich Pesetsky’s theory so that
it could account for the \ili{Sakha} partitive -- maybe even ways that are not
intrinsically more complex than how we have enriched our configurational theory
in~\eqref{ex:12.14} -- but one would have to evaluate specific proposals
carefully to see if they succeed in retaining what is initially attractive
about Pesetsky's proposal, and whether the enriched proposal more or less
converges with ours.\footnote{For example, an anonymous reviewer suggests that
    maybe v\textsubscript{imp} transfers its category feature to V, the head of
    its complement, and this affects the nature of the V feature transferred to
    the object, with the composite [v\textsubscript{imp} +V] feature on NP
    being realized as partitive\is{partitive case}.  It goes beyond the scope of this paper to
    consider what consequences this richer theory of composite category
features might have within Pesetsky’s overall system.}

In contrast, our version of the configurational approach to case assignment
does have the resources to handle \isi{partitive case} in Sakha -- both the
fact that it adds to the other possibilities for case marking direct objects in
the language, rather than replacing one of them, and the fact that it is
limited to one very specific type of clause. Partitive case in \ili{Sakha} can
thus be treated as a structural case, as long as structural case is rethought
along configurational lines.

\printchapterglossary{}

\section*{Acknowledgements}

We thank the participants in the May 2016 MA lectures in the linguistics
program at the North East Federal University in Yakutsk for their judgments and
input, as well as Ken Safir, Viviane Deprez, the participants of the Syntactic
Theory at Rutgers group, participants at a symposium on case at MIT in February
2017, and two anonymous reviewers for comments and suggestions. Special thanks
to Christina Tikhonova for additional \ili{Sakha} judgments. Any errors are our
responsibility.

{\sloppy\printbibliography[heading=subbibliography,notkeyword=this]}

\end{document}
