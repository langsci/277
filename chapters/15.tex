\documentclass[output=paper]{langsci/langscibook}
\ChapterDOI{10.5281/zenodo.4680306}

\author{Henk C. van Riemsdijk\affiliation{Tilburg University}}
\title{Case mismatches and match fixing cases}

\abstract{Matching and mismatching are names for a fairly wide variety of
    phenomena in the grammar of many, perhaps most, languages. Given the fact
    that inflection is a crucial element in (mis-)matching phenomena, the
    overall attention that these phenomena have attracted has been fairly poor.
    The present article attempts to tackle one specific aspect of
    (mis-)matching phenomena that we may suspect could be a key to a broader
    set of facts in this domain. Specifically, the article examines the
    relationship between case matching and case attraction. The former is
    frequently found in the syntax of free \isi{relative clauses}, while the second
    is often  a characteristic of \isi{relative clauses} headed by pronominal
    elements. As there are good reasons to consider these two sets of phenomena
    to be closely related, an attempt will be made here to show that matching
    and attraction are indeed two sides of the same coin. The crucial argument
will be to pursue the analysis of headed and headless \isi{relative clauses} in terms
of what has come to be called \enquote{grafting}.}


\begin{document}\glsresetall
\maketitle
\begin{refcontext}

\section{Case matching and case attraction in relative clauses}

This article will address certain phenomena concerning
    morphological case\is{case!morphological case} in a number of relative clause\is{relative clauses} constructions, in
    particular case (non-)attraction and case (mis-)matching.\footnote{There
        are similar issues in many other domains of grammar. To give just one
        example, in various constructions involving \isi{coordination} we find both
        matching requirements and mismatches. For a discussion of such
        phenomena in \textit{right node raising} constructions, for example, see
        \citet{Larson2012}. In the present article I use the term \emph{case
        (mis-)matching} to refer to case conflicts independently of whether
        they occur in a single position or in two (usually adjacent or close)
        positions. To distinguish the two, I use \emph{case attraction} (two
    positions interacting) and \emph{case superimposition} (two different cases
that fight for a single position).} The main puzzle that I would like to
discuss is the question of how many positions are involved. In case attraction
we are dealing with a head of the relative clause\is{relative clauses} and the wh-phrase in the
Spec,CP of the relative clause\is{relative clauses}: two separate positions. In \glspl{FR}, however,
it seems as if in some cases at least there is just a single position in which
a case is realized that the matrix environment and the relative clause
environment fight about determining.

\begin{sloppypar}Starting with case attraction, let us look at some examples from Ancient
Greek.\footnote{The examples given here are adapted from
    \textcite{Hirschbuhler1976} and were cited in
    \Citet{GroosVanRiemsdijk1981}. I use superscripts to indicate the case
    imposed by the item in question and subscripts to indicate the actual case
borne by the element in question.}\end{sloppypar}

\ea\label{ex:15.1}
    \ea
        \gll    pro t\=on kak\=on \emph{ha} oida\\
                instead-of\textsuperscript{\Gen{}}  the evils\tss{\Gen{}}   which\tss{\Acc{}} I-know\textsuperscript{\Acc{}} \\
        \glt    ‘instead of the evils which I know’\\
    \ex pro\textsuperscript{\Gen{}}  t\=on kak\=on\tss{\Gen{}}  \emph{h\=on}\tss{\Gen{}} oida\textsuperscript{\Acc{}}
    \z
\z
In (\ref{ex:15.1}a) the head of the relative clause\is{relative clauses} has the genitive\is{genitive case} case imposed by the
preposition in the matrix while the relative pronoun\is{relative pronouns} has the accusative
case\is{accusative case} imposed by the embedded verb ‘know’. In (\ref{ex:15.1}b) however, the case of the
relative pronoun has been changed from accusative to genitive\is{genitive case}, the case of the
head. This is called case attraction.

\ea\label{ex:15.2}\ili{Greek}
    \ea
        \gll    \dots{} ekpiein  sun         \emph{toutois}    \emph{hous}        malista phileis\\
                {} to-drink with\textsuperscript{\Dat{}} those\tss{\Dat{}} whom\tss{\Acc{}} best you-love\textsuperscript{ACC}\\
        \glt    ‘to drink with those whom you love best’
    \ex \dots{} ekpiein sun\textsuperscript{\Dat{}} \emph{hois}\tss{\Dat{}} malista phileis\textsuperscript{\Acc{}}
    \z
\z
(\ref{ex:15.2}a) is a headed relative clause\is{relative clauses} in which the head is in the dative\is{dative case} case
according to the requirements by the matrix preposition while the relative
pronoun appears in the \isi{accusative case} thereby fulfilling the case requirements
of the verb in the relative clause\is{relative clauses}. (\ref{ex:15.2}b) is the corresponding FR\@. As there is
only one single relative pronoun\is{relative pronouns}, that is, only one position to express case
morphology, a conflict arises between the dative\is{dative case} required by the matrix and the
accusative imposed by the relative clause\is{relative clauses}: a case mismatch\is{case!case mismatches}. In some languages
this would lead to a conflict that cannot be resolved. In such languages an
example like (\ref{ex:15.2}a) could not be expressed by means of a FR\@. In Ancient Greek,
however, the conflict is resolved by means of a kind of radical form of case
attraction which we might call case superimposition. In (\ref{ex:15.2}b) the matrix dative\is{dative case}
supersedes the embedded accusative.

The question as to whether a case conflict in a given language results in
ungrammaticality or whether it can be resolved by case attraction (or
superimposition) is a complicated one. For \ili{Ancient Greek},
\textcite{Hirschbuhler1976} proposed a case hierarchy:\footnote{See also
    \citet{Harbert1983} for extensive discussion, including \ili{Gothic}.}

\ea\label{ex:15.3}
    \Nom{} $>$ \Acc{} $>$ \Dat{} $>$ \Gen{}
\z
This hierarchy goes from least oblique to most oblique. And the corresponding
principle is as in \eqref{ex:15.4}.

\ea\label{ex:15.4}
    In situations of case superimposition the more oblique case wins.
\z

This will  correctly predict that in (\ref{ex:15.2}b) it is the dative\is{dative case} that wins and
suppresses the accusative.

German may well be the language for which this issue has been studied in the
greatest detail.\footnote{See among many others \citet{Vogel2001}.} There is
considerable variation in the judgments ranging from those who allow very few
case mismatches to those who allow virtually all of them.\footnote{This is just
    scratching the surface. As an anonymous reviewer points out, \ili{Polish} does
    not resolve case mismatches. To circumvent ineffability problems, however,
    \ili{Polish} makes extensive use of so-called \enquote{light headed relatives},
    that is, \isi{relative clauses} with a pronominal head. See \citet{Citko2004}.
Furthermore, it appears that in modern Greek the matrix case always wins, cf.\
\citet{Daskalaki2011} and \citet{Spyropoulos2007}.\label{fn:15.6}}

This is not, however, the question that I mean to discuss in this paper.
Instead, the issue I want to address here is what it means to say that “in the
FR there is only one position to realize case”. Take the following examples of
\glspl{FR} in Standard High German.\footnote{These examples are from \textcite[15, ex. 22a,b]{Vogel2001}.}

\ea\label{ex:15.5}\ili{German}
    \ea[]{%
        \gll    Wen\tss{\Acc{}}     du  einlädst\textsuperscript{\Acc{}} wird auch kommen\textsuperscript{\Nom{}}.\\
                who-\Acc{} you invite          will   also  come\\
        \glt    ‘Whoever you invite is sure to come too.’}
    \ex[*]{
        \gll    Sie zerstört\textsuperscript{\Acc{}}, wer\tss{\Nom{}}       ihr            begegnet\textsuperscript{\Nom{}}.\\
                she  destroys     who-\Nom{}  her-\Dat{}  meets\\
        \glt    ‘She destroys whoever meets her.’}
    \z
\z
At first sight, there is a relative clause\is{relative clauses} without a head and a relative
pronoun in the relative clause\is{relative clauses}. So, ostensibly, there is only one pronoun that
has a slot for case morphology. Suppose, however, that \glspl{FR} do have a
head just like headed relatives but that the head is silent.\footnote{This was
the analysis proposed in \Citet{GroosVanRiemsdijk1981}.} In that case we could
say that there are two slots for case morphology, but at spell-out there is
only one in which case can be overtly expressed.

As I will suggest at the end of \Cref{sec:15.3}, there is only one
syntactic position which is \enquote{shared} by the relative clause\is{relative clauses} and the
matrix clause. An anonymous reviewer remarks that from a semantic point of view
the FR-pronoun is not a shared argument: the argument of the relative predicate
is the FR-pronoun but the argument of the matrix predicate is the FR as a
whole. Notice, however, that on a raising analysis of relative
clauses\is{relative clauses!raising analysis} the head
of the relative clause\is{relative clauses} is similarly shared between the relative clause\is{relative clauses} and the
matrix clause. Space prevents a more extensive discussion here.

\section{One position for case or two?}

While there are language particular differences in the case hierarchies and the
way they determine case attraction and case superimposition, the similarities
are nevertheless considerable. And the fact that they affect both attraction
and superimposition strongly suggests that the structures to which they apply
should be sufficiently similar in order to allow for the generalization to be
expressed. It follows, apparently, that the silent head analysis of \glspl{FR}
should be preferred as the adoption of that analysis implies the presence of
two positions in both constructions: case attraction and case superimposition.
Simplifying, the structure of  (\ref{ex:15.5}a) would be roughly like \eqref{ex:15.6}.

\ea\label{ex:15.6}
    {}[\tss{DP} [ $\varnothing$ ]\tss{\Nom{}} ]  [\tss{CP} [\tss{Spec,CP}
    [\tss{WhP} [\tss{Wh} wen ]\tss{\Acc{}} ]\tss{i}  du  einlädst  t\tss{i} ]  wird auch kommen]
\z
The nominative\is{nominative case} case feature on the silent head and the \isi{accusative case} feature
on the relative pronoun\is{relative pronouns} now have to fight about which one of them can be
realized on the only available host, the relative pronoun\is{relative pronouns} \emph{wen}. In case
attraction situations, which are now structurally identical except that the
head is lexically realized, not silent, each case feature can be realized on
its host, but nevertheless the two case features\is{case!case features} may “feel the necessity to
create a closer bond between them”, resulting in a copy of one of the two case
features being superimposed on the other one. And that is case attraction.

Unfortunately the situation is somewhat more complicated than that. I have
argued (cf.~\Cite{VanRiemsdijk2006a})\footnote{See this chapter for an ample
overview of the relevant literature. An updated version of this chapter has
appeared in \Textcite{VanRiemsdijk2017}.} that \glspl{FR} should be treated in
terms of what I call \emph{grafting}. Let me first introduce the notion of
\enquote{graft} and then show how \glspl{FR} could be analysed in terms of
graft structures.

There are ample arguments for grafts \Parencite[cf.][]{VanRiemsdijk2000}. A
more \enquote{authoritative} view is presented in \Citet{VanRiemsdijk2006b}. As
an illustration of simple cases, consider a DP like \eqref{ex:15.7}:

\ea\label{ex:15.7}
    a far from simple matter
\z
It is quite easy to see that assigning a structure to such a DP is, indeed, a
far from simple matter. Clearly we have a head noun \textit{matter}. To the left there
is an attributive AP\@. But there are two adjectives: \textit{far} and \textit{simple}.
Assuming that \textit{from simple} is a PP, that PP is presumably a complement of
\textit{far}. That is, we might assume that the structure of that PP in \eqref{ex:15.7} is
equivalent to that of \eqref{ex:15.8}.

\ea\label{ex:15.8}
    far from the airport
\z
But this leads immediately to a serious problem in that \eqref{ex:15.9} is ungrammatical:

\ea[*]{a far from the airport hotel}\label{ex:15.9}
\z
The reason is quite straightforward. The head of the AP, \textit{far}, is not left
adjacent to the head noun \textit{hotel}. That they must be adjacent has been argued
in \citet{Emonds1985,Emonds1976}, \citet{Williams1982},
\Citet{VanRiemsdijk1993}, \citet{BibHolRob2014}. As \eqref{ex:15.7} is grammatical, we are
led to assume that \textit{simple} is the head. This assumption also makes sense
semantically in that the meaning of \eqref{ex:15.7} is something like \textit{a not really simple
matter}, where \textit{not really} is a modifier of the head \textit{simple}.\footnote{Note
    also, that, as an anonymous reviewer observes, in \eqref{ex:15.7} the postnominal
    position for the AP is ungrammatical: *\emph{a matter far from simple}
    while in \eqref{ex:15.9} the postnominal position of the AP makes the phrase
grammatical: \emph{a hotel far from the airport.} } In short, we have a
paradox, if we want to express the structure of \eqref{ex:15.7} taking all these
considerations  into account. The notion of graft (which I have argued is
simply a special case of merge, cf.~\Cite{VanRiemsdijk2006b}) offers a solution (see \figref{fig:ex:15.10}).

\begin{figure}\caption{\label{fig:ex:15.10}A simple graft}
    \begin{tikzpicture}[baseline]

        \begin{scope}[xshift=.9995cm]
        \Tree
                [.DP
                    [.D a ]
                    [.N$'$
                        [.AP
                            [.A
                                \node (a) {simple};
                            ]
                        ]
                        [.N
                            matter
                        ]
                    ]
                ]
        \end{scope}
        \begin{scope}[yshift=-8.5cm, grow'=up]
        \Tree
            [.AP
                [.A
                    far
                ]
                [.PP
                    [.P
                        from
                    ]
                    [.AP
                        \node (A) {A};
                    ]
                ]
            ]
        \end{scope}

        \draw (a) -- (A);

    \end{tikzpicture}
\end{figure}

Cases like \eqref{ex:15.7} alone would not suffice to justify this type of approach. But
there is considerable evidence
(cf.~\Citealt{VanRiemsdijk2001,VanRiemsdijk2006a,VanRiemsdijk2006b,VanRiemsdijk2006c,VanRiemsdijk2010})
for grafts from a number of constructions including \glsdesc{FR}s (\glspl{FR})
and particularly a special type of FR called \glspl{TFR}.

On this view, \glspl{FR} will be analysed along the following
lines \REF{ex:15.11}:

\ea\label{ex:15.11}\ili{German}
    \ea[]{%
        \gll    Ich gebe\textsuperscript{\Dat{}} die Belohnung wem\tss{\Dat{}} eine  gebührt\textsuperscript{\Dat{}}.\\
                I     give       the reward       to-whom   one  deserves\\
        \glt    ‘I give the reward to who deserves one.’}
    \ex[]{%
        \gll    Ich gebe\textsuperscript{\Dat{}} die Bel. *wer\tss{\Nom{}}/*wem\tss{\Dat{}} eine verdient\textsuperscript{\Nom{}}.\\
                I    give       the reward    who/whom   one  deserves\\
        \glt    }
    \ex[*]{%
        \gll    Wem\tss{\Dat{}} /*wer\tss{\Nom{}} eine Belohnung gebührt\textsuperscript{\Dat{}} bekommt\textsuperscript{\Nom{}} eine.\\
                whom who a   reward deserves   receives         one\\
        \glt    }
    \z
\z
(\ref{ex:15.11}a), which incidentally illustrates the case matching effect, would roughly
be assigned the following structure under a graft approach (\figref{fig:ex:15.12}).

%%please move the includegraphics inside the {figure} environment
%%\includegraphics[width=\textwidth]{vanRiemsdijkrevised-img9.png}
%\emph{\figref{fig:2}:FR analysis by grafting}

\begin{figure}\caption{\label{fig:ex:15.12}\gls{FR} analysis by grafting}
    \begin{tikzpicture}[baseline]
        \begin{scope}
            \node [text width=5cm] (treeA) {input tree A (matrix/host):};
        \end{scope}
        \begin{scope}[xshift=3cm]
            \Tree   [.\node(vbar){V$'$};
                        [.DP \edge [roof]; {die Belohnung} ]
                        [.V geb- ]
                    ]
        \end{scope}
        \begin{scope}[yshift=-3cm]
            \node [text width=5cm] (treeB) {tree B (grafted):};
        \end{scope}
        \begin{scope}[xshift=1cm, yshift=-3cm]

            \Tree   [.\node(cp){\textbf{CP}};
                        \edge[draw=none]; {}
                        [.\node(ip){\textbf{IP}};
                            \edge[draw=none]; {}
                            [.\emph{v}P
                                [.\node(dp1){DP}; eine ]
                                [.VP
                                [.\node(dp2){DP}; wem ]
                                    [.V gebühr- ]
                                ]
                            ]
                        ]
                    ]

        \end{scope}
        \begin{scope}[xshift=6.5cm, yshift=-2.0cm]

            \Tree   [.\node(vp){\textbf{VP}};
                        \edge[draw=none]; {}
                        [.V$'$
                            [.DP \edge [roof]; {die B.} ]
                            [.V geb- ]
                        ]
                    ]

        \end{scope}

        \draw [->, shorten >=1mm, shorten <=1mm] (vbar) -|
            node[below left]{graft}(vp);

        \draw [dashed] (ip.south).. controls +(south west:1.25) and
            +(north east:.75)..node[left=.25cm, solid, draw]{1}(dp1.north);

        \node at (1, -8) [inner sep=0mm] (control) {};

        \draw [dashed] (cp.south).. controls +(south west:3.0)
            ..node[left=.25cm, solid, draw]{2}(control);

        \draw [dashed] (control).. controls +(south east:1.25) and
            +(north:1)..(dp2.north);

        \draw [dashed] (vp.south)..controls +(south:2) and
            +(north:2)..node[above left=.25cm, solid, draw]{3}(dp2.north);

    \end{tikzpicture}
\end{figure}

The strongest arguments for a graft/multi-dominance approach come from
\glspl{TFR}.  Below I will summarize some of the major properties of
\glspl{TFR} to show what these arguments are.\footnote{Some of these
    observations are due to \citet{Wilder1998} and some are my own, see
\Citet{VanRiemsdijk2001,VanRiemsdijk2006a,VanRiemsdijk2006b}.}

\begin{itemize}

    \item 	 \glspl{FR} are definite or free choice universal as in \eqref{ex:15.13} –
        \glspl{TFR} are typically indefinite, cf. \eqref{ex:15.14}, that is, it is the
        predicate nominal (PN) that determines the indefiniteness of the
        \gls{TFR}, not the \emph{wh}{}-word.

\ea\label{ex:15.13}
    I eat what is on the table.
\z
\ea\label{ex:15.14}
    \ea I ate what they euphemistically referred to as a steak.
    \ex There is what I suspect is a meteorite on the front lawn.
    \z
\z

    \item 	 (\ili{English}) number agreement: \emph{what} determines singular
        agreement inside and out in the \gls{FR} (\ref{ex:15.15}a), but it is the \gls{PN}
        that determines the actual agreement in the \gls{TFR} (\ref{ex:15.15}b,c).

\ea\label{ex:15.15}
    \ea What pleases\textbf{/}*please me most adorns/*adorn the living\\ room wall.
    \ex What *seems/seem to be some meteorites *was/were \\ lying there.
    \ex What seems/*seem to be a meteorite was/*were lying\\ there.
    \z
\z

    \item Adjectival agreement in Dutch is present in attributive adjectives
        but not in predicative adjectives. The \gls{PA} in a
        \gls{TFR} inflects like an adjective when the \gls{TFR} is adnominal.
        That is, the \gls{PA} is the shared element.


\ea\label{ex:15.16}\ili{Dutch}\\
    \gll    een  wat    ik zou     noemen eenvoudig-*(e) oplossing  \\
            a      what  I  would call        simple               solution\\
    \glt
\z

    \item Idiom chunks: the \gls{PN} in the \gls{TFR} can complete a matrix
        idiom.\is{idioms}

\ea\label{ex:15.17}
    \ea The headway they made was impressive.
    \ex They didn't make what can reasonably be considered headway.
    \z
\z

    \item Bound anaphors in the \gls{PN} of the \gls{TFR} can be bound by a matrix
antecedent, showing again that the \gls{PN} is the shared element.

\ea\label{ex:15.18}
    \ea They live in what is often referred to as each other's backyard.
    \ex She was what can only be interpreted as proud of herself.
    \z
\z

\ea\label{ex:15.19}
    \ea Bush\tss{i} would never acknowledge what Cheney\tss{j} refers to as [each other’s]\tss{i+j} mistakes.
    \ex John\tss{i} hates to discuss what Mary\tss{j} calls [each other’ s]\tss{i+j} sexual deficiencies.
    \z
\z

    \item Case matching is required on the \gls{PN}. The examples are from German.

    \ea\label{ex:15.20}\ili{German}
    \ea[]{%
        \gll    Er hat was   man einen\tss{\Acc{}} Halunken     nennt\textsuperscript{\Acc{}} festgenommen\textsuperscript{\Acc{}}.\\
                he has what one   a             scoundrel     calls  apprehended\\
        \glt    ‘He has apprehended what they call a scoundrel.’}
    \ex[*]{%
        \gll    Er ist was man einen\tss{\Acc{}}  / einem\tss{\Dat{}} Halunken nennt\textsuperscript{\Acc{}}   auf den  Leim gegangen\textsuperscript{\Dat{}}.\\
                he  is  what one    a          /      a           scoundrel calls          on  the  glue   gone\\
        \glt    ‘He has been hoodwinked by what they call a scoundrel.'}
    \z
\z
\end{itemize}
In (\ref{ex:15.20}a) the case requirements by the matrix clause and by the \gls{TFR} are
identical, they match. But note that the shared element that has to satisfy the
double case requirement is the \gls{PN}, not the \emph{wh}{}-word. This is
shown by (\ref{ex:15.20}b) where the case requirements on the \gls{PN} do not match. Note
also that case syncretism, which can resolve case mismatches in \glspl{FR} as
in \eqref{ex:15.21} also does so in \glspl{TFR}, cf. \eqref{ex:15.22}:

\ea\label{ex:15.21}\ili{German}
    \ea[*]{%
        \gll    Wen\tss{\Acc{}}   du  liebst\textsuperscript{\Acc{}} ist\textsuperscript{\Nom{}} ein Halunke.\\
                whom you love         is         a   scoundrel\\
        \glt    }
    \ex[]{%
        \gll    Was\tss{NOM/ACC} du  liebst\textsuperscript{\Acc{}} ist\textsuperscript{\Nom{}} Pasta.\\
                what            you love        is        pasta\\
        \glt    }
    \z
\z
The \emph{wh}{}-word \emph{wen} in (\ref{ex:15.21}a) can only be an accusative, hence we
have a case-mismatch which causes ungrammaticality. But in (\ref{ex:15.21}b) the
\emph{wh-}word \emph{was} is syncretic in that it can be both a nominative\is{nominative case} and
an accusative. Thereby the mismatch is avoided. Perhaps the most convincing
indication that in \glspl{TFR} it is the \gls{PN} that is the shared element
between the matrix clause and the (transparent) free relative is the fact
that the \gls{PN} shows syncretic behavior just like the \emph{wh-}word in
\glspl{FR}.\footnote{(\ref{ex:15.22}a) is an example of a case mismatch\is{case!case mismatches} in which the
    accusative wins over the nominative\is{nominative case}. This is considered more or less
    grammatical by many speakers of German, see \citet{Vogel2001} for
discussion.}

\ea\label{ex:15.22}\ili{German}
    \ea[]{%
        \gll    Was  viele  einen\tss{\Acc{}} geilen\tss{\Acc{}} Wagen nennen\textsuperscript{\Acc{}} wird      oft             gekauft\textsuperscript{\Nom{}}.\\
                what many a             sexy          car       call             is   frequently bought\\
        \glt    }
    \ex[*]{Was viele ein\tss{\Nom{}} geiler\tss{\Nom{}} Wagen nennen\textsuperscript{\Acc{}} wird oft gekauft\textsuperscript{\Nom{}}.}
    \ex[]{Was viele ein\tss{NOM/ACC} geiles\tss{\Nom{}} Auto nennen\textsuperscript{\Acc{}} wird oft gekauft\textsuperscript{\Nom{}}.}
    \z
\z
The important fact here is that, while \emph{Wagen and Auto} are synonymous,
\emph{Wagen} is a masculine noun while \emph{Auto} is neuter. In the paradigm
for masculine nouns the nominative\is{nominative case} and the accusative are distinct, but in the
paradigm for neuter nouns they are not, in other words there is syncretism in
the case morphology. Accordingly the case mismatch\is{case!case mismatches} in (\ref{ex:15.22}b) causes
ungrammaticality, but in (\ref{ex:15.22}c) the mismatch is avoided by syncretism.

The important thing about \glspl{TFR}, then, is that it is perfectly evident
that it is the \gls{PN}/PA of the \gls{TFR} that acts as the shared element,
i.e.\ the element that is also part of the matrix clause. There does not appear
to be an obvious way to posit a second position alongside the \gls{PN} which
could be used as the locus for a second case morpheme as in example \eqref{ex:15.6} above.

A graft approach directly expresses the notion that the \gls{PN} (or the PA) is
simultaneously part of the \gls{TFR} and of the matrix structure. By way of
illustration, here is a simplified graft derivation of a simple \gls{TFR}:

\ea\label{ex:15.23}
    I ate what they called a steak.
\z\largerpage[2]

\begin{figure}[H]
    \caption{\label{fig:ex:15.24}TFR analysis by grafting}
    \begin{tikzpicture}[baseline]
        \begin{scope}
            \node [text width=5cm] (treeA) {input tree A (matrix/host):};
        \end{scope}
        \begin{scope}[xshift=3cm]
            \Tree   [.\node(v){V};
                        eat-
                    ]
        \end{scope}
        \begin{scope}[yshift=-3cm]
            \node [text width=5cm] (treeB) {tree B (grafted):};
        \end{scope}
        \begin{scope}[xshift=1cm, yshift=-2.5cm]

            \Tree   [.\node(cp){\textbf{CP}};
                        \edge[draw=none]; {}
                        [.\node(ip){\textbf{IP}};
                            \edge[draw=none]; {}
                            [.\emph{v}P
                                [.\node(dp1){DP}; they ]
                                [.VP
                                    [.V call- ]
                                    [.SC
                                        [.\node(dp2){DP}; what ]
                                        [.\node(dp3){DP}; \edge[roof]; {a
                                            steak} ]
                                    ]
                                ]
                            ]
                        ]
                    ]

        \end{scope}
        \begin{scope}[xshift=6.5cm, yshift=-2.0cm]

            \Tree   [.\node(vbar){\textbf{V$'$}};
                        [.V eat- ]
                        \edge[draw=none]; {}
                    ]

        \end{scope}

        \draw [->, shorten >=1mm, shorten <=1mm] (v) -|
            node[below left]{graft}(vbar);

        \draw [dashed] (ip.south).. controls +(south west:1.25) and
            +(north east:.75)..node[left=.25cm, solid, draw]{1}(dp1.north);

        \node at (1, -8) [inner sep=0mm] (control) {};

        \draw [dashed] (cp.south).. controls +(south west:3.0)
            ..node[left=.25cm, solid, draw]{2}(control);

        \draw [dashed] (control).. controls +(south east:2.5) and
            +(north:1)..(dp2.north);

        \draw [dashed] (vbar.south)..controls +(south east:2.5) and
            +(north:2)..node[left=.5cm, solid, draw]{3}(dp3.north);

    \end{tikzpicture}
\end{figure}

At this point we can draw three interim conclusions:

\begin{description}
\item[Interim conclusion 1:] Matching effects (and mismatches) in
\glspl{FR} and \glspl{TFR} must be dealt with in terms of a single position,
that is, the shared element.

\item[Interim conclusion 2:] Case attraction as well as its absence is a
process that occurs between two positions.

\item[Interim conclusion 3:] The phenomena of (mis-)matching and case
(non-)at\-trac\-tion are sufficiently similar to regard a theory in which we
need two separate treatments as a failure, hence we must study ways in which we
can interpret both phenomena as two sides of the same coin. We might call this
\textsc{the theoretician's dilemma}.
\end{description}

\section{Can we have our cake and eat it too?}\label{sec:15.3}

There is a simple and straightforward way to solve the theoretician’s dilemma.
We have been tacitly assuming that grafting applies to maximal projections, to
phrases. This is not only a simplification, but it is, in fact, wrong. First,
as I have argued in \Citet{VanRiemsdijk2006b} grafting is not an exotic new
enrichment of the power of the theory but simply an instance of merge. Indeed,
a stipulation would be necessary to prevent merge from applying to, for
example, the adjective \emph{simple} with the noun \emph{matter} in \figref{fig:ex:15.10}.
But observe that limiting grafting to maximal phrases would also require a
stipulation that is unwarranted both from a theoretical perspective and for
empirical reasons.

This does not alter the fact that grafting is a powerful mechanism.  There are
two reasons why this is unavoidable. First, I believe grafting is unavoidable
if we are to present cogent analyses for constructions like \glspl{FR} and
\glspl{TFR} (and many others such as Horn-amalgams, cf.\
\Cite{VanRiemsdijk2006c}). There are many other cogent reasons for making
merge the central operation in syntax. As I have argued \Parencite{VanRiemsdijk2006b}
grafting is an inevitable consequence of the introduction of merge.  What seems
to be realized much less is that the adoption of merge inexorably initiates a
new program to search for powerful limitations of the descriptive power in much
the same way that the introduction of transformations in the 60s defined a
program to restrict them severely. If the program to restrict merge turns out
to be as fruitful as the program to restrict transformations, generative syntax
may look forward to a very bright future indeed. As for grafting, a very modest
attempt at restricting its power is presented in \Citet{VanRiemsdijk2010}.

Returning now to the \enquote{theoretician’s dilemma}, consider the fact, for
example, that a \gls{TFR} can be inserted in the middle of a DP as in:

\ea\label{ex:15.25}
    John has three what I would call gas guzzlers in his garage.
\z
In this example the shared element is the compound \emph{gas guzzler}.  Inside
the matrix DP (\emph{three gas guzzlers}) the compound is not a complete DP
but, presumably, just N. In the \gls{TFR}, however, the \gls{PN} is a complete
DP\@. A very simplified tree structure for \eqref{ex:15.25} shows this (\figref{fig:ex:15.26}).

\begin{figure}
    \caption{\label{fig:ex:15.26}`Attributive' \glspl{TFR}}
    \begin{tikzpicture}[baseline]
    \begin{scope}[xshift=0cm]
    \Tree
            [.DP
                [.NUM three ]
                [.N \node (gg) {gas guzzlers}; ]
            ]
        \end{scope}
    \begin{scope}[yshift=-8.5cm, xshift=-2.455cm, grow'=up]
    \Tree
        [.CP
            [.SpeCP what$_i$ ]
            [.IP
                [.DP I ]
                [.VP
                    [.V {would call} ]
                    [.SC
                        t$_i$
                        [.DP \node (GG) {N}; ]
                    ]
                ]
            ]
        ]
    \end{scope}

    \draw (gg) -- (GG);

    \end{tikzpicture}
\end{figure}

In our discussion about \enquote{one position or two}, what we are talking
about is positions  in which the case features\is{case!case features} (or their ultimate spellout) are
located.  And when we talk about case attraction and case (mis-)matching, these
positions are usually characterized as \enquote{K} (for Kase, to avoid
confusion between the ordinary word case and the grammatical term case). Before
showing how this would work for \glspl{TFR} with matching or mismatching case
such as those in \eqref{ex:15.22}, let us look at a simple case which shows
that this is typical and necessary for grafts involving inflectional
morphology.

Recall the third argument for a grafting analysis of \glspl{TFR} presented
above, cf.\ example \eqref{ex:15.16}. In Dutch attributive adjectives are inflected. The
rule is very simple. The \gls{AI} marker is always -ə
(spelled ‘-e’) unless the head noun is indefinite neuter singular, as in
(\ref{ex:15.27}e):\footnote{I have left out adjectives with non-count nouns. It should also
    be pointed out that in Dutch spelling an adjective like \emph{groot} when
    suffixed by –\emph{e}  is spelled with a single ‘o’ (because the syllable
is open). For more detailed discussion, see
\textcites[11--13]{Broekhuis2013a}.}

\ea\label{ex:15.27}\ili{Dutch}
    \ea een groot-*(e) woning\hfill indef.\ masc.\ sing.\\
        (a large apartment)
    \ex twee groot-*(e) woningen\hfill indef.\ masc.\ pl.\\
        (two large apartments)
    \ex de groot-*(e) woning\hfill def.\ masc.\ sing.\\
        (the large apartment)
    \ex de groot-*(e) woningen\hfill def.\ masc.\ pl.\\
        (the large apartments)
    \ex \textbf{een groot-(*e) huis}\hfill indef.\ neuter sing.\\
        (a large house)\\
    \ex twee groot-*(e) huizen\hfill indef.\ neuter pl.\\
        (two large houses)
    \ex het groot-*(e) huis\hfill def.\ neuter sing.\\
        (the large house)
    \ex de groot-*(e) huizen\hfill def.\ neuter pl.\\
        (the large houses)
    \z
\z
Example \eqref{ex:15.16}, repeated here as \eqref{ex:15.28}, can now be represented quite simply as
\figref{fig:ex:15.29}, where the AIs remain outside the shared adjective which is
grafted.\footnote{Not unexpectedly the same \gls{TFR} with a neuter noun is
    perfectly grammatical as neither the matrix nor the \gls{TFR} requires a
\emph{-e} ending: \emph{een wat ik zou noemen groot huis}.}

\ea\label{ex:15.28}\ili{Dutch}\\
    \gll    een wat    ik zou     noemen eenvoudig-*(e) oplossing  \\
            a      what I  would call         simple               solution\\
    \glt
\z

%%please move the includegraphics inside the {figure} environment
%%\includegraphics[width=\textwidth]{vanRiemsdijkrevised-img12.png}
 %
% Dear copy{}-editor/ formatter, in \figref{fig:5} there is a small but important error which, due to my insufficient knowledge and means, I cannot correct.  Underneath the blue oval there is a node label that reads   AI{\textbar}. The last vertical stroke should be removed. That is, this label should simply read   AI. I hope you can do this for me. Thanks.
%

%\emph{\figref{fig:5}: Mismatch avoidance with attributive adjectives}

\begin{figure}[p]
    \caption{\label{fig:ex:15.29}Mismatch avoidance with attributive adjectives}
    \begin{tikzpicture}[baseline]
    \begin{scope}[xshift=3.60cm]
    \Tree
            [.DP
                [.DET een ]
                [.NP
                    [.AP
                        [.A \node [text width=1.8cm] (a) {eenvoudig-}; ]
                        [.AI \node (b1) {-e}; ]
                    ]
                    [.N oplossing ]
                ]
            ]
    \end{scope}
    \begin{scope}[yshift=-11.15cm, xshift=-0.0cm, grow'=up]
    \Tree
        [.CP
            [.SpecCP wat$_i$ ]
            [.IP
                [.DP ik ]
                [.VP
                    [.V {zou noemen} ]
                    [.SC
                        [.{t$_i$} \node [text width=1.8cm] (A) {A}; ]
                        [.AP
                        [.AI \node (b2) {$\emptyset$}; ]
                        ]
                    ]
                ]
            ]
        ]
    \end{scope}

    \draw (a) -- (A);
    \node [draw, thick, fit = (b1) (b2), inner sep=0mm] (box) {};

    \end{tikzpicture}
\end{figure}

\begin{figure}[p]
    \caption{\label{fig:ex:15.31}Case mismatch with \gls{TFR}}
    \begin{tikzpicture}[baseline]
    \begin{scope}[xshift=0cm,frontier/.style={distance from root=150pt}]
        \tikzset{level 1/.style={sibling distance=40pt}}
    \Tree
            [.IP
                [.KP
                    [.DP
                        [.DET \node (det) {ein}; ]
                        [.NP
                            [.AP \node [text width=1cm] (ap) {geiler}; ]
                            [.N \node [text width=1cm] (n) {Wagen}; ]
                        ]
                    ]
                    [.K [.\node(nom){NOM}; ] ]
                ]
                \edge [roof]; {wird oft gekauft}
            ]

    \end{scope}
    \begin{scope}[yshift=-13.75cm, xshift=-6.05cm, grow'=up,
        frontier/.style={distance from root=241pt}]
    \Tree
        [.CP
            [.SpecCP Was$_i$ ]
            [.IP
                [.DP viele ]
                [.VP
                    [.SC
                        [.DP {t$_i$} ]
                        [.KP
                            [.DP
                                [.\node(DET){DET}; ]
                                    [.NP
                                        [.\node[text width=1cm](AP){AP}; ]
                                        [.\node[text width=1cm](N){N}; ]
                                    ]
                            ]
                            [.K [.\node(acc){ACC}; ] ]
                        ]
                    ]
                    [.V nennen ]
                ]
            ]
        ]
    \end{scope}

    \draw (det) -- (DET);
    \draw (ap) -- (AP);
    \draw (n) -- (N);

    \node [draw, thick, fit = (nom) (acc), inner sep=0mm] (box) {};

    \end{tikzpicture}
\end{figure}


We see that what looked like a morphological mismatch is resolved in structure
\figref{fig:ex:15.29} as we have two separate positions.  A conflict is avoided because one of
the two AI positions is empty.\footnote{For discussion of other cases involving
agglutinative morphology and also an extension to the issue of how the theta
criterion can be maintained in grafting structures, see \Citet{VanRiemsdijk2010}.} With this in hand, we can address the issue of case
(mis-)matches, for example in \glspl{TFR}.\clearpage

Take example (\ref{ex:15.22}b), repeated here as \eqref{ex:15.30}.

\ea\ili{German}\\
    \gll   \llap{*}Was viele ein\tss{\Nom{}} geiler\tss{\Nom{}} Wagen nennen\textsuperscript{\Acc{}} wird oft gekauft\textsuperscript{\Nom{}}.\\
            what many a          sexy          car       call            is      often bought\\
    \glt    ‘What many would call a sexy car is frequently bought.’\label{ex:15.30}
\z
The structure for such a \gls{TFR} would be roughly as in \figref{fig:ex:15.31}.

The case mismatch\is{case!case mismatches} can now be localized in the box, where \Nom{} and \Acc{} are
in conflict with each other. In this example the matrix case \Nom{} has won,
which results in ungrammaticality. If the \gls{TFR} case \Acc{} wins, as in
(\ref{ex:15.22}a) there is still a conflict, but according to the case
hierarchy \Acc{} supersedes \Nom{}. And indeed, this example is perfect for
some varieties of German and definitely much better than (\ref{ex:15.22}b)
for all speakers (see also example \eqref{ex:15.5} above and
\cref{fn:15.6}).

This solution closes the circle in that case (mis-)matching in \glspl{FR} can
be treated in a completely parallel way. Take the example (\ref{ex:15.5}a)
above, repeated here as \eqref{ex:15.32}. \figref{fig:ex:15.33} is a very
simplified tree depicting the relevant structure.\largerpage[2]

\ea\label{ex:15.32}\ili{German}\\
    \gll    Wen\tss{\Acc{}}     du  einlädst\textsuperscript{\Acc{}} wird kommen\textsuperscript{\Nom{}}.\\
            who-\Acc{} you invite          will   come\\
    \glt
\z

\begin{figure}[H]
    \caption{\label{fig:ex:15.33}Case mismatch resolved by superimposition}
    \begin{tikzpicture}[baseline]
    \begin{scope}[frontier/.style={distance from root=100pt}]
    \tikzset{level 1/.style={sibling distance=90pt}}

        \Tree
            [.CP
                [.SpecCP \node (nom) {wer\textsubscript{NOM\emph{i}}}; ]
                [.IP
                    [.DP$_j$\textsuperscript{NOM}
                        e$_i$
                        wird
                    ]
                    [.VP kommen ]
                ]
            ]

    \end{scope}
    \begin{scope}[yshift=-7.7cm, xshift=-1.7cm, grow'=up,
        frontier/.style={distance from root=100pt}]
%    \tikzset{level 1/.style={sibling distance=30pt}}

        \Tree
            [.CP
                [.SpecCP \node (acc) {wen\textsubscript{ACC\emph{j}}}; ]
                [.IP
                    [.DP du ]
                    [.VP
                        {DP$_j$\textsuperscript{ACC}}
                        einlädst
                    ]
                ]
            ]

    \end{scope}
    \node [draw, thick, fit = (nom) (acc), inner sep=0mm] (box) {};
    \end{tikzpicture}
\end{figure}

This is a typical example of a case mismatch\is{case!case mismatches} that is, however, accepted by many
speakers of German. As there is only one position in which a wh-word can be
spelled out, the mismatch must be resolved. It is resolved in the rectangle in
that the accusative wins over the nominative\is{nominative case}, as predicted by the Case
Hierarchy. In very strict versions of German, which do not accept this
mismatch, the battle has no winner and the derivation crashes as both
wh-words\is{wh-words}
cannot be spelled out simultaneously.\footnote{The question arises as to whether
the resolution of case conflicts that ultimately determines the spell-out takes
place in narrow syntax or post-syntactically, as an anonymous reviewer asks.
The answer has to be that this must be a matter of post-syntactic spell-out.
The most convincing considerations arguing for this view have to do with the
way that syncretism helps resolve case conflicts. Space prevents me from going
into the details here however.}\\

\section{Conclusion}

We started out with a puzzle. Case attraction and case (mis-)matching in normal
and transparent free relatives are sufficiently similar to aim for a unified
treatment of both. But case attraction involves an interaction between two
positions while case (mis-)matches seemingly involve only one position, at
least if, as I have argued, they are accounted for in terms of grafting. What
I hope to have shown is that there are good independent reasons for adopting
analyses in terms of sub-phrasal grafts which allow us to have two tree
positions for the matching or conflicting morphological elements, but only a
single spell-out position. Thereby we are an important step closer to a unified
theory of attraction and (mis)matching.

\printchapterglossary{}

\section*{Acknowledgements}
Parts of this article were presented at the workshop
    \emph{Insufficient strength to defend its case: Case attraction and related
    phenomena} at the University of Wrocław in September 2015. Thanks are due
    to the audience for interesting discussion.  In particular I would like to
    thank  Joanna Błaszczak and Philomen Probert for having invited me to this
    conference which gave me a chance to clarify my thoughts on attraction and
    matching. Thanks are also due to two anonymous reviewers.  The more general
    background for these issues is the antithesis of two very general forces
    that manifest themselves in many ways and in many aspects of the physical
    world: attraction and repulsion, see~\Textcite{VanRiemsdijk2019}.

{\sloppy
\printbibliography[heading=subbibliography,notkeyword=this]
}

\end{refcontext}
\end{document}
