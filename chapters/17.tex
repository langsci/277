\documentclass[output=paper]{langsci/langscibook}
\author{Tarald Taraldsen}
\title{Rethinking (un)agreement}

% \chapterDOI{} %will be filled in at production

\abstract{The labeling\is{labelling} algorithm proposed by \citet{Chomsky2013} has
consequences overlapping with formal agreement and is taken as a starting point
for developing a new analysis of sentences with plural DPs as subjects of verbs
with \Fpl{} or \Spl{} agreement in \ili{Spanish} and some other languages.}


\begin{document}\glsresetall
\maketitle

\section{Interpretable agreement features}\label{sec:key:17.1}

In most languages, a finite verb with a plural DP as its subject must be in its
\Tpl{} form. The contrast in~(\ref{ex:key:17.1}) exemplifies this for \ili{Italian}:

\ea\label{ex:key:17.1} \ili{Italian}
    \ea[]{%
    \gll I giocatori vanno a Parigi.\\
        the players go-\Tpl{} to Paris\\
%    \glt \enquote*{the players go to Paris}}
    \glt}
    \ex[*]{%
    \gll I giocatori andiamo a Parigi.\\
    the players go-\Fpl{} to Paris\\
    \glt}
    \z
\z
The standard assumption is that this follows from~(\ref{ex:key:17.2}):

\ea\label{ex:key:17.2}
    \ea Person and number features\is{person features}\is{number features} on a verbal functional head\is{functional items}, e.g. I, are
    uninterpretable and unvalued.
    \ex Hence, they must be valued under Agree with a DP.
    \z
\z
But it is a priori conceivable that person and number features\is{person features}\is{number features} on I could be
interpreted as imposing a semantic restriction on the applicability of the
verbal predicate, e.g.\ \emph{andiamo a Parigi} ‘go-\Fpl{} to Paris’
in~(\ref{ex:key:17.1}b) might translate as in (\ref{ex:key:17.3}), where [x = \Fpl{}]
restricts the range of the λ-expression:

\ea\label{ex:key:17.3}
    λx [ x = \Fpl{}]. x go to Paris
\z
If so,~(\ref{ex:key:17.1}b) would translate as \eqref{ex:key:17.4}, which would be okay as long as
\emph{i giocatori} ‘the players’ happens to denote a set of individuals
containing the speaker, since x = \Fpl{} means that the argument of
(\ref{ex:key:17.3}) must denote a set containing the speaker plus “others”:

\ea\label{ex:key:17.4}
    λx [ x = \Fpl{}]. x go to Paris (the players)
\z
But nothing stops a \Third{}rd person DP from denoting a set containing the
speaker:

\ea\label{ex:key:17.5}
    We are the champions.
\z
So, taking person and number features on I to be interpretable as
in~\eqref{ex:key:17.3} seems to yield the incorrect prediction that \eqref{ex:key:17.1}b should be
fine, and therefore one might be led back to \eqref{ex:key:17.2}. But this leaves open the
question why UG should rule out the option illustrated by \eqref{ex:key:17.3}.

Also, \ili{Spanish} (and some other languages) allows sentences
like~(\ref{ex:key:17.1}b):

\ea\label{ex:key:17.6}
    \gll    Los jugadores vamos a París.\\
            the players      go-\Fpl{} to Paris\\
    \glt    ‘We players are going to Paris.’\\
\z
The ‘we players’ part of the translation, i.e.\ the entailment that the set of
individuals denoted by \emph{los jugadores} ‘the players’ includes the speaker,
would follow from construing the verbal predicate as in~\eqref{ex:key:17.3}.

Sentences like~\eqref{ex:key:17.6} are sometimes classified descriptively as
instances of “un\-agreement”.

\section{Labelling and agreement}\label{sec:key:17.2}

A route to an analysis of the Spanish~\eqref{ex:key:17.6} based on \eqref{ex:key:17.3} which still
excludes the Italian~(\ref{ex:key:17.1}b) is suggested by the approach to labelling
taken by \citet{Chomsky2013}:

\ea\label{ex:key:17.7}
    If the syntactic object X is built by merging Y and Z, the label of X is a set
    of features associated with the head closest to the root of X.
\z
There are two cases to consider:

\ea\label{ex:key:17.8}
    \ea X = [ A [\tss{BP} \dots{} B \dots{} ]]\hfill(A is the head closest to the root)
    \ex X = [[\tss{AP} \dots{} A \dots{} ] [\tss{BP} \dots{} A \dots{} ]]\hfill(no head is closest to the root)
    \z
\z
Taking “closest” to be defined in terms of asymmetric
c-command,~(\ref{ex:key:17.8}a), where A is a head, is unproblematic. But
in~(\ref{ex:key:17.8}b), where two phrases have been merged, neither head
c-commands the other. To provide a label for X in~(\ref{ex:key:17.8}b),
\textcite{Chomsky2013} proposes that the tie is resolved as
in~\eqref{ex:key:17.9}:

\ea\label{ex:key:17.9}
    \ea In~(\ref{ex:key:17.8}b), the label of X is the set of features shared by the
    heads A and B.
    \ex If A and B have no feature in common,~(\ref{ex:key:17.8}b) is unlabelled,
    hence ill-formed.
    \z
\z
Adding a Specifier to IP is an instance of~(\ref{ex:key:17.8}b):

\ea\label{ex:key:17.10}
    X = [ [\tss{DP} \dots{} D \dots{} ] [\tss{IP} I \dots{} ]] (no head closest to the root)
\z
Hence, an IP can have a subject DP analyzed as SpecIP just in case D and I
share some feature F leading to:

\ea\label{ex:key:17.11}
    X = [\tss{FP} [\tss{DP} \dots{} D\tss{F} \dots{} ] [\tss{IP} I\tss{F} \dots{} ]]
\z
Thus, labelling\is{labelling} imposes a requirement similar to agreement as induced
by~\eqref{ex:key:17.2} without invoking a distinction between interpretable and
uninterpretable features.

This leads to the suggestion in~\eqref{ex:key:17.12} for~(\ref{ex:key:17.1}b) vs.\
\eqref{ex:key:17.6}:

\begin{exe}
    \exi{\eqref{ex:key:17.1}} \ili{Italian}
    \begin{xlist}
    \exi{b.}[*]{%
    \gll I giocatori andiamo a Parigi.\\
    the players go-\Fpl{} to Paris\\
    \glt}
    \end{xlist}
\end{exe}
\begin{exe}
\exi{\eqref{ex:key:17.6}} Spanish\\
    \gll    Los jugadores vamos a París.\\
            the players      go-\Fpl{} to Paris\\
    \glt    ‘We players are going to Paris.’\\
\end{exe}

\ea\label{ex:key:17.12}
    \ea The Italian~\eqref{ex:key:17.1}b corresponds to an instance of
    \eqref{ex:key:17.9} where D and I have no feature in common.
    \ex The Spanish~\eqref{ex:key:17.6} corresponds to an instance of
    \eqref{ex:key:17.9} where D and I have a feature F in common, as
    in~\eqref{ex:key:17.10}.
    \z
\z
But what is F?

\section{The feature composition of \First/\Spl{} pronouns and
Agr}\label{sec:key:17.3}

I will adopt the following partially uncontroversial general assumptions:

\ea\label{ex:key:17.13}
    \ea \emph{We} means ‘the speaker plus others’
    \ex \emph{We} has two features, a person feature\is{person features} π and a feature \#
    \ex \# introduces a set S of individuals (the ‘others’)
    \ex π ($=$ 1 $o$r 2) adds the speaker or the hearer to S
    \z
\z
How many values π should have and what exactly they are, will be immaterial
to what follows. The value for π in \First{}st and \Second{}nd person
pronouns will simply be given as 1 ( = the speaker) or 2 ( = the
hearer).~(\ref{ex:key:17.13}d) may be thought of in the following way: π
introduces the singleton set \{1\} or \{2\}, and \# introduces another set S of
individuals, and when π and \# co-occur, the union of the two sets is
formed and used as the restriction on x as in~\eqref{ex:key:17.3}. (In \Cref{sec:key:17.4}, I
suggest that \# does not occur in singular 1/2 pronouns, and in this case,
π alone determines the restriction on x.)

To this I add:

\ea\label{ex:key:17.14}
    \Fpl{} and \Spl{} verbal inflection (on I) are composed just like \emph{we}
    and \emph{you}, i.e. has the same two features\is{person features}\is{number features} π and \#, both interpretable
    as in~\eqref{ex:key:17.3} above.
\z
The link to labelling\is{labelling} provided by~\eqref{ex:key:17.7} suggests that the \ili{Spanish}
\eqref{ex:key:17.6} is grammatical because of~\eqref{ex:key:17.15}:

\begin{exe}
\exi{\eqref{ex:key:17.6}}
    \gll    Los jugadores vamos a París.\\
            the players      go-\Fpl{} to Paris\\
    \glt    ‘We players are going to Paris.’\\
\end{exe}

\ea\label{ex:key:17.15}
    The Spanish~\eqref{ex:key:17.6} corresponds to an instance of \eqref{ex:key:17.9}
    where D and I have a feature F in common, as in~\eqref{ex:key:17.10}.
\z
Taking a DP like \emph{los jugadores} ‘the players’ to have the feature \#, but
not a π feature, we then have:

\ea\label{ex:key:17.16}
    \ref{ex:key:17.6} = [\tss{\#P} [\tss{DP} \dots{} D\tss{\#} \dots{} ] [\tss{IP} I\tss{\#} \dots{} ]]
\z
Correspondingly, we can exclude the Italian~(\ref{ex:key:17.1}b)
via~\eqref{ex:key:17.17}:

\begin{exe}
    \exi{\eqref{ex:key:17.1}}
    \begin{xlist}
    \exi{b.}[*]{%
    \gll I giocatori andiamo a Parigi.\\
    the players go-\Fpl{} to Paris\\
    \glt}
    \end{xlist}
\end{exe}
\ea\label{ex:key:17.17}
    In \ili{Italian}, π and \# associated with verbal inflection behave as a unit
    with respect to labelling\is{labelling}.
\z
That is, the label of X = (\ref{ex:key:17.1}b) might be the feature complex
consisting of both π and \#, but not only \#:

\ea\label{ex:key:17.18}
    \ea[*]{(\ref{ex:key:17.1}b) = [\tss{\{}\tss{π}\tss{\#\}P} [\tss{DP} \dots{} D\tss{\textcolor{red}{\#}} \dots{} ] [\tss{IP} I\tss{\{}\tss{π}\tss{,\#\}} \dots{} ]]}
    \ex[*]{(\ref{ex:key:17.1}b)  = [\tss{\#P} [\tss{DP} \dots{} D\tss{\#} \dots{} ] [\tss{IP} I\tss{\{}\tss{π}\tss{,\#\}} \dots{} ]]}
    \z
\z
But since the DP \emph{i giocatori} ‘the players’ does not have the person
feature π, D does not share \{π, \#\} with I in~(\ref{ex:key:17.18}a), and so the
required labelling\is{labelling} is disallowed.

\section{Plural vs.\ singular}\label{sec:key:17.4}

The Spanish~\eqref{ex:key:17.6} has no singular counterpart:

\begin{exe}
\exi{\eqref{ex:key:17.6}}
    \gll    Los jugadores vamos a París.\\
            the players      go-\Fpl{} to Paris\\
    \glt    ‘We players are going to Paris.’
\end{exe}

\ea[*]{%
    \gll El jugador voy a París.\\
        the player go-\Fsg{} to Paris\\
    \glt}\label{ex:key:17.19}
\z
So, what’s wrong with~\eqref{ex:key:17.20}?:

\ea\label{ex:key:17.20}
    \eqref{ex:key:17.19} = [\tss{\#P} [\tss{DP} \emph{el}\tss{\#} \emph{jugador} ] [\tss{IP} \emph{voy}+ I\tss{\#} \emph{a París} ]]
\z
One might adopt~\eqref{ex:key:17.21} as an axiom:

\ea\label{ex:key:17.21}
    The feature \# only co-occurs with π in the plural forms of pronouns
    and verbal inflections.
\z
The singular interpretation of \emph{yo} ‘I’ and \emph{tú} ‘you (sg.)’ then
follows from π = 1 or 2 by itself only denoting a single individual.

But one might also decide to take ‘others’ seriously in ‘\emph{we} = the
speaker plus others’ restricting the \# combining with π to denote sets not
containing the speaker:

\ea\label{ex:key:17.22}
    In pronouns, \# cannot introduce a set containing the speaker or the hearer.
\z
By~\eqref{ex:key:17.14},~\eqref{ex:key:17.22} extends to verbal inflections.

Then, even if \# can denote singletons, as in~\eqref{ex:key:17.23}, π acting on
the denotation of \# in accordance with~\eqref{ex:key:17.13} will create a
plurality, i.e. \{1, y\}, since y ${\neq}$ 1:

\ea\label{ex:key:17.23}
    λx [ x = \tn{pi}{π}(\tn{num}{\#}) ]. x go to Paris\\
    \begin{tikzpicture}[remember picture, overlay]
        \node (1) [below =.3cm of pi] {1};
        \node (y) [below =.3cm of num] {y $\to$ \{1, y\}};
        \draw (pi) to (1);
        \draw (num) to (y);
    \end{tikzpicture}
\z

\begin{exe}
\exi{\eqref{ex:key:17.13}}
\begin{xlist}
    \exi{c.} \# introduces a set S of individuals (the ‘others’)
    \exi{d.} π (= 1 or 2) adds the speaker or the hearer to S
\end{xlist}
\end{exe}
By assumption, this makes the verbal predicate applicable only to DPs denoting
pluralities, which \emph{el jugador} ‘the player’ does not.

(\ref{ex:key:17.22}) is also instrumental in ruling out sentences where a \Fpl{}
or \Spl{} subject co-occurs with a verb with \Tpl{} inflection: Since the
\Tpl{} inflection contains \#, but not π (= 1 or 2), and \# can only introduce
a set S not containing the speaker or the hearer, the x introduced by λx can
only range over sets not containing the speaker or the hearer when the verbal
inflection is \Tpl{}, hence not over sets associated with \Fpl{} or \Spl{}
subject pronouns. (Merging a \Fpl{} subject with a IP with \Spl{} verbal
inflection is ruled out because \Spl{} inflection like \Spl{} pronouns don’t
denote sets containing the speaker so that π = 2 in the \Spl{} inflection also
restricts λx to range over sets not including the speaker.)

From the perspective of this analysis, the grammaticality of sentences similar
to~\eqref{ex:key:17.19} in \ili{Greek} is unexpected. But as observed by
\textcite{Hoehn2016}, such sentences differ from the \ili{Greek} counterparts
of~\eqref{ex:key:17.6} by imposing specific requirements on the noun inside the
singular subject, suggesting that they call for a special account in any event.

\section{\emph{otros}}\label{sec:key:17.5}

The strong forms of \ili{Spanish} \emph{we} and \emph{you} (pl.) contain
\emph{otros}/\emph{otras} ‘other’:

\ea\label{ex:key:17.24}
    \emph{we} = \emph{nosotros}, \emph{you} (pl.) = \emph{vosotros}
\z
The strong forms of \emph{I} and \emph{you} (sg.) do not:

\ea\label{ex:key:17.25}
    \emph{I} = \emph{yo}(*\emph{otro}), \emph{you} (sg) = \emph{tu}(*\emph{otro})
\z
Taking \emph{otro}(\emph{s}) `other' to relate to \# we can see it as an overt
reflex of~\eqref{ex:key:17.22}:

\begin{exe}
\exi{\eqref{ex:key:17.22}}
    In pronouns, \# cannot introduce a set containing the speaker or the hearer.
\end{exe}
That is:

\ea\label{ex:key:17.26}
    In combination with π ( = 1/2), \emph{otro}(\emph{s}) reflects the presence
    of \# introducing a set containing only individuals ‘other than the
    speaker/hearer’.
\z
Then, the forms with \emph{otro} in~\eqref{ex:key:17.25} are excluded the same way
as

\begin{exe}
\exi{\eqref{ex:key:17.19}}[*]{%
    \gll El jugador voy a París.\\
        the player go-\Fsg{} to Paris\\
    \glt}
\end{exe}
Again, the interaction between~\eqref{ex:key:17.22} and~(\ref{ex:key:17.13}d) will
force *\emph{yootro} and *\emph{tuotro} to denote a plurality, and we may
assume that this is only possible with the plural pronouns \emph{nos} ‘we, us’
and \emph{vos} ‘you (pl.)’:

\begin{exe}
\exi{\eqref{ex:key:17.13}}
\begin{xlist}
    \exi{c.} \# introduces a set S of individuals (the ‘others’)
    \exi{d.} π (= 1 or 2) adds the speaker or the hearer to S
\end{xlist}
\end{exe}
Notice that this leads to the conclusion that singular \First{}st/\Second{}nd
pronouns and inflections cannot have the feature \#. So,~\eqref{ex:key:17.21} does
hold, but for a reason:

\begin{exe}
\exi{\eqref{ex:key:17.21}}
    The feature \# only co-occurs with π in the plural forms of pronouns
    and verbal inflections.
\end{exe}
As regards spell-out, I take it that \First{}st/\Second{}nd pronouns and verbal
inflections take the plural form if and only if \# is present in the structure.

\section{DP-internal \First/\Spl{} pronouns}\label{sec:key:17.6}

Spanish also has:

\ea\label{ex:key:17.27}
    \ea
    \gll    nosotros los jugadores\\
            we          the players\\
    \glt    \enquote*{we players}
    \ex
    \gll    vosotros los jugadores\\
            you          the players\\
    \glt    \enquote*{you players}
    \z
\z
But these have no singular counterparts:

\ea[*]{%
    \gll yo / tú el jugador\\
    I {} you.\Sg{} the player\\
    \glt}\label{ex:key:17.28}
\z
Consider the labelling\is{labelling} of X = (\ref{ex:key:17.27}a) taking the pronouns to be
phrasal:

\ea\label{ex:key:17.29}
    (\ref{ex:key:17.27}a) = [\tss{\#P} [ \emph{nos}\tss{\#} \emph{otros} ] [\tss{DP} \emph{los}\tss{\#} \emph{jugadores} ]]
\z
The labelling\is{labelling} in~\eqref{ex:key:17.29} is legitimate for the same reason as the
labelling of~\eqref{ex:key:17.6} in~\eqref{ex:key:17.30}, since \# can be used as a
label independently of π in \ili{Spanish}:

\begin{exe}
\exi{\eqref{ex:key:17.6}}
    \gll    Los jugadores vamos a París.\\
            the players      go-\Fpl{} to Paris\\
    \glt    ‘We players are going to Paris.’
\end{exe}

\ea\label{ex:key:17.30}
    \eqref{ex:key:17.6} = [\tss{\#P} [\tss{DP} \emph{los}\tss{\#} \emph{jugadores}] [\tss{IP} \emph{vamos}+I\tss{\#}  \emph{a París}]]
\z
Consider now an attempt to label~\eqref{ex:key:17.28} as in~\eqref{ex:key:17.31} taking
\emph{yo} ‘I’ to be phrasal as well:

\ea\label{ex:key:17.31}
    \eqref{ex:key:17.28} = [\tss{\#P} [ \emph{yo} ] [\tss{DP} \emph{el}\tss{\#} \emph{jugador} ]]
\z
\eqref{ex:key:17.31} presupposes that \# can co-occur with π in the singular
\First{}st and \Second{}nd person pronoun. But we have concluded that this is
not the case:

\begin{exe}
\exi{\eqref{ex:key:17.21}}
    The feature \# only co-occurs with π in the plural forms of pronouns
    and verbal inflections.
\end{exe}
Hence, merging \emph{yo} (or \emph{tu}) with a DP results in a structure that
cannot be labelled.

Italian cannot have forms like~\eqref{ex:key:17.27}:

\ea\label{ex:key:17.32}
    \ea[*]{%
    \gll    noi i giocatori\\
            we  the players\\}
    \ex[*]{%
    \gll    voi i giocatori\\
            you.\Pl{} the players\\}
    \z
\z
This is for the same reason that \ili{Italian} does not allow~(\ref{ex:key:17.1}b):

\begin{exe}
    \exi{\eqref{ex:key:17.1}}
    \begin{xlist}
    \exi{b.}[*]{%
    \gll I giocatori andiamo a Parigi.\\
    the players go-\Fpl{} to Paris\\
    \glt}
    \end{xlist}
\end{exe}
The attempt to label~(\ref{ex:key:17.1}b) as in~(\ref{ex:key:17.18}a) fails because
the D does not have the feature π, hence not \{π, \#\},
while~(\ref{ex:key:17.18}b) fails because of~\eqref{ex:key:17.17}:

\begin{exe}
\exi{\eqref{ex:key:17.18}}
\begin{xlist}
    \exi{a.}[*]{(\ref{ex:key:17.1}b) = [\tss{\{}\tss{π}\tss{\#\}P} [\tss{DP} \dots{} D\tss{\#} \dots{} ] [\tss{IP} I\tss{\{}\tss{π}\tss{,\#\}} \dots{} ]]}
    \exi{b.}[*]{(\ref{ex:key:17.1}b)  = [\tss{\#P} [\tss{DP} \dots{} D\tss{\#} \dots{} ] [\tss{IP} I\tss{\{}\tss{π}\tss{,\#\}} \dots{} ]]}
\end{xlist}
\end{exe}

\begin{exe}
\exi{\eqref{ex:key:17.17}}
    In \ili{Italian}, π and \# associated with verbal inflection behave as a unit
    with respect to labelling\is{labelling}.
\end{exe}
Correspondingly, the forms in~\eqref{ex:key:17.32} are excluded, if we
generalize~\eqref{ex:key:17.17} to (\ref{ex:key:17.33}) as already suggested by
(\ref{ex:key:17.14}):

\ea\label{ex:key:17.33}
    In \ili{Italian}, π and \# associated with verbal inflection or a pronoun behave
    as a unit with respect to labelling\is{labelling}.
\z

\ea\label{ex:key:17.34}
    \ea[*]{(\ref{ex:key:17.32}a) = [\{\tss{\#,\#\}P} [ \emph{noi}\tss{\{}\tss{π}\tss{,\#\}}] [\tss{DP} \emph{i}\tss{\#} \emph{giocatori} ]]\hfill(\emph{noi} and D don’t share π)}
    \ex[*]{(\ref{ex:key:17.32}a) = [\tss{\#P} [ \emph{noi}\tss{\{}\tss{π}\tss{,\#\}}] [\tss{DP} \emph{i}\tss{\#} \emph{giocatori} ]]\hfill(because of~\eqref{ex:key:17.33})}
    \z
\z
This leaves open the question how one is to analyze the \ili{Italian}
\emph{noi}/\emph{voi giocatori} ‘we/you (pl.) players’. But if we adopt
\citegen{Hoehn2016} idea that \emph{noi} and \emph{voi} sit in D here, there is
no labeling\is{labelling} problem, since D is a head merging with a phrase (NP) bringing us
into scenario~(\ref{ex:key:17.8}a) where A (here the pronoun) doesn’t have to
share any feature with B (here N). This line of analysis provides a link back
to~\eqref{ex:key:17.33}: If \emph{noi} and \emph{voi} can be heads, the two
features π and \# must bundle together on the same head, e.g. D, and this may
explain why \# cannot be used for labeling\is{labelling} separately from π.\is{person features}\is{number features}

To exclude *\emph{noi}/\emph{voi i giocatori} ‘we/you the players’ vs.\ the
Spanish \emph{nosotros/vos\-otros los jugadores}, we must then say that
the position above D filled by the pronoun in \ili{Spanish} must be in SpecDP
(deviating from Höhn’s analysis) and can only be filled by a phrasal
constituent, and if \emph{noi, voi} (parsed as non-branching phrases) are
merged in SpecDP, the outcome cannot be labeled. (As for *\emph{io}/\emph{tu
giocatore} ‘I/you player’, it may be that D must be associated with a feature
bundle containing \#. which, as we have seen, cannot be part of a 1/2 \Sg{}
pronoun.)

On this analysis, \ili{Spanish} would differ from \ili{Italian} by associating π and
\# with different heads. (Adherence to the labeling\is{labelling} algorithm assumed in
section 2, then requires that \# is higher than π.) If so,
\emph{nosotros} and \emph{vosotros} are phrasal and cannot be in D, but can
be in a Spec position above D. If D cannot be silent, this excludes
*\emph{nosotros}/\emph{vosotros jugadores} ‘we/you players’ in \ili{Spanish}.

\section{Comparison with a different analysis}\label{sec:key:17.7}

\textcite{Hoehn2016} (who also refers to earlier work by \citealt{Hurtado1985}
and \citealt{AckemaNeeleman2013}) offers a different account of the apparent
case of “\isi{unagreement}” in the Spanish~\eqref{ex:key:17.6}, by proposing that
\eqref{ex:key:17.6} is to be analyzed as~(\ref{ex:key:17.35}a) with an unpronounced
counterpart of the overt \emph{nosotros} ‘we’ that appears
in~(\ref{ex:key:17.35}b):

\begin{exe}
\exi{\eqref{ex:key:17.6}}
    \gll    Los jugadores vamos a París.\\
            the players      go-\Fpl{} to Paris\\
    \glt    ‘We players are going to Paris.’
\end{exe}

\ea\label{ex:key:17.35}
    \ea {}[\tss{IP} [\tss{PersP} NOSOTROS [\tss{DP} los jugadores ]] [\tss{IP} vamos a París ]]
    \ex
    \gll    [\tss{IP} [\tss{PersP} nosotros [\tss{DP} los jugadores ]] [\tss{IP} vamos a París ]]\\
            {} {} we {} the players {} go-\Fpl{} to Paris\\
    \z
\z
Then, *\eqref{ex:key:17.19} correlates directly with *\eqref{ex:key:17.28}:

\begin{exe}
\exi{\eqref{ex:key:17.19}}[*]{%
    \gll El jugador voy a París.\\
        the player go-\Fsg{} to Paris\\
    \glt}
\end{exe}

\begin{exe}
\exi{\eqref{ex:key:17.28}}[*]{%
    \gll yo / tú el jugador\\
    I {} you.\Sg{} the player\\
    \glt}
\end{exe}
And the Italian~(\ref{ex:key:17.1}b) is ungrammatical because \ili{Italian} does not
allow~(\ref{ex:key:17.32}a):

\begin{exe}
    \exi{\eqref{ex:key:17.1}} \ili{Italian}
    \begin{xlist}
    \exi{b.}[*]{%
    \gll I giocatori andiamo a Parigi.\\
    the players go-\Fpl{} to Paris\\
    \glt}
    \end{xlist}
\end{exe}

\begin{exe}
\exi{\eqref{ex:key:17.32}}
\begin{xlist}
    \exi{a.}[*]{%
    \gll    noi i giocatori\\
            we the players\\
    \glt}
\end{xlist}
\end{exe}
Taking the Spanish~\eqref{ex:key:17.6} to have the structure in~(\ref{ex:key:17.35}a),
Höhn concludes that \isi{unagreement} is an illusion.

But Höhn has nothing to say about:

\ea\label{ex:key:17.36}
    \ea What excludes~\eqref{ex:key:17.28}?
    \ex What excludes~(\ref{ex:key:17.32}a) in Italian?
    \z
\z
The line of analysis followed here, however, has led to answers to the two
questions in~\eqref{ex:key:17.36}, based on the labelling\is{labelling} algorithm in
\citet{Chomsky2013}, with no recourse to agreement. My analysis also ties
grammatical \eqref{ex:key:17.19} to ungrammatical (\ref{ex:key:17.28}), like Höhn’s
analysis, and relates grammatical \eqref{ex:key:17.6} in \ili{Spanish} to grammatical
(\ref{ex:key:17.27}) and ungrammatical (\ref{ex:key:17.1}b) in \ili{Italian} to
ungrammatical (\ref{ex:key:17.32}a). This suggests that \isi{unagreement} is an
illusion because agreement also is an illusion (in the range of cases
considered here).

\section{A potential extension}\label{sec:key:17.8}

\citet{BosqueMoreno1984} discuss a peculiar fact about interrogative
infinitival clauses in \ili{Spanish}. Like \ili{English}, \ili{Spanish} allows the fairly
unexciting type of sentence exemplified in~\eqref{ex:key:17.37}:

\ea\label{ex:key:17.37}
    \ea
    \gll    No sabemos cuando ir a París.\\
            not know-\Fpl{} when  go to Paris    \\
    \glt    ‘We don’t know when to go to Paris.’
    \ex     pro\tss{i} no sabemos [\tss{CP} cuando [\tss{IP} PRO\tss{i}  ir a París ]]
    \z
\z
But unlike \ili{English} and, apparently, most other languages, \ili{Spanish} also has
infinitival interrogatives like~\eqref{ex:key:17.38}:

\ea\label{ex:key:17.38}
    \gll    No sabemos quiénes ir a París.\\
            not  know-\Fpl{} which-\Pl{} go to Paris\\
    \glt    ‘We don’t know which ones of us will go to Paris’
\z
The ungrammatical \ili{English} counterpart of~\eqref{ex:key:17.38} is supposed to be
ungrammatical because the trace (or lower copy) of the wh-phrase is not in a
case-marked position:

\ea\label{ex:key:17.39}
    \ea[*]{We don’t know [\tss{CP} [\tss{whP} which ones]\tss{i} [\tss{IP} \textbf{t}\tss{i} [\tss{IP} to go to Paris ]]]}
    \ex[*]{We don’t know [\tss{CP} [\tss{whP} which ones]\tss{i} [\tss{IP} PRO [\tss{IP} to go \textbf{t}\tss{i} to Paris ]]]}
    \z
\z
In~(\ref{ex:key:17.39}a), the trace is in the subject position of the infinitival
clause. In~(\ref{ex:key:17.39}b), it is in a lower position, e.g.\ Spec\emph{v}P
or the object position, but still presumably not case-marked. So, the question
is how the \ili{Spanish} (\ref{ex:key:17.38}) overcomes this problem.

Sentences like~\eqref{ex:key:17.38} have two properties in common with sentences
like~\eqref{ex:key:17.6}. The first has to do with the meaning
of~\eqref{ex:key:17.38}.  The denotation of the matrix subject restricts the
domain of \emph{quienes} ‘which ones’ as indicated by ‘which ones \textbf{of
us}’ in the translation of \eqref{ex:key:17.38}. This holds even when
\emph{quienes} is accompanied by an overt restriction as in
\eqref{ex:key:17.40}:

\ea\label{ex:key:17.40}
    \gll    No sabemos quiénes de los jugadores ir a París.\\
            not know-\Fpl{} which-\textsc{pl} of the players go to Paris\\
    \glt    ‘We don’t know which ones of the players will go to Paris.’
\z
\eqref{ex:key:17.40} entails that the speaker is one of the players.

This recalls the fact that~\eqref{ex:key:17.6} entails that the speaker is one of
the players:

\begin{exe}
\exi{\eqref{ex:key:17.6}}
    \gll    Los jugadores vamos a París.\\
            the players      go-\Fpl{} to Paris\\
    \glt    ‘We players are going to Paris.’\\
\end{exe}
The second property is revealed by the contrast between~\eqref{ex:key:17.38}
and~\eqref{ex:key:17.41}, which is ungrammatical even though run-of-the mill
infinitival interrogatives like (\ref{ex:key:17.37}a) allow the subject to be
\First{}st/\Second{}nd \Sg:

\ea[*]{%
    \gll    No sé quién ir a París.\\
            not know-\Fsg{} which one go to Paris\\
        }\label{ex:key:17.41}
\z
This recalls the fact that~\eqref{ex:key:17.6} also has no singular counterpart:

\begin{exe}
\exi{\eqref{ex:key:17.19}}[*]{%
    \gll El jugador voy a París.\\
        the player go-\Fsg{} to Paris\\
    \glt}
\end{exe}
This suggests that the analysis of~\eqref{ex:key:17.38} should be assimilated to the
analysis of~\eqref{ex:key:17.6}, a link also suggested by Bosque and Moreno.

To capture the two properties of~\eqref{ex:key:17.38} just mentioned, we might
begin by reanalyzing \emph{PRO} as a covert counterpart of the “agreement”
inflection on finite verbs, while continuing to require that the subject of the
infinitival clause (in SpecIP) must be unpronounced. This is indicated by the
strike-through in~\eqref{ex:key:17.42} proposed as a partial analysis of
(\ref{ex:key:17.37}a):

\ea\label{ex:key:17.42}
    no sabemos\tss{\Fpl} [\tss{CP} cuando\tss{i} [\tss{IP} \sout{DP} [\tss{IP} ir-PRO\tss{\Fpl} a París t\tss{i} ]]]
\z
I will also assume that \emph{PRO} must have the same features as the
inflection on the matrix verb, i.e. π (= 1) and \#, as indicated by the
subscripted \Fpl{} in~\eqref{ex:key:17.42}. For the infinitival IP to have a label,
the unpronounced \emph{DP} must then also have the feature π ( = 1) in addition
to \# in a language like \ili{Italian} or \ili{English}. In \ili{Spanish}, however, this need not
be the case, since \ili{Spanish} allows the \# of 1/2 \textsc{pl} inflections and
pronouns to be used as a label independently of the π.\is{person features}\is{number features}

In the light of this, consider~\eqref{ex:key:17.43} (similar to~\eqref{ex:key:17.39}a) as
a representation of the \ili{Spanish} (\ref{ex:key:17.40}):

\ea\label{ex:key:17.43}
    no sabemos\tss{\Fpl} [\tss{CP} [\tss{whP} quiénes de [\tss{DP}  los jugadores ]] [\tss{IP} \sout{DP} [\tss{IP} ir-PRO\tss{\Fpl} a París ]]]
\z
The \sout{DP} in~\eqref{ex:key:17.43} is now to be taken as the trace the DP
\emph{los jugadores} ‘the players’, which combines with \emph{quiénes} ‘which
ones’ only after movement to SpecCP, as in \citet{Sportiche2005}.  Therefore,
the labeling\is{labelling} of the infinitival IP only depends on the feature \# of
\emph{PRO}\tss{\Fpl} being able to be used as a label independently of
the π. Since \ili{Spanish} allows this,~\eqref{ex:key:17.43} is fine as far as labeling\is{labelling} is
concerned for exactly the same reason~\eqref{ex:key:17.6} is.

Similarly,~\eqref{ex:key:17.41} is ungrammatical for the same reason
as~\eqref{ex:key:17.19}. The infinitival IP remains unlabeled in
(\ref{ex:key:17.44}), because π does not combine with \# in singular pronouns or
inflections:

\ea[*]{%
    no sé\tss{\Fsg} [\tss{CP} [\tss{whP} quién de [\tss{DP} los jugadores]]] [\tss{IP} \sout{DP} [\tss{IP} ir-PRO\tss{\Fsg} a París ]]]
    }\label{ex:key:17.44}
\z
The fact that~\eqref{ex:key:17.40} entails that the speaker is one of the players,
follows from \emph{PRO}\tss{1pl} making the predicate \emph{ir a París}
applicable to \emph{DP} only if \emph{DP} in~\eqref{ex:key:17.44} denotes a
plurality including the speaker, i.e.\ for the same reason \emph{los jugadores}
‘the players’ must denote a set containing the speaker in~\eqref{ex:key:17.6}.

Finally, the case problem may be resolved if we take the covert \emph{DP} in
SpecIP as case-marked in~\eqref{ex:key:17.42} and the following representations
where \emph{PRO} acts as verbal inflection, effectively treating this covert
\emph{DP} as \emph{PRO} itself has been treated in classical analyses of
\isi{control} infinitivals.

To exclude the English~\eqref{ex:key:17.45} along with~(\ref{ex:key:17.39}a) and their
equally ungrammatical counterparts in many other languages, e.g.\ \ili{Italian}, we
must now also assume that \emph{PRO} has a π feature even when π
does not have the value 1 or 2:\is{person features}\is{number features}

\ea[*]{%
    They don’t know [\tss{CP} [\tss{whP} which ones]\tss{i} [\tss{IP} \textbf{t}\tss{i} [\tss{IP} to go to Paris ]]]
    }\label{ex:key:17.45}
\z
Then,~\eqref{ex:key:17.45} is also excluded because no label can be provided for the
infinitival IP in~\eqref{ex:key:17.45} in a language where \# combining with π
cannot be used for labeling\is{labelling} independently of π.

The assumption that \emph{PRO} can have a π ${\neq}$ 1 or 2 is based on
the conjecture that \emph{PRO} is like a reflexive pronoun in conjunction
with the common assumption that reflexive pronouns such as \ili{Romance} and
\ili{Slavic}
\Third{}rd person reflexives like \emph{se}/\emph{si} form a natural class
with the \First{}st and \Second{}nd person pronouns (\emph{me}/\emph{mi},
\emph{te}/\emph{ti}) to the exclusion of non-reflexive ``\Third{}rd
person'' pronouns and determiners (no π in the analysis developed here).

Quite obviously, this is just a sketchy beginning of a story line that might
bring~\eqref{ex:key:17.6} and~\eqref{ex:key:17.38} together, and it rests on extra
assumptions in need of justification and refinement in addition to the
hypotheses appealed to in the preceding sections. Even more importantly, it
remains to be seen whether~\eqref{ex:key:17.6} and~\eqref{ex:key:17.38} cluster
cross-linguistically as tightly as my proposal would predict.

\section{A conclusion of sorts}\label{sec:key:17.9}

Throughout, I have argued that a set of otherwise puzzling facts can be made
sense of building on the idea that the person and number features\is{person features}\is{number features} associated
with verbal inflection are really interpretable as in~\eqref{ex:key:17.3}. This
represents a clear break with main stream thinking about subject/verb
agreement.

It remains to be seen whether agreement along the lines of~\eqref{ex:key:17.2}
is still necessary for other cases of agreement such as adjective or participle
agreement. But the fact that \citegen{Chomsky2013} theory of
labeling\is{labelling} largely predicts the effects of~\eqref{ex:key:17.2},
makes this unlikely.

Finally, I have led contrasts between \ili{Spanish} and other languages back to an
assumption about the relation between the two features π and \#\is{person features}\is{number features} of pronouns and
inflections: In \ili{Spanish}, \# can be used for labeling\is{labelling} independently of π, but in
Italian and most other languages this is not possible. A suggestion as to why
Spanish and \ili{Italian} behave differently in precisely this way, has been offered
at the end of \Cref{sec:key:17.6}, but it is not unlikely that there are better ways of
understanding what exactly it means to say that the two features come
prepackaged in \ili{Italian} in a way they don’t in \ili{Spanish}.\is{person features}\is{number features}

\printchapterglossary{}

{\sloppy
\printbibliography[heading=subbibliography,notkeyword=this]
}

\end{document}
