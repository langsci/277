\documentclass[output=paper]{langsci/langscibook}
\ChapterDOI{10.5281/zenodo.4680298}
\author{Jan-Wouter Zwart\affiliation{University of Groningen} and Charlotte Lindenbergh\affiliation{University of Groningen}}
\title{Rethinking alignment typology}

\abstract{Considering the standard typological distinction between ergative and
    accusative \isi{alignment}, this article argues that the variety of phenomena
    suggests the need for a more fine-grained classification of \isi{alignment}
    types. We start from the observation that grammatical processes may or may
    not apply to all the grammatical functions, leading to a basic division in
    complete and incomplete types. It follows that \enquote{ergative} is just
    one of 18 \isi{alignment} types, while some incomplete \isi{alignment} types that look
    ergative are in fact different, and closer to the family of accusative types.}


\begin{document}\glsresetall
\maketitle

\section{Introduction}\label{sec:11.1}

\textsc{alignment} is the grouping of \isi{grammatical functions} (such as
subject, object; henceforth \glsunset{GF}\glspl{GF}) across transitive and
intransitive clauses. As is well known, the subject of an intransitive clause
(S\textsuperscript{I}) may be grouped, in terms of case-marking, control of
verbal agreement, syntactic position, etc., with either the subject
(S\textsuperscript{T}) or the object (O) of a transitive clause. With
S\textsuperscript{I}/S\textsuperscript{T} grouping we get \emph{accusative}
alignment, with S\textsuperscript{I}/O grouping \emph{ergative} \isi{alignment}
\parencite{Plank1979,Dixon1994,Deal2015}.\footnote{The transitive subject,
    intransitive subject, and object are conventionally referred to as A, S,
    and O (or P), respectively, after \textcite[xxiii]{Dixon1972}, but we
    refrain from utilizing these symbols here in order to stay as close as
    possible to the cumbersome but appropriate locutions \enquote{subject of a
    transitive/ intransitive clause}. We are also not committed to the view,
    often underlying the use of A/S/O, that these symbols stand for “universal
    syntactic-semantic primitives” \citep[6]{Dixon1994}.}

The two \isi{alignment} types are named after the morphological case\is{case!morphological case} of the outlier
in each type of grouping: O in the S\textsuperscript{I}/S\textsuperscript{T}
grouping accusative type, S\textsuperscript{T} in the S\textsuperscript{I}/O
grouping ergative type. Thus in \ili{German} \eqref{ex:11.1}, an accusative language (where case
is marked on the determiner), the determiner of the
S\textsuperscript{I}/S\textsuperscript{T} \emph{der Mann} ‘the man’ is
invariably nominative\is{nominative case} \emph{der}, whereas the determiner of the O \emph{den
    Mann} in (\ref{ex:11.1}b) is marked differently with accusative
    \emph{den}.\footnote{Glosses are abbreviated according to the Leipzig
        Glossing Rules
        (\url{https://www.eva.mpg.de/lingua/pdf/Glossing-Rules.pdf}), and have
    been adjusted from our sources for reasons of consistency.}

\ea%1
    \label{ex:11.1}\ili{German} ({cf.\ \citealt{Curme1952}})\\
	\ea
		\gll de-\textbf{r}        Mann    schwimm-t\\
			\Det-\M.\Nom{}  man(\M)  swim-\Tsg{}\\
		\glt ‘The man is swimming.’
	\ex
		\gll de-\textbf{r}        Mann    sieh-t  de-\textbf{n}      Hund\\
			\Det-\M.\Nom{}  man(\M)  see-\Tsg{} \Det-\glossN.\Acc{}  dog(\glossN)\\
		\glt ‘The man sees the dog.’
	\z
\z

Contrasting with this, in \ili{Coast Tsimshian} \eqref{ex:11.2}, an
ergative language (where case is marked on predicate markers cliticizing to the
constituent to their left), the S\textsuperscript{I} \emph{üüla} ‘the seal’ in
(\ref{ex:11.2}a) and the O \emph{hoon} ‘the fish’ in (\ref{ex:11.2}b)
are marked by the absolutive predicate marker \emph{-a}, whereas the
S\textsuperscript{T} \emph{duus} ‘the cat’ in (\ref{ex:11.2}b) is marked
differently with the ergative predicate marker \emph{-da}.

\ea%2
    \label{ex:11.2}\ili{Coast Tsimshian} {\parencite[32]{Mulder1994}}\\
	\ea
		\gll yagwa  hadiks-\textbf{a}    üüla\\
			\Prs{}    swim-\Abs{}    seal\\
		\glt ‘The seal is swimming.’
	\ex
		\gll yagwa-t    huum-\textbf{da}    duus-\textbf{a}    hoon\\
			\Prs{}-\Tsg.\Sbj{}  smell-\Erg{}    cat-\Abs{}    fish\\
		\glt ‘The cat is sniffing the fish.’
	\z
\z

Our discussion in this article starts from the assumption that the
characterization of elements as subjects or objects in the relevant languages
is uncontroversial. On this assumption it is clear that the ergative\is{ergative alignment} \isi{alignment}
type cuts across grammatical functions, grouping S\textsuperscript{I}/O
together to the exclusion of S\textsuperscript{T}.

In this introductory section we have followed the usual practice of calling a
language with ergative\is{ergative alignment} \isi{alignment} for some grammatical phenomenon \enquote{ergative}.
But the usefulness of \isi{alignment} as a typological characteristic has been
questioned, most notably by \citet{DeLancey2004}, who observes that ergative
patterning shows too much variation to allow us to identify an ergative subset
of languages in any theoretically interesting way. Somewhat in line with this,
\citet{Deal2015} decomposes \isi{ergativity} into three \isi{ergativity} properties, listed
in \eqref{ex:11.3}.

\ea%3
    \label{ex:11.3}Ergativity properties \citep{Deal2015}\\
	\ea The ergative property\\
        S\textsuperscript{T} ≠ S\textsuperscript{I} for some grammatical
        generalization(s)
	\ex The absolutive property\\
        S\textsuperscript{I} = O for some grammatical generalization(s)
	\ex The argument-structural property\\
        As the ergative property, but restricted to S\textsuperscript{I} of
        unaccusative predicates\is{unaccusativity}
	\z
\z

\enquote{Canonical} \isi{ergativity}, as illustrated in \eqref{ex:11.2} for
\ili{Coast Tsimshian} case, combines the ergative (\ref{ex:11.3}a) and absolutive (\ref{ex:11.3}b)
properties, but there is room for less canonical shades of \isi{ergativity}, where
one or more of the properties in \eqref{ex:11.3} may be missing. In fact,
certain grammatical phenomena are generally (perhaps universally) aligned
according to (\ref{ex:11.3}b) or (\ref{ex:11.3}c), as argued by \textcite{Queixalos2013}, suggesting
that the components of \isi{ergativity} are not restricted to ergative
languages.\footnote{\textcite{Queixalos2013} mentions in this connection
    deverbal nominalization/adjectivalization, orientation of secondary
    predicates, control of verbal number and honorific agreement, \isi{raising} of
embedded arguments in causative constructions. See also
\citet{Moravcsik1978,Plank1979}.\label{fn:3}} Conversely,
\citet{VerbekeWillems2012} argue that special behavior of S\textsuperscript{T}
in \ili{Indo-Aryan} languages (i.e.\ property \ref{ex:11.3}a) is not
necessarily a marker of \isi{ergativity}.

We want to add to this discussion by showing that the typological
characterization of \isi{alignment} is generally complicated by an unwarranted
idealization which assumes that all grammatical functions
(S\textsuperscript{I}/S\textsuperscript{T}/O) partake in the relevant
grammatical phenomena (case, agreement, wh-movement, etc.). Very often, this is
not the case, and it is not immediately clear how \isi{alignment} generalizations
carry over when it is not, or, conversely, how incomplete phenomena are to be
characterized in terms of \isi{alignment} typology. We argue for the recognition of a
different typological dimension, \emph{completeness}, ranging over the extent
to which \isi{grammatical functions} participate in grammatical processes, and
consider its consequences for \isi{alignment} typology.

Based on the parameter of completeness, we can identify 18 different
\isi{alignment} types, which may be grouped in four families (ergative,
accusative, indifferent, and residual). We show that the ergative property
(\ref{ex:11.3}a) is found in both the ergative and the accusative family,
and that the absolutive property (\ref{ex:11.3}b), while restricted to the
ergative family, is found in both complete and incomplete types.

Having outlined the basic typology of \isi{alignment} patterns, we illustrate the
phenomena in a number of more or less complicated languages, turn to the
puzzling \enquote{tripartite} \isi{alignment} type, and reconsider the notion of
ergative as a \enquote{\isi{dependent case}} \citep{Marantz1991}, instrumental
to a discussion of the relation between case and agreement in accusative and
ergative languages in \citet{Bobaljik2008}.

\section{Completeness}\label{sec:11.2}

In \ili{German} \eqref{ex:11.1} we saw that both subjects and objects are marked for case, along
the lines of accusative \isi{alignment} (S\textsuperscript{I}/S\textsuperscript{T} vs
O). However, verbal agreement is triggered only by subjects (in fact alike by
both S\textsuperscript{I} and S\textsuperscript{T}), as can be seen when we
manipulate the number of the noun phrases:

\ea%4
    \label{ex:11.4}\ili{German}\\
	\ea
		\gll d-\textbf{ie}        Män-\textbf{ner}  seh-\textbf{en}  de-n      Hund\\
			\Det-\Nom.\Pl{}  man-\Pl{}    see-\Pl{} \Det-\glossN.\Acc{}  dog(\glossN)\\
		\glt ‘The men see the dog.’
	\ex
		\gll de-r        Mann    sieh-\textbf{t}  d-\textbf{ie}      Hund-\textbf{e}\\
			\Det-\M.\Nom{}  man(\M{})  see-\Tsg{} \Det-\Acc{}  dog-\Pl{}\\
		\glt ‘The man sees the dogs.’
	\z
\z

In fact, there is never any reflection of the grammatical features of the
object on the verb in \ili{German}. This is different from, say, \ili{Swahili}
where both the subject (always) and the object (under circumstances) trigger
verbal agreement:\footnote{See \citet[235--236]{Creissels2000} for a discussion
    of the conditions favoring object agreement\is{agreement!object agreement}
    marking in \ili{Bantu} languages. This touches on the phenomenon of
    \isi{differential object marking}, which we cannot discuss in any detail
    within the confines of this article. Suffice it to say here that
    \isi{differential object marking} may affect the
    completeness/incompleteness typology in various ways, depending on the
    factor that determines the marking. To take the example of object
    agreement\is{agreement!object agreement} in \ili{Bantu} languages, in some
    cases, where only topics\is{topic} trigger object
    agreement,\is{agreement!topic agreement}\is{agreement!object agreement} one
    might argue that object agreement\is{agreement!object agreement} is
    qualitatively different from subject agreement,\is{agreement!subject
    agreement} and agreement would no longer be complete. On the other hand, in
    cases where object agreement\is{agreement!object agreement} is a function
    of definiteness of the object, as in \ili{Swahili}, we may take object
marking to involve an overt/covert opposition, still within the complete
type.}\newpage

\ea%5
    \label{ex:11.5}\ili{Swahili} \parencite[18]{Barrett-Keach1980}\\
    \gll Juma    a-li-(ki)-soma  ki-tabu\\
        1.Juma  1-\Pst-7-read    7-book\\
    \glt ‘Juma read the book.’
\z

Both \ili{German} and \ili{Swahili} show accusative \isi{alignment} for
agreement, but the languages clearly differ in that in \ili{Swahili} all
grammatical functions participate in agreement, whereas agreement is restricted
to subjects in German. To refer to this difference, we will say that
\ili{Swahili} is complete and German incomplete, for verbal agreement.

Characterizing languages as complete or incomplete is complicated by the
circumstance that morphological oppositions typically involve markedness, where
an unmarked member of the opposition may be zero. This is not a simple matter,
but we proceed on the assumption that the distinction between zero marking and
nonparticipation can be made. In \ili{Swahili}, for instance, it makes sense to
describe the optional presence of the object agreement\is{agreement!object agreement} marker \emph{ki} in \eqref{ex:11.5}
in terms of a \emph{ki}/${\varnothing}$ opposition, so that the object will
participate in agreement even in the case of absence of object agreement\is{agreement!object agreement}
morphology. No such argument can be made for object agreement\is{agreement!object agreement} in
German.\footnote{See \citet[146]{Nordlinger1998} for discussion of this
    question in the context of \ili{Wambaya} object agreement\is{agreement!object agreement}. In \ili{Wambaya}, the
form of the auxiliary\is{auxiliaries} is sensitive to the presence or absence of object
agreement, allowing Nordlinger to conclude that third person object marking is
absent rather than zero.\label{fn:5}}

Completeness or incompleteness can also be demonstrated in the domain of case,
as in \ili{Spanish}, where only objects (under certain conditions) can ever be marked
by the preposition \emph{a}:\footnote{The discussion applies to \ili{Spanish}
    nonpronominal noun phrases only. Case-marking of personal pronouns in
    \ili{Spanish} is complete, with different forms for subject and object
pronouns.\label{fn:6}}

\ea%6
    \label{ex:11.6}\ili{Spanish} {\parencite[80]{Leonetti2004}}\\
    \gll busc-a    (a)  un  médico\\
    look-\Tsg{} \hphantom{(}\Obj{} \Indef{}  doctor\\
    \glt ‘S/he is looking for a (particular) doctor.’
\z

\largerpage[2]
Since subjects are never marked by \emph{a} (or any other particle), we have to
say that only objects participate in case-marking, so that \ili{Spanish}, unlike
\ili{German} and \ili{Coast Tsimshian}, is incomplete for case.\footnote{In
    this connection we should refer to \citegen{Jakobson1936} theory of
    case-marking, in which the nominative\is{nominative case} is basically the case for the noun
    (phrase) in isolation, not signaling any opposition to a marked
    counterpart. If so, the nominative\is{nominative case} may be characterized as absence of case
in the grammar of many languages \citep{Zwart1988}, suggesting that
incompleteness for case is more widespread than commonly assumed.}

To see how completeness complicates \isi{alignment} typology, consider the case of
\ili{Paumarí} (\citealt{ChapmanDerbyshire1991}), a language characterized as
ergative. \ili{Paumarí} has a case-marker \emph{-a} that appears only with
S\textsuperscript{T}:

\ea%7
    \label{ex:11.7}\ili{Paumarí} {\parencite[164]{ChapmanDerbyshire1991}}\\
    \gll Dono-\textbf{a}  bi-ko’diraha-’a-ha      ada    isai    hoariha\\
        Dono-\Erg{}  \Tsg.\Tr{}-pinch-\Asp-\Th:\M{}  \Dem:\M{}  child  other\\
    \glt ‘Dono pinched the other boy.’
\ex%8
    \label{ex:11.8}\ili{Paumarí} {\parencite[163]{ChapmanDerbyshire1991}}\\
    \gll soko-a-ki      hida    mamai\\
        wash-\Detr-\Nth{}  \Dem:\glossF{}  mother\\
    \glt ‘Mother is washing.’
\z

This would appear to be a tell-tale sign of \isi{ergativity} (property \ref{ex:11.3}a).
However, we should be careful, as the case system is incomplete: only the
immediate preverbal noun phrase gets marked
\parencite[250]{ChapmanDerbyshire1991}, and the unmarked word orders are
S\textsuperscript{T}{}-V-O and V-S\textsuperscript{I}. Marked orders do occur,
such as S\textsuperscript{T}{}-O-V \eqref{ex:11.9}, and S\textsuperscript{I}{}-V \eqref{ex:11.10}, and
in these cases the system is again incomplete, with O marked by \emph{{}-ra},
S\textsuperscript{I} by zero, and S\textsuperscript{T} not participating.

\ea%9
    \label{ex:11.9}\ili{Paumarí} {\parencite[197]{ChapmanDerbyshire1991}}\\
    \gll bano    pa'isi  o-sa'a-\textbf{ra}      anani-hi\\
        piranha  small  \Fsg{}-finger-\Obj{}  bite-\Th{}\\
    \glt ‘A small piranha bit my finger.’
\ex%10
    \label{ex:11.10}\ili{Paumarí}
    {\parencite[197]{ChapmanDerbyshire1991}}\\
    \gll Morosi  va-a-kaira-ha-’a-ha\\
        Morosi  \Tpl{}-\Vblz{}-guava-\Prt-\Asp-\Th{}\\
    \glt ‘Morosi c.s. went to get guava.’
\z

The only analysis that unifies the marked and unmarked word orders is a
tripartite analysis, with different markings for each of
S\textsuperscript{T}/S\textsuperscript{I}/O in the immediate preverbal
position. But in unmarked orders \ili{Paumarí} is apparently incomplete rather than
ergative, as only S\textsuperscript{T} participates in case-marking.

We have to be similarly careful in the analysis of \ili{Paumarí} agreement. In the
third person singular, there is a special agreement marker \emph{bi-} for
S\textsuperscript{T}, once more suggesting \isi{ergativity} (see \ref{ex:11.7} vs. \ref{ex:11.8}).
However, in all other feature specifications, there is a single agreement
prefix for S\textsuperscript{T} and S\textsuperscript{I} (e.g.\ \Tpl{}
\emph{va-} in intransitive \eqref{ex:11.10} and transitive \eqref{ex:11.11}).

\ea%11
    \label{ex:11.11} \ili{Paumarí} \parencite[281]{ChapmanDerbyshire1991}\\
    \gll ija'ari  va-ipohi-ki-a \textbf{va}-ka-abada-bada-risaha-khama-ha\\
        people  \Tpl{}-many-\Desc-\Erg{} \Tpl-\Tr.\Distr{}-touch-\Red{}-\Iter-\Distr-\Th{}\\
    \glt ‘Each of the many people was in turn touching him.’
\z

On the other hand, O never triggers person/number agreement on the
verb.\footnote{The object does trigger gender agreement on the verb,
determining the choice of the verb-final theme affix, but so can any other
postverbal noun phrase (\citealt[288]{ChapmanDerbyshire1991}).} It seems,
therefore, that the pattern is basically accusative (agreement only with
S\textsuperscript{T}/S\textsuperscript{I}), and that on top of that verbal
agreement is sensitive to transitivity (in the 3rd person singular).

The example of \ili{Paumarí} shows that the question of completeness must precede the
question of \isi{alignment} typology. It also shows another thing, namely that
special treatment of S\textsuperscript{T} (the ergative property (\ref{ex:11.3}a)) is not
enough to decide that the system is ergative. In the case of \ili{Paumarí} agreement,
we observe that a particular grammatical relation, verb agreement, is
incomplete, applying to subjects only
(S\textsuperscript{T}/S\textsuperscript{I} vs. O). Moreover, the morphological
realization of the relation (at least in the third person singular) shows
sensitivity to transitivity (i.e.\ S\textsuperscript{T} ≠ S\textsuperscript{I}).
To adequately characterize the nature of \ili{Paumarí} case and agreement, then, we
need a more fine-grained descriptive apparatus, one that takes completeness
into account and distinguishes between relations and realizations of these
relations.

\section{Completeness prolegomena}\label{sec:11.3}

The first question to ask is whether a particular grammatical phenomenon
applies to all of S\textsuperscript{T}, S\textsuperscript{I}, and O, or just to
a subset.\footnote{Throughout the discussion, we ignore the grammatical
    function of indirect object, as is standard in the analysis of \isi{alignment}
    typology. However, as a reviewer correctly points out, indirect objects do
    participate in case-marking and verbal agreement. We leave the implications
    of this fact for further research. Likewise, we consider only basic
    transitive and intransitive constructions, and leave the application of the
concept of completeness to ditransitives, causatives, applicatives, etc. for
future research.} If a grammatical process $\pi $ in language λ involves the
complete set \{S\textsuperscript{T}, S\textsuperscript{I}, O\}, we will say
that λ is \textsc{complete} for $\pi$. If the process involves just a subset of
\{S\textsuperscript{T}, S\textsuperscript{I}, O\} the language is
\textsc{incomplete} for that process. If a process in a language λ applies to
none of \{S\textsuperscript{T}, S\textsuperscript{I}, O\}, we will say that λ
is \textsc{neutral} for that process.

If a grammatical process applies to the full set of \{S\textsuperscript{T},
S\textsuperscript{I}, O\}, the next question to ask is whether the process is
realized in identical ways with S\textsuperscript{T}, S\textsuperscript{I}, and
O. Here the possibilities are (where \enquote{=} indicates identical
realization and \enquote{≠} different realization):

\NumTabs{5}
\ea%12
    \label{ex:11.12}Complete types\\
	\ea S\textsuperscript{T} = S\textsuperscript{I} = O \tab{\textit{identical}}
	\ex S\textsuperscript{T} = S\textsuperscript{I} ≠ O \tab{\textit{accusative}}
	\ex S\textsuperscript{T} ≠ S\textsuperscript{I} = O \tab{\textit{ergative}}
	\ex S\textsuperscript{T} = O ≠ S\textsuperscript{I} \tab{\textit{intransitive}}
	\ex S\textsuperscript{T} ≠ S\textsuperscript{I} ≠ O \tab{\textit{tripartite}}
	\z
\z

The names of the types (\ref{ex:11.12}b,c) are derived from the case that
would normally mark the single element.

Next we can illustrate the incomplete \isi{alignment} types, where we have twelve
logically possible combinations, of which the types that involve two
participating \isi{grammatical functions} (a--c) all represent three possibilities
(the \enquote{$>$} indicates which of the elements is morphologically more
marked).

\ea%13
    \label{ex:11.13}Incomplete types\\
    \ea only S\textsuperscript{T}/S\textsuperscript{I}
        \tab{\makebox[0pt][r]{i.}\quad S\textsuperscript{T} = S\textsuperscript{I}} \tab{\textit{subjective}}\\
        \tab{\makebox[0pt][r]{ii.}\quad S\textsuperscript{T} $>$ S\textsuperscript{I}} \tab{\textit{transitive subjective}}\\
        \tab{\makebox[0pt][r]{iii.}\quad S\textsuperscript{T} $<$ S\textsuperscript{I}} \tab{\textit{intransitive subjective}}
	\ex only S\textsuperscript{I}/O
        \tab{\makebox[0pt][r]{i.}\quad S\textsuperscript{I} = O} \tab{\textit{absolutive}}\\
        \tab{\makebox[0pt][r]{ii.}\quad S\textsuperscript{I} $>$ O} \tab{\textit{intransitive absolutive}}\\
        \tab{\makebox[0pt][r]{iii.}\quad S\textsuperscript{I} $<$ O} \tab{\textit{transitive absolutive}}
	\ex only S\textsuperscript{T}/O
        \tab{\makebox[0pt][r]{i.}\quad S\textsuperscript{T} = O} \tab{\textit{transitive}}\\
        \tab{\makebox[0pt][r]{ii.}\quad S\textsuperscript{T} $>$ O} \tab{\textit{subjective transitive}}\\
        \tab{\makebox[0pt][r]{ii.}\quad S\textsuperscript{T} $<$ O} \tab{\textit{objective transitive}}
    \ex only O                    \tab{} \tab{\textit{objective}}
    \ex only S\textsuperscript{T} \tab{} \tab{\textit{narrow ergative}}
    \ex only S\textsuperscript{I} \tab{} \tab{\textit{narrow intransitive}}
	\z
\z

Referring to the \isi{ergativity} properties of \citeauthor{Deal2015}
(\citeyear{Deal2015}; cf.\ \ref{ex:11.3}), we may say that a language
that combines the ergative (\ref{ex:11.3}a) and absolutive
(\ref{ex:11.3}b) properties for some grammatical generalization γ is
complete for γ and in fact ergative (\ref{ex:11.12}c).  But a language that
has the ergative property (\ref{ex:11.3}a) but not the absolutive property
(\ref{ex:11.3}b) for γ can be either complete or incomplete for γ,
depending on whether O participates in γ. If so, the language is complete for γ
and in fact tripartite (\ref{ex:11.12}e, e.g.\ \ili{Paumarí} for case),
but if not, the language is incomplete for γ, and in fact subjective
(\ref{ex:11.13}a, e.g.\ \ili{Paumarí} for agreement).

Both tripartite and what we have called subjective are typically considered to
be ergative variants (\enquote{three-way ergative}, cf.\ \citealt{Deal2015}),
perhaps because they are not obviously affiliated with the accusative type. But
from the perspective proposed here, considering completeness first, we may
question which variants among the complete and incomplete types might be
meaningfully grouped together under the rubrics of \enquote{ergative} or
\enquote{accusative}.  It seems to us that this grouping should be as in \tabref{tab:fromex:11.14},
calling the groupings \enquote{families}.

\begin{table}
\caption{Alignment types\label{tab:fromex:11.14}}
    \begin{tabular}{llll}
    \lsptoprule
    family               & complete types                  & incomplete types          & other types \\
    \midrule
    \textsc{accusative}  & accusative (\ref{ex:11.12}b)   & subjective (\ref{ex:11.13}a)          & \\
                         &                                 & objective  (\ref{ex:11.13}d)          & \\
    \textsc{ergative}    & ergative (\ref{ex:11.12}c)  & absolutive (\ref{ex:11.13}b)          & \\
                         &                                 & narrow ergative (\ref{ex:11.13}e)     & \\
    \textsc{indifferent} & identical (\ref{ex:11.12}a)    & & neutral \\
                         & tripartite (\ref{ex:11.12}e)                &                           & \\
    \textsc{residual}    & intransitive (\ref{ex:11.12}d) & transitive (\ref{ex:11.13}c)          & \\
                         &                                 & narrow intransitive (\ref{ex:11.13}f) & \\
    \lspbottomrule
    \end{tabular}
\end{table}

To illustrate the logic behind this grouping, consider the subjective type
(\ref{ex:11.13}a). This is one of the incomplete types, where only
S\textsuperscript{T}/S\textsuperscript{I} participate in γ. This creates a
subject--object opposition typical of the accusative family of types. Within the
subjective type, further divisions are possible, depending on whether γ is
realized identically for S\textsuperscript{T} and S\textsuperscript{I} or not.
What \citet{Deal2015} calls the ergative property (\ref{ex:11.3}a) may in fact be
identified as (transitive) subjective in those cases where the language is
incomplete for the relevant grammatical generalization.

\section{Some illustrations}\label{sec:11.4}

In this section we illustrate the completeness-based typology for the data
introduced above and for a number of other cases from the
literature.\footnote{This research started as an investigation of agreement in
    split-ergative languages, for which we used a convenience sample based on
    data extracted from the \emph{World Atlas of Language Structures}
    (\citealt{DryHas2013}, accessed April 2014). The languages included in the
    sample were: \ili{Chamorro} (Austronesian), \ili{Georgian} (Kartvelian),
    \ili{West Greenlandic} (Eskimo-Aleut), \ili{Hunzib} (North Caucasian),
    \ili{Lak} (North Caucasian), \ili{Marathi} (\ili{Indo-European}), \ili{Ngiyambaa}
    (Australian), \ili{Paumarí} (Arauan), \ili{Pitjantjatjara} (Australian),
    \ili{Suena} (Trans New Guinea), \ili{Coast Tsimshian} (Penutian),
    \ili{Wambaya} (Australian), \ili{Yidiny} (Australian), \ili{Yup’ik}
    (Eskimo-Aleut). These were supplemented by data from \ili{Nez Perce}
(Penutian) and \ili{Shipibo} (Panoan), and from familiar \ili{Indo-European}
languages such as \ili{German} and \ili{Spanish}. No claim of representative coverage of
the languages of the world is made.}

German \eqref{ex:11.1} is complete for case and in fact
\textsc{accusative}, and incomplete for agreement, in fact \textsc{subjective}
(as shown in \ref{ex:11.4}). Subjective being in the accusative family
(cf.\ \tabref{tab:fromex:11.14}), we may identify \ili{German} as an accusative
language.

Coast Tsimshian \eqref{ex:11.2} is complete for case and in fact
\textsc{ergative}.  However, the phenomena are considerably more complicated,
as discussed in great detail in \citet{Mulder1994}.\footnote{Our data reflect
    the reduced system observed by Mulder in everyday speech
\citep[39]{Mulder1994}.} First, tense and aspect are relevant (p.\ 85), and
secondly, things differ when the noun phrase is a name (p.\ 39). In the past,
the ergative predicate connective \emph{-da} becomes \emph{-a}, yielding an
\textsc{identical} pattern (p.\ 85). With names the cake is cut differently:
the predicate marker for S\textsuperscript{T}/S\textsuperscript{I} is
\emph{-as} and for O \emph{-at}, yielding an \textsc{accusative} pattern; but
in the imperfective/present, S\textsuperscript{T} has its own predicate marker
\emph{-dit}, yielding a \textsc{tripartite} pattern (p.\ 40--41). So while
\ili{Coast Tsimshian} is invariably complete for case, it ranges over four
different complete types, leaving only the (rare) intransitive type unused. To
complicate matters further, while free pronouns behave like (non-name) noun
phrases (p.\ 66), clitic pronouns have their own system (p.\ 54--55). Clitics
are taken from one of three series, called subjective (preverbal), objective
(postverbal) and definite objective (postverbal). In the subjunctive, these are
organized along \textsc{ergative} lines, S\textsuperscript{T} taken from the
subjective series and S\textsuperscript{I}/O from the objective series. In the
indicative, various types occur depending on the relative \isi{animacy} of
S\textsuperscript{T}/S\textsuperscript{I}/O, including even the rare
\textsc{intransitive} type (S\textsuperscript{T}/O: S\textsuperscript{I}). So
much for \ili{Coast Tsimshian} case. Agreement is much more restricted, being
controlled only by the person feature of S\textsuperscript{T}, and limited to
the past tense (\textsc{narrow ergative}) (p.\ 68); outside the past, no verbal
agreement occurs (\textsc{neutral}) (p.\ 69).\footnote{We take apparent cases of
    number agreement in \ili{Coast Tsimshian} to instantiate the phenomenon of
pluractionality (one of the \enquote{ubiquitous} \isi{ergativity} traits of
\citealt{Queixalos2013}, cf.\ \cref{fn:3}).} All in all \ili{Coast Tsimshian}
is predominantly ergative, though sometimes veering to one of the other
complete types.

Swahili \eqref{ex:11.5} is \textsc{neutral} for case and complete, in fact
\textsc{accusative}, for agreement.

Spanish \eqref{ex:11.6} is incomplete for case (modulo \cref{fn:6}), in fact \textsc{objective}.
It is also incomplete for agreement, in fact \textsc{subjective}. All in all a
clear accusative language.

Paumarí (\ref{ex:11.7}--\ref{ex:11.11}, cf.\ \citealt{ChapmanDerbyshire1991}) is complicated, as we
have seen, at least for case. If we consider unmarked orders only, \ili{Paumarí} is
incomplete for case, in fact \textsc{narrow ergative}. If we include marked
orders also, \ili{Paumarí} is a combination of the \textsc{tripartite} and the
\textsc{neutral} types: the immediate preverbal element has different markings
for each of S\textsuperscript{T}/S\textsuperscript{I}/O, but in all other
positions no case-marking occurs. Case-marking for pronouns is even more
restricted, affecting only O (which is always in preverbal position), an
instantiation of the \textsc{objective} type. Verbal agreement is incomplete,
being controlled by S\textsuperscript{T}/S\textsuperscript{I} only, i.e.
\textsc{subjective}; only if the subject is \Tsg{} do we get a further
specialization (\emph{bi-} for S\textsuperscript{T}, zero for
S\textsuperscript{I}), making the language \textsc{transitive subjective} for
\Tsg{} agreement (p.\ 287).

In \ili{Wambaya} (\citealt{Nordlinger1998}; cf.\ fn.~\ref{fn:5}), case is marked on
S\textsuperscript{T} and obliques, and zero on S\textsuperscript{I}/O (p.\ 80);
since the language is rich in case (p.\ 81), it is more plausible to think of the
absolutive as being zero than absent. This makes \ili{Wambaya} complete, in fact
\textsc{ergative}, for case.\footnote{The ergative pattern is also visible in
the nouns’ gender markings, which are taken from one of two series, absolutive
(for S\textsuperscript{I}/O) and non-absolutive (elsewhere).\label{fn:13}} With
pronouns, though, we do not see an
S\textsuperscript{T}/S\textsuperscript{I}{}-distinction: in the singular all
subject and object pronouns are alike (though different from oblique pronouns),
hence \textsc{identical}, and in the dual and plural subject pronouns differ
from object and oblique pronouns, hence \textsc{accusative} (p.~126). Verbal
agreement is expressed by bound pronouns on the auxiliary\is{auxiliaries}, and is controlled by
both subjects (identically for S\textsuperscript{T}/S\textsuperscript{I}) and
objects in first/second person, hence complete and in fact \textsc{accusative}
(p.\ 139). In the 3rd person, no object agreement\is{agreement!object agreement} shows up, and
\citet{Nordlinger1998} has an ingenious argument showing that object agreement\is{agreement!object agreement}
is absent rather than zero (see \cref{fn:5}). For 3rd person agreement, then,
Wambaya is incomplete, in fact \textsc{subjective}. Moreover, in \Tsg{} there
is a special agreement marker for transitive subjects, making the type more
particularly \textsc{transitive subjective}. All in all \ili{Wambaya} seems clearly
ergative for case of noun phrases, and accusative for case of pronouns and for
agreement.

\begin{sloppypar}
To add another example not mentioned so far, but typologically interesting and
well represented in the \isi{ergativity} literature (e.g.\ \citealt{Legate2008,Barany2015}), \ili{Marathi} \citep{Pandharipande1997} shows a sensitivity to the
tense/aspect of the clause: outside the past tense, and ignoring oblique
subject constructions, \ili{Marathi} has no case-marking for
S\textsuperscript{T}/S\textsuperscript{I} and case-marking by \emph{-la} for O
(under conditions) (p.\ 283f).\footnote{The object is marked by \emph{-la},
regardless of tense/aspect, when it refers to a human or specific indefinite
entity \citep[287--288]{Pandharipande1997}.} This puts the language in the
accusative ballpark (i.e.\ \textsc{accusative} or \textsc{objective}, depending
on whether we take subject case to be zero or absent). In the past tense, a 3rd
person S\textsuperscript{T} is marked by \emph{-ne}, making the system
\textsc{tripartite} (if complete) or \textsc{transitive} (if incomplete; p.\
284); with first/second person subjects the language remains accusative/objective
also in the past (p.~284).\footnote{Here we differ from \citet{Legate2008} and
    \textcite{Barany2015}, who assume zero-marked \isi{ergative case} for first/second
    person subjects in the past tense. The Legate/Bárány analysis is supported
    by the observation that first/second person subjects do not trigger agreement in
    the past tense (\citealt[130]{Pandharipande1997}, although they may in some
varieties, see the references in \cref{fn:13}), which we may have to analyse as a
form of analogical leveling.\label{fn:15}} Verbal agreement is triggered by both subjects
and objects, though typically restricted to a single controller, according to a
hierarchy that prefers subject agreement\is{agreement!subject agreement} over object agreement\is{agreement!object agreement}
(p.\ 446).\footnote{The restriction applies to Standard \ili{Marathi}, but not to
    certain varieties, such as Pune \ili{Marathi} and Nagpuri \ili{Marathi}, where we see a
    combination of subject and object agreement\is{agreement!object agreement}. See \textcite[262]{Bloch1970}
    and \textcite[412]{Pandharipande1997}. See also \textcite[250]{Magier1983}
    for \ili{Marwari}, \textcite[216]{VerbekeWillems2012} for \ili{Kashmiri}, and and
\textcite{GroszPatelGrosz2014} for \ili{Kutchi Gujarati}.\label{fn:16}}
Furthermore, oblique elements (including ergative elements) never trigger
agreement (p.\ 446). This restriction has the effect that a 3rd person
S\textsuperscript{T} does not control verbal agreement in the past tense, so
that object agreement\is{agreement!object agreement} resurfaces. Other than that, there is no sensitivity to
transitivity, making the system \textsc{accusative}. All in all, \ili{Marathi}
seems very much in the accusative corner, and we assume this carries over to
related languages with comparable typological features (see also
\citealt{VerbekeWillems2012}).
\end{sloppypar}

Finally, consider the case of \ili{Nez Perce}, as analysed in \citet{Deal2010}.
Nez Perce has both caseless clauses (\textsc{neutral}) and case-marked clauses,
where S\textsuperscript{T} is mark\-ed by \emph{-(n)im}, O by \emph{-ne}, and
S\textsuperscript{I} is unmarked (p.\ 74--75). \citet{Deal2010} shows that the
choice between the two systems hinges on the presence of object agreement\is{agreement!object agreement} on
the verb, object agreement\is{agreement!object agreement} forcing the case-marked variant.
\citet{Lindenbergh2015} suggests that the logic entails that the unmarked case
on S\textsuperscript{I} (in the case-marked variant) is absence of case rather
than presence of zero case, since intransitive clauses by definition lack
object agreement. This would make \ili{Nez Perce} in the case-marked variant
incomplete, in fact \textsc{transitive}, for case. With pronouns, a distinction
between S\textsuperscript{T} and S\textsuperscript{I} exists only in the 3rd
person, 1st and 2nd person showing no subject case even in case-marked clauses
(p.\ 78). Depending on whether case on S\textsuperscript{T} is zero or absent,
the system for case of pronouns would remain transitive or be reduced to
\textsc{objective}.\footnote{\citet{Deal2010} describes it as
nominative--accusative (our accusative), assuming the system to be complete,
with zero marking on unmarked subjects.}  Verbal agreement in \ili{Nez Perce} is
triggered by subjects in all (i.e.\ caseless and case-marked) clauses, without
any sensitivity to transitivity. Object agreement, on the other hand, is
restricted to case-marked clauses (p.\ 79--80). Inevitably, agreement in caseless
clauses, lacking object agreement\is{agreement!object agreement}, is of the incomplete variety, in fact
\textsc{subjective}, and agreement in case-marked clauses is complete, in fact
\textsc{accusative}.\footnote{First/second person subjects and objects are not
    overtly marked, but \citet{Lindenbergh2015} argues that agreement with
    first/second person objects must be zero rather than absent, to maintain
    Deal’s generalization that \isi{ergative case} is conditioned by the presence of
    object agreement\is{agreement!object agreement}, given the fact that ergative subjects do occur with
first/second person objects. A fortiori, then, we may assume first/second
person subject agreement\is{agreement!subject agreement} to be zero as well.}

\section{Some consequences}\label{sec:11.5}

\subsection{The ergative property}\label{sec:11.5.1}

It is now clear that special behavior of the transitive clause subject
S\textsuperscript{T} (i.e.\ the ergative property (\ref{ex:11.3}a)) can come about in
various ways, depending on completeness and morphological realization.

If a language is complete for a grammatical phenomenon γ, and γ is realized in
one way on S\textsuperscript{T} and in a different way on
S\textsuperscript{I}/O, the language is complete and in fact ergative for γ. We
saw this illustrated for case in \ili{Coast Tsimshian} \eqref{ex:11.2}. \ili{Wambaya} is also ergative
in this sense, at least for case on (nonpronominal) noun phrases. Languages
that are complete and ergative for agreement are also widely attested,
illustrated here for Malimiut Iñupiaq \citep{Lanz2010}:

\ea%15
    \label{ex:11.15}\ili{Malimiut Iñupiaq} {\parencite[188]{Lanz2010}}\\
	\ea
		\gll iγlaq-tu-ŋa\\
			laugh-\Intr.\Ind-\Fsg{}\\
		\glt ‘I am laughing.’
	\ex
		\gll aŋuti-m  tusa:-γ-a:-ŋa\\
			man-\Erg{}  hear-\Tr.\Ind{}-\Tsg{}-\Fsg{}\\
		\glt ‘The man hears me.’
	\z
\z

Another way in which the ergative property may arise is when the language is
incomplete for γ, with S\textsuperscript{I}/O not participating. This is the
narrow ergative type (\ref{ex:11.13}e). We saw this for case in \ili{Paumarí} unmarked word
orders (where only the preverbal element S\textsuperscript{T} participates in
case-marking) and for agreement in the \ili{Coast Tsimshian} past
tense.\footnote{\citet[305]{Bobaljik2008} takes this narrow ergative agreement
type to be absent from the languages of the world.} This narrow ergative type
is still within the ergative family (cf.\ \tabref{tab:fromex:11.14}).

However, the ergative property may also arise in the accusative family, in
particular when the language is incomplete with only subjects
(S\textsuperscript{T}/S\textsuperscript{I}) participating in γ, and γ being
realized differently in S\textsuperscript{T} and S\textsuperscript{I}
(transitive subjective, if S\textsuperscript{T} is more marked than
S\textsuperscript{I}, cf.\ (13aii)). We saw this with \Tsg{} agreement in
Paumarí and \ili{Wambaya}. In \ili{Paumarí}, O never controls agreement, which is clearly a
subjective grammatical feature then, and while
S\textsuperscript{T}/S\textsuperscript{I} mostly control agreement in identical
fashion, there is further specialization when S\textsuperscript{T} is \Tsg{}.
Wambaya is in fact complete for agreement except in the 3rd person (see note
5), where agreement is incomplete, in fact subjective, and there too we see
special treatment of S\textsuperscript{T}.

Our limited data do not show any clear cases of transitive subjective
case-marking at this point, but cases where only S\textsuperscript{T} is
case-marked are well-attested (e.g.\ in Mizo; \citealt{Chhangte1989}). These
are typically described as ergative, and would be narrow ergative in our
typology.  In principle we cannot exclude that this type is in fact transitive
subjective, with a marked vs. zero opposition between S\textsuperscript{T} and
S\textsuperscript{I}, and O not participating. But the subjective type, very
common for agreement, seems rare for case, where morphological realization,
when incomplete, appears to gravitate towards O rather than
S\textsuperscript{T}/S\textsuperscript{I}.

\subsection{The absolutive property}\label{sec:11.5.2}

The absolutive property (\ref{ex:11.3}b), like the ergative property
(\ref{ex:11.3}a), shows up in both complete and incomplete types, but all
these types stay within the ergative family (\tabref{tab:fromex:11.14}).

Identical treatment of S\textsuperscript{I} and O is one of the characteristics
of the complete ergative type (\ref{ex:11.12}c), which we have seen for
case in \ili{Coast Tsimshian} \eqref{ex:11.2} and also in \ili{Wambaya}
(except for pronouns). For agreement the complete ergative type is illustrated
in \ili{Malimiut Iñupiaq} \eqref{ex:11.15}.

\largerpage[2]
The incomplete absolutive type (\ref{ex:11.13}b) shows up when
S\textsuperscript{T} does not participate in γ. This type is not
represented by any of the languages discussed so far, neither for case, nor for
agreement. We know of no languages that show the absolutive pattern for
case-marking.\footnote{As noted by an anonymous reviewer, a case in point may
    be initial consonant mutation in \ili{Nias}, which
    \parencite[342--343]{Brown2001} shows to be a \gls{GF}-marking device
applying to S\textsuperscript{I} and O, but not S\textsuperscript{T}.} On the
other hand, the absolutive pattern for agreement is well attested, e.g.\ in
\ili{Tsez} \parencite[344--345]{Polinsky2014}:

\ea%16
    \label{ex:11.16} \ili{Tsez} \parencite[345]{Polinsky2014}\\
	\ea
		\gll isi        y-ay-s \\
            snow(\Ii):\Abs{} \Ii-come-\Pst.\Evid{}\\
		\glt ‘It snowed.’
	\ex
		\gll uži-z-ä t’ek y-is-si\\
			boy(\Ig)-\Pl.\Obl-\Erg{}  book(\Ii)  \Ii-take-\Pst.\Evid{}\\
		\glt ‘The boys bought a book.’
	\z
\z

Agreement here is gender/number agreement, controlled by S\textsuperscript{I}
(\ref{ex:11.16}a) or O (\ref{ex:11.16}b).

Languages of the type of \ili{Marathi}, discussed above, are also usually included in
this category (e.g.\ \citealt[305]{Bobaljik2008}). In these languages,
agreement is normally controlled by S\textsuperscript{T}/S\textsuperscript{I},
but in the past tense, where S\textsuperscript{T} is marked with ergative case,
S\textsuperscript{T} fails to control agreement, which is then controlled by O
instead. In our terms, the language alternates between two incomplete types
(for agreement), subjective (default) and absolutive (in the past).

However, two factors conspire to yield the absolutive pattern here: (i)
morphologically case-marked noun phrases in \ili{Marathi} never control agreement,
and (ii) the verb must show agreement with a single controller (in most
varieties, see \cref{fn:16}). That morphologically case-marked noun phrases do not
control agreement is a general rule, applying not just to ergative subjects but
also to oblique elements and accusative-marked objects
\citep[446]{Pandharipande1997}. That the verb must show agreement is evidenced
by the appearance of default agreement in the absence of an eligible
controller. Therefore, one way to explain O-controlled agreement in \ili{Marathi}
would be to say that O takes over when S\textsuperscript{T}, because of its
ergative case, is no longer eligible, as an option preferred over the last
resort default agreement. On this explanation, agreement in Marathi-type
languages is complete, and the fact that O controls agreement only secondarily
when S\textsuperscript{T} is not available as an agreement controller suggests
an organization along the lines of accusativity.\footnote{This leaves the
    \ili{Tsez} type as the only clear example we have seen of agreement along
    absolutive lines. Agreement in \ili{Tsez} is gender/number agreement, a
    phenomenon found across Northwest Caucasian, always triggered by the
    absolutive element alone. Person agreement on the other hand is very
    limited in Northwest Caucasian, and completely absent in \ili{Tsez}, but where it
    exists, as in \ili{Hunzib} (\citealt{VandenBerg1995}), it is sensitive to a
    person hierarchy and may be triggered by various \glspl{GF}\is{grammatical functions}. This suggests that the distinction between person agreement and
    number/gender agreement may lead to different agreement \isi{alignment} patterns
within a single language.\label{fn:21}}

\subsection{The tripartite type}\label{sec:11.5.3}

In the tripartite system (\ref{ex:11.12}e), S\textsuperscript{T}, S\textsuperscript{I} and
O are each treated differently. We saw some examples of this above: the
predicate connectives with names in \ili{Coast Tsimshian} imperfective and present
tense clauses are \emph{-dit} (S\textsuperscript{T}), \emph{-as}
(S\textsuperscript{I}) and \emph{-at} (O), and \ili{Paumarí} has different
case-markers for S\textsuperscript{T} (\emph{-a}), S\textsuperscript{I}
(\emph{-ra}) and O (zero) in immediate preverbal position. We have seen no
cases of tripartite agreement systems in our limited data.

With all \glspl{GF}\is{grammatical functions} participating in tripartite
case-marking, this \isi{alignment} type is complete, and it seems to combine elements
of both ergative (marked S\textsuperscript{T}) and accusative (marked O)
alignment patterns.  Above, we have grouped it in the indifferent family though
(see \tabref{tab:fromex:11.14}), the family of \isi{alignment} types that treat all
\glspl{GF}\is{grammatical functions} on a par (i.e.\ all the same or all
different).

\begin{sloppypar}
Tripartite \isi{alignment} is much rarer than accusative or ergative\is{ergative alignment} \isi{alignment}
\citep[40]{Dixon1994}, and the cases we have seen invariably involve
differential marking as a function of a noun phrase \isi{animacy} hierarchy. Consider
the example of \ili{Kham} as discussed in \textcite{Watters2002}:
\end{sloppypar}

\ea%17
\label{ex:11.17} \ili{Kham} \parencite[66--67]{Watters2002}\\
	\ea
		\gll la:-${\varnothing}$        si-ke\\
			leopard-\Abs{}  die-\Pfv{}\\
		\glt ‘The leopard died.’
	\ex
		\gll no:-\textbf{ye}    la:-${\varnothing}$        səih-ke-o\\
			\Tsg.\Erg{}  leopard-\Abs{}  kill-\Pfv{}-\Tsg{}\\
		\glt ‘He killed a leopard.’
	\ex
		\gll ŋa:-${\varnothing}$    no-\textbf{lai}    ŋa-r:h-ke\\
			\Fsg-\Nom{} \Tsg-\Acc{} \Fsg{}-see-\Pfv{}\\
		\glt ‘I saw him.’
	\z
\z

As can be seen, S\textsuperscript{T} receives a special case-marking in (\ref{ex:11.17}b),
while S\textsuperscript{I} in (\ref{ex:11.17}a) and O in (\ref{ex:11.17}b) are zero-marked. However,
the ergative marking is absent with S\textsuperscript{T} in (\ref{ex:11.17}c), and O is
marked by a special \isi{accusative case} in (\ref{ex:11.17}c), yielding what looks like an
accusative pattern. The ergative and accusative patterns can also be mixed:

\ea%18
    \label{ex:11.18} \ili{Kham}\\
	\ea
		\gll ge:-\textbf{${\varnothing}$}    em-tə    mi:-rə-\textbf{${\varnothing}$}      ge-ma-ra-dəi-ye\\
			we-\Nom{}  road-on  person-\Pl.\Abs{} \Fpl-\Neg-\Tpl{}-find-\Ipfv{}\\
		\glt ‘We met no people on the way.’
	\ex
		\gll g:h-\textbf{ye}  ŋa-\textbf{lai}  duhp-na-ke-o\\
			ox-\Erg{}    I-\Acc{}  butt-\Fsg-\Pfv-\Tsg{}\\
		\glt ‘The ox butted me.’
	\z
\z

As \citet[69]{Watters2002} explains, the marking of both S\textsuperscript{T}
and O in \ili{Kham} is sensitive to \isi{animacy}, such that low \isi{animacy}
S\textsuperscript{T} and high \isi{animacy} O require marking.\footnote{Since marked
and unmarked S\textsuperscript{T} and O can be freely mixed, the marking does
not reflect a subject--object dependency: O is not marked because it is high
animate relative to S\textsuperscript{T}, or S\textsuperscript{T} because it is
low animate relative to O, but marking reflects high or low \isi{animacy} relative to
the expected \isi{animacy} of the relevant \gls{GF}. Note that the cut-off point
in the \isi{animacy} hierarchy is different for S\textsuperscript{T} and O, as 3rd
person definite elements count as low for the subject hierarchy and as high for
the object hierarchy (so they will always be marked in S\textsuperscript{T}/O
position).} Interestingly, S\textsuperscript{I} is never marked, regardless of
animacy, suggesting that \ili{Kham} case-marking is more properly characterized
as incomplete, involving only S\textsuperscript{T}/O, hence of the type we
called transitive (\ref{ex:11.13}a).\footnote{On the analysis of
\citet{Lindenbergh2015}, this applies to \ili{Nez Perce}, another language
described as tripartite for case, as well.} Differential subject or object
marking then decides whether the construction at hand is subjective,
(\ref{ex:11.17}b), or objective transitive, (\ref{ex:11.17}c), or in
fact both, as in (\ref{ex:11.18}b).

Animacy sensitivity seems to be invariably involved in tripartite case-marking
\citep{Zwart2006a}. In principle, tripartite \isi{alignment} may be incomplete,
as in \ili{Kham}, or may be a hierarchy-driven adjustment of an accusative
system (with special marking for S\textsuperscript{T} by differential subject
marking) or of an ergative system (with special marking for O by differential
object marking).  We leave this as an avenue for further study.

\subsection{Case and agreement}\label{sec:11.5.4}

A separate question is how case-marking and agreement control are related, if
at all. Our limited data suggest that there is no straightforward connection.

One possible connection would be that completeness in case entails completeness
in agreement (or vice versa). This, however, does not seem to be the case. As
we have seen, \ili{Coast Tsimshian} is complete for case (in various ways), but
at best incomplete (in fact, narrow ergative) for agreement, and even neutral
outside the past tense. Likewise, \ili{Wambaya} is complete for case, but not
always for agreement (accepting Nordlinger’s argument that 3rd person object
agreement is absent rather than zero, see \cref{fn:5}). Conversely, \ili{Nez
Perce} is incomplete for case in case-marked clauses (accepting Lindenbergh’s
argument that case on S\textsuperscript{I} is absent rather than zero, see
\Cref{sec:11.4}), but complete for agreement.

We can also ask whether a language that is incomplete for case will show the
same incompleteness for agreement. Again, this does not seem to be the case.
Spanish, for instance, is incomplete for case and agreement, but objective for
case and subjective for agreement. Likewise, \ili{Paumarí} is incomplete for
case in an unusual way, restricting case-marking to the immediate preverbal
element, whereas agreement is incomplete in the more standard subjective
\isi{alignment} type.

Our data also allow us to track agreement \isi{alignment} as a potential
function of case \isi{alignment} by differentiating between case for full noun
phrases and pronouns. As we have seen, case \isi{alignment} often differs
between full noun phrases and pronouns, at least in the languages discussed
here. It turns out, then, that in these languages agreement \isi{alignment}
does not typically covary with the case \isi{alignment} of noun phrases and
pronouns. For example, in \ili{Paumarí} the case \isi{alignment} type becomes
objective with pronouns, but the agreement \isi{alignment} type remains subjective.

One possible connection between case and agreement \isi{alignment} could be
that incomplete case \isi{alignment} and incomplete agreement \isi{alignment}
are each other’s inverse. This would be the case if a language is narrow
ergative for case and absolutive for agreement, or objective for case and
subjective for agreement.  This would require that we analyse \ili{Tsez}, which
has absolutive agreement, as (incomplete) narrow ergative for case, rather than
(complete) ergative, an unlikely move given the rich case system of \ili{Tsez}
\citep{Polinsky2014}.\footnote{Another case could be \ili{Marathi} (and similar
    languages), which shows agreement controlled by S\textsuperscript{I}/O in
    the past tense, where S\textsuperscript{T} is ergative. However, the
    situation of \ili{Marathi} can be analyzed differently, as discussed in the
    text (\Cref{sec:11.4}).  Also, the absolutive-looking agreement pattern shows up in all past
    tense clauses, even when S\textsuperscript{T} is not ergative (as with
first and second person pronouns, see \cref{fn:15}).} Objective case and
subjective agreement do go hand in hand in some cases discussed here, such as
\ili{Spanish} and \ili{Paumarí} (with object pronouns), but subjective
agreement being relatively widespread, we cannot ascribe these cases to a
systematic mirror image relation between incomplete case and agreement types.

In short, the data we have looked at do not allow us to set up any
correspondence between case and agreement \isi{alignment}.

\subsection{Syntactic ergativity}\label{sec:11.5.5}

Our discussion so far has been restricted to morphosyntactic \isi{alignment} in
the domains of case and agreement. When ergative\is{ergative alignment} \isi{alignment} is observed
for some syntactic process, we speak of syntactic ergativity\is{ergativity!syntactic ergativity}
(see \citealt{Deal2016} for a survey of the phenomena and the issues involved).

Syntactic \isi{ergativity} can take various forms: ergative S\textsuperscript{T} may
not participate in a particular syntactic process \eqref{ex:11.19}, or the elements
participating in the syntactic process are tracked morphologically (e.g.\ on
the verb) along an ergative\is{ergative alignment} \isi{alignment} pattern
\eqref{ex:11.20}.

\ea%19
    \label{ex:11.19}\ili{West Greenlandic} {\parencite[55]{Bittner1994}}\\
    \ea[]{%
		\gll miiqqa-t [ — sila-mi pinnguar-tu-t ]\\
			child-\Pl.\Abs{} {}  〈S\textsuperscript{I}〉  outdoors-\Loc{}  play-\Rel.\Intr-Pl{}\\
        \glt ‘the children who are playing outside’}
        \ex[]{
		\gll miiqqa-t    [  Juuna-p    —    paari-sa-i  ]\\
			child-\Pl.\Abs{} {} Juuna-\Erg{}  〈O〉  look.after-\Rel.\Tr-\Tsg.\Pl{}\\
        \glt ‘the children that Juuna is looking after’}
	\ex[*]{%
    \gll angut      [  —    aallaat    tigu-sima-sa-a  ]\\
			man.\Abs{}  {} 〈S\textsuperscript{T}〉  gun.\Abs{}  take-\Pfv-\Rel.\Tr-\Tsg.\Sg{}\\
        \glt intended: ‘the man who took the gun’}
	\z
\ex%20
    \label{ex:11.20}Tongan \parencite[81]{Otsuka2006}\\
    \ea[]{%
		\gll e    fefine  [  na'e    —    tangi  ]\\
            \Def{}  woman  {}  \Pst{} 〈S\textsuperscript{I}〉{} cry\\
        \glt ‘the woman who cried’}
    \ex[]{%
		\gll e    fefine  [  na'e  fili      —    ‘e    Sione  ]\\
			\Def{}  woman  {}  \Pst{}  choose    〈O〉  \Erg{}  John\\
        \glt ‘the woman who John chose’}
    \ex[*]{
		\gll e    fefine  [  na'e  fili      ‘a    Sione  —    ]\\
			\Def{}  woman  {}  \Pst{}  choose    \Abs{}  John  〈S\textsuperscript{T}〉\\
        \glt intended: ‘the woman who chose John’}
	\z
\z

In both \ili{West Greenlandic} \eqref{ex:11.19} and \ili{Tongan}
\eqref{ex:11.20}, straightforward relativization of S\textsuperscript{T} is
ungrammatical. In West Greenlandic, the solution is to detransitivize the
clause to be relativized, by application of the antipassive:

\ea%21
    \label{ex:11.21}\ili{West Greenlandic} {\parencite[58]{Bittner1994}}\\
    \gll angut      [  —    aallaam-mik    tigu-si-sima-su-q  ]\\
        man.\Abs{}  {}    〈S\textsuperscript{I}〉  gun.\Ins{}   take-\Antip{}-\Pfv-\Rel.\Intr{}-\Sg{}\\
    \glt ‘the man who took the gun’
    \z

The antipassive turns a transitive clause into an intransitive clause, so that
the relativized subject becomes S\textsuperscript{I} instead of
S\textsuperscript{T}. Effectively, then, this type of syntactic ergativity\is{ergativity!syntactic ergativity} is
incomplete, in fact absolutive (\ref{ex:11.13}b).

In \ili{Tongan}, the solution is to morphologically mark relativization of
S\textsuperscript{T} (by \emph{ne}):

\ea%22
    \label{ex:11.22}\ili{Tongan} {\parencite[81]{Otsuka2006}}\\
    \gll e    fefine  [  na'a  ne  fili      ‘a    Sione  —    ]\\
        \Def{}  woman {}   \Pst{}  \Tsg{}  choose    \Abs{}  John  〈S\textsuperscript{T}〉\\
    \glt ‘the woman who chose John’
    \z

In this type, relativization is complete and in fact ergative
(\ref{ex:11.12}c). Other languages that show morphological tracking of
A$'$-moved elements along ergative lines include \ili{Abaza}, \ili{Selayarese},
and \ili{Gitksan} \citep[180--181]{Deal2016}.

From our perspective, these two types of syntactic ergativity\is{ergativity!syntactic ergativity} represent two
different \isi{alignment} types, both within the ergative family (\tabref{tab:fromex:11.14}), namely
absolutive (affecting only S\textsuperscript{I}/O), for \ili{West Greenlandic}, and
ergative (S\textsuperscript{T} vs. S\textsuperscript{I}/O), for \ili{Tongan}.

\subsection{Ergativity generalizations}\label{sec:11.5.6}

It has been observed that syntactic ergativity\is{ergativity!syntactic ergativity} is limited to
morphologically ergative\is{ergativity!morphological ergativity} languages
\citep[172]{Dixon1994}. In other words, morphological alignments of the
accusative family types (cf.\ \tabref{tab:fromex:11.14}) do not give rise to
syntactic differentiation of S\textsuperscript{T} and S\textsuperscript{I}. One
way to explain this would be to assume that accusative \isi{alignment} (of any
type) is a function of syntactic derivation, merging subjects of all stripe in
identical positions. Conversely, ergative\is{ergative alignment} \isi{alignment} (of any type), while
not reflecting any different syntactic derivation, must be the result of an
additional, marked process, which is reflected in morphology, and possibly
(though by no means necessarily) also in syntax.

From this perspective, it is interesting to note that morphological
differentiation between S\textsuperscript{T} and S\textsuperscript{I} is not
wholly absent in the accusative \isi{alignment} types. In particular, the
transitive subjective type (13aii), while being in the accusative family, does
show transitivity sensitivity leading to marked S\textsuperscript{T} (we saw
this in third person agreement in \ili{Paumarí} and \ili{Wambaya}). It would be
interesting to see if this morphological differentiation has syntactic
side-effects, but these questions have to be put off for now.

More generally, typological universals related to \isi{ergativity} (as discussed
recently in \citealt{Sheehan2014} and \citealt{Deal2015}) may be evaluated anew
in the context of the more refined \isi{alignment} typology contemplated here. For
example, \citet[668]{Deal2015} observes that \isi{ergative case} is invariably
overtly marked. This follows trivially in two of the three ergative family
alignment types (cf.\ \tabref{tab:fromex:11.14}): in the absolutive type (only
S\textsuperscript{I}/O), S\textsuperscript{T} does not participate, so no
ergative case is involved, and the narrow ergative type (only
S\textsuperscript{T}) could not exist without ergative marking of
S\textsuperscript{T}. So the only type to consider is the complete ergative
type (S\textsuperscript{T} vs. S\textsuperscript{I}/O), but this type would
reduce to the absolutive type if S\textsuperscript{T} were not overtly marked.
The generalization therefore turns out to be inevitable.

We expect that a close investigation of the \isi{ergativity} generalizations listed
in \citet{Sheehan2014} and \citet{Deal2015}, from the perspective of our more
refined typology, may shed further light on their status, reason away apparent
exceptions, and perhaps provide a more fundamental explanation. However, any
further attempt in this direction would lead us beyond the scope of this
article.

\section{Ergative a dependent case?}\label{sec:11.6}

We noted in \Cref{sec:11.5.4} that no correspondence between case and
agreement \isi{alignment} could be set up. That conclusion is at variance with a
proposal in \citet{Bobaljik2008}, who argues for a conditional relation between
case-marking and eligibility for agreement control. We conclude by evaluating
this argument in the context of the system contemplated here.

\citet[296]{Bobaljik2008} acknowledges that agreement \isi{alignment} is often
incomplete, and proposes that incomplete agreement is sensitive to a
\gls{GF}-hierarchy (subject $>$ object; cf.\ \citealt{Moravcsik1978}), such
that the higher element on the hierarchy is the preferred agreement
controller.\footnote{Bobaljik’s definition: \enquote{The controller of
    agreement on the finite verbal complex (Infl+V) is the highest accessible
NP in the domain of V} (p. 296). \enquote{Domain} refers to considerations of
locality which are irrelevant to the discussion in this article. Accessibility
is subject to an implicational hierarchy\is{implicational relations} captured in Bobaljik’s generalization
discussed below (see \ref{ex:11.24}).} This has the effect that
subjective agreement may co-occur with \isi{ergative case} \isi{alignment}, a common
enough situation, illustrated here by the case of \ili{Wambaya}.

Beyond the \gls{GF}-hierarchy governing agreement control eligibility,
\citet{Bobaljik2008} also assumes the case hierarchy in \eqref{ex:11.23},
where \enquote{dependent case} may be accusative or ergative (following
\citealt{Marantz1991}), and \enquote{unmarked case} nominative\is{nominative
case} or absolutive.

\ea%23
    \label{ex:11.23}
    unmarked  $>$  dependent  $>$  lexical/oblique
\z

The conditional relation between case-marking and eligibility for agreement
control can then be formulated as in \eqref{ex:11.24}, which we refer to as Bobaljik’s
generalization \citep[303]{Bobaljik2008}.

\ea%24
    \label{ex:11.24}
    If in a language λ \isi{dependent case} noun phrases control agreement, then
    unmarked noun phrases in λ must also control agreement.
\z

\citet{Bobaljik2008} does not discuss why dependent case-marked elements may or
may not control agreement. The generalization in \eqref{ex:11.24} merely states what we can
expect if they do.\largerpage[2]

From our perspective, Bobaljik’s generalization ranges over (complete or
incomplete) \isi{alignment} types, and serves to exclude the incomplete types of
objective agreement (when O is accusative and controls agreement) and narrow
ergative agreement (when S\textsuperscript{T} is ergative and controls
agreement); in these situations \eqref{ex:11.24} tells us that the unmarked case elements
control agreement as well, yielding complete agreement types.\footnote{Strictly
    speaking, Bobaljik’s generalization (by its conditional nature) does not
    predict anything about agreement control by unmarked case-marked elements
    when the condition is not met (i.e.\ when the accusative and ergative
    elements do not control agreement). For the implicit assumption that we
expect the absolutive agreement type to show up in this situation, see the
text.}

However, objective agreement is also predicted not to occur by the
\gls{GF}-hi\-er\-ar\-chy (subject $>$ object), which limits incomplete agreement to
the subjective type (controlled by S\textsuperscript{T}/S\textsuperscript{I}
alone). Bobaljik’s generalization is redundant here. Narrow ergative agreement
(controlled by S\textsuperscript{T} alone) is also consistent with the
\gls{GF}-hi\-er\-ar\-chy, if we allow for some transitivity sensitivity in this
department. This incomplete agreement type seems uncommon, but, as we saw, it
is represented in our limited data set by past tense clauses in \ili{Coast
Tsimshian} \citep[68]{Mulder1994}.

It seems, then, that the explanatory value of \eqref{ex:11.24} is somewhat
limited.  \citet{Bobaljik2008} mentions the incomplete absolutive agreement
type (controlled by S\textsuperscript{I}/O, represented by \ili{Tsez} and
perhaps languages of the \ili{Marathi} type, like \ili{Hindi}), as consistent
with his generalization \eqref{ex:11.24}, because agreement control by
absolutive case-marked elements is a situation we might expect to occur when
ergative case-marked S\textsuperscript{T} fails to control agreement. However,
absolutive agreement of the type found in languages like \ili{Marathi} is only
inconsistent with a \gls{GF}-based theory of agreement control, if we choose to
ignore the generalization that morphologically case-marked elements (not just
ergative elements) never control agreement in these languages (cf.\
\citealt[446]{Pandharipande1997}; \citealt{Woolford2000b}). If we take this
generalization into account, agreement control by ergative case-marked
S\textsuperscript{T} is ruled out by an independent language particular
constraint, and the situation in \ili{Marathi} does not argue against a
\gls{GF}-based theory of agreement control.\footnote{On absolutive agreement in
the \ili{Tsez} type of languages, see \Cref{sec:11.5.2} above and
\cref{fn:21}.}

If this is correct, we may maintain that agreement control and case are subject
to different organizational principles, agreement being sensitive to
grammatical function much more so than case (see also \citealt{Legate2008}).
This conclusion would cast doubt on the usefulness of the definition of
ergative case as a \isi{dependent case} \citep{Marantz1991}.\footnote{As an anonymous
    reviewer rightly points out, the concept of ergative as a \isi{dependent case}
    has been put to profitable use in the literature many times since
    \citet{Marantz1991}, among others in \citegen{Baker2015} analysis of
    differential case-marking. As addressing these implementations is not
    possible in the context of this article, we restrict ourselves here to a
    discussion of the conceptual appeal of the \isi{dependent case} hypothesis.} On
    the view of \citet{Marantz1991}, now widely shared, the difference between
    ergative and \isi{accusative case} \isi{alignment} is due to a morphological
    mechanism of \enquote{dependent case} assignment, targeting O in accusative
    languages and S\textsuperscript{T} in ergative languages. Assuming a
    hierarchical organization of cases like \eqref{ex:11.23}, it then
    follows that \isi{grammatical functions} are differently ranked in the two types
    of languages, as in \eqref{ex:11.25}.

\ea%25
    \label{ex:11.25}
    \ea accusative \tab{S $>$ O $>$ other}
	\ex ergative \tab{S\textsuperscript{I}/O  $>$  S\textsuperscript{T}  $>$ other}
	\z
\z

An alternative to the Marantzian approach to \isi{ergativity} would be to deny
any meaningful grouping of ergative S\textsuperscript{T} and accusative O, and
to assign the status of a universal to the \gls{GF}-based grouping in
(\ref{ex:11.25}a). On this approach, the ergative would still be a
morphologically marked phenomenon, but differently from the accusative. Without
the S\textsuperscript{T}/O grouping inherent in the \isi{dependent case} premiss, we
do not expect Bobaljik’s generalization to make any predictions, beyond what is
already predicted by a \gls{GF}-based analysis.

From a derivationalist perspective, the characterization of ergative as a
dependent case strikes us as incongruous. We take dependency to be a function
of syntactic hierarchy (\citealt{Zwart2004} et seq.), itself a function of the
structure generating procedure \isi{Merge} of \citet{Chomsky1993}. In the spirit of
\citet{Epstein1999}, we assume that in any pair (α, δ) resulting from Merge, δ
is the dependent of α (the antecedent), and the dependency can be
morphologically realized on any term of δ \citep{Zwart2006b}. Accusative case,
on this view, is the morphological realization of a subject--object dependency,
essentially signaling the presence of a higher (antecedent) grammatical
function \citep{Zwart2006a}, a view that goes back to
\citet{Jakobson1936}.\footnote{To be more exact, a marker of the dependency
between the subject and its sister, realized on the object as a term of the
subject’s sister.\label{fn:29}} It is unclear how \isi{ergative case} may be defined
as dependent on this approach, but certainly its dependency must be different
from that of the \isi{accusative case}, as the ergative is itself the subject.
Flipping the dependency relation such that the object becomes the antecedent
for the subject would be incompatible with the definition of dependency as a
function of \isi{Merge}.\footnote{A related question is whether \isi{ergative case} should
    be characterized as structural or inherent. Since (if we are right)
    \isi{ergative case} can come about in a variety of ways (see
    \Cref{sec:11.5.1}), it is unlikely that this question can be given a
uniform answer, and we propose to leave it for further study.\label{fn:30}}

\section{Conclusion}\label{sec:11.7}

In this article we have argued for a more fine-grained \isi{alignment} typology, in
which  the canonical ergative\is{ergative alignment} \isi{alignment} type is just one of five so-called
complete types, and one of 18 types overall. We have shown that some of the
incomplete types that look ergative, especially the transitive subjective type,
are in fact not in the ergative family of types, involving special treatment of
transitive subjects within a basically accusative \isi{alignment} system.

We submit that the new \isi{alignment} typology with its 18 possible types is better
suited to describe the attested variation in \isi{alignment} patterns than the
conventional \isi{alignment} typology, and provides a basis for understanding
existing \isi{alignment} generalizations as discussed in \citet{Sheehan2014} and
\citet{Deal2015}.\largerpage

Following up on \citet{DeLancey2004}, our analysis calls into question the
existence of a theoretically significant concept \enquote{ergativity}, and
suggests that attempts at identifying an \enquote{ergativity parameter} as the
locus of variation between an \enquote{ergative system} and an
\enquote{accusative system} may well remain futile. Therefore, it is important
that syntactic approaches to ergativity pick up on the amount of variation
attested in \isi{alignment} patterns, and rethink their analyses accordingly.

\printchapterglossary{}

\noindent When not followed by \Sg{} or \Pl{}, numbers refer to noun classes.

\section*{Acknowledgements}

We like to thank Ian Roberts for his many contributions, over more than three
decades, to our field, setting an example of thinking and rethinking that this
gratulatory effort can only begin to approach. Earlier versions of this article
were presented at the University of Konstanz, Germany, in May 2014, and at the
February 2015 TIN-dag in Utrecht. The authors would like to thank the audiences
at these occasions, in particular Frans Plank, Heidi Klockmann, and Bernat
Bardagil-Mas, as well as András Bárány and Jonathan Bobaljik. Thanks are also
due to the editors of this volume and two anonymous reviewers. This research
grew out of the NWO-funded project \enquote{Dependency in Universal Grammar},
Jan-Wouter Zwart principal investigator, and benefited from preliminary
research carried out in that context by Aysa Arylova.

{\sloppy\printbibliography[heading=subbibliography,notkeyword=this]}
\end{document}
