\documentclass[output=paper]{langsci/langscibook}
\ChapterDOI{10.5281/zenodo.4680316}
\author{Dominique Sportiche\affiliation{University of California, Los Angeles}}
\title{Rethinking French dative clitics in light of frozen scope effects}


\abstract{Frozen scope effects as found in double object constructions in
    \ili{English} are shown to be found in French too. They arise when an
    indirect object is doubled with a dative clitic as in clitic left or right
    dislocation but not otherwise. This minimally suggests that dative
\isi{clitics} do not simply represent the counterpart of prepositional indirect
objects, which do not exhibit frozen scope effects.}


\begin{document}\glsresetall
\maketitle

\section{Introduction}

English has both a \gls{PDC}\is{prepositional dative construction} and a
\gls{DOC}\is{double object construction} with different
properties.\footnote{Some terminology: I will discuss pairs such as {\it Mary
    sent Bill flowers, Mary sent flowers to Bill}.  I will call the latter the
    prepositional dative\is{dative case} construction (\gls{PDC}) and the former the double
    object construction (\gls{DOC}). I will call \gls{IO} the DP interpreted as
    the goal\slash recipient\slash intended possessor\slash benefactives\slash malefactives, namely
{\it Bill} here. I will call \gls{DO} the DP that interpreted as the
theme/patient, here {\it flowers}.} One distinguishing property is the frozen
scope effect only found in the DOC.  First, this note documents that this
effect is sometimes found in \ili{French} too and concludes that \ili{French}, like
English, has a distinction between \glspl{PDC} and \glspl{DOC}, as suggested in
\citet{anagnostopoulou2005cross}.  It next discusses the fact that this effect
is only found in the presence of a dative\is{dative case} clitic, suggesting that dative
clitics are only available for \glspl{IO} in DOC constructions and not in
\gls{PDC} constructions and  discusses how \glspl{DOC}\is{double object
construction} surface in \ili{French} and concludes they do not.

\section{CLLD}

\subsection{Basics}\label{clld-htld}

To illustrate the frozen scope effect in \ili{French}, I will use \gls{CLLD}. I
could have equally well used \gls{CLRD} and will make scattered remarks about
it.  French \gls{CLLD}\is{clitic left dislocation} is illustrated below and can
affect any XP which can be associated with a clitic, a weak pronominal form
(with a different distribution than its non pronominal counterparts):

\ea
\ea
	\gll  {\bf Jean}, {\bf il} est parti.\\
John, he is left\\
\ex
	\gll {\bf Jean}, on  {\bf le} connait.\\
John, we him know\\
\ex
	\gll {\bf A} {\bf Paris}, on  {\bf y} va souvent.\\
To Paris, we there go often\\
\ex
	\gll {\bf Triste}, Albert pourrait {\bf le} devenir.\label{predle}\\
Sad, Albert could it become \\
\z
\z
%
Several properties distinguish the sometimes superficially similar
\gls{CLLD}\is{clitic left dislocation} from hanging topic left dislocation
(HTLD), e.g.\ the following two (cf.\ \citealp{AlexiadouLD}, or
\citealp{Krapova2008}):

\begin{itemize}
\item HTLD can only be found in root contexts, \gls{CLLD}\is{clitic left dislocation} appears in both root and non-root contexts.
\item There can be more than one \gls{CLLD}-ed XP in a clause, but no more than one HTLD-ed DP.
\end{itemize}
Accordingly, all the \gls{CLLD}\is{clitic left dislocation} sentences to come
should be considered subordinate clauses, or follow an independent Topic,
making the relevant element the second of two consecutive Topics, even if this
is not explicitly indicated.\is{topic}

\subsection{CLLD is movement}

We now show that \gls{CLLD}\is{clitic left dislocation} is \isi{movement}, without
worrying about the kind of \isi{movement} involved. A fuller discussion of the
derivation properties is given in \citet{Angelopoulo2017}.

\subsubsection{Scope reconstruction}

A (further) difference between \gls{CLLD}\is{clitic left dislocation} and HTLD
is the presence vs. absence of connectivity effects. It can be observed if Case
is differentially marked: hanging Topics\is{topic} do not exhibit Case connectivity with
the resumptive element and consequently the Topic appears in the default case,
unlike what happens with CLLD. Most telling among connectivity effects however
is the fact that reconstruction effects are observed with CLLD,  demonstrating
that \gls{CLLD}\is{clitic left dislocation} is, or can be, a \isi{movement}
dependency between a left peripheral\is{topic} and a clause internal position.
Indeed, reconstruction of a high-XP  to a low-XP position as in the adjacent
tree arises if and only if  low-XP is the trace of  high-XP.

\ea
    \begin{tikzpicture}[baseline, level distance=15pt]

        \Tree 	[
                    \node (h) {\textsc{high-xp}\tss{\emph{k}}};
                    [
                        {\dots}
                        [
                            \textsc{qp}
                            [
                                \dots{}
                                \node (l) {\textsc{low-xp}\tss{\emph{k}}};
                            ]
                        ]
                    ]
                ]

        \draw [arrow, ->, bend right=60] (h.south west) to (l.south west);

    \end{tikzpicture}
\z
%
Here, it will suffice to show that (total) reconstruction is possible for
pronominal \isi{binding}. In the tree above, if the high-XP contains a pronoun bound
by the QP, with the QP not outscoping it, total reconstruction of high-XP is
required to within the scope of this QP to put the pronoun in the scope of QP.
This thus diagnoses the presence of a trace in the c-command domain of the
quantifier. Reconstruction is said to be total iff low-XP is interpreted and
high-XP is not interpreted at all.  This is standardly illustrated by:

\ea
    \gll \hphantom{[ }Quelle {photo de lui} est-ce-que personne (ne) vend (~photo de lui)?\\
    {[ Which} \sout{picture of his}\tss{\emph{k}}~]\tss{\emph{j}} did {\bf
        nobody\tss{\emph{k}}} \hphantom{(}\Neg{} sell [~picture of his\tss{\emph{k}}~]\tss{\emph{j}}\\
\z
%
Here {\it picture of his} (in fact possibly {\it which picture of his}) is
totally reconstructed to its trace as {\it nobody} cannot outscope the clause
initial position of the wh-phrase.

\subsubsection{CLLD reconstruction}

Reconstruction of \gls{CLLD}-ed constituents for pronominal \isi{binding} can be
readily illustrated. First, a \gls{CLLD}-ed DO or IO for example can totally
reconstruct below the subject of its clause.\footnote{In all cases of
    pronominal \isi{binding}, we choose embedded pronouns rather than high
    possessors. High possessors display ununderstood properties, cf.\

\begin{exe}
    \exi{(i)}
    \begin{xlist}
    \ex[?]{His grades  persuaded every boy to work harder}
    \ex[✔]{The grades he got persuaded every boy to work harder}
    \end{xlist}
\end{exe}}
%
Note that we typically (but not exclusively) use \enquote{negative} quantifiers
to prevent the possibility of them outscoping the preposed \gls{CLLD}-ed XP:

\ea \label{f1}
\ea
	\gll  [ La  prof de sa\tss{\emph{j}} fille~]\tss{\emph{k}},  aucun parent\tss{\emph{j}}  (ne) la\tss{\emph{k}} 	connait bien.\\
    {} the teacher of his daughter, no parent \hphantom{(}\Neg{} her knows well\\
\ex
	\gll [ À la prof  de 	sa\tss{\emph{j}} fille~]\tss{\emph{k}},  aucun parent\tss{\emph{j}} 	lui\tss{\emph{k}} 	a parl\'{e}.\\
    {} to the teacher of her daughter, no parent to-her has spoken\\
\z
\z
%
This shows that the \gls{CLLD}-ed XP has been moved from below the QP subject.

This extends to long distance cases: pronominal \isi{binding}, shown here with
a \gls{CLLD}-ed subject or object, is allowed from a quantifier in the source
clause (that containing the clitic), or in a clause higher  than the source
clause.

\ea \label{20.3}
\ea
\gll [ Les  louanges   pour   son\tss{\emph{j}}  dernier livre~]\tss{\emph{k}},
aucun auteur\tss{\emph{j}} ne     pense     qu'    elles\tss{\emph{k}}  seront
ignor\'{e}es.\\
    {} the    praises       for  his    last      book,     no       author
    \Neg{}  thinks    that   they   {will be} ignored\\
\ex \gll [ Les  louanges   pour   son\tss{\emph{j}}  dernier
livre~]\tss{\emph{k}},     je pense     qu'    aucun auteur\tss{\emph{j}} ne
les\tss{\emph{k}}  ignorait.\\
    {}  the    praises       for  his    last      book,     I  think    that
    no author \Neg{} them ignored\\
\z
\z
%
\subsection{CLLD reconstruction asymmetries}

As shown above, a \gls{CLLD}-ed XP can reconstruct, hence can have been moved.
More specifically, these examples illustrate reconstruction under subjects:
examples (\ref{f1}a) and \eqref{20.3} show that a DO can reconstruct under a
subject; example (\ref{f1}b) shows that an IO can reconstruct under a subject;
and example (\ref{20.3}a) shows that a subject can reconstruct under a subject.
Is it possible to show reconstruction under a non-subject? The answer is
positive, but there is a surprising gap.

\subsubsection{Background on French PDC}

With \glspl{DO} and \glspl{IO}, \ili{French} superficially shows only
\gls{PDC}\is{prepositional dative construction} constructions.  Furthermore, in
such \glspl{PDC} without  \isi{movement}, IOs and DOs behave as c-commanding
each other: informally, they behave as if they were under each
other.\footnote{This is independently interesting and telling about the
    derivational history of \glspl{PDC}, and \glspl{DOC}\is{double object
    construction} for that matter. This is  not discussed here but is in
\textcite{Sportiche2017}.}

\ea \label{doio1}
\ea IO c-commands DO\\
\gll On {a pr\'{e}sent\'{e}} l'habilleur de \bf{son} partenaire \`{a}
\textbf{chaque} / \textbf{aucune} danseuse.\\
We introduced {the dresser} of her partner to each {} no dancer.\glossF{} \\
\ex  DO c-commands IO\\
\gll On {a pr\'{e}sent\'{e}} \textbf{chaque} / \textbf{aucune} danseuse \`{a}
\bf{son} partenaire.\\
We introduced each {} no dancer.\glossF{} to her partner\\
\z
\z
%
Here the bold face pronoun can be bound by the bold faced quantifier.

This remains true under some \isi{movement} operation, e.g.\ wh-movement:

\ea \label{doio2}
\ea IO reconstructs under DO\\
\gll Quel habilleur de \bf{son} partenaire on {a pr\'{e}sent\'{e}} \`{a} \textbf{chaque} / \textbf{aucune} danseuse?\\
Which dresser of her {partner did} we introduced to each {} no dancer.\glossF{} \\
\ex DO reconstruct under IO\\
\gll  Auquel de  \bf{ses} partenaires on {a pr\'{e}sent\'{e}} \textbf{chaque} / \textbf{aucune} danseuse?\\
{To which} of her {partners did} we introduced each {} no dancer.\glossF{} \\
\z
\z

\subsubsection{Can CLLD-ed DOs and IOs totally reconstruct
under each other?}\label{frozenscope}

We are now in a position to show that \gls{CLLD}-ed \glspl{DO} can totally
reconstruct under an  IO. The observation is that a pronoun contained in a
\gls{CLLD}-ed \gls{DO} can be bound by a quantifier contained in the \gls{IO}.

\ea
    \gll [ La     note   de   {\bf son\tss{\emph{j}}}  dernier
devoir~]\tss{\emph{k}},       le     professeur        l\tss{\emph{k}}' {a rendue}
\`{a}  \textbf{chaque} / \textbf{aucun}   \'{e}l\`{e}ve\tss{\emph{j}}.\\
    {}  the    grade  on  his    last       assignment  the   professor   it
     gave      to  each {} no        student\\
\z
%
Surprisingly, the symmetric situation does not hold: a pronoun contained in a
\gls{CLLD}-ed IO {\it cannot} be bound by a quantifier contained in the DO.
This shows that  \gls{CLLD}-ed \glspl{IO} cannot totally reconstruct.
I will return below to the question of why. Note that the DO/IO reconstruction
contrast also shows that total reconstruction is indeed involved in the DO
case, rather than the QP somehow outscoping a higher position (namely XP$_k^2$
of the tree in \figref{fig:fromex:clitics}).

\ea * CLLD-ed IOs in the scope of an unmoved DO \hfill{Indirect object  {\it lui}} \label{scofree}
\ea[]{
	\gll  on {a pr\'{e}sent\'{e}}    \textbf{chaque} / \textbf{aucun}
    professeur\tss{\emph{j}} aux      parents     de {\bf son\tss{\emph{j}}}
    meilleur \'{e}tudiant.\\
we  introduced each {} no professor to-the parents of his best student\\}
\ex[*]{
	\gll  [ Aux      parents     de {\bf son\tss{\emph{j}}}   meilleur
    \'{e}tudiant~]\tss{\emph{k}},     on      leur\tss{\emph{k}}        a
    pr\'{e}sent\'{e}     \textbf{chaque} / \textbf{aucun}
    professeur\tss{\emph{j}}.\\
    {} {to-the}    parents     of  his    best        student,      we
     them\tss{\Dat} have introduced     each {} no       professor\\}
\z
\z
%
Superficially, \glspl{IO} look like PPs, unlike \glspl{DO} that do reconstruct.
Their failure to reconstruct, however, is not due to this (potential)
categorial difference with \glspl{DO} (or subjects). Indeed, other
\gls{CLLD}-ed PPs clearly contrast with \glspl{IO}, as illustrated below:

\ea \gls{CLLD}-ed genitive\is{genitive case} PPs in the scope of an unmoved DO \hfill{Locative {\it en}}
\ea[]{
	\gll      On     a   \'{e}loign\'{e}  aucune fille\tss{\emph{j}} [ de
    sa\tss{\emph{j}}   meilleure amie]\tss{\emph{k}}.\\
we   have   removed   no girl {} from her   best        friend\\}
\ex[?]{
    \gll
  [ De sa\tss{\emph{j}}   meilleure amie~]\tss{\emph{k}}, on  en\tss{\emph{k}}   a   \'{e}loign\'{e}  aucune fille\tss{\emph{j}}.\\
          {} From her   best        friend, we   of-her have   removed   no girl  \\}
\z
\ex \gls{CLLD}-ed genitive\is{genitive case} PPs in the scope of an unmoved IO \hfill{About {\it en}}
\ea[]{
	\gll      On     a   parl\'{e} \`{a} aucune fille\tss{\emph{j}} [ de
    sa\tss{\emph{j}}   meilleure amie]\tss{\emph{k}}.\\
we   have   spoken to   no girl {} about her   best        friend\\}
\ex[?]{
    \gll
  [ De sa\tss{\emph{j}}   meilleure amie~]\tss{\emph{k}}, on  en\tss{\emph{k}}   a   parl\'{e} \`{a}  aucune fille\tss{\emph{j}}.\\
          {} About her   best        friend, we   of-her have   spoken to  no girl  \\}
\z
\ex \gls{CLLD}-ed locative PP  \hfill{Locative {\it y}} \label{48}
\ea[]{
	\gll     Ils n'  ont   renvoy\'e  aucune lettre$_m$ [ \`a l'adresse de
    son$_m$ exp\'editeur~].\\
    they \Neg{} have {sent back} no letter {} to {the address} of its sender
      \\}
      \ex[?]{
	\gll [ \`A l'adresse de son$_m$ exp\'editeur~]\tss{\emph{k}}, ils n' y\tss{\emph{k}}  ont   renvoy\'e  aucune lettre$_m$.\\
    {} To {the address} of its sender, they \Neg{} there have {sent back} no letter  \\}
\z
\z
%
While \glspl{IO} contrast with PPs, the PP facts are a bit less clear than the
DP cases: they are better than \glspl{IO}, perhaps just good. The same point
can be made clearly with CLitic Right Dislocation, \gls{CLRD}\is{clitic right
dislocation}, only briefly mentioned here, which shares all the relevant
properties with \gls{CLLD}\is{clitic left dislocation} (they differ in the
surface position of the dislocated constituent):

\begin{exe}
    \ex * CLRD-ed IO in the scope of an unmoved DO \hfill{Indirect object  {\it lui}}
    \begin{xlist}
        \ex[]{
            \gll  On {a   pr\'{e}sent\'{e}}     \textbf{chaque} /
            \textbf{aucun} professeur\tss{\emph{j}} aux      parents     de
            {\bf son\tss{\emph{j}}}   meilleur \'{e}tudiant.\\
                We  introduced each {} no professor to-the parents of his best student\\}
\ex[*]{
	\gll  On      leur\tss{\emph{k}}        a   pr\'{e}sent\'{e}
    \textbf{chaque} / \textbf{aucun} professeur\tss{\emph{j}}, [ aux
    parents     de {\bf son\tss{\emph{j}}}   meilleur \'{e}tudiant~]\tss{\emph{k}}.\\
         We  them\tss{\Dat} have introduced     each {} no       professor,
     {} {to-the}    parents     of  his    best        student \\}
    \end{xlist}
    \ex ✔ \gls{CLRD}\is{clitic right dislocation}-ed genitive\is{genitive case}
PPs in the scope of an unmoved DO \hfill{Locative {\it en}}
    \begin{xlist}
    \ex
	\gll      On     a   \'{e}loign\'{e}  aucune fille\tss{\emph{j}} [ de
    sa\tss{\emph{j}}   meilleure amie~]\tss{\emph{k}}.\\
    We   have   removed   no girl {}  from her   best        friend\\
    \ex
    \gll On  en\tss{\emph{k}}   a   \'{e}loign\'{e}  aucune fille\tss{\emph{j}}, [ de sa\tss{\emph{j}}   meilleure amie~]\tss{\emph{k}}.\\
    We   of-her have   removed   no girl {} from her   best        friend  \\
    \end{xlist}
\end{exe}
%
Furthermore, reconstructability extends to other categories, e.g.\ to
predicates as in \eqref{xyz} (in fact they {\it must} totally reconstruct as low
as can be tested, as preposed predicates generally do).

\ea \label{xyz}
\gll [ Fier d' un \'{e}tudiant~]\tss{\emph{j}},    Pierre
l\tss{\emph{j}}' a     [ souvent [ \'{e}t\'{e} t]].\\
{}  Proud of a student      Peter  it has  {} often {} been	{}\\
\sn \gll
    (✔ souvent $>$ un \'{e}tudiant)\\
    (✔ often $>$ a student)\\
\z
%
As shown, reconstruction to below the adverb {\it souvent/often} is possible.

\section{Analyzing the CLLD reconstruction asymmetries}

\subsection{\glspl{DOC} in French}

We have established that \gls{CLLD}-ed constituents can all totally
reconstruct,  except for \gls{CLLD}-ed \glspl{IO} which alone fail to totally
reconstruct to the surface position they apparently occupy when not moved,
namely the dative\is{dative case} position of a \gls{PDC}. Why do \glspl{IO} behave
differently?  There are two analytical options as to  why a constituent M would
fail to reconstruct to some position P:

\begin{enumerate}
\item M has moved to a position disallowing reconstruction.
\item M has not moved from P.
\end{enumerate}

\subsubsection{Exploring option 1} Given  that all \gls{CLLD}-ed constituents can
totally reconstruct to some position, it can't be that properties of the
\gls{CLLD}\is{clitic left dislocation} surface position itself prevent
reconstruction. The difference between \glspl{IO} and others XPs must thus come
from somewhere else. One option is to attribute the IO/DO difference to
different properties of the \isi{clitics} themselves.  \citet{Angelopoulo2017} show
that \gls{CLLD}\is{clitic left dislocation} of \glspl{DO} and \glspl{IO} (e.g.)
in a simple clause is at least a two-step operation proceeding roughly as shown
in the tree \figref{fig:fromex:clitics}, where XP$_k^2$ is possibly in a spec/head relation
with CL (as in \citealt{Sportiche1996}).

\begin{figure}
    \caption{\Glsdesc{CLLD} of direct or indirect objects\label{fig:fromex:clitics}}
    \begin{tikzpicture}[baseline=(root.base), scale=1.0]

        \Tree 	[.\node(root){TopicP};
                    \node(xp1){XP$^{1}_{k}$};
                    [.TP
                        subject
                        [.T$'$
                            \dots{}
                            [.\dots{}
                                \node (xp2) {XP$^2_{k}$};
                                [.\dots{}
                                    CL\tss{\emph{k}}
                                    [.\dots{}
                                        \dots{}
                                        [.VP
                                            \dots{}
                                            \node (xp3) {\dots{} \sout{XP$^3_{k}$} \dots{}};
                                        ]
                                    ]
                                ]
                            ]
                        ]
                    ]
                ]

        \draw [arrow, ->, solid, bend left=60] (xp3.south west) to (xp2.south
        west);

        \draw [arrow, ->, solid, bend left=60] (xp2.south west) to (xp1.south
        west);

    \end{tikzpicture}
\end{figure}

In the context of this analysis, we can interpret the reconstruction
possibilities as follows: if XP=DO, total reconstruction is possible from
XP$_k^1$ either to XP$_k^2$   (lower than the subject) or to XP$_k^3$ (lower
than an unmoved IO). If XP=IO, total reconstruction to  XP$_k^2$ (lower than
the surface subject) is possible, but not lower, hence not in the scope of a
DO.  A version of option 1 would attribute to dative\is{dative case} \isi{clitics} themselves the
prevention of such total reconstruction. But while there is some plausibility
to the existence of systematic differences between dative\is{dative case} \isi{clitics} and all
others (e.g.\ datives must be animate, mostly, or personified, unlike other
clitics), it is unclear why this should have the requisite interpretive effect
(of blocking reconstruction). I therefore conclude against option 1 and in
favor of option 2.

\subsubsection{Exploring option 2} According to option 2, \gls{CLLD}-ed \glspl{IO}
have not moved from a position in the scope of DO. Since \glspl{IO} can
reconstruct at least to XP$_k^2$ (cf.\ example \ref{f1}b), they must have
been \gls{CLLD}-ed from a position L intermediate between \glspl{DO} and
(surface) subjects. There is evidence that this position is relatively low.
Indeed consider the derivational path of \gls{CLLD}-ed elements as it is
described in the tree (\figref{fig:fromex:clitics}). The example (\ref{f1}b) with {\it aucun}
shows that L must be lower than the position in which an {\it aucun NP} subject
must be interpreted. Such DPs are indefinites in the scope of negation which
must totally reconstruct from their surface position to such a position so we
can conclude that L is also in the scope of clausal negation (which excludes
the surface subject position). L might well be the XP$_k^2$
position.\footnote{By the same reasoning, L must be in the scope of e.g.\ a
    conditional modal: in the example (i) \emph{Les secrets de
    ses\tss{\emph{k}} amis, [un homme loyal]\tss{\emph{k}} les garderait pour
lui.} \enquote*{The secrets of his friends, a loyal man would keep them to
himself.}, the subject must reconstruct under the modal but can still bind the
pronoun in the CLLD-ed  phrase. This means this phrase can totally reconstruct
below the subject hence below the modal.}\multiplefootnoteseparator\footnote{It
is difficult to decide whether the lowest position L \glspl{IO} can reconstruct
to is higher or lower than the VP internal subject position. What we can
conclude is that it is lower than negation or a modal but higher than the
highest A-position a quantificational DO can occupy. If (case~\#1) such a
position is lower than the VP internal subject position, L could be higher than
both.  If (case~\#2) such a position is higher  than the VP internal subject
position, L would have to be VP external.  Under the assumption that both
\glspl{DO} and \glspl{IO} can A-scramble to the same "neighborhood", the
behavior of \glspl{DO} could help. DO QPs can't seem to A-scramble past the VP
internal subject, as backwards \isi{binding}  (as {\it a friend of
his\tss{\emph{k}} mother invited every child\tss{\emph{k}}}) triggers a
\glsunset{WCO}\gls{WCO} effect.\is{weak crossover} So we are in case \#1.}

With \gls{CLLD}-ed \glspl{IO}, we must then have an underlying structure with
IO asymmetrically c-commanding DO, and necessarily outscoping DO. But this is
nothing other than a DOC, and failure of reconstruction of the \gls{CLLD}-ed IO
simply illustrate the frozen scope effect familiar from \ili{English} double
object constructions (cf.\ e.g.\ \citealt{Larson:1988})! This effect is illustrated
below:\footnote{Frozen scope can also be illustrated with  impossible
    pronominal \isi{binding} as in {\it Marta owed a peasant who raised it every
    horse} where trying to bind {\it it} with {\it every  horse} triggers much
stronger deviance than a weak crossover effect.\is{weak
crossover}}\is{DOC|see{double object construction}}\is{PDC|see{prepositional dative construction}}

\ea \label{20}
    \ea Marta owed a peasant a horse.
    \ex Marta owed a peasant every horse.
    \z
\z
%
In (\ref{20}b), the DO cannot outscope the IO: the reading {\it every}
\textgreater {\it a} is unavailable. This means that, despite the lack of
surface evidence,  \ili{French} does have a \gls{DOC}\is{double object
    construction} (with frozen scope) in addition to a
    \gls{PDC}\is{prepositional dative construction} (without frozen scope):
    this corroborates a conclusion  reached in \citet{anagnostopoulou2005cross}
    based on a study of Datives in \ili{Greek}, \ili{French}, \ili{Japanese}
    and \ili{Spanish}.

\citet{anagnostopoulou2005cross} concludes that all these  languages have both
\glspl{DOC} and \glspl{PDC}, with some variation as to the categorial
realization of \glspl{IO} in both \glspl{DOC}\is{double object construction} and \glspl{PDC} (PPs vs DPs).  It
takes the crucial  property of \glspl{DOC}\is{double object construction} to be  the association of \glspl{IO}
with extra functional structure (i.e. light applicative heads) in \glspl{DOC}
but not in \glspl{PDC}.{\interfootnotelinepenalty10000\footnote{While the present article fundamentally agrees
    on the essentials of these conclusions, there may be some disagreement
    about details not affecting, in fact possibly further reinforcing,
    \citeauthor{anagnostopoulou2005cross}'s
    (\citeyear{anagnostopoulou2005cross}) central conclusions.

    Thus it is not entirely clear that \glspl{IO} introduced by \emph{à} always
    are DPs, rather than possibly ambiguous between DPs and PPs. This would
    agree both with \citet{Kayne1975} and \citet{Vergnaud:1974} (and there may be
    variation among French speakers here). The debate hinges on the
    wellformedness of \emph{à} introducing a bare DP conjunction. It seems to
    me that such coordinations oscillate between a mild  intermediate status
    and fine cf.\ (i) \emph{On a donn\'e un livre \`a Jean et
        $^{?}$/$^{✔}$(\`a) Marie.} \enquote*{We gave a book to John and
        (to) Mary.}; (ii) \emph{$^{?}$On a donn\'e un livre \`a Marie et son
        fr\`ere.} \enquote*{we gave a book to Mary and her brother.}; (iii)
        $^{✔}$\emph{On donn\'e un livre \`a l'homme et (\`a) la femme qui
    se sont vus hier.} \enquote*{We gave a book to the man and the woman who met
    yesterday}. In particular, the equivalent of (i) and (ii) with
    genitive\is{genitive case} introducing {\it de} are much worse, while the
    equivalent of (iii) with a relative clause is better. Finally, the
    repetition of \emph{à} favors a distributive reading of the conjunction,
    and absence of it a group reading. This may play a role in
    \citeauthor{Jaeggli:1982}'s \citeyear{Jaeggli:1982} reported deviance of
    (iv) *\emph{Ils ont parl\'e \`a Marie et le directeur.} \enquote*{They talked to Mary
    and the director.} which I find overstated: its intermediate status
improves substantially when it is read with the conjunction of DPs forming a
single intonational phrase and a group reading is intended.}}

\subsection{Consequences}

\subsubsection{Dative clitics}

This conclusion is now informative about dative\is{dative case} \isi{clitics}.
Indeed, if dative\is{dative case} clitics could equally well stand for
\glspl{IO} in \glspl{DOC}\is{double object construction} and \glspl{PDC}, we
would not expect scope freezing, since scope freezing is generally not found in
\glspl{PDC}, as (\ref{doio1}b) illustrates for \ili{French}, and \ili{English}.
I conclude that \gls{PDC}\is{prepositional dative construction} \glspl{IO}
never cliticize as datives, only \gls{DOC} \glspl{IO} do.\footnote{This leaves
open the question of how this generalizes to other cases not obviously
involving \glspl{DOC}\is{double object construction} or \glspl{PDC}.}

This corroborates a conjecture made in \citet{Charnavel2015} regarding
antilogophoricity effects in clitic clusters. \citet{Charnavel2015} propose the
following descriptive generalization:

\ea \label{CLR} Logophoric restriction (CLR): When a third person IO clitic and a DO clitic co-occur in a cluster, the DO clitic cannot corefer
with a logophoric center.
\z
%
To account for the deviance of examples such as (\ref{CM}b), where the
accusative clitic is coreferential with {\it Anna}, a logophoric center (here
an attitude holder):

\ea \label{CM} \ea
	\gll Anne$_i$ croit qu' on  la$_i$ recommandera au patron.\\
        Anna$_i$ thinks that s.o. her.\Acc{} {will recommend} {to the} boss \\
    \trans `Anna$_i$ thinks that they will recommend her$_i$ to the boss.'
    \ex
	\gll  Anne$_i$ croit qu' on la$_{*i/j}$ lui\tss{\emph{k}} recommandera, [ au patron~]\tss{\emph{k}}.\\
    Anna$_i$ thinks that s.o. her.\Acc{} \Third.\Dat{} {will recommend} {} {to the} boss.\\
    \trans `Anna$_i$ thinks that they will recommend her$_i$ to him, the boss.'
\z
\z
%
They propose to derive \eqref{CLR} by assuming that (i) there cannot be two
perspective centers within the same minimal syntactic domain; (ii) in
(\ref{CM}b), the accusative clitic is a perspective center by virtue of being
coreferential with one ({\it Anna}) and (iii) the dative\is{dative case} clitic is one
inherently because the cliticized IO must correspond to the IO found in
DOC.\footnote{That \glspl{IO} in \glspl{DOC}\is{double object construction} must be perspective/logophoric
    centers is independently justified by their being able to antecede
    logophors in \ili{Japanese}, or in \ili{English} for example.} Their point (iii) is
    exactly what we found independent evidence for.

\subsubsection{Are French \glspl{DOC} visible on the surface?}

Although I did not not distinguish high (benefactives/malefactives) from low
(goals, possessors) applicatives, unambiguous scope (scope freezing) effects
are found with both in \gls{CLLD}\is{clitic left dislocation} cases. Low
applicatives have already been illustrated, cf.\ (\ref{scofree}). Here is a
case with a high applicative:

\ea\label{doio6}
    \ea DO c-commands IO in the order DO IO\\
    \gll \llap{?}On {a pr\'{e}par\'{e}} \textbf{chaque} / \textbf{aucun} {plat
    de viande \'epic\'e} \`{a} la cliente qui {\bf l'} {a command\'e}.\\
     we prepared each {} no {spicy meat dish}  to the customer who it {asked for}.\\
    \ex DO cannot bind into IO in IO CLLD\\
    \gll \llap{*}À la cliente qui \textbf{l'} {a commandé}, on lui a  préparé
        \textbf{chaque} / \textbf{aucun} {plat de viande épicé}.\\
         to the customer who it {asked for} we to-her {} prepared each {} no
        {spicy meat dish}.\\
\z
\z

\subsubsubsection{Low applicatives}

\citet{anagnostopoulou2005cross} uses contrasts reported in \citet{Oehrle:1976}
to detect \glspl{PDC}:

\ea
    \ea[]{The war years gave Mailer his first big success.}
    \ex[*]{The war years gave his first big success to Mailer.}
    \z
\ex \ea[]{Katya taught Alex Russian.}
\ex[]{Katya taught Russian to Alex.}
\z
\ex
\ea[]{This trip taught Alex patience.}
\ex[*]{This trip taught patience to Alex.}
\z
\z

This illustrates that the \gls{PDC}\is{prepositional dative construction}
requires agentive subjects while the \gls{DOC} does not (an intriguing
observation, unexplained  I believe).  \citet{anagnostopoulou2005cross}
conjectures that the order V IO DO in \ili{French} exemplifies the
\gls{DOC}\is{double object construction} but Oehrle's contrasts are
inconclusive in \ili{French} as the translation of the above examples yields pairs
that are equally fine:

\ea \ea Les ann\'ees de guerre ont donn\'e \`a Mailer son premier gros
succ\`es.
\ex Les ann\'ees de guerre ont donn\'e son premier gros succ\`es \`a Mailer.
\z
\ex \ea Katya a appris \`a Alex le russe.
\ex Katya a appris le russe  \`a Alex.
\z
\ex \ea Ce voyage a appris \`a Alex la patience.
\ex Ce voyage a appris la patience  \`a Alex.
\z
\z
%
Furthermore, if the order V IO DO exemplified a \gls{DOC}\is{double object construction} construction, we
would expect scope freezing. This is not observed as illustrated below:

\ea \label{doio4}
\ea  DO c-commands IO in the order DO IO\\
\gll On {a pr\'{e}sent\'{e}} {\bf chaque} / {\bf aucune} danseuse de ce nouveau ballet \`{a} \bf{son} futur partenaire.\\
We introduced each {} no dancer.\glossF{} of this new ballet to her future partner\\
\ex  DO c-commands IO in the order IO DO\\
\gll On {a pr\'{e}sent\'{e}} \`{a} {\bf son} futur partenaire {\bf chaque} / {\bf aucune} danseuse de ce nouveau ballet.\\
We introduced to her future partner each {} no dancer.\glossF{} of this new ballet\\
\z
\z
%
That the pronouns in the IO can be licitly bound by the quantified DO shows
that the DO can outscope the IO. This can also be illustrated with two
    quantifiers:

\begin{exe}
    \ex\label{ex:key:20.28}%i{(i)}
    \begin{xlist}
    \ex
        \gll On {a  pr\'{e}sent\'{e}} {\bf chaque} / {\bf aucune} danseuse de ce nouveau ballet \`{a} {\bf un} mentor de \bf{son} futur partenaire.\\
        we introduced each {} no dancer.\glossF{} of this new ballet to a mentor of her future partner.\\
        \ex  \gll On {a  pr\'{e}sent\'{e}}  \`{a} {\bf un} mentor de {\bf son} futur partenaire {\bf chaque} / {\bf aucune} danseuse de ce nouveau ballet.\\
        we introduced to a mentor of her future partner each {} no dancer.\glossF{} of this new ballet.\\
    \end{xlist}
\end{exe}
%
Here, the universal quantifier {\it chaque} \enquote*{each} can outscope the
existential {\it un/a} thereby licensing pronominal \isi{binding}.

The contrast between \ili{French} (\ref{ex:key:20.28}b) and \ili{English}
\glspl{DOC}\is{double object construction} (\ref{ex:key:20.28}c) is striking:

\begin{exe}
    \exi{\eqref{ex:key:20.28}}
    \begin{xlist}
    \exi{c.}[*]{We {} showed  a mentor of her future partner each/no dancer.\glossF{} of this new ballet.}
    \end{xlist}
\end{exe}
%
Finally, the order IO DO is most natural if DO has sufficient weight, an
observation suggesting that Extraposition or Heavy NP shift is involved in
shifting the DO to derive this order.

\subsubsubsection{High applicatives}
We now turn to (some) high applicatives, e.g.
bene\slash male\-factives introduced high\-er in the structure. The preferred option is
for benefactive to be introduced by \emph{pour}/\emph{for} but they can appear
introduced by \emph{à} with variable acceptability results.\footnote{Because
    benefactives can use {\it pour}, benefactives can be slightly degraded,
    hence it is preferable to use malefactives. These constructions (with
    \emph{à}) seem by no means to be productive, and results also seem
    sensitive to the nature of the direct object in ways that remain obscure.
    Results, however, are uniformly good and productive if the applied object
is a dative\is{dative case} clitic.} Both orders again (IO DO and DO IO) are
allowed but \isi{binding} is fine in either:

\ea \label{doio25}
\ea  DO c-commands IO in the order DO IO\\
\gll On {a enlev\'{e}} \textbf{chaque} / \textbf{aucun} enfant maltrait\'e
\`{a} {\bf ses} parents.\\
We {took away} each {} no child mistreated from its parents\\
\ex  DO c-commands IO in the order IO DO\\
\gll On {a enlev\'{e}} \`{a} {\bf ses} parents \textbf{chaque} / \textbf{aucun} enfant maltrait\'e.\\
We {took away} from its parents each {} no child mistreated\\
\z
\ex \label{doio26}
\ea  DO c-commands IO in the order DO IO\\
\gll Elles {ont  jou\'{e}} [ {\bf chaque} / {\bf aucun} morceau qu'on leur {a
appris}~]\tss{\emph{k}} \`{a} {\bf son}\tss{\emph{k}} compositeur.\\
They played {} each {} no piece we to-them {} taught to its composer\\
\ex  DO c-commands IO in the order IO DO\\
\gll Elles  {ont jou\'{e}}  \`{a} {\bf son}\tss{\emph{k}} compositeur [
\textbf{chaque} / \textbf{aucun} morceau qu'on leur {a appris}~]\tss{\emph{k}}.\\
They played to its composer {} each {} no piece we to-them taught \\
\z
\z
%
From this, two conclusions are possible. Either the order V DO IO is ambiguous
between a \gls{DOC}\is{double object construction} structure and a
\gls{PDC}\is{prepositional dative construction} structure so that we cannot
observe Oehrle's effects or scope freezing (since each tests one structure but
the other structure is also available); and in fact this could also be true of
the V IO DO order. Or \glspl{DOC}\is{double object construction} in
\ili{French} cannot surface unless the IO is cliticized. In the case of low
applicatives, the first option is reasonable as a \gls{PDC}\is{prepositional
dative construction} structure where the IO in fact instantiates a different
thematic structure with the IO being e.g.\ a locative (cliticizing as {\it y}).
In the case of high applicatives, however, it is hard to see what alternative
thematic structure there could be. This suggests that the
\gls{PDC}\is{prepositional dative construction} realization of high
applicatives is not ambiguous with a \gls{DOC}\is{double object construction}
and that, in turn, \glspl{DOC} are not just applied objects. We are thus led to
modify \citeauthor{anagnostopoulou2005cross}'s
(\citeyear{anagnostopoulou2005cross}) conclusion that the crucial  property of
\glspl{DOC}\is{double object construction} is  the association of \glspl{IO}
with extra functional structure such  as light applicative heads. Rather, such
structures may be necessary but not sufficient: \glspl{IO} in \glspl{DOC} are
applied objects with an additional property.\footnote{In the absence of this
    additional property, there may be Case differences between the two objects,
    but no deep c-command asymmetry in terms of \isi{binding} or scope.} This
    would explain why, whereas (Standard American) \ili{English} \glspl{IO} in
    \glspl{DOC} do not tolerate being wh-moved, high applicatives, even though
    they are applied objects, are not subject to such a prohibition:

\ea
\ea[]{We gave Mary a book. / We baked John a cake.}
\ex[*]{Who did you give a book? / *Who did you bake a cake?}
\ex[]{
	\gll On {a fait} un sale coup \`a Jean. / \`A qui   on {a fait} un sale coup.\\
     we played a dirty trick to Jean / To whom we played a dirty trick\\
    \trans `We played a dirty trick on Jean. / Who did we play a dirty trick on?'}
\z
\z
%
I tentatively conclude that \glspl{PDC}\is{prepositional dative construction}
do indeed realize high or low applicatives, but that they simply do not
instantiate the in principle (surface) possible \gls{DOC} re\-al\-iza\-tion
(which, alone, would show a scope freezing effect).

This would mean that in \ili{French}, there is no  clitic-less candidate for a
\gls{DOC} realization of applicatives. This would make \ili{French} similar to
Spanish, in which \glspl{IO} in \glspl{DOC}\is{double object construction} must be clitic doubled (cf.\
\citealp{anagnostopoulou2005cross}  and references therein). Given the
derivation in \figref{fig:fromex:clitics}, the closest \ili{French} correspondent to a
\gls{DOC}\is{double object construction} is the structure in which the IO has
moved to the position XP$_k^2$ in \figref{fig:fromex:clitics}, a \isi{movement} requiring the
presence of an associate clitic, but to a position where \ili{French} does not allow
an XP to surface. This would explain why, just like \glspl{IO} in \ili{English}
\glspl{DOC}, \gls{CLRD}\is{clitic right dislocation}-ed Datives (or
Accusatives\footnote{Conversely, we should expect to find all the properties
    associated with \gls{DOC}\is{double object construction} \glspl{IO} in
    \ili{English} to also be available with DOs. In general, this is not going
    to be easy to detect since DOs, unlike \glspl{IO} in \glspl{DOC}\is{double
object construction}, do not have to move so high: a plausible place to look is
of course \glspl{DO} in verb-particle constructions in the order V DO Part.})
have to be specific, see \citet{Sportiche2017c}, a requirement imposed in
\ili{French} by the mandatory presence of the clitic.

This means that the closest equivalent to \ili{English} \glspl{DOC}\is{double
object construction} in \ili{French} is either \gls{CLRD}\is{clitic right
dislocation} (briefly mentioned earlier) where the IO is linearized to the
right as in \eqref{clrd}, or \gls{CLLD}\is{clitic left dislocation} where the
IO has moved to the left periphery of its clause:

\ea \label{clrd}
\ea
    \gll  On {a pr\'{e}sent\'{e}} Jean \`a Pierre.\\
    we introduced Jean to Peter\\
\ex
	\gll  On      lui\tss{\emph{k}} a   pr\'{e}sent\'{e}    Jean,  [ \`a Pierre~]\tss{\emph{k}}.\\
    we  him\tss{\Dat} have introduced     Jean, {} to Pierre \\
\z
\z
%
But neither CLLD nor \gls{CLRD}\is{clitic right dislocation} are exact
equivalents of \glspl{DOC}, even if they share with \glspl{DOC}\is{double object construction} some properties
characteristic of Topics\is{topic} (see \citealt{Polinsky:1996}). Indeed, \gls{CLLD}-ed
constituents are higher than subjects, and \gls{CLRD}\is{clitic right
dislocation} constituents must be understood as backgrounded Topics\is{topic} and thus
can't be contrasted, unlike \glspl{IO} in \glspl{DOC}.

\section{Conclusion}

I have shown that \ili{French} displays mandatory scope freezing effects in the
presence of dative\is{dative case} \isi{clitics} in what superficially look like \glspl{PDC}. I have
attributed these effects to the presence of hidden \glspl{DOC}\is{double object construction} in \ili{French}, which
alone allow an IO to cliticize as a Dative. I have further suggested that
\glspl{DOC} do not surface in \ili{French}, but they constitute an intermediate
derivational step  involved in \gls{CLLD}\is{clitic left dislocation} and
\gls{CLRD}\is{clitic right dislocation}.

Many questions, left unaddressed here, remain.

\begin{enumerate}
\item If the distribution of floated Qs off a DP reveals the presence of traces
    of this DP as in  \citet{Sportiche:1988}, the following type of data:

\ea
\ea
	\gll  On leur avait  (\`a) tous montr\'e le film.\\
    we to-them had \hphantom{(}to all shown the movie.\\
    \trans we had shown the movie to them all
\ex
	\gll On leur avait montré le film *(\`a) tous.\\
    we to-them had  shown the movie \hphantom{*(}to all\\
    \trans we had shown the movie to them all
\z
\z
%
suggests that the distribution of floated Qs off objects interacts with the
derivational steps involved in \glspl{DOC}\is{double object construction} (in a
way reminiscent of what \citealp{Sportiche2017c}, suggests for \ili{English}).

\item The syntax of IO reflexives in some versions of \ili{French} (or in \ili{Italian}),
    where they trigger participle agreement, suggests that the derivational
    steps involved in IO reflexive cliticization display an A-movement syntax:
    a connection with \ili{French} \glspl{DOC}\is{double object construction} suggests itself that merits
    investigation.

\item If the conclusion above is correct, the distribution of \ili{French} Dative
    Clitics only indirectly relates to \glspl{PDC}\is{prepositional dative construction}; the connection
    is instead mediated by \glspl{DOC}. One area on which this indirect
    connection should have a direct bearing is that of causative constructions.

\end{enumerate}

\printchapterglossary{}

\section*{Acknowledgements}

\`A Ian, who once crossed the Irish sea with his gang for a talk and a pint. Or
vice versa.

Thanks to Danny Fox, Elena Anagnostopoulou, Nikos Angelopoulos, and the
participants in my 2017 UCLA proseminar on Scrambling. This work is supported
in part by the NSF under grants 1424054 and 1424336 and by the ANR under grant
12CORP-0014-01.

{\sloppy
\printbibliography[heading=subbibliography,notkeyword=this]
}

\end{document}
