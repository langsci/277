\documentclass[output=paper]{langsci/langscibook}
\author{Marcel den Dikken\affiliation{Eötvös University and RIL HAS}\and
Éva Dékány\affiliation{RIL HAS}}
\title{Rethinking ``defective goal'': Clitics and noun incorporation}

\usepackage{tabularx}

\newcommand{\posscite}[1]{\citeauthor{#1}'s \citeyearpar{#1}}

% \chapterDOI{} %will be filled in at production

\abstract{In this paper we revisit \posscite{mithun84} classic typology of
    noun-incorporation\is{noun incorporation} (NI) constructions and offer an analysis for the various
    types of \gls{NI} using Roberts'  (\citeyear{Roberts2010}) notion of
    \enquote{defective goal}. We suggest that the cross-linguistic variation
    across \gls{NI} types can be captured by three parameters: i) whether the host of
    the incorporated nominal element is V or \emph{v}, ii) whether the
    incorporate is \emph{n} or D with a referential index, and iii) whether the
    object is a \enquote{defective goal} or not.}

\maketitle

\begin{document}\glsresetall
\begin{refcontext}

In his most recent book, \cite{Roberts2010} unfolds a perspective on sundry cases
of \isi{head movement} that is centred on what he calls a \enquote{defective goal}.
The idea is that in syntactic configurations in which a probe π engages in
an Agree relation with a goal γ whose feature content is a proper subset
of that of π, the effects of chain formation and displacement arise without
movement needing to be involved: thanks to the subsective probe--goal relation,
a chain is formed between π and γ, with exponence at π as a
simple result of chain reduction (which as a rule singles the highest member of
a chain out for phonological realisation\footnote{\citet[61]{Roberts2010}:
    \enquote{Usually, the \enquote{head} of the chain -- that is, the
        position that asymmetrically c-commands all the others -- is the one
        non-deleted position because this is the locus of the most
        feature-checking/valuing relations.}}). This approach to \enquote{head
    movement} in terms of subsective probe–goal relations at the same time
    makes the phenomenon squarely syntactic in nature (it is a result, after
    all, of a syntactic Agree relationship) and has the potential to take
    \enquote{movement} out of the equation entirely.\footnote{We write
        \enquote{has the potential to take} rather than simply
        \enquote{takes} because of a lack of clarity on this point in the
        book. On the one hand, \citeauthor{Roberts2010}'
        (\citeyear[160]{Roberts2010}) prose makes it perfectly clear that he
        believes that \enquote{given that copying the features of the
            \isi{defective goal} exhausts the feature content of the goal,
        Agree/Match is in effect indistinguishable from movement. For this
    reason we see the PF effect of movement.} Yet on the other hand, the trees
    that he presents still make it look like movement is involved (as
    \citealt{matushanskyrobertsreview} also notes in her review of the book).
    We take the prose and not the trees to reflect the true nature of Roberts’
thinking on the matter.}

The centrepiece of Roberts' application of the \isi{defective goal} approach to head
movement is his analysis of object cliticisation\is{clitics} in the \ili{Romance} languages. In
our contribution to this volume, we would like to present some thoughts, in the
general spirit of Roberts’ approach but fine-tuning them in a number of ways,
on the syntax of definite direct objects, object \isi{clitics}, and noun
incorporation. Starting out from Roberts' (\citeyear{Roberts2010}) own proposal
for object cliticisation\is{clitics} and his suggestions regarding \isi{noun incorporation}, we
proceed in \Cref{sec:16.2} by reviewing \posscite{mithun84} typology of
noun-incorporation constructions, and develop an explanatory account of this
typology in which Roberts' analysis of \isi{clitics} as defective goals is mobilised
to maximum effect as a point of variation among noun-incorporating languages,
in conjunction with two other microparameters that fit naturally into the
system: the locus (\emph{v} or V) and size (\emph{n} or D) of the incorporated
nominal element. In object cliticisation\is{clitics} and a subset of N-incorporation
constructions, the combination of \emph{v} and a nominal element attached to it
forms a complex probe with a \isi{defective goal}. How \isi{clitic doubling} fits into this
perspective is addressed in \Cref{sec:16.3}. In \Cref{sec:16.4}, we
explore an analysis of object pro-drop afforded by the logical possibility for
\emph{v} by itself to be a proper featural superset of its goal, sanctioning
the latter's silence. After \Cref{sec:16.5} takes a brief look at
definiteness agreement and person, we close in \Cref{sec:16.6} with a note
on an important difference between two ways in which a functional head\is{functional items} can be a
proper featural subset of a c-commanding functional head\is{functional items}: through extended
projection (which does not implicate probing), or via a probe--goal relationship
involving two different extended projections.

We believe that these thoughts taken together enhance, at the empirical level,
the efficacy of Roberts' \enquote{defective goal} approach to \isi{head
movement}
phenomena and, at the theoretical level, the delimitation and significance of
subsective probe--goal relations in the morphosyntax.

%%%%%%%%%%%%%%%%%%%%%%%%%%%%%%%%%
%%%%%%%%%%%%%%%%%%%%%%%%%%%%%%%%%
\section{Clitics}
\label{sec:16.1}

\cite{Roberts2010} takes a novel approach to the problem posed by clitic
constructions, with particular reference to object \isi{clitics}, as in
\ili{French}
\eqref{1b}, the clitic counterpart to~\eqref{1a}, with a full-fledged definite
object-DP.

\ea \ili{French}
    \ea \gll j'ai surpris \emph{les} filles \label{1a}\\
        {I have} surprised the girls\\
        \glt `I surprised the girls'
    \ex \gll je \emph{les} ai surprises\label{1b}\\
            I them have surprised.\glossF.\Pl\\
        \glt `I surprised  them' (said of feminine direct object)
    \z
\z

%
%\begin{exe}
%\ex \begin{xlist}
%        \ex
%\gll j'ai surpris \emph{les} filles\\
%     I have surprised the girls\\
%\glt `I surprised the girls'
%        \ex
%\gll je \emph{les} ai surprises\\
%I them have surprised.\textsc{fpl}\\
%\glt `I surprised  them' (said of feminine direct object)
%            \end{xlist}
%\end{exe}

Roberts argues that in a structural configuration in which a probe π
c-commands a goal γ and the feature content of γ is a proper
subset of that of π, we get the effect of displacement: in such a defective
(π,γ) relationship, only one of the partners is spelled out --
typically the structurally higher one (i.e., the probe, π). Roberts takes
an object clitic to be just a bundle of φ{}-features, and to thereby
constitute a proper subset of \emph{v} -- which, apart from the
φ{}-features that match those of the object also has a category feature
and possibly plenty of other formal features as well.\footnote{For the purpose
    of linearisation, we are representing the clitic as a φ{}-feature
    bundle adjoined\is{adjunction} to \emph{v} and forming a complex probe
    φ{}+\emph{v}. It may be that linearisation can be dealt with in ways
    not exploiting adjunction;\is{adjunction} but for simplicity and transparency, we will use
    adjunction structures throughout the paper. In the adjunction structures
    employed in this paper, the feature content of the adjunction complex is
the sum of the feature bundles of the host and the adjunct.\is{adjunction}  Throughout, we
annotate this as follows: [\tss{\emph{v}} [X\tss{\{FFx\}}]
[\emph{v}\tss{\{FFv\}}]]\tss{\{\{FFx\}, \{FFv\}\}}.\label{fn:19.3}}

\ea {}[\tss{\emph{v}P} [\tss{\emph{v}} [ \hspace{-.6ex}φ\tss{\{φ, [+N]\}}] [ \hspace{-.7ex}\emph{v}\tss{\{[+V], \Acc{}, \ldots{}\}}]]\tss{\{\{φ, [+N]\}, \{[+V], \Acc{}\}\}} [\tss{VP} V φ{}P\tss{\{φ, [+N], \Acc{}\}} \ldots{}]]\label{2}
\z
For Roberts, this explains why the object clitic is spelled out on \emph{v}
rather than in the A-position that non-clitic objects find themselves in: the
probe \emph{v} and the clitic, its \isi{defective goal}, form a chain which, through
chain reduction, gets an exponent (in the form of a pronominal element
representing the φ{}-feature bundle) in the position of the structurally
higher member of the chain, \emph{v}.

\section{Noun incorporation}\glsunset{NI}
\label{sec:16.2}
In Section 4.2.2 of his book, \cite{Roberts2010} unfolds what he calls
\enquote{a note on noun incorporation} (NI), whose purpose it is \enquote{to
    sketch how Baker's data and results concerning \gls{NI} and related issues might
    be captured in terms of the general approach to head-movement advocated
    here, and what some of the consequences of that may be} (p.\ 188). In his
    brief discussion, while rightly stressing the similarity between noun
    incorporation and object cliticisation, Roberts suggests that nouns that
    incorporate are \emph{n}'s associated with an object that projects no
    further up than \emph{n}P. This leads to~\eqref{19.3} as the representation of
    \isi{noun incorporation} constructions along the lines pursued by Roberts:

\ea {}[\tss{\emph{v}P} [\tss{\emph{v}} [ \hspace{-.7ex}\emph{n}\tss{\{[+N]\}}] [ \hspace{-.7ex}\emph{v}\tss{\{[+V], \Acc{}, \ldots{}\}}]]\tss{\{\{[+N]\}, \{[+V], \Acc{}\}\}} [\tss{VP} V \emph{n}P\tss{\{[+N], \Acc{}\} \ldots{}}]]\label{19.3}
\z
Once again, the object is a defective goal: its features form a proper subset
of the features on the complex probe \emph{v}. The noun will therefore be
spelled out on the verb, and the \emph{n}P in VP, the \isi{defective goal}, remains
silent.

Roberts intends this note as an indication of how his notion of a \isi{defective goal} might be of
service in the account of a nominal displacement phenomenon close in nature to object cliticisation.
And indeed, it seems to us that in a proper understanding of the complexities of \isi{noun incorporation},
defective probe--goal relations play an important role. But~\eqref{19.3} is only the tip of the iceberg.

In the ensuing subsections, we will show that (3) does indeed have a place in
the syntax of noun-incorporation\is{noun incorporation} constructions: it accounts well for one
subtype of Type I in Marianne \posscite{mithun84} classic typology of
noun-incorporation phenomena. But Mithun’s typology features several other
members, which also need to be analysed. The goal of the remainder of
\Cref{sec:16.2} is to present an account of Mithun’s complete typology of noun
incorporation, in a theoretically parsimonious way, and mobilising Roberts’
notion of \enquote{defective goal} as fruitfully as possible.\largerpage[-1]

\subsection{The typology of \isi{noun incorporation} and pseudo-incorporation}
\posscite{mithun84} monumental study of noun-incorporation\is{noun incorporation} phenomena in a wide range of different
languages resulted in a typology of four distinct types of N-incorporation cases. Of these, the first
has two subtypes, which we will refer to in this paper as Types I\emph{a} and I\emph{b}; the latter has taken on the
title \enquote{pseudo-incorporation} in the more recent literature on noun
incorporation (see e.g. \citealt{massam01}), and we will often use this label
ourselves when talking about Type I\emph{b}.\footnote{Mithun herself does not
    use the labels \enquote{Type I\emph{a}} and \enquote{Type I\emph{b}}, or
    \enquote{pseudo-incorporation}. She does, however, make an explicit
    distinction among Type I noun-incorporating languages between morphological
    compounding\is{compounding} cases and cases in which the verb and the noun are
    \enquote{simply juxtaposed to form an especially tight bond}. In our
    structural analysis, Types I\emph{a} and I\emph{b} will turn out to be
    quite different: there is no obvious sense, from our point of view, in
    which they should be grouped together as subtypes of one basic
incorporation type. But we will follow Mithun’s classification for the sake of
transparency and straightforward comparison.}

\ea\leavevmode\\[-1\baselineskip]
\begin{tabularx}{.92\textwidth}{lX}
{} & \emph{Descriptive typology of \isi{noun incorporation} phenomena} (based on \citealt{mithun84})\\
I\emph{a} & lexical compounding:\is{compounding} the incorporated noun is non-referential, generic; the incorporation
complex denotes a conventional, institutionalised activity\\
I\emph{b} & \enquote{pseudo-incorporation}: the incorporated noun is non-referential, but shows a much
greater degree of morphosyntactic independence than in lexical compounding\is{compounding}\\
II & the incorporated noun lacks argument status, and does not usurp the verb’s structural
case-assigning capacity, which is redirected to a phrase in the external syntax\\
III & the incorporated noun can be referential and absorbs case, but cannot be associated
with modifiers in the external syntax\\
IV & the incorporated noun can be referential and absorbs case, and can be associated with
modifiers in the external syntax\\
\end{tabularx}\label{mithun}
\z
We will argue in this section that for an understanding of this typology, three
things are essential:

\ea \label{supp}
    \ea  the host of the incorporated nominal element -- V or \emph{v}\label{5a}
    \ex  the nature of the incorporated nominal element -- \emph{n} or D$^i$ (\enquote{i} = \enquote{referential index})\label{5b}
    \ex  the status of the object -- \enquote{defective goal} or not \label{5c}
    \z
\z
When the incorporated nominal element is attached to \emph{v}, it can form an
integral part of the \emph{v} probe that is a proper featural superset of a
defective goal in VP, in the sense of \cite{Roberts2010}. This is what we argue
is the case in noun-incorporation\is{noun incorporation} cases of Types I\emph{b} and III. In Types
I\emph{a}, II, and IV, the object is not a \isi{defective goal} -- either (as in
Types I\emph{a}, IV) because the incorporated element is not attached to
\emph{v} but to V (which is not a probe) or because the object is not a proper
featural subset of the feature content of the complex probe formed by \emph{v}
and the incorporated element adjoined to it.\is{adjunction}

The structural translation of the typology in~\eqref{mithun} that~\eqref{supp}
offers is given in~\eqref{our}, which sums up in a nutshell the proposal that
will be spelled out in the subsections to follow.\footnote{In~\eqref{our} and
    throughout the paper, \enquote{D$^i$} stands for a D(eterminer) with a
    referential index. For simplicity,~\eqref{our} adopts a structural
representation of the object-of relationship in which the object is the
complement of V; but nothing in what follows is incompatible with a
representation of the Theme as the specifier of VP; as in \cite{halekeyser} et
passim.}

\ea\leavevmode\\[-1\baselineskip]
\begin{tabularx}{.92\textwidth}{llX}
{} & \multicolumn{2}{l}{\emph{Structural typology of \isi{noun incorporation} constructions (this paper)}}\\
I\emph{a} & host: &V\\
{} & guest: &\emph{n}\\
{} & object: &none\\
{} & \multicolumn{2}{l}{[\tss{\emph{v}P} \emph{v}\tss{\{[+V], \Acc{}, \ldots{}\}} [\tss{VP} [\tss{V} [\emph{n}\tss{\{[+N]\}}] [V]]]]}\\
I\emph{b} & host:  &\emph{v}\\
{} & guest:  &\emph{n}\\
{} & object:  &defective goal\\
{} & \multicolumn{2}{l}{[\tss{\emph{v}P} [\tss{\emph{v}} \emph{n}\tss{\{[+N]\}} [\emph{v}\tss{\{[+V], \Acc{}, \ldots{}\}}]]\tss{\{\{[+N]\}, \{[+V], \Acc{}, \ldots{}\}\}} [\tss{VP} V \emph{n}P\tss{\{[+N]\}}]]}\\
II & host: &\emph{v}\\
{} & guest: &\emph{n}\\
{} & object:& non-defective goal\\
{} & \multicolumn{2}{l}{[\tss{\emph{v}P} [\tss{\emph{v}} \emph{n}\tss{\{[+N]\}} [\emph{v}\tss{\{[+V], \Acc{}, \ldots{}\}}]]\tss{\{\{[+N]\}, \{[+V], \Acc{}, \ldots{}\}\}} [\tss{VP} V D$^i$P\tss{\{D, φ, [+N]\}}]]}\\
III & host: &\emph{v}\\
{} & guest:& D$^i$\\
{} & object: &defective goal\\
{} & \multicolumn{2}{l}{[\tss{\emph{v}P} [\tss{\emph{v}} D$^i$\tss{\{D, φ, [+N]\}} [\emph{v}\tss{\{[+V], \Acc{}, \ldots{}\}}]]\tss{\{\{D, φ, [+N]\}, \{[+V], \Acc{}, \ldots{}\}\}} [\tss{VP} V \emph{n}P\tss{\{[+N]\}}]]}\\
IV & host:& V\\
{} & guest: &D$^i$\\
{} & object: & non-defective goal\\
{} & \multicolumn{2}{l}{[\tss{\emph{v}P} \emph{v}\tss{\{[+V], \Acc{}, \ldots{}\}} [\tss{VP} [\tss{V} D$^i$\tss{\{ D, φ, [+N]\}} [V]] xNP]]}\\
\end{tabularx}\label{our}
\z
Note that in (6.I\emph{a}) and (6.IV), \emph{v} is included for parallelism
with the other structures -- but while \emph{v} is a necessary ingredient of
the other structures, it can freely be absent from (6.I\emph{a}) and (6.IV).
This will be important later, in the discussion of the transitivity restriction
on \isi{noun incorporation}.

\subsection{Incorporated nouns associated with defective goals}
\label{2.1}

The notion of \enquote{defective goal} is particularly helpful in the analysis of
noun incorporation of Type III, but it also plays a role in the account of Type
I\emph{b}. Let us start with the latter, usually referred to as
\enquote{pseudo-incorporation}.

\subsubsection{Type I\emph{b} pseudo-incorporation}

The representation in~\eqref{7} differs from~\eqref{2} in the size of the object (φ{}P in~\eqref{2}, but a mere \emph{n}P in~\eqref{7})
and in the size of the feature bundle represented by the element adjoined\is{adjunction} to \emph{v} (\{φ{}, [+N]\} in~\eqref{2}, but
just \{[+N]\} in~\eqref{7}: the only feature that \emph{n} contributes is a categorial feature).

\ea {}[\tss{\emph{v}P} [\tss{\emph{v}} \emph{n}\tss{\{[+N]\}} [ \hspace{-1ex}\emph{v}\tss{\{[+V], \Acc{}, \ldots{}\}}]]\tss{\{\{[+N]\}, \{[+V], \Acc{}, ...\}\}} [\tss{VP} V \emph{n}P\tss{\{[+N]\}}]] \label{7}
\z
In both~\eqref{2} and~\eqref{7} the feature content of the complex probe \emph{v}
is a proper superset of that of the object. So the object is a defective goal
in both structures. The representation in~\eqref{7} is the equivalent of Roberts'
(\citeyear{Roberts2010}) suggestion for the syntax of \isi{noun incorporation} in
general, given in~\eqref{19.3}.

\eqref{7} is useful for the analysis of what has been called
pseudo-incorporation. In a typical pseudo-incorporation construction, the
clause shows the valency and case pattern characteristic of intransitives, and
the object is non-referential, lacking a referential index. On the assumption
that referential indices are located on D, this means that pseudo-incorporated
nouns must lack the D layer. But it can be modified (as in~\eqref{8}, from
\ili{Niuean}), indicating that it does not form a complex head with the verb.
\cite{massam01} argues that pseudo-incorporation in \ili{Niuean} involves
determinerless noun phrases.~\eqref{7} translates this structurally by analysing
the internal argument as \emph{n}P, specified for category and hence eligible
for adjectival modification, but not as large as DP. \emph{n}P is not subject
to the Case Filter\is{case!Case Filter}, which is why in~\eqref{8} \isi{absolutive case} is available for
the external argument.

\ea \ili{Niuean}
    \sn\gll ne inu kofe kono i Sione \label{8}\\
    \Pst{} drink coffee bitter \Abs{} Sione\\
    \glt \enquote*{Sione drank bitter coffee.}
\z
The question of whether or not pseudo-incorporated objects form a complex head
with the verb depends, given the proposal in~\eqref{7}, on whether chain
reduction singles out the bottom or the top of the chain for
exponence.\footnote{The question of which member of the chain is spelled out in
    turn depends, at least in part, on whether \emph{n}P remains in VP or makes
    its way into a position outside the c-command domain of the
    \emph{n}+\emph{v} probe by the time of spell-out.  Exponence of \emph{n}P
    will make the incorporation \enquote{covert}, but still ensures that the object
    and the verb are spelled out in close proximity to one another: \emph{n}P,
because of its minimal size, is not eligible for \enquote{scrambling} into positions
beyond \emph{v}P.} When \emph{n}P is the term that is subject to exponence, the
incorporated noun will accept attributive modifiers, as in~\eqref{8}. In cases of
pseudo-incorporation in which the noun does not accept dependents or modifiers,
it will be the \emph{v}-adjoined member of the chain that is singled out for
phonological exponence, with \emph{n}P fully silenced because it is a defective
goal to the \emph{n}+\emph{v} probe.\footnote{We take \ili{Niuean}~\eqref{8} to
    represent the typical pseudo-incorporation pattern. But as our reviewers
    rightly point out, the term \enquote{pseudo-incorporation} has been applied with
    reference to a wide variety of phenomena. The use of bare morphologically
    accusative objects as \enquote{verbal modifiers} in \ili{Hungarian} (as in \emph{János
    verset ír} \enquote{János poem.\Acc{} writes}) has been treated under this
    rubric (see fn.\ \ref{fn:19.13}, below), as have the \enquote{weak definites} of \ili{Germanic}
    (\emph{John plays the double bass}). For the latter, an approach along the
    lines of~\eqref{7} would require a treatment of the article as something
    different from D (see \citealt{zamparelli}). The behaviour of bare singular
    objects in \ili{Norwegian} (\emph{Anna kj\o{}pte bil} \enquote{Anna bought car}), for
    which \cite{kallullidiss} argues that they establish discourse referents
    yet lack the DP-layer (which is arguably why they cannot serve as subjects
of secondary predication: \emph{Anna kj\o{}pte bil*(en) ny} \enquote*{Anna
bought car(\Def{}) new}), might also be folded into~\eqref{7} -- but then the
ability to introduce a discourse referent must (for \ili{Norwegian}, at least) be
divorced from D.\label{fn:19.7}}

\subsubsection{Type III noun incorporation}
In Mithun’s Type III noun-incorporation\is{noun incorporation} languages (which include \ili{Ainu},
\ili{Chukchi}, \ili{Mapudungun}, and \ili{Nahuatl}), the incorporated noun can be fully referential,
playing an active role in the discourse.  \citet[145--6]{bakeretal05}
illustrate this clearly for \ili{Mapudungun}.

\ea \label{9}\ili{Mapudungun}
    \ea[]{\gll ngilla-waka-n; fey lang\"um-fi-\~n\label{9a}\\
    buy-cow-\Ind.\Fsg.\Sbj{} then kill-\Third.\Obj{}\normalfont{-}1\Sbj{}\\
    \glt \enquote*{I bought a cow; then I killed it.}}
    \ex[\#]{\gll ti ullcha domo pe-fi-y ti ay\"u-domo-le-chi wentru\label{9b}\\
    the young woman see-\Third.\Obj{}-\Ind.\Tsg.\Sbj{} the
love-woman-\Stat-\Adj{} man\\}\newpage
    \ex[]{\gll ti ullcha domo \~ni chaw pe-fi-y ti ay\"u-domo-le-chi wentru\label{9c}\\
    the young woman \Third.\Poss{} father see-\Third.\Obj{}-\Ind.\Tsg.\Sbj{} the love-woman-\Stat-\Adj{} man\\
    \glt \enquote*{the young woman$^{\#{}}$('s father) saw the man who loved the
woman}}
    \z
\z
In~\eqref{9a}, we see that in \ili{Mapudungun} an incorporated noun can set up a new
discourse referent and serve as the antecedent for a referentially dependent
element. In the contrast between~\eqref{9b} and~\eqref{9c}, we discover a Principle
C effect similar to the one found in the \ili{English} translations, which suggests
that the incorporated object behaves in syntax like an independent referential
expression does in languages such as \ili{English}.

The fact that the incorporated noun in Type III constructions can be fully
referential suggests that such \isi{noun incorporation} should be given a different
analysis from the one proposed in the previous subsection for
pseudo-incorporation, with the difference lying in the size of the object.
While for pseudo-incorporation a bare \emph{n}P, as in~\eqref{7}, seems right on
target, for Type III \isi{noun incorporation} we need an object that can harbour a
referential index. If, as is standardly assumed, D is the locus of referential
indices, the D-head must be active in the syntax of noun-incorporating
languages of the \ili{Mapudungun} type, Mithun’s Type III. We introduce this D-head
(D$^i$, where \enquote{i} is the referential index) directly on \emph{v}, serving as
the incorporated element, as shown in~\eqref{10}. This D forms a discontinuous
object with the \emph{n}P in the θ{}-position. The noun lexicalises
the  D-head, which is what gives rise to physical incorporation into the verb.
(We will return to lexicalisation in \Cref{sec:16.2.4}.)

\ea {}[\tss{\emph{v}P} [\tss{\emph{v}} D$^i$\tss{\{D, φ, [+N]\}} [ \hspace{-.4em}\emph{v}\tss{\{[+V], \Acc{}, \ldots{}\}}]]\tss{\{\{D, φ, [+N]\}, \{[+V], \Acc{}, \ldots{}\}\}} [\tss{VP} V \emph{n}P\tss{\{[+N]\}}]] \label{10}
\z
In Type III constructions there can be no \enquote{modifier stranding}, which means
that it is impossible for the constituent situated in the object position of
the verb to harbour any modifiers associated to the incorporated
object.\footnote{We use \enquote{modifier stranding} as the familiar descriptive
    term for this, even though it will emerge later in the paper that we do not
    actually take modifiers of an incorporated noun that occur outside the
    incorporation complex to have literally been stranded (by movement of the
    noun). We would also like to emphasise that under \enquote{modifier stranding}
    we do NOT understand the presence of external possessors: this is a
different phenomenon, often associated with \enquote{possessor ascension}. See
\citet[168]{bakeretal05} for discussion of the concerns raised by \enquote{possessor
stranding/ascension}.} This will follow immediately if in the syntax of Type
III noun-incorporation\is{noun incorporation} constructions, the object position is structurally
occupied by a \isi{defective goal} of the \emph{v} probe.

Because the \isi{defective goal} is destined to complete silence under  Roberts'
(\citeyear{Roberts2010}) proposal, it cannot harbour any modifiers of the
incorporated noun. In the structure in~\eqref{10}, the \emph{n}P in the verb's
object position is, by Roberts' logic, a \isi{defective goal} that remains completely
silent at PF. Any modifiers merged inside \emph{n}P will be silenced along with
the rest of \emph{n}P. Adjunction\is{adjunction} of modifiers to \emph{n}P itself is
impossible because \emph{n}P occupies a θ{}-position: adjunction to
θ{}-role bearers is impossible (\citealt[6]{Chomsky1986},
\citealt[57]{mccloskey96}).

In their detailed comparative study of noun-incorporating languages,
\cite{bakeretal05} find that in Type III languages, the verb does not engage in
morphological agreement with the incorporated object. The structure in~\eqref{10}
derives this -- in part on principled grounds, and in part by executive
decision. The principled part of the agreement story is the relation between
the \emph{v}-adjoined D (which is the locus of the referentiality of the
object) and the \emph{v} probe: since \emph{v} does not c-command the D
adjoined\is{adjunction} to it, it cannot establish an Agree-relation with this D. But \emph{v}
does c-command the object, to which D is linked and with which it forms a
discontinuous object. If this object were as large as φ{}P, it should be
able to control φ{}-feature agreement with \emph{v}, which is not what
we find in the languages studied by Baker et al. In these languages, the object
position of the verb, to which V assigns its θ{}-role, is occupied by
something too small (\emph{n}P) to be able to engage in a morphological
φ{}-agreement relationship with \emph{v}.

But though the size of the nominal construct in the object-of-V position
in~\eqref{10} must be such that it is a \isi{defective goal} for the D+\emph{v} probe,
it is not guaranteed to be as small as \emph{n}P: the syntax of \eqref{10} would
be convergent also if the object were a φ{}P. Our analysis of Type III
noun-incorporation constructions thus leads us to suspect that the correlation
that \cite{bakeretal05} found between absence of \enquote{modifier stranding} and
absence of agreement with the object is not necessarily absolute: there could
be Type III noun-incorporating languages which do evince φ{}-feature
agreement with the object. Whether such languages exist is something we are not
in a position to confirm at this time.

\posscite{bakeretal05} third hallmark of Type III incorporating languages is
that in these languages, incorporation of the (deep) object into an
unaccusative\is{unaccusativity} verb is impossible.\footnote{Barring (in some languages)
    meteorological predicates and constructions in which the incorporated noun
    is associated with a possessor. \cite{bakeretal05} have an account for
these cases -- one which does not directly carry over to the proposal
in~\eqref{10}. We have no immediate solution to offer for these exceptions.}
Baker et al. derive this in a rather complicated way, with an appeal to
φ{}-feature deletion on the \enquote{trace} of the incorporated noun, in
conjunction with a particular interpretation of the
\glsunset{EPP}\gls{EPP}.\is{Extended Projection Principle} For us, the
correlation between absence of \enquote{modifier stranding} and the ban on
incorporation of unaccusative\is{unaccusativity} objects is also expected to necessarily be an
absolute one. And as a matter of fact, from our proposal it follows much more
straightforwardly than it does from Baker et al.’s: in the analysis of Type III
incorporating languages in~\eqref{10}, the locus of incorporation is \emph{v},
and this element is either not present in the syntax of
unaccusative\is{unaccusativity}
constructions at all, or too weak to be able to support incorporated nominal
elements.\footnote{\posscite{chomskymp} original \emph{v}-hypothesis had it
    that \emph{v} is responsible for the checking of \isi{accusative case} AND for
    the assignment of an external θ{}-role to the subject of a
    transitive clause. More recent work has extended the distribution of
    \emph{v} to all things verbal, making a distinction between \emph{v}* (the
    \enquote{strong} \emph{v} that occurs in transitive constructions and assigns an
    external θ{}-role) and \enquote{unstarred} \emph{v} (the \enquote{weak}
    \emph{v} found everywhere else). On that approach, the strong correlation
    between absence of \enquote{modifier stranding} and absence of incorporation in
    unaccusative\is{unaccusativity} constructions can still be made to follow from~\eqref{10}, on
the assumption that \enquote{\emph{v}} here is specifically the transitivising
\emph{v}*.}

\subsection{Incorporated nouns not associated with defective goals}
\label{sec:16.2.3}
In Type IV noun-incorporating languages, incorporation of the (deep) object of
unaccusative\is{unaccusativity} verbs is unrestricted. \cite{bakeretal05} find that in Type IV
languages it is also quite generally possible to strand modifiers, unlike in
Type III languages. These things suggest that the host of the incorporate is
different in Type IV languages, and that the object in these languages is not a
defective goal.

\subsubsection{Type IV noun incorporation}
Type III and Type IV noun-incorporating languages are on a par (and as a pair
differ in this regard from the other noun-incorporation\is{noun incorporation} types) when it comes to
the referentiality of the incorporated noun. \citet[287--8 and sect.
7.4.3]{baker96} shows for Mohawk, in the same way that  \cite{bakeretal05}
later did this for \ili{Mapudungun} (recall~\eqref{9}), that the incorporated noun is
fully active in the discourse.  From our point of view, this means that a
D-head is involved in Type IV noun-incorporation\is{noun incorporation} constructions, just as it is
in Type III. It is important to establish that this is something the two types
have in common.

But besides this parallel, \cite{bakeretal05} demonstrate that Mithun’s Type IV
noun-incorporating languages (including \ili{Mayali}, \ili{Mohawk},
\ili{Southern Tiwa}, and \ili{Wichita}) are diametrically opposed to Type III in three respects. We just
mentioned that Type IV languages, unlike those of Type III, allow \enquote{modifier
stranding} and incorporation in unaccusative\is{unaccusativity} contexts; in addition, in Type IV
languages but not in Type III ones, the verb agrees morphologically with the
incorporated noun.  What might the difference between Types III and IV be, in
analytical terms, such that these divergences fall out?

Our hypothesis regarding Type IV noun-incorporation\is{noun incorporation} constructions is that the
incorporated D (spelled out as a noun) is attached, not to \emph{v} but to V,
as shown in~\eqref{11}:

\ea  {}[\tss{\emph{v}P} \emph{v}\tss{\{[+V], \Acc{}, \ldots{}\}} [\tss{VP}
[\tss{V} D$^i$ \tss{\{D, φ, [+N]\}} [ \hspace{-.7ex}V]] xNP]]\label{11}
\z
D does not form a discontinuous object with xNP (some extended projection of N)
in the object position: although they can be interpretively linked (in a
relationship of specification), the two are merged independently of one
another. Importantly, in its V-adjoined position,\is{adjunction} D is not in a position to
probe anything because its host, V, is not itself a probe. xNP, therefore, is
not a \isi{defective goal}, and not doomed to silence. This means that when both are
present in the structure simultaneously, xNP and D can both be spelled out. xNP
can harbour modifiers that are semantically associated with the incorporated
object, creating the impression of \enquote{modifier stranding} -- although the
modifier, included in xNP, is not actually being \enquote{stranded} by
anything.\footnote{The proposed approach to \enquote{modifier stranding} is
    compatible with \posscite{rosen89} representation of \enquote{stranded}
    modifiers as associated with a silent noun, though it is not necessarily
    dependent on that representation.

A treatment of \enquote{modifier stranding} that does not take this term literally
is recommended by the fact that the external-syntactic material associated with
the incorporated noun in Type IV languages is not necessarily representable as
a subconstituent of the noun phrase of which the incorporated noun is supposed
to be the (moved) head. The external material in Type IV is characterised by
Mithun as \enquote{classificatory} material. Its function is to specify the content
of the incorporated noun further. This can be done by modifiers in the
traditional sense (\enquote{red} as further specifying the content of \enquote{car}), but
it can also be achieved by another, more specific nominal expression
(\enquote{Cadillac} as a further specification of the content of \enquote{car}). The
generalisation covering external material in Type IV languages is that it is
specificational -- regardless of how the content specification that it brings
about is syntactically represented (i.e., irrespective of whether or not it can
be mapped into a noun phrase).}

Besides the possibility of \enquote{modifier stranding} (or, better put, the
presence of \enquote{classificatory} or specificational material in the external
syntax),~\eqref{11} also correctly predicts the fact that the incorporated object
(the V-adjoined D) in Type IV noun-incorporation\is{noun incorporation} constructions enters into an
agreement relation with the verb and checks \emph{v}’s case feature. This is
thanks to the fact that \emph{v} in~\eqref{11} c-commands the \emph{v}-adjoined
D$^i$ and can hence engage in an Agree relationship with D$^i$.\is{adjunction}

Thirdly,~\eqref{11} also makes it immediately understandable that in Type IV
noun-incorporating languages, it is possible for the (deep) object of
unaccusative\is{unaccusativity} verbs to incorporate. After all, nothing in~\eqref{11} implicates
\emph{v} in the incorporation process: the incorporated element (D with its
referential index \enquote{i}) is attached to V; this should be possible regardless
of whether \emph{v} is present or not (or on the featural properties of
\emph{v} when present).

A clear prediction made by~\eqref{11} that is not raised by \cite{bakeretal05}
but which is indeed fulfilled is that in Type IV noun-incorporating languages
the incorporated object must be a thematic dependent of the incorporator.
Consider in this context the Mayali examples in~\eqref{12} \citep{evans94}:

\ea \label{12}\ili{Mayali}
    \ea[*]{\gll an-barndadja gu-wukku ngarri-\emph{mim}-wo-ni\label{12a}\\
    III-owenia\_vernicosa \Loc{}-water \First\textsc{a}\normalfont{-}fruit-put-\textsc{pi}\\}
    \ex[]{\gll an-barndadja ngarri-\emph{mim}-bo-wo-ni\label{12b}\\
    III-owenia\_vernicosa \First\textsc{a}\normalfont{-}fruit-water-put-\textsc{pi}\\
    \glt \enquote*{we used to put the fruit of Owenia vernicosa in the water}}
    \z
\z
What we see in~\eqref{12a} is that \emph{mim} \enquote*{fruit} cannot be incorporated into the
verb \emph{wo} \enquote*{put} by itself. This is because \emph{mim}, in the structure
of a \enquote{put}-type construction, is not a direct argument of the verb: the
predicate for \emph{mim} is \emph{gu-wukku} \enquote*{in the water}, or, on a
Larson/Hale \&{} Keyser-style approach, the complex predicate \emph{gu-wukku}
\emph{wo} \enquote*{put in the water}, not the verb by itself. Since Type IV
incorporation, on our analysis in~\eqref{11}, involves the adjunction of the
incorporated noun directly to the verbal root V, and since by hypothesis such
adjunction\is{adjunction} is legitimate only if there is a direct thematic relationship
between V and the incorporated material, it is impossible in \ili{Mayali}~\eqref{12a}
to incorporate \enquote*{fruit} into \enquote*{put}. Interestingly, it is possible to
incorporate \enquote*{fruit} when \enquote*{water} forms a complex verb with \enquote*{put}, as
in~\eqref{12b}. This is immediately understandable as well: \emph{bo-wo}, the
head-level combination of \enquote*{put} and \enquote*{water} that we find in~\eqref{12b},
takes \enquote*{fruit} as its argument, and can therefore serve as a host for
\emph{mim} at
the level of V.\footnote{For completeness, we mention here that the complex
    verb \emph{bo-wo} \enquote*{put in water} can also take \emph{mim} as its
    argument externally, as in~\eqref{i}. Note that the form of the element
    glossed as \enquote*{water} is very different in~\eqref{12a} (\emph{wukku}) from
    the form found in~\eqref{12b} and~\eqref{i} (\emph{bo}). We take this to
    suggest that \emph{bo} in~\eqref{12b} and~\eqref{i} is not an incorporated
    locative but rather a base-generated subpart of a complex verb \enquote*{put in
    water}.

\begin{exe}
    \exi{(i)} \ili{Mayali}
    \sn
    \gll an-barndadja an-mim ngarri-bo-wo-ni\label{i}\\
    III-owenia\_vernicosa III-fruit \First\textsc{a}\normalfont{-}water-put-\textsc{pi}\\
\end{exe}} The Mayali data in~\eqref{12} thus support the idea that noun
incorporation in Type IV languages involves a thematic relation between the
incorporated noun and its verbal host, V.

This is a good moment to mention that in our approach to the difference between Type III
and Type IV noun-incorporating languages, we take a stance that is almost exactly the opposite of
the one taken by \cite{rosen89} in her lexicalist analysis of \isi{noun incorporation}. For \cite{rosen89},
the difference between Type III and Type IV languages is that in the former, the incorporated noun
saturates a θ{}-role in the verb's argument structure whereas in the latter it modifies that role, allowing
for the assignment of the (modified) θ{}-role to a phrase in the external syntax. For us, on the other
hand, Type IV languages are characterised precisely by the fact that the incorporated noun (adjoined
directly to V)\is{adjunction} receives a θ{}-role from V. The material in the external syntax that the incorporated noun
may be associated with in Type IV languages (\enquote{xNP} in~\eqref{11}) is not, on our analysis, a thematic
dependent of the verb: rather, it stands in a specificational relationship to the incorporated noun.

\subsubsection{Type I\emph{a} noun incorporation}
For noun-incorporation\is{noun incorporation} cases of Types II–IV, there has always been much debate in the literature
regarding the question of whether they should be given a lexical or a syntactic treatment. In the mainstream
generative literature, \posscite{rosen89} paper is the primary representative of the lexicalist
approach, and Baker’s (\citeyear{baker88}, \citeyear{baker96}) work is the main champion of the syntactic approach. For Type
I\emph{a}, on the other hand, there has never been any doubt as to how it should be treated: there is a broad
consensus that this is a case of lexical compounding.\is{compounding}

In standard, pre-1990s work on the syntax/lexicon distinction, the term
\enquote{lexical compounding}\is{compounding} used to make reference to cases in which a
lexical element is attached to another lexical element in the lexicon, i.e.,
prior to entering the syntactic component. But in a theory in which there is no
distinction, in the realm of derivational processes, between the lexicon and
the syntax (i.e., in a theory in which \enquote{lexical word-formation
operations} are part and parcel of the syntactic component), we can no longer
appeal to a difference in timing between \enquote{lexical compounding}\is{compounding} and the
kind of \isi{noun incorporation} seen in Type IV languages. There is just a single
derivational engine, called \enquote{syntax}. So if the term \enquote{lexical
compounding} is to mean anything in a single-engine theory of morphological and
syntactic derivation, it can only make reference to the size of the elements
combined: \enquote{lexical compounding}\is{compounding} involves the combination of two
elements that are both \enquote{lexical}; Type IV \isi{noun incorporation} combines a
lexical element with something that is not \enquote{lexical}.

Let us make this more precise. What we are describing is a difference between
two types of \isi{noun incorporation}, Types I\emph{a} and IV. Both specifically
involve nouns -- the process of \isi{noun incorporation} is to be distinguished from
cases of preposition incorporation or verb incorporation. So at a minimum, the
incorporated element in all cases of \enquote{noun incorporation} must be
categorised as being nominal. If we take the bare root (\enquote{N}) to be
acategorial (as is standard in current mainstream generative morphosyntactic
theorising), then in all cases of \enquote{noun incorporation} the adjoined\is{adjunction}
element must minimally be as large as \emph{n}, the categorising
\enquote{little head} that identifies the root as a nominal one. For
\enquote{lexical compounding}\is{compounding} (i.e., Type I\emph{a} incorporation), this is
exactly what we take the incorporated element to be: a \enquote{little
\emph{n}} adjoined directly to the verbal root, as in~\eqref{13}. What makes Type
I\emph{a} incorporation different from Type IV incorporation, as analysed
in~\eqref{11}, is thus not the nature of the host (V in both cases) or the timing
of the adjunction to V, but the size of the adjunct (lexicalised as a noun in
both cases; see \Cref{sec:16.2.4}): \emph{n} in Type I\emph{a}, and D$^i$
in Type IV.\is{adjunction}

\ea  {}[\tss{\emph{v}P} \emph{v}\tss{\{[+V], \Acc{}, \ldots{}\}} [\tss{VP}
[\tss{V} [ \hspace{-.7ex}\emph{n}\tss{\{[+N]\}}] [ \hspace{-.7ex}V]]]]\label{13}
\z
Both Type I\emph{a} and Type IV \isi{noun incorporation} are characterised by the
fact that the incorporated noun is attached directly to V, a lexical root.
Viewed from the perspective of the host, then, we could call both Type
I\emph{a} and Type IV incorporation \enquote{lexical}. The difference between
them lies in the size of the nominal adjunct. Due to the fact that the
incorporated nominal element is a mere \emph{n}, it is not a referential
element in Type I\emph{a} incorporation. The combination of \emph{n} and V is
entirely devoid of morphosyntactic content besides the adjunct's category\is{adjunction}
feature. Since the incorporated element is no larger than \emph{n}, it cannot
be associated with anything in the external syntax with which it forms a
discontinuous object: \emph{n} is itself the lowest point in the functional
sequence. So \enquote{modifier stranding} or external specification is
impossible in Type I\emph{a}.

In Type I\emph{a} \isi{noun incorporation}, as in Type IV, the locus of the
incorporated nominal element is V. In our discussion of Type IV cases in the
previous subsection, we noted that this derives an important fact about such
cases: that the incorporated element must bear a thematic relation to the
incorporator. For Type I\emph{a} incorporation, this holds as well -- as a
matter of fact, this is something that \cite{halekeyser} draw prominent
attention to in their discussion of conversion in \ili{English}, which on their
syntactic approach is an instance of Type I\emph{a} \isi{noun incorporation}.

\cite{halekeyser} point out a striking regularity in the pattern of denominal
verb formation in \ili{English} (and similar languages). In the pairs
in~\eqref{14}--\eqref{16}, we see that it is systematically impossible to base
denominal verbs on the nominal head of the Theme argument of a complex
predicate -- despite the fact that the denominal verbs in the b-examples do
exist independently (see the expressions immediately below them), they cannot
be used in resultative secondary predication constructions in which the nominal
base of the verb serves as the Theme of the complex predicate of which the
constituent to the right of the verb is a part.

\ea \label{14}
    \ea[]{to shelve a book}
    \ex[*]{to book on a shelf}
    \ex[]{to book a ticket}
    \z
\z

\ea \label{15}
    \ea[]{to clear a screen}
    \ex[*]{to screen clear}
    \ex[]{to screen a movie}
    \z
\z

\ea \label{16}
    \ea[]{to coat a house (with paint)}
    \ex[*]{to house with a coat (of paint)}
    \ex[]{to house a family}
    \z
\z
A denominal verb can be formed out of an abstract verb (like
\enquote{\textsc{put}} in~\eqref{14}, \enquote{\textsc{make}} in~\eqref{15},
and \enquote{\textsc{provide}} in \eqref{16}) and a secondary predicate with
which it combines, as in the a-examples; but when the element incorporated into
the abstract verb is an argumental noun whose θ{}-role is not assigned to it by
the abstract verb by itself, as in the b-cases (where the incorporated noun
that serves as the base for the denominal verb is the Theme argument of
\enquote{\textsc{put}}, \enquote{\textsc{make}} or \enquote{\textsc{provide}}
plus the secondary predicate that follows the verb), the output is
ungrammatical. The regularity of the pattern discovered by \cite{halekeyser}
strongly suggests that \isi{noun incorporation} of Type I\emph{a} is subject to a
thematic restriction -- one that follows straightforwardly from an analysis in
which the locus of incorporation is the verbal root \enquote{V} (as in Type
IV).

\subsubsection{Type II noun incorporation}

The two cases of \isi{noun incorporation} just discussed (Types I\emph{a} and IV) are
both characterised by the attachment of the incorporated noun directly to the
verbal root, which makes these \enquote{lexical} incorporation cases in the
relevant sense of the term. Thanks to its being attached directly to V, the
incorporated noun in Type I\emph{a} and Type IV is an argument of the predicate
head. In Type II constructions, by contrast, the incorporated noun does not
have argument status. The fact that the incorporate lacks argument status
vis-\`a-vis the verb indicates that it is not attached to the verbal root: if
it were, it would necessarily get the root's internal θ{}-role assigned to it.
So from the incorporate's non-argument status, we conclude that Type II noun
incorporation must be like Types I\emph{b} and III in having the incorporate
attached to \emph{v} rather than to V.

A defining property of Type II that sets it apart from Type III is that the incorporate does not
absorb \emph{v}'s case.\footnote{In Type I\emph{b} pseudo-incorporation of the
    \ili{Niuean} type (recall~\eqref{8}), the pseudo-incorporate also does not absorb
\emph{v}'s case. The incorporate in both~\eqref{7} and~\eqref{17} is a \emph{n}; and in~\eqref{7} even the occupant of the complement-of-V position is
just a \emph{n}P. In fn.\ \ref{fn:19.7}, we mentioned that \ili{Hungarian} \enquote{verbal modifier}
constructions such as \emph{János verset ír} \enquote*{János poem writes} could be
treated as pseudo-incorporation constructions of Type I\emph{b}. Here we see an
explicitly case-marked nominal object (\emph{vers-et} \enquote*{poem-\Acc{}}),
classified as a \enquote{mere} \emph{n}P. It is quite generally possible in
Hungarian for morphological case\is{case!morphological case} to be hosted by things that are not
necessarily as large as a full-blown DP (even non-nominal constituents can bear
morphological case: \emph{Mari jót futott} \enquote*{Mari good.\Acc{} ran,
i.e., Mari had a good run}; \emph{Marit szépnek tartom} \enquote*{I consider
Mari.\Acc{} pretty.\Dat{}}; \emph{szépnek, Mari szép} \enquote*{(as for)
pretty.\Dat{}, Mari is pretty}). But there is no universal requirement
that \emph{n}P have case: Universal Grammar\is{Universal Grammar} only demands that DPs have case
(the Case Filter\is{case!Case Filter}). In Type II incorporation constructions, by contrast, the DP
present in VP must necessarily engage in a case-checking
Agree-relationship\is{Agree}
with \emph{v}.\label{fn:19.13}} If the incorporate were as large as D$^i$, this would be hard
to account for: a D with a referential index wants case (i.e., is subject to
the Case Filter\is{case!Case Filter}). From this, we conclude that Type II incorporation involves a
\emph{n} adjoined to \emph{v} (see~\eqref{17}). In this regard, Type II is like
Type I\emph{b}.\is{adjunction}

\ea  {}[\tss{\emph{v}P} [\tss{\emph{v}} \emph{n}\tss{\{[+N]\}} [ \hspace{-.7ex}\emph{v}\tss{\{[+V], \Acc{}, \ldots{}\}}]]\tss{\{\{[+N]\}, \{[+V], \Acc{}, \ldots{}\}\}} [\tss{VP} V D$^i$P\tss{\{D, φ, [+N]\}}]]\label{17}
\z
Unlike in the case of pseudo-incorporation (Type I\emph{b}; recall~\eqref{7}),
however, the complex probe [$_v$ \emph{n}+\emph{v}] is not a proper featural
superset of the object, which is a full DP originating in the object position
merged independently of the incorporated object. In \citeauthor{mithun84}'s
(\citeyear[859]{mithun84}) terms, \enquote{[i]nstead of simply reducing the
valence of the V by one, [Type II] permits another argument of the clause to
occupy the case role vacated by the IN} (i.e., the incorporated noun). The
b–examples in~\eqref{18} and \eqref{19}, from Yucatec Mayan (adapted from
\citealt[858]{mithun84}), illustrate this:

\ea\label{18}\ili{Yucatec Mayan}
    \ea \gll k-in-\v{c}'ak-k \v{c}e' i\v{c}il in-kool\\
    \Incmpl{}\normalfont{-}I-chop-\Ipfv{} tree in my-cornfield\\
    \glt \enquote*{I chop the tree in my cornfield.}
    \ex \gll k-in-\v{c}'ak-\v{c}e'-t-ik in-kool\\
    \Incmpl{}\normalfont{-}I-chop-tree-\Tr-\Ipfv{} my-cornfield\\
    \glt \enquote*{I clear my cornfield.}
    \z
\z

\ea\label{19}\ili{Yucatec Mayan}
    \ea \gll k-in-wek-k ha'\\
    \Incmpl{}\normalfont{-}I-spill-\Ipfv{} water\\
    \glt \enquote*{I spill water.}
    \ex \gll k-in-wek-ha'a-t-ik \emph{pro}\\
    \Incmpl{}\normalfont{-}I-spill-water-\Tr-\Ipfv{}\\
    \glt \enquote*{I splash him.}
    \z
\z
In Type II incorporation cases (which resemble applicative constructions of the
Bantu type, as \cite{rosen89} also notes), the feature sets of \emph{n} and DP
each get their own exponents: the \enquote{associate} of the incorporate is
not a \isi{defective goal}, and is not condemned to silence. In its base position,
the DP can check the verb's \isi{accusative case} feature, and behaves in every way
like an ordinary object. This accounts for all the properties of Type II
incorporation.

\subsection{On discontinuous objects and spanning}
\label{sec:16.2.4}
At the end of this survey of the typology of noun-incorporation\is{noun incorporation} constructions, we address two analytical
details to which we have so far paid scant attention but which are vital ingredients of the account.


In the structures of Type III and Type IV \isi{noun incorporation}, the incorporated nominal
element is represented as a D (attached to \emph{v} in Type III and to V in Type IV). In Type III cases, this
D is associated with a \emph{n}P in the object position. Two questions arise in connection with this:

\vspace{1ex}
\begin{tabularx}{\textwidth}{lX}
\emph{(a)} & how can D, a determiner head, have a noun as its exponent (as desired)?\\
\emph{(b)} & how can D be associated with the \emph{n}P in object position in Type III constructions?\\
\end{tabularx}


\vspace{1ex}

Let us start with question \emph{(a)}. The key idea here is that, in
noun-incorporation languages of Types III and IV, lexical nouns can serve as
exponents of a \enquote{span} (in the terminology of \isi{nanosyntax}).  A span is a
series of heads in head--complement relations. The languages in question have
lexical entries that can expone the entire nominal functional sequence, from
\emph{n} all the way up to D.  This is correlated with the typological fact
that polysynthetic languages as a rule lack true determiners
\citep{baker96}.\footnote{For \emph{ti} in \ili{Mapudungun}~\eqref{9b} and~\eqref{9c}, we
assume that it is not a true determiner but more like a demonstrative.}
Determinerlessness is a result of the lexical noun's representing the entire
string of functional heads\is{functional items} in the extended projection of N, up to and including
D. When N and D are in a contiguous span in the tree, they can and therefore
must be co-lexicalised by a single morpheme, the \enquote{lexical noun}. This is the
result of an economy principle variously known as Minimise Exponence
(\citealt{siddiqi}, cf. also \citealt{noyer93}), the Union Spellout Mechanism
\citep{muriungiunionspo}, or Maximise Span \citep{pantchevalocationsourcegoal}.
The D attached to the verb in Types III and IV harbours the feature content of
this entire functional sequence, and, in the languages in question, receives
the lexical noun as its exponent.\footnote{When D and N are not in a contiguous
    sequence, they can, in principle, both be separately exponed by the lexical
    noun, provided that \emph{n} or N is not a \isi{defective goal} to a probe with D
attached to it. For further relevant discussion of \enquote{spanning}, see
\cite{ramchand08}, \cite{taraldnguni}, \cite{merchant15} and
\cite{svenonius16}.}

Regarding question \emph{(b)}, in the noun-incorporation\is{noun incorporation} structure
in~\eqref{10}, for Type III, the DP that serves as the object of the verb is
discontinuous: its D- and φ{}-portions are base-generated directly on \emph{v},
very much like an object clitic like \emph{les} in \ili{French}~\eqref{1b} (a
determiner with φ-feature\is{φ-features} content); the rest of the noun phrase (\emph{n}P)
occupies the object position in VP, where the noun phrase hooks up to the
thematic role that it requires for interpretation as an argument of the verb.
The discontinuity of the definite object, with D generated outside VP, is
directly in the spirit of work by \cite{sportiche98} and \cite{lin00}. In the
configuration in~\eqref{10}, D is part of a D+\emph{v} complex that is a
featural superset of the \emph{n}P in object position, which serves as a
defective goal for the D+\emph{v} probe.  Chain reduction leads to the
silencing of the \isi{defective goal}, and exponence of the object in
\emph{v}-adjoined position.\is{adjunction}

\subsection{Noun incorporation: Summary}
In this section, we have presented a proposal for the typology of \isi{noun incorporation} that preserves
and extends \posscite{bakeretal05} major results, recasting their main parameters and supplementing
them with Roberts' (\citeyear{Roberts2010}) notion of \enquote{defective goal}, thereby achieving greater descriptive adequacy
than either Roberts or Baker et al. would have been able to attain by themselves.

Noun incorporation constructions of Types I\emph{a} and IV are united in our
analysis by their choice on~\eqref{5a}: they both pick V rather than \emph{v} as
the host. The other three types of \isi{noun incorporation} all have the incorporated
element hosted by \emph{v}. Types I\emph{a} and IV differ in the nature (and
concomitantly the size of the feature bundle) of the incorporate~\eqref{5b}:
\emph{n} versus D$^i$. Types I\emph{b} and III are distinct from one another in
this way as well. Type II is like Type I\emph{b} with respect to the choices
on~\eqref{5a} and \eqref{5b}; but in Type II the object in VP is not a defective
goal, in the sense of \cite{Roberts2010}, for the \emph{n}+\emph{v} probe: it is
a full-blown argumental and referential DP. So~\eqref{5c} is what makes the
difference between Type II \isi{noun incorporation} and pseudo-incorporation (Type
I\emph{b}), the latter behaving with regard to~\eqref{5c} like Type III noun
incorporation.

\begin{exe}
    \exi{(5)}
    \begin{xlist}
    \ex  the host of the incorporated nominal element -- V or \emph{v}
    \ex  the nature of the incorporated nominal element -- \emph{n} or D$^i$ (\enquote{i} = \enquote{referential index})
    \ex  the status of the object -- \enquote{defective goal} or not
    \end{xlist}
\end{exe}
Taken together,~\eqref{5a}--\eqref{5c} provide just the right parameters to
differentiate between the various distinct types of noun incorporation
identified in the literature. With just~\eqref{5a} and~\eqref{5b}, we would have
been able to describe most of the differential properties of noun incorporation
that \cite{bakeretal05}  manage to account for in their important work --
albeit in a non-trivially different way: where Baker et al. bank heavily on a
parameter regarding the deletion of φ{}-features from the \enquote{trace} of noun
incorporation, the present analysis eschews movement (hence \enquote{traces} or
multiple copies) altogether and capitalises on two formal properties of the
incorporated element (its host and the size of its feature bundle). It is
thanks to our third parameter,~\eqref{5c}, that we get a purchase on the
difference between Types I\emph{b} and II, and, more generally, on the
distribution of external-syntactic material associated with the incorporated
element (\enquote{modifier stranding}). \cite{bakeretal05} explicitly set Type II
aside, and do not talk about pseudo-incorporation at any length. For a full
perspective on the typology of \isi{noun incorporation}, Roberts’
(\citeyear{Roberts2010}) notion of \enquote{defective goal} (which Baker et al. did
not have the benefit of) is essential.




%%%%%%%%%%%%%%%%%%%%%%%%%%%%%%%%%
%%%%%%%%%%%%%%%%%%%%%%%%%%%%%%%%%
\section{On doubling}
\label{sec:16.3}

In the syntax of Type II \isi{noun incorporation}, the incorporated noun (a \emph{n}
attached to \emph{v}) can freely cooccur with an overt DP object in VP because
the \emph{n}+\emph{v} probe is not a proper featural superset of the DP in
object position. In Type IV, the incorporate is itself a large feature set (D);
but because it attaches low, to V rather than \emph{v}, and because D+V is not
a probe, an object in VP is never going to be a \isi{defective goal} in the sense of
\cite{Roberts2010} either. In Type III noun-incorporation\is{noun incorporation} constructions, by
contrast, the incorporated element is a D and its host is \emph{v} -- so here
we get a complex probe D+\emph{v} that is a proper featural superset of any
object inside VP, thereby turning any object in VP into a \enquote{defective
goal} and forcing it to be silent.

For object-clitic constructions in languages of the \ili{Romance} type, in which
there is a clear formal identity between object \isi{clitics} and definite
determiners, we will adopt an analysis in which the clitic is a D attached to
\emph{v} -- very much as in the analysis of Type III noun-incorporation
constructions in \Cref{sec:16.2}. The syntax of object-clitic constructions
thus looks as in~\eqref{200}, where the \emph{v}-adjoined D\is{adjunction} is associated with a
nominal constituent (some extended projection of N, \enquote{xNP}; in
French~\eqref{1b} this is φ{}P, controlling φ{}-agreement with the
participle, but in \ili{Romance} varieties without clitic agreement it may be just
\emph{n}P) in the object-of-V position that is a \enquote{defective goal} for
the D+\emph{v} probe. Since the \ili{Romance} languages have determiners, the
exponent of the D attached to \emph{v} will be a definite article (\emph{les}
in~\eqref{1b}), not a lexical noun (as in Type III/IV noun-incorporation
languages; recall \Cref{sec:16.2.4}).

%\setcounter{ExNo}{19}
%Example (20)
\ea  {}[\tss{\emph{v}P} [\tss{\emph{v}} D$^i$\tss{\{ D, φ, [+N]\}} [ \hspace{-.7ex}\emph{v}\tss{\{[+V], \Acc{}, \ldots{}\}}]]\tss{\{\{D, φ, [+N]\}, \{[+V], \Acc{}, \ldots{}\}\}} [\tss{VP} V xNP]] \label{200}
\z
In light of our discussion of the syntax of Type III noun
incorporation,~\eqref{200} leads us to expect that the clitic should not be able
to be associated with any overt material in the external syntax. This is
certainly not dramatically inaccurate -- but \isi{clitic doubling} does exist (see
e.g.\ Rioplatense \ili{Spanish} \eqref{21}, from \citealt[32]{jaeggli86}), and needs to
be accounted for.\largerpage[-4]

\ea \ili{Spanish}
    \sn\gll lo vimos a Juan\label{21}\\
    we saw P\textsubscript{DAT} Juan\\
    \glt \enquote*{we saw Juan}
\z
When D(=\Cl{})+\emph{v} co-occurs with an object, as in \isi{clitic doubling}
constructions, the associate of the clitic cannot be placed anywhere in the
complement of \emph{v}, c-commanded by D+\emph{v}. Clitic doubling must instead
involve the placement of the associate in a position outside the c-command
domain of \emph{v} -- arguably the very same position used in
\enquote{differential object marking} (\glsunset{DOM}\gls{DOM}) and
\enquote{object shift} constructions. The fact that in \ili{Spanish} the associate of
a clitic in a \isi{clitic doubling} construction is adorned with the same marker (the
dative preposition \emph{a}) as a DOM-object goes along with this directly. We
identify the spell-out position of the associate of the clitic in clitic
doubling constructions as an outer specifier of \emph{v}P, as
in~\eqref{22}.\footnote{It is entirely possible that the DOM position is the
    specifier of a functional projection outside \emph{v}P (rather than an
    outer Spec\emph{v}P). See e.g. \cite{manzinifranco}  for a concrete
    proposal which also sheds light on the function of the prepositional
    element \emph{a}. The fact that this element may be omitted in certain
    clitic-doubling varieties (e.g. Porte\~no \ili{Spanish};
    \citealt[399--400]{suner88}) seems to us not to affect the proposal
    in~\eqref{22}: whether or not xNP is marked by a prepositional element is a
low-level point of variation, not a core-syntactic one. We thank a reviewer for
raising this point as well as the issue addressed in the next paragraph in the
main text.}

\ea  {}[\tss{\emph{v}P} [\tss{xNP} associate]$_i$ [\tss{\emph{v}P} [\tss{\emph{v}} [\tss{D} D=\Cl{}] [ \hspace{-.7ex}\emph{v}]] [\tss{VP} \ldots{} \sout{xNP} \ldots{}]]]\label{22}
\z
Note that the clitic, in its \emph{v}-adjoined position,\is{adjunction} does not receive a
θ{}-role from V. The associate must hence be the thematic member of
the clitic-doubling complex. This compels xNP to bind a silent copy in a
θ{}-position inside VP. The θ{}-role that xNP's silent copy
receives does not have to be one assigned by V: as \cite{sportiche96} points
out (citing \ili{Greek} examples from Schneider-Zioga's work), clitic-doubled objects
can be subjects of (small) clauses in V's complement. This is unproblematic for
our proposal, as long as the spell-out position of the associate is outside
\emph{v}'s command.

Placement of a \enquote{double} of the incorporated object in a position outside the
search domain of the \emph{v} probe is a logical possibility for
noun-incorporating languages as well. As \citet[165]{bakeretal05} point out
(following \citealt{baker96}), doubling is indeed a different matter from
\enquote{modifier stranding} in noun-incorporating languages:

\begin{quotation}
All polysynthetic languages allow overt NPs to be dislocated, standing in a
relation of resumption to pronouns expressed as agreement morphemes on the
verb. Some languages expand upon this, allowing dislocated NPs to stand in a
relation of resumption to \ldots{} an IN [incorporated noun] as
well.\footnote{What Baker calls \enquote{resumption}, we would prefer to refer
    to as \enquote{specification}. The \enquote{double} is typically more
    specific than the incorporate. The relation between the two has often been
    likened to classifier constructions\is{classifiers} -- both \cite{mithun84} and
\cite{rosen89} appeal to this notion. It seems to us that
\enquote{specification} is a more appropriate term, not raising expectations
about fundamental similarities with complex noun phrases involving
\isi{classifiers}.}
\end{quotation}

Among both Type III and Type IV noun-incorporating languages (which differ with
respect to the legitimacy of \enquote{modifier stranding}), we find cases in which
the incorporated noun can be \enquote{doubled} by a noun phrase in the external
syntax that is descriptively richer than the incorporated element.  Like Baker,
we treat these \enquote{doubles} as being located outside the c-command domain
of \emph{v} (i.e., outside VP). They can be in a dislocated position (an
\=A-position in the left or right periphery), or serve as appositions, or
function as DOM-objects \`a la~\eqref{22}.

%%%%%%%%%%%%%%%%%%%%%%%%%%%%%%%%%
%%%%%%%%%%%%%%%%%%%%%%%%%%%%%%%%%
\section{Object pro-drop and defective goals}
\label{sec:16.4}

In many of the empirical cases reviewed so far in this paper, adjunction of a\is{adjunction}
nominal element to \emph{v} turns \emph{v} into a \enquote{super-probe}: a probe
whose feature content is a superset of that of the goal, which is thereby
declared defective in Roberts' (\citeyear{Roberts2010}) sense of the term.
Imagine now that there could be languages, or situations within languages, in
which \emph{v}  is a featural superset of the goal all by itself, without the
help of a nominal element attached to it. Concretely, imagine a situation in
which \emph{v}  in \eqref{23} possesses all of the formal features
\{α{}FF\} borne by the object-DP. Will this turn the object into a
defective goal, forcing it to be silent?

\ea  {}[\tss{\emph{v}P} \emph{v}\tss{\{α FF, \ldots{}\}} [\tss{VP} V [\tss{DP} D\tss{\{α FF\}} \ldots{}] \ldots{}]] \label{23}
\z
Whenever DP in~\eqref{23} is not a common-noun phrase with idiosyncratic,
encyclopedic properties that are not included in the feature bundle
\{α{}FF\} possessed by \emph{v} (more on this at the end of this
section), we cannot prevent the silencing of DP in this structure: DP is a
proper featural subset of \emph{v} and c-commanded by \emph{v}. This will then
be a case where Agree between \emph{v} and the object, the latter a defective
goal, leads to pure silence in the object position. This reads exactly like the
description of object pro-drop licensed in the absence of a clitic: in
languages whose \emph{v} has such featural wealth as to make it a superset of
the object (with at least some of the object's features spelled out on the
verb, in the form of agreement morphology), it licenses the dropping of the
object by turning the object into its \isi{defective goal}.

For languages that have object \isi{clitics} but no (general) object pro-drop, it is
possible for the object to be silenced \emph{only} when it is the associate of
a D attached to \emph{v}: only the presence of this D (the clitic) gives
\emph{v} the morphological feature content that makes it a featural superset of
D's associate φ{}P in the object position.\largerpage[-3]

\vspace{-1\baselineskip}
For languages whose inflected \emph{v} by itself is rich enough to take the
object as a \isi{defective goal}, we will want any overt objects to be outside the
c-command domain of \emph{v} -- in the \enquote{DOM} position in~\eqref{22}, or in an
\=A-position elsewhere in the tree. The silent object inside VP is recoverable
by the local c-command relation with the coindexed object outside VP. The
subsective probe--goal relation between \emph{v} and the VP-internal object
guarantees the latter's silence.\footnote{A reviewer asks how this account of
    object pro-drop languages can allow such languages to have non-specific
    lexical objects, which are not expected to be positionable in the \enquote{DOM}
    position. If in a particular object-drop language \emph{v} is
    \emph{systematically} in possession of all of the formal features borne by
    the object, non-specific objects will always be silent, and overt objects
    will always be interpreted specifically. There may be languages that work
    like this -- languages in which the verb will need to be antipassivised in
    order for a non-specific \enquote{object} to be introduced. But our proposal
    does not predict that all object-drop languages should work this way: in
    languages in which \emph{v} \textsc{can} possess all the formal features of
    the object, there is no reason to assume that it \textsc{must}, under all
    circumstances. Objects can be spelled out in VP and be overt whenever they
are not defective goals -- i.e., whenever \emph{v} does not bear all of the
object’s features.}

What are the features that can be included in the \{α{}FF\} on
\emph{v} in~\eqref{23}? Obviously the familiar φ{}-features -- but
probably also idiosyncratic lexical properties such as [edible] or [spherical].
Such lexical properties of roots are addressed by the functional heads\is{functional items} within
the extended projection of the nominal root: \isi{classifiers} are typically highly
sensitive to geometric properties such as [spherical], for instance. These are
also implicated in selectional restrictions: [edible] is relevant for the
object of verbs like \emph{eat}; [spherical] is for the internal argument of
verbs such as \emph{roll}. Such selectional restrictions are idiosyncratic
properties of individual roots, hence most likely the province of V. But V is
not a probe, so if selection involves a probe--goal dependency (which is not
necessarily the case but not seldom assumed), \emph{v} will be the probe in the
case of the \enquote{object of} relation: \emph{v} will inherit the relevant
selectional features from the root, and take care of their checking. A root
such as \emph{eat} will then combine with a \emph{v} specified for the feature
[edible], requiring that the object bear the matching feature; similarly, the
\emph{v} combining with the root \emph{roll} will be specified for [spherical].

More microscopic encyclopedic properties of objects (such as sweet or tart,
soft or hard, tender or chewy, for objects of \emph{eat}; bouncy or not for
objects of \emph{roll}) are not usually active in selectional relations:
\emph{eat} cares about its object being edible but not about its sweetness or
hardness; a classifier\is{classifiers} for spherical objects combines equally well with bouncy
and non-bouncy spheres. In a late insertion theory, these encyclopedic
properties are added only at spell-out, not fed into the syntax, and never
involved in probe--goal relations or Roberts-style silencing under
defectiveness.

The defective probe--goal approach to object drop allows the silent object of
verbs such as \emph{eat} to be specified as [edible], and that of \emph{roll}
as [spherical], as desired: a dropped object must meet the verb's selectional
restrictions. But more specific encyclopedic properties of the dropped object
are not morphosyntactically recoverable. When such encyclopedic features are
not retrievable from the surrounding discourse, they must be made explicit in
the form of an overt object. In object pro-drop languages, that object must be
located outside the probing domain of \emph{v}, for otherwise it would be a
morphosyntactically \isi{defective goal} for \emph{v}, destined to silence. The
\enquote{DOM} position in~\eqref{22} or some \=A-position elsewhere in the tree will
be the syntactic locus in languages sanctioning object pro-drop for overt
objects whose formal features (i.e.,  \{α{}FF\} in~\eqref{23}) match
those of \emph{v}.

%%%%%%%%%%%%%%%%%%%%%%%%%%%%%%%%%
%%%%%%%%%%%%%%%%%%%%%%%%%%%%%%%%%
\section{Definiteness agreement and person}
\label{sec:16.5}

For the so-called \enquote{definite/objective conjunction} of Hungarian, illustrated
in~\eqref{24}, an analysis can be proposed along the lines of the approach to
Romance-style object cliticisation\is{clitics} taken above.\footnote{In Den
    \cite{dikkenkeneseifs}, an extended argument is presented for the clitic
    status of \enquote{definite agreement} in \ili{Hungarian} (as well as Proto-Uralic).}

\ea \label{24}\ili{Hungarian}
    \ea \gll l\'at-j-\emph{a} (\emph{\H{o}t}) / *(\emph{\H{o}ket})\\
    see-\textsc{j}-\Def{} (s)he.\Acc{} {} \phantom{*(}they.\Acc{}\\
    \glt \enquote*{(s)he sees him/her/them}
     \ex \gll   l\'at-t-\emph{a} (\emph{\H{o}t}) / *(\emph{\H{o}ket})\\
    see-\Pst-\Def{} (s)he.\Acc{} {} \phantom{*(}they.\Acc{}\\
    \glt \enquote*{(s)he saw him/her/them}
    \z
\z
On such an approach, the \enquote{definiteness agreement} marker on the verb is a D
attached to \emph{v} (undergoing vowel harmony with the verb). When no overt
object is present, the D+\emph{v} complex is associated with a defective goal
in VP and licenses its silence -- this is what is usually referred to for
Hungarian as \enquote{object pro-drop}, now actually assimilated to object
cliticisation, with D attached to \emph{v}.

It is interesting to note that number is not recoverable from D=\emph{a}: the
Hungarian definite article has no plural form (\emph{az \'ev} \enquote*{the year},
\emph{az \'evek} \enquote*{the year.\Pl{}}; not *\emph{azok \'evek}
\enquote*{the.\Pl{} year.\Pl{}}).  Only definiteness and (default) third
person are retrievable from D. So the combination of D=\emph{a} and \emph{v}
cannot take the third person plural pronoun as a \isi{defective goal} because this
goal has something that D=\emph{a} does not have: number (\emph{a} represents D
and person, not number). As a consequence, third person plural objects cannot
be dropped in Hungarian: *(\emph{\H{o}ket}) in~\eqref{24}. This falls out
directly from the \isi{defective goal} hypothesis.

Interestingly, first and second person object pronouns can be dropped both in
the singular and in the plural, even though nothing about them is recoverable
from verbal inflection (from the subjective/indefinite conjugation):

\ea \label{25}\ili{Hungarian}
    \ea \gll lát (\emph{engem}) / (\emph{minket})\\
    see.\Indef{} \phantom{(}me {} \phantom{(}us\\
    \glt \enquote*{(s)he sees me/us}
    \ex \gll látott (\emph{engem}) / (\emph{minket})\\
    saw.\Indef{} \phantom{(}me {} \phantom{(}us\\
    \glt \enquote*{(s)he saw me/us}
    \z
\z

\ea \label{26}\ili{Hungarian}
    \ea \gll lát (\emph{téged}) / (\emph{titeket})\\
    see.\Indef{} \phantom{(}you\textsubscript{\Sg{}} / \phantom{(}you\textsubscript{\Pl}\\
    \glt \enquote*{(s)he sees you\textsubscript{\textsc{sg/pl}}}
    \ex \gll látott (\emph{téged}) / (\emph{titeket})\\
    saw.\Indef{} \phantom{(}you\textsubscript{\Sg{}} / \phantom{(}you\textsubscript{\Pl}\\
    \glt \enquote*{(s)he saw you\textsubscript{\textsc{sg/pl}}}
    \z
\z
In light of the preceding discussion, we are led to conclude that~\eqref{25}
and~\eqref{26} do not involve a defective probe--goal relation. The dropping of
first and second person object pronouns must be licensed discursively; it
cannot be sanctioned morphosyntactically. More generally, Baker's
(\citeyear[877, fn.\ 3]{baker11}) conjecture that \enquote{agreement for first and
second person can never take place under mere Agree}, but requires the
Spec-Head relation (a conjecture that is confirmed and derived from a
structural representation of the feature [person] in Den
\citealt{dikkengetegratalk}) leads us to draw the conclusion that the dropping
of person-marked objects can never involve a Roberts-style defective probe--goal
relation when the object is structurally represented inside \emph{v}'s
complement.

When a person-marked object is structurally represented in the specifier
position of \emph{v}P (the \enquote{DOM} position in~\eqref{22}), the object's
silence can be morphosyntactically licensed by \emph{v} if \emph{v}'s feature
set includes [person] and if the Spec-Head relation is a probe--goal
configuration (\enquote{upward Agree} or \enquote{downward valuation};
\citealt{BjoZei2019}, \citealt{premingerpolinsky}). Whenever \emph{v} does not
probe the person-marked object, it can remain unexpressed only if the discourse
makes it recoverable, as in the \ili{Hungarian} case illustrated above.

\section{Agreement inside extended projections}
\label{sec:16.6}

In configurations involving an object that serves as a \isi{defective goal}, the
complex \emph{v} is a \enquote{super-probe} for the \isi{defective goal} inside VP,
sanctioning its silence and giving rise to the effect of \isi{head movement}
(cliticisation or \isi{noun incorporation}). Inside the complex noun phrase
in~\eqref{27}, D is also a featural superset of the functional projections below
it: D has a specification for the feature [D(efinite)] as well as for the
φ{}- and categorial features of the complex noun phrase (which are
visible on DP).

\ea  {}[\tss{DP} D\tss{\{D, φ, [+N]\}} [\tss{φP} φ{}\tss{\{φ, [+N]\}}
[\tss{\emph{n}P} \emph{n}\tss{\{[+N]\}} [\tss{NP} N]]]\label{27}
\z
Similarly, in the clause, C has a specification for [force] as well as for the
φ{}- and categorial features of the finite verb. But plainly, the fact
that D and C are featural supersets of the functional projections in their
complement does not force the latter to be silent. Why not?\largerpage[-4]

Although D and C are featural supersets of the φ{}P and TP in their
complement, they do not probe the feature bundles in the heads of their
complements. D and φ{} are part of one and the same extended projection,
and so are C and T. While functional heads\is{functional items} in a continuous extended projection
are arguably always a proper featural superset of the functional heads\is{functional items} they
immediately c-command,\footnote{This will provide a very simple explanation for
    the fact that the complement of C/D is immobile (i.e., cannot engage in
    filler-gap dependencies: cf. *[\emph{John is smart}], \emph{I don't think
    that}, and *[\emph{book}], \emph{I didn’t read the}). On the text approach,
    this becomes a specificity effect. The higher FP (i.e., CP or DP) has all
    the features of the lower FP (TP, φ{}P); therefore, if an external
    probe seeks to engage in an Agree relation for the features shared by the
    two FPs, it will pick the more inclusive and more directly accessible of
    the two phrases (i.e., the higher one) as its goal. (Cases like \emph{books
        I have none} (Lord Mansfield in the House of Lords; 18th century)
    do not involve subextraction -- the \enquote{stranded} portion of the DP in
these cases can always constitute a noun phrase by itself: contrast \emph{books
I have none} with *\emph{books I have no}.)} they do not stand in a probe--goal
or selectional relation with them. The various functional heads\is{functional items} in the extended
projection of a head all belong to the same family, and have matching genes
because of this family relation. No functional head\is{functional items} can establish a probe--goal
relation with a lower functional head\is{functional items} in the same extended projection because
the feature content of the lower functional head\is{functional items} could not have been disjoint
from that of the higher functional head\is{functional items}. By definition there is feature
matching throughout the spine of an extended projection. Because feature
matching is thus guaranteed, probing is generally futile.\footnote{For VP
    \isi{topicalization} (placement of an extended projection of V in the specifier
    position of a functional category in the clausal left periphery), no
    exception to this general statement needs to be made if, as is plausible,
    the clause is a combination of two extended projections, one of V (incl.
    \emph{v} and presumably also a functional head\is{functional items} for Aktionsart aspect) and
    one of T (incl. C and the information-structural F-cats familiar from
    cartographic work). The need to split the full clause into two separate
    extended projections becomes compelling once it turns out that elements in
    the functional sequence of the high left periphery (outside TP) rear their
    heads also in the low left periphery (between T and \emph{v}P). Thus, if it is
    true that TopP occurs both outside TP and outside \emph{v}P (see
    \citealt{belletti04} for relevant discussion of low topic positions), and
    if it is true (as the facts of \ili{Hungarian} suggest) that within the
    functional sequence of a single extended projection TopP can never occur
    below FocP, then it must be the case that a low TopP outside \emph{v}P and
    a high TopP outside TP and FocP (see the schematic structure
    in~(i)) belong to different extended projections -- the extended
    projections of V and T, respectively. Any functional head\is{functional items} in the extended
    projection of T is then welcome to probe for some extended projection of V.
    VP \isi{topicalization} thus does not involve probing within a single extended
    projection.

\begin{exe}
    \exi{(i)} {}[\tss{CP} C [\tss{TopP} Top [\tss{FocP} Foc \ldots{} [\tss{TP}
    T \ldots{} [\tss{TopP} Top \ldots{} [\tss{\emph{v}P} \emph{v} [\tss{VP} V \ldots{}]]]]]]] \label{fnexx}
\end{exe}
The kind of VP-raising at work in predicate-initial languages such as
\ili{Niuean},
for which \cite{massam01pred} argues that T is the probe and SpecTP the
landing-site, is also unproblematic from this perspective: with T defining its
own extended projection, such VP-raising does not involve a probe--goal relation
within one single extended projection.}

On the other hand, across different extended projections, feature matching is
not guaranteed: it can arise only as a function of a probe--goal relation
between the terms of these different extended projections. One can refer to
both the feature matching within extended projection and the feature matching
resulting from probing agreement by the cover term \enquote{Agree}.\footnote{If one
    finds it confusing to apply the term \enquote{Agree} both to feature matching
    under probing and to the definitional feature matching found within
extended projections, one could alternatively express the feature sharing found
in functional sequences in terms of spans, a notion introduced in the
nanosyntax literature and exploited in \Cref{sec:16.2.4}.} But because the
former kind of feature matching does not involve a probe--goal relation, it does
not lead to chain formation and concomitant chain reduction (i.e., silencing of
the goal, in the case of a \isi{defective goal}).

In the complex noun phrase \emph{les filles} in~\eqref{1a}, repeated below
(along with \eqref{1b}) and analysed as in~\eqref{27}, D and φ{} are part of a
single extended projection, so feature sharing is guaranteed, and no
probe--goal relations are established within this complex object. A functional
head F$_n$ in an extended projection of some lexical root cannot engage in a
probe--goal relation with a functional head\is{functional items} F\tss{\emph{n−1}} in its immediate
c-command domain, so the D-head in~\eqref{27} cannot probe φ{}. Despite the
fact that in the structure in~\eqref{27} φ{}P is a proper featural subset of
DP, we are not dealing with a \isi{defective goal} because there is no probing among
the members of a single extended projection. φ{}P is not forced to be silent in
\eqref{1a}, therefore.

\begin{exe}
    \exi{(1)}\ili{French}
    \begin{xlist}
    \ex \gll j'ai surpris \emph{les} filles \\
        {I have} surprised the girls\\
    \glt \enquote*{I surprised the girls}
    \ex \gll je \emph{les} ai surprises\\
    I them have surprised.\glossF.\Pl{}\\
    \glt \enquote*{I surprised  them} (said of feminine direct object)
    \end{xlist}
\end{exe}
For~\eqref{1b}, it might a priori seem attractive to represent \emph{les} as the
exponent of D inside a complex noun phrase in which the complement of D remains
silent:~\eqref{28} achieves a generalisation over definite common noun phrases
and object \isi{clitics} that accounts for the form-identity of the definite article
and the clitic.

%\setcounter{ExNo}{27}
%(28)
\ea \begin{tabular}[t]{ll}
{} & [\textsubscript{DP} D=\emph{les} [\textsubscript{φ{}P} φ{} [\textsubscript{nP} \emph{n} [\textsubscript{NP} N]]]\\
(1a): & φ{}P = \emph{filles}\\
(1b): & φ{}P = $\varnothing$\\
\end{tabular}\label{28}
\z
But~\eqref{28} raises the questions of why \emph{les}, when unaccompanied by
any overt material in φ{}P, must cliticise, how it goes about the business of
cliticising to a verb, and, perhaps most fundamentally, how the φ{}P in
\eqref{1b} can be silenced in the first place. Since this φ{}P is part of the
same extended projection as D, and since Roberts' notion of \enquote{defective
goal} is not applicable within the confines of an extended projection (because
no probe--goal relations are established among the members of the functional
sequence that constitutes the extended projection), it cannot be that φ{}P
in~\eqref{1b} is silenced due to its being a \isi{defective goal}.

So the occurrence of \emph{les} by itself, as an object clitic that is a
portmanteau for D and φ{}, cannot be accounted for straightforwardly if
the clitic is taken to originate in the object position. This emphasises the
need to approach \isi{clitics} in a manner different from the one presented
in~\eqref{1b}, and seems to make it inevitable to base-generate the clitic
outside VP (on \emph{v}, as in~\eqref{200}), where it can be the exponent of
D+φ{} and form a discontinuous object with a \isi{defective goal} in the
θ{}-position inside VP\@. This is the essence of Roberts'
(\citeyear{Roberts2010}) approach to object cliticisation, which we have
defended, refined and expanded in this paper to cover not just cliticisation
but also the full range of \isi{noun incorporation} constructions reported in the
literature on polysynthetic languages.

\printchapterglossary{}

\section*{Acknowledgements}
We would like to express our gratitude to George Soros, and to two anonymous
reviewers for their perceptive and constructive comments on an earlier version
of this paper. The work reported here has been  supported in part by Dékány's
HAS Premium Postdoctoral Grant, which is hereby gratefully acknowledged. It is
with profound appreciation for his many fundamental contributions to linguistic
analysis that we offer these notes to the wonderful colleague who inspired
them, one of the true giants of generative linguistics in Europe.
%

\printbibliography[heading=subbibliography,notkeyword=this]

\end{refcontext}
\end{document}
